
\section{Introduction}%
\label{sec:Introduction}

% Software development is a highly collaborative activity. 
Development teams typically collaborate with the aid of a version control system (VCS) like Git, Subversion, or Darcs~\cite{DBLP:conf/haskell/Roundy05}. 
These systems maintain a branching history of \emph{commits} to a source code repository, each consisting of a \emph{patch} together with various metadata, e.g. a human-readable commit message. Patches are imperative programs written in a 
\emph{patch language} defining a set of primitive editing commands.
The standard POSIX \li{patch} language, for example, specifies commands for inserting and deleting specified lines of text at specified line numbers within a file. 

Developers do not typically express program edits using the patch language directly, nor in any case do version control systems typically have access to a log of the developer's edits. Instead, version control systems must \emph{synthesize} patches from the file system state using heuristic algorithms, such as the classic \li{diff} algorithm that synthesizes a patch that minimizes the edit distance between two file system states \cite{DiffAlgorithm}.

When two patches, developed concurrently in branches based on a common ancestral commit, must be merged, version control systems deploy a \emph{three-way merge algorithm}. The standard approach is to apply the \emph{local patch} first, then modify the \emph{remote patch} by shifting its line numbers to account for the local patch's line insertions and deletions. This algorithm is an \emph{operational transform}~\cite{DBLP:conf/sigmod/EllisG89}. 
Character-level operational transforms are similarly used for real-time collaborative editing, e.g. in tools like Google Docs and in Visual Studio Code's Live Share feature. 

\paragraph{Problem 1: Merging Sensibly} 
When merging patches, conflicts (e.g. due to different modifications to the same location) are unavoidable. However, 
standard three-way merge algorithms for line-based patch languages commonly identify spurious conflicts, fail to identify legitimate conflicts, or silently duplicate or misplace code. Let us briefly review some classic problems with these systems. 
%Appendix~\ref{appendix:merge-problems} shows examples of the problematic default behavior of Git in each of these scenarios.\todo{write this, can move to supplement if needed}

The \textbf{granularity problem} 
arises when both commits make edits to different locations within a single line of code. For example, one patch might rename a function argument while the other adds a new argument. 
Similarly, if one patch renames an type while another 
makes unrelated changes to code that references that type,
there will be conflicts at every line of code shared between the two patches. 
% A notable special case is the \textbf{wrapping problem}, which arises when one patch changes the nesting of code structures, e.g. by wrapping a code block within a new control flow construct, thereby changing the indentation of every line in the block. These changes can cause conflicts.
 
The \textbf{relocation conflict problem} arises when two patches relocate a code block to different locations. 
Standard patch languages operationalize code relocation as simply a deletion paired with an insertion. The merge will therefore fail to identify this legitimate location conflict and instead silently duplicate the code block at both locations.

The \textbf{relocation modification problem} arises when one commit relocates a code block that another commit modifies. 
A naive approach would silently leave the modifications at the original location. A more sophisticated block-based approach, like that deployed by \li{git}, might indicate a conflict when an insertion occurs within the bounds of a code block that has been deleted.

These classic problems have motivated research into richer patch languages, more sophisticated patch synthesis algorithms, and corresponding improvements to three-way merge algorithms. 
For example, systems like Git address the nesting problem by allowing indentation changes from one patch to be merged with other changes that do not modify indentation. 
More sophisticated systems deploy parsers and tree differencing algorithms~\cite{DBLP:conf/sigmod/ChawatheG97, DBLP:journals/tse/FluriWPG07,DBLP:conf/kbse/FalleriMBMM14,DBLP:conf/doceng/Lindholm04,DBLP:conf/fase/NguyenNPN10,DBLP:journals/scp/SchwagerlUW15} to better
address the granularity problem.

Addressing the relocation-related problems is more difficult.
A common approach is to enrich the patch language to make code relocation a primitive command.
However, correctly synthesizing code relocation commands given only the initial and final state of the repository requires heuristics. 
The typical approach is to assume that a matching deletion and insertion is due to code relocation. However, relocated code is often also modified. 
In these cases, it is difficult to determine whether a deletion and a similar but not identical insertion are related by relocation, rather than coincidental code similarity. 
Developers often intentionally copy and then modify code, so there may be multiple partially matching insertions for a given deletion and there is no clear way to decide which, if any, are related by relocation. 
The developer's actual actions, e.g. cuts, copies, and pastes, are not persisted into the file system, nor are there persistent identifiers associated with code structures represented as text, so text-file-based systems have no choice but to deploy imperfect heuristics.

\paragraph{Contribution 1: Grove: A Collaborative Structure Editor Calculus} 
This paper considers the problem of collaborative editing for \emph{structure editors},
which eschew text editing. Instead, developers code by applying tree edit actions directly to a continuously evolving \emph{program sketch}, i.e. a syntax tree with holes, shown projected visually in various ways to the developer. 
Structure editing has been studied since the 1980s with the Cornell Program Synthesizer~\cite{DBLP:journals/cacm/TeitelbaumR81}
and research continues to this day, with numerous active projects including Scratch~\cite{maloney2010scratch}, Snap!, and other block-based editors (which are widely used in educational and end-user programming settings), Jetbrains MPS (which has been deployed in industry)~\cite{voelter2011language}, and Hazel (a live functional programming environment rooted in a structure editor calculus called Hazelnut, which serves as 
an active research platform)~\cite{DBLP:conf/popl/OmarVHAH17}. 

Inspired by Hazelnut, this paper introduces Grove, a \emph{collaborative structure editor calculus} for arbitrary syntax trees that does not suffer from the problems just outlined. 
This is in large part due to a substantial simplification of the overall collaborative
editing architecture. In particular, we \emph{eliminate patch synthesis (i.e. diff algorithms) entirely}, 
instead deriving patches directly from the log of edits performed by the developer. 
We also \emph{eliminate the need for three-way merge algorithms (i.e. operational transforms) entirely}.
Instead, we define the patch language such that all edits commute, 
so remote patches can be applied without transformation. 
We prove a \emph{convergence theorem} that ensures that branches of a repository will converge to the same state 
when the same set of patches are applied, regardless of the order in which they are applied.
The patch language forms what is known as a \emph{commutative replicated data type (CmRDT)}~\cite{preguicca2018conflict,shapiro2011conflict}.

Defining a structure editor calculus that supports code insertion, deletion, and relocation using only commutative edits, and avoiding the problems outlined above, is not trivial. The Hazelnut action language is neither commutative nor does it support relocation. 
Relocation is particularly challenging because of the potential for relocation conflicts,
including the potential for cyclic relocation (when one commit relocates node $A$ beneath node $B$, and the other \emph{vice versa}). 

This need to represent conflicting states means that we cannot use a single syntax tree as Grove's 
core data structure. Instead, we use a directed labeled multi-graph. A vertex corresponds to a syntax tree node and is labeled with a \emph{unique identifier (UID)} and a \emph{constructor}, e.g. \li{Plus}. An edge is also labeled with a UID and establishes a parent-child relationships at a labeled \emph{position} for the parent vertex's constructor, e.g. at the \li{L} or \li{R} position of the \li{Plus} constructor. We refer to a parent vertex and position collectively as a \emph{location}.

The \emph{Grove patch language} is quite simple: a patch can insert or delete an edge (which might cause the creation of a mentioned vertex if it did not already exist). 
To ensure commutativity, deletion of an edge is permanent, i.e.
the edges form a two-phase set (2P-Set) CmRDT. The edit actions that users perform are given meaning by a straightforward translation to a graph patch. 
Relocation simply translates to 
edge deletion and insertion. Critically, vertices are \emph{not} deleted during relocation,
so we do not need to deploy heuristics to identify relocated nodes.

Structure editors are designed to provide editing affordances for trees, not graphs where there may be any number of children at a given location, vertices may have any number of parents, and where there may be cycles. 
To support conventional tree-based structure editing, 
we define a \emph{decomposition} of our program graph into a \emph{grove}, which is a set of 
programs with holes, local conflicts, and conflict references between them to account for motifs that arise when editing collaboratively. 
In particular, a location with no out-edges decomposes to a \emph{hole}. 
More than one out-edge at a given location decomposes to an explicitly represented \emph{local conflict}. 
More than one in-edge indicates that a vertex has a \emph{relocation conflict}, so we leave a \emph{relocation conflict reference} at each of the conflicting locations. 
Finally, cycles are broken during decomposition by leaving a \emph{unicycle conflict reference} at an arbitrary (but deterministically chosen) edge. 
Resolving these various conflicts simply requires 
manipulating these constructs like any other syntactic construct, e.g. deleting all but one relocation conflict reference to determine a unique location for a node. 

\paragraph{Problem 2: Semantic Gaps During Conflict Resolution}
When working with traditional version control systems,
resolving conflicts can take time and require reasoning about 
syntax, types, and program behavior.
Traditionally, however, conflicts are indicated by inserting extra-linguistic markers
into files. 
These markers are not typically understood by the parser, and 
because they include conflicting alternatives, they cannot generally be removed 
or concatenated to result in a sensible program. 
Consequently, language services that require a well-formed, meaningful program
(e.g. type error localization, go-to-definition, live evaluation and so on) 
either fail to operate or exhibit gaps in service, e.g. because they are relying on 
data from a compile prior to the merge attempt. Developers are left to reason without the aid of much of their tooling during conflict resolution. This is an instance of the more general \textbf{semantic gap problem} when programming tools encounter incomplete programs~\cite{DBLP:conf/snapl/OmarVHSGAH17}. 

The previous work on the Hazelnut structure editor calculus addresses the semantic gap problem in the single-user setting by 
defining a type system and type error localization system (the \emph{marked lambda calculus}) for incomplete programs, i.e. programs with holes~\cite{DBLP:journals/pacmpl/ZhaoMDBPO24}. 
Notably, type error localization is proven to be \emph{total}, i.e. the system is able to assign static meaning to \emph{every} syntactically well-formed expression by inserting \emph{marks} to localize errors. Marking employs local type inference as codified by a \emph{bidirectional type system}~\cite{dunfield2019} as well as \emph{gradual typing}~\cite{siek2015}, i.e. the theory of \emph{type holes}, to recover from situations where type errors make it impossible to 
determine a known type. 
A separate type hole filling phase deploys unification-based (i.e. non-local) type inference to fill type holes when possible, 
or allows the user to interactively select from hole fillings that partially satisfy generated constraints when there are type conflicts.


\paragraph{Contribution 2: Total Type Error Localization and Recovery for Groves}

This paper extends this work to develop a total type error localization system for groves, introduced in Contribution 1 above as the result of decomposing a commutatively edited graph into a set of terms with empty holes, local conflicts, relocation conflict references, and unicycle conflict references. This paper develops a type (and type error localization) discipline for handling these novel constructs. We follow the marked lambda calculus in rooting our \emph{marked grove calculus} in bidirectional type checking, deploying gradual typing when conflicts do not allow a single type to be inferred, and then layering on a unification-based type inference system to opportunistically fill holes or suggest partial solutions when there are conflicting types due to conflicting syntax.

\paragraph{Paper Outline}

\autoref{sec:Grove By Example} introduces Grove by example, demonstrating its behavior in each of the problematic scenarios named above. 
\autoref{sec:Formalism} then formally defines Grove's graph structure, commutative patch language, grove decomposition procedure, and edit action language. We establish key metatheoretic properties using the Agda proof assistant. 
\autoref{sec:Implementation} describes our implementation of the Grove Workbench, which is defined modularly to allow it to be instantiated with arbitrary syntax trees. 
\autoref{sec:Marking System} instantiates Grove with a simply typed lambda calculus, then defines a bidirectional type and type localization system for groves and proves totality and other key metatheoretic properties using the Agda proof assistant. 
\autoref{sec:Related Work} reviews related work in more detail. 
\autoref{sec:Discussion and Conclusion} concludes with a discussion and directions for future work on collaborative editing.


% A lot of collaborative editing of other sorts (not necessarily programming) requires a tree model,
% But that problem doesn't seem to have been studied directly (maybe except JSON CRDT).

% A \emph{collaborative structure editor} is an editor that
% (1) allows multiple concurrent users to work on a shared document while also
% (2) providing structure-aware editor services such as projectional editing, syntax highlighting, or automatic code folding.
% %
% Collaborative editing research focuses on the design and implementation of real-time, multi-user, character-based communication systems,
% whereas structure editors typically presume a more complex document schema and then focus on some other aspect of the user experience.
% In both settings, preservation of user intent is a core technical challenge.
% %
% Although collaborative editors and structure editors have overlapping goals (optimal user experience)
% and complementary design challenges (subject-subject versus subject-object harmony),
% To our knowledge, there is no comprehensive, principled account of their combined use.

% Since collaborative editors are essentially distributed systems, existing work tends to focus on extensions to distribution protocols.
% Lots of examples using OT. (Google Docs)
% OT is complex and largely textual.
% OT can make sense for real-time systems: users typically change one character at a time, and instant feedback can help to prevent conflicts.
% On the other hand, OT system designs can be difficult to extend.

% Alternatively, there's CRDT. (Peritext?)
% CRDT is easier to implement but harder to design for.
% There's an All-CRDT editor---it turned out to be not so realistic. (what's it called again?)

% Structure editing has been a recurring theme in computer science literature since at least Engelbart's ``Mother of All Demos.''
% Provides automation for domain experts and reduces the barrier to entry for everyone else.
% Popular for editing programs, i.e., for programming language-specific editors.

% However, modern program editors typically disable editor services (like what?) when the document is not in a consistent state,
% a phenomenon called the ``gap problem.''
% Of course, in the presence of multiple concurrent users, the problem gets worse.

% In a collaborative setting,
% Hazel is a structure editor with support for advanced editing services, (e.g., semantic actions?).





% \newpage




% Motivation:

% - collaborative editing (both synchronous ala Google Docs and asynchronous version control)
% is good and important as computing grows

% - semantic structure editing is good because it solves the gap problem (semantic editor services
% are always available) -- cite Hazelnut papers (talk about holes)

% - previous approaches to collaborative editing have limitations

% - diff/merge-based approaches (trying to solve the inverse problem based on final states --
% you lose the actual actions that were performed and have to reconstruct them or an approx.
% of them i.e., add line/delete line actions -- would need to adapt this to structure editing,
% Some papers have started to look at that, but fundamentally, we don't want to throw away the
% knowledge we have about the edits!)

% - operational transforms (complexity, you have to patch previous actions based on new actions)

% - CRDT-based collaborative editing (that's all been on text, not PL semantics) -- this is good
% because it is relatively simple: you just send all the edits to all the replicas and they are
% convergent by design

% - we want to have the same convergence for a CRDT-based collaborative structure editor that maintains
% the sensibility invariant of Hazelnut, i.e., every editor state has meaning. mention that maintaining sensibility
% allows scaling of semantic editor services in the presence of a large number of collaborators (in contrast,
% using VS Code or other collaborative text editors with large numbers of collaborators means that almost always
% The semantic editor services will be disabled because the program is going to be broken in multiple places
% transiently)

% This is tricky because:

% - some edits might be conflicting -- solve this with "conflict holes"

% - adding cut/paste or delete/restore allows for degenerate programs (cycles, multiple parents, etc.)

% - since we are commutative, we solve both synchronous and async collaborative editing

% - and this resolves issues around merges and conflicts

% - contribution of this paper is to solve these problems from type-theoretic first principles:

% - ...

% - Hazel

% \subsection{Contributions and Paper Organization}%
% \label{sec: Contributions and Paper Organization}
