
\section{Related Work}%
\label{sec:Related Work}

Collaboration requires defining a patch language, a method for synthesizing patches from user actions, and an approach for merging patches.

\subsection{Patch Languages}
There have been many of different designs for patch languages and indeed many imperative data structures and their associated operations can be construed as patch languages in the most general sense. In the context of collaborative coding, patch languages can differ in the data structures they operate on (e.g. line-based text, character-based text, tree-structured data). 
They can also differ in how they identify locations within the data (e.g. by using numeric offsets, unique identifiers, or paths through a tree).
Finally, patch languages differ in which specific actions are supported explicitly. Insertion and deletion are common, while code relocation, copying, undo, and other operations are variously also included.

In this paper, our focus was on syntax trees with holes (i.e. program sketches~\cite{DBLP:conf/aplas/Solar-Lezama09, DBLP:conf/popl/OmarVHAH17}) and explicit conflicts, which we represented as directed graphs. We identify locations using unique IDs. We have a two-level patch language, with a low level graph patch language supporting only edge insertion and deletion and a higher-level user edit action language focused on insertion, deletion, and relocation, with some additional narrative consideration of copying and undo (a fuller account of which we leave to future work).  Our user edit language therefore forms a structure editor calculus, inspired closely by recent work on the Hazelnut structure editor calculus (which did not support relocation)~\cite{DBLP:conf/popl/OmarVHAH17} and patch languages for other tree-based data structures~\cite{DBLP:conf/sigmod/ChawatheG97, DBLP:journals/tse/FluriWPG07,DBLP:conf/kbse/FalleriMBMM14,DBLP:conf/doceng/Lindholm04,DBLP:conf/fase/NguyenNPN10,DBLP:journals/scp/SchwagerlUW15}.

\subsection{Patch Synthesis}
The most common approach to patch synthesis is to use a differencing algorithm to compare two states, e.g. from the file system, to generate a patch. 
The classic \li{diff} algorithm~\cite{DiffAlgorithm}, for example, minimizes edit distance for a patch language involving line insertions and deletions.

In the setting of tree-based editing, there have been a number of tree differencing algorithms described in the literature \cite{DBLP:conf/esa/Klein98,DBLP:journals/tcs/Bille05,DBLP:journals/talg/DemaineMRW09,DBLP:journals/fuin/AratsuHK10,DBLP:conf/sigmod/ChawatheG97, DBLP:journals/tse/FluriWPG07,DBLP:conf/kbse/FalleriMBMM14,DBLP:conf/doceng/Lindholm04,DBLP:conf/fase/NguyenNPN10,DBLP:journals/scp/SchwagerlUW15}. 
As described in \autoref{sec:Introduction}, synthesizing insertions and deletions is well-understood, but synthesizing relocations is more complex and requires heuristics.

The approach we explore in this paper is far simpler: we directly translate from the log of user edit actions to graph patches, without needing a differencing algorithm at all. This is only possible with a structure editor integrated with the version control system, but the benefit of direct visibility into the edit action log is that we do not need heuristics to synthesize relocations.

\subsection{Merging}
\subsubsection{Operational Transforms}
The most common approach to merging concurrently developed patches is to deploy an operational transform~\cite{DBLP:conf/sigmod/EllisG89} whereby locations in a remote patch are modified based on the action of a local patch. Standard three-way merge algorithms in version control systems deploy this approach, as do real-time collaborative editors. There have been a number of papers studying the algebraic properties of merging patches in this style. 

For example, the Darcs~\cite{DBLP:conf/haskell/Roundy05} and Pijul version control systems, like Grove, represent repositories using sets of patches. However, unlike Grove, not all orderings of patches are algebraically sensible (e.g. commutative and associative), so the theory of these systems define and algebraically characterize when operational transforms do satisfy these properties. Recent work on homotopical patch theory \cite{DBLP:journals/jfp/AngiuliMLH16} has similarly developed an abstract algebraic framework for distinguishing sensible merges. 

\subsubsection{CRDTs}

The observation that commutativity is a particularly helpful property when dealing with concurrent systems, including version control systems, has led to the development of a number of data structures centered on commutativity. These are known as CRDTs, which stands variously for \emph{conflict-free}, \emph{convergent} (CvRDT), or \emph{commutative replicated datatypes} (CmRDT) depending on particular details about the operations and the state representation~\cite{shapiro2011conflict}.
Our approach draws directly from this line of work: the Hazel patch language forms a CmRDT on directed graphs, from which we observe that we can derive a CmRDT for trees with explicit conflicts.

There have been other recent efforts to develop CmRDTs for tree data structures~\cite{DBLP:conf/icdcs/PreguicaMSL09, Highly-Available}. However, the focus in this work has been on avoiding conflicts and cycles entirely by applying \emph{ad hoc} heuristics for conflict resolution at merge-time, e.g. using reported timestamps or favoring particular directions in the tree. Our approach instead embraces manual conflict resolution, as is common practice in software projects where arbitrarily losing code is not acceptable.



% \url{https://www.waitingforcode.com/big-data-algorithms/conflict-free-replicated-data-types-flags-graphs-maps/read}



% We are using action based instead of tree diff

% We have to deal with merge

% We have to deal with edits from multiple people

% J. W. Hunt and M. D. McIlroy. 1976. An Algorithm for Differential File Comparison. Technical Report CSTR 41. Bell Laboratories, Murray Hill, NJ.

% Type-directed diffing of structured data \url{https://dl.acm.org/doi/10.1145/3122975.3122976}

%   Approximating Tree Edit Distance through String Edit Distance \url{https://dl.acm.org/doi/10.5555/3118232.3118518}

%   Meaningful change detection in structured data \url{https://dl.acm.org/doi/10.1145/253260.253266}

%   An optimal decomposition algorithm for tree edit distance \url{https://dl.acm.org/doi/10.5555/2394539.2394560}

%   Diff/TS: A Tool for Fine-Grained Structural Change Analysis \url{https://dl.acm.org/doi/10.1109/WCRE.2008.44}

% An efficient algorithm for type-safe structural diffing \url{https://dl.acm.org/doi/10.1145/3341717}

%   Precise Version Control of Trees with Line-Based Version Control Systems \url{https://dl.acm.org/doi/10.1007/978-3-662-54494-5_9}

%   A survey on tree edit distance and related problems \url{https://dl.acm.org/doi/10.1016/j.tcs.2004.12.030}

%   Cycle-aware minimization of acyclic deterministic finite-state automata \url{https://dl.acm.org/doi/10.1016/j.dam.2013.08.003}

%   Computing the Edit-Distance between Unrooted Ordered Trees \url{https://dl.acm.org/doi/10.5555/647908.740125}

%   Type-safe diff for families of datatypes \url{https://dl.acm.org/doi/10.1145/1596614.1596624}

%   A Categorical Theory of Patches \url{https://dl.acm.org/doi/10.1016/j.entcs.2013.09.018}

%   Type-directed diffing of structured data \url{https://dl.acm.org/doi/10.1145/3122975.3122976}

%   The Semantics of Version Control \url{https://dl.acm.org/doi/10.1145/2661136.2661137}

%   The Tree-to-Tree Correction Problem \url{https://dl.acm.org/doi/10.1145/322139.322143}

%   Generic Diff3 for algebraic datatypes \url{https://dl.acm.org/doi/10.1145/2976022.2976026}

% \subsection{Version Control}

% Git \url{https://git-scm.com/}

% Darcs \url{https://darcs.net/}

%   Darcs: distributed version management in haskell \url{https://dl.acm.org/doi/10.1145/1088348.1088349}

% Hg? \url{https://www.mercurial-scm.org/}

% SVN \url{https://subversion.apache.org/}

% Pijul and (Anu is a rewrite of Pijul and seems to have been subsumed into Pijul)

%   \url{https://pijul.org/}
%   \url{https://pijul.org/manual/theory.html}

%   \url{https://tahoe-lafs.org/~zooko/badmerge/simple.html}

% \subsection{Collaborative Editing}

% Collaborative Structure Editing

% SmallTalk collaboration with images

% TouchDevelop papers

% Lots of list-of-chars or list-of-list-of-chars (we ignore these except to discuss them here)

% \subsection{CRDTs}
% (Are we a known CRDT?)

% List of CRTD papers: \url{https://crdt.tech/papers.html}

% Bottom = Tombstone

% \url{https://www.waitingforcode.com/big-data-algorithms/conflict-free-replicated-data-types-flags-graphs-maps/read}
%  - Add-Remove Partial Order data type
%  - 2P2P-Sets
%  - Replicated Growable Array

% \url{https://github.com/PsychoLlama/graph-crdt}
%  - Graph CRDT
%  - Uses a LWW-E-Set

% \url{https://martin.kleppmann.com/2020/07/06/crdt-hard-parts-hydra.html} (overview talk)
%  We don't have interleaving problems because
%    - everything is relative to a specific ID not a position
%    - we don't try to auto resolve
%    - we are tree not list
%    Part three: moving sub tree
%    - Last parent writter wins (prevents cycles)
%      Equivalent to us if we filter multiparent edges, different for cycles, no way to delete?

% A commutative replicated data type for cooperative editing
%   \url{https://hal.inria.fr/inria-00445975/document}
%   Describes TreeDoc but this uses a document model that is a list (tree is just how it is implemented)

% Logoot : a Scalable Optimistic Replication Algorithm for Collaborative Editing on P2P Networks
%   \url{http://pagesperso.lina.univ-nantes.fr/~molli-p/pmwiki/uploads/Main/weiss09.pdf}

% Specification and Complexity of Collaborative Text Editing
%   \url{https://www.microsoft.com/en-us/research/wp-content/uploads/2016/07/podc16-complete.pdf}

% LSEQ: an Adaptive Structure for Sequences in Distributed Collaborative Editing,
%   \url{https://hal.archives-ouvertes.fr/file/index/docid/921633/filename/fp025-nedelec.pdf}

% Data consistency for P2P collaborative editing
%   \url{https://hal.archives-ouvertes.fr/file/index/docid/108523/filename/OsterCSCW06.pdf}

% Interleaving anomalies in collaborative text editors
%   \url{https://martin.kleppmann.com/papers/interleaving-papoc19.pdf}

% Moving Elements in List CRDTs
%   \url{https://martin.kleppmann.com/papers/list-move-papoc20.pdf}

% A highly-available move operation for replicated trees and distributed filesystems
%   \url{https://martin.kleppmann.com/papers/move-op.pdf}

% ...
%  commutative replicated data types CmRDT
%  convergent replicated data types, or CvRDTs
%  Delta state CRDTs[12][13] (or simply Delta CRDTs

% The Causal Graph CRDT for Complex Document Structure
%   \url{https://dl.acm.org/doi/10.1145/3209280.3229110}

% \url{https://en.wikipedia.org/wiki/Conflict-free_replicated_data_type}

% G-Set
% PN-Set
% 2P-Set
% LWW-element-Set (Last-Write-Wins)
% OR-Set
% MV-Register: Multi-Value Register
% U-Set (this is what we are for edges, not OR-set due to duplicates)
% Add-Remove Partial Order data type
%   2P-Set for vertices, and a G-Set for edges.

% A comprehensive study of Convergent and Commutative Replicated Data Types (2011)
%   \url{https://hal.inria.fr/inria-00555588/document}

% CRDTs: Consistency without concurrency control
%   \url{https://arxiv.org/abs/0907.0929}

% https://medium.com/@amberovsky/crdt-conflict-free-replicated-data-types-b4bfc8459d26
%  removeVertex() has priority, all incident edges are removed
%  addEdge() has priority, all removed vertices are re-added
%  Delay removeVertex() execution till all concurrent removeVertex() are executed.
% First one is 2P2P-Set

% https://crdt.tech/papers.html

% Mahsa Najafzadeh, Marc Shapiro, and Patrick Eugster. Co-design and verification of an available file system. In 19th International Conference on Verification, Model Checking, and Abstract Interpretation, VMCAI 2018, pages 358--381. Springer LNCS volume 10747, January 2018. [ bib | DOI | .pdf ]
%   \url{http://dx.doi.org/10.1007/978-3-319-73721-8_17}
%   https://pages.lip6.fr/Marc.Shapiro/papers/VMCAI-2018-filesys.pdf

% Martin Kleppmann and Alastair R Beresford. A conflict-free replicated JSON datatype. IEEE Transactions on Parallel and Distributed Systems, 28(10):2733--2746, April 2017. [ bib | DOI | arXiv ]
%   http://dx.doi.org/10.1109/TPDS.2017.2697382
%   http://arxiv.org/abs/1608.03960

% Vinh Tao, Marc Shapiro, and Vianney Rancurel. Merging semantics for conflict updates in geo-distributed file systems. In 8th ACM International Systems and Storage Conference, SYSTOR 2015. ACM, May 2015. [ bib | DOI | .pdf ]
%   http://dx.doi.org/10.1145/2757667.2757683
%   https://pages.lip6.fr/Marc.Shapiro/papers/geodistr-FS-Systor-2015.pdf

% Mehdi Ahmed-Nacer, Stéphane Martin, and Pascal Urso. File system on CRDT. Research Report RR-8027, INRIA, July 2012. [ bib | arXiv | http ]
%   http://arxiv.org/abs/1207.5990
%   https://hal.inria.fr/hal-00720681/

% Stéphane Martin, Pascal Urso, and Stéphane Weiss. Scalable XML collaborative editing with undo. In On the Move to Meaningful Internet Systems (OTM), pages 507--514. Springer LNCS volume 6426, October 2010. [ bib | DOI | arXiv ]
%   \url{http://dx.doi.org/10.1007/978-3-642-16934-2_37}
%   http://arxiv.org/abs/1010.3615



% 2P-Sets
% Anomaly: Creating a lone deleted vertex requires create and delete of otherwise unneeded edge

% Operational Transforms

% Etherpad

% Live Share

% \subsection{Synchronization}

% Unison
% \url{https://www.cis.upenn.edu/~bcpierce/unison/}
% \url{https://www.cis.upenn.edu/%7Ebcpierce/papers/index.shtml#File%20Synchronization}

% \subsection{Homotopical Patch Theory}

% Homotopical Patch Theory: \url{https://www.cambridge.org/core/journals/journal-of-functional-programming/article/homotopical-patch-theory/42AD8BB8A91688BCAC16FD4D6A2C3FE7}
% Homotopical patch theory: \url{https://dl.acm.org/doi/10.1145/2628136.2628158}

% - diff
%     - A file comparison program 1985
%     - myers an O(ND) difference algorithm and its variations
% - merging
%     - a state-of-the-art survey on software merging
% - homotopy patch theory
%     - 7 10 15 16 28 31
%     - general algebraic framework for discussing patch languages
% - darcs
%     - unordered series of patches (but dependencies between them)
%     - we don't focus on inversion in this paper
%     - some properties of darcs patch theory
%         - http://urchin.earth.li/darcs/ganesh/darcs-patch-theory/theory/formal.pdf
%     - A formalization of darcs patch theory using inverse
% semigroups
%         - 2, 3, 5, 11, 18
%         - 15 = selective undo https://ww3.math.ucla.edu/camreport/cam09-83.pdf
%         - patch commutation requires an operational transform! https://ww3.math.ucla.edu/camreport/cam09-83.pdf
%         - (we also define it as a semigroup)
% - dagit - type correct changes
%     - Type-correct changes—a safe approach to version control
% implementation
%     - MS thesis
%     - Darcs: Distributed version management in haskell.
%     - "patches that modify different parts of the code are considered, by default, independent"
%     - patience diff
% - A categorical theory of patches
% - OT = "Concurrency control in groupware systems"
%     - Formal design and
% verification of operational transformation algorithms for copies convergence
% - The Semantics of Version Control
%     - https://dl.acm.org/doi/abs/10.1145/2661136.2661137?casa_token=hKqrcG9g7_UAAAAA:vN0l9AJhdnjc_2wnAIBl3TumxJBgy_4JYYSsG7522fYnbdfzh7Mpc0PIwUKcsX2VbwA0M67Cy_Y
%     - cite for this is a version control system
%     - CACM article by o'sullivan
%         - https://dl.acm.org/doi/pdf/10.1145/1562164.1562183
%         - "We are not by any means near the end
% of the road in the evolution of revisioncontrol systems. The field has received
% only fitful attention from academia.
% Much work could be done on its formal foundations, which could lead to
% more powerful and safer ways for developers to work together. Alas, I know
% of only one notable publication on the
% topic in the past decade.1
%  As a simple
% example, when merging potentially
% conflicting changes, almost everybody
% uses either three-way merging, which
% is decades old, or unpublished ad hoc
% approaches in which there is little reason to be confident."
% - touchdevelop
%     - https://dl.acm.org/doi/pdf/10.1145/2846661.2846672
% - tree CRDTs
%     - A Highly-Available Move Operation for Replicated Trees
%         - "last writer wins" conflict resolution not explicit conflict representation
%         - using timestamps
%         - Martin et al 28, 29 CRDT for XML data + Kleppman and Beresford CRDT for JSON but no moves
%         - [31] outline approaches to handling conflicts on trees but no algor
%         - treedoc 34 also no move
%         - 37, 38, 39 Ts for trees
%         - 40 defines OT with a move operation but need a total order broadcast (not commutative)
%     - Maram
%         - The price to pay is that some move operations “lose”, i.e., have no
% effect; achieving the same end result as previous correct approaches but at a lower cost
%         - https://arxiv.org/pdf/2103.04828
%     - UDR tree -- ordering
%     - Najafzadeh et al Subtree_1 requires locks
%     - Tao et al [17]
%         - . Merging semantics for conflict updates in
% geo-distributed file systems
%         - Can lead to duplication
%         - 
% - tree diff based approaches
%     - https://dl.acm.org/doi/10.1007/978-3-662-54494-5_9
%         - 14, 15, 21, 24, 25
%     - Change distilling: tree differencing
% for fine-grained source code change extraction.
%     - Fine-grained
% and accurate source code differencing.
%     - Tree-based Version Control in Envision
%     - A three-way merge for XML documents
%     - Operation-Based, Fine-Grained Version Control Model for Tree-Based Representation
%     - https://www.sciencedirect.com/science/article/pii/S0167642315000532
% - do not guess!
%     - https://dl.acm.org/doi/pdf/10.1145/2508075.2508092 (brief note)
%         - Cedalion: a language for language-oriented programming
% - variability aware execution
%     - https://dl.acm.org/doi/abs/10.1145/2786805.2803208
% - text CRDTs?
