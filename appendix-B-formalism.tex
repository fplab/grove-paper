\section{Formalism}
\label{sec:Formalism}

%%%%%%%%%%%%%%%%%%%%%%%%%%%%%%%%%%%%%%%%%%%%%%%%%%%%%%%%%%%%%%%%%%%%%%%%%%%%%%%%

\subsection{Graphs}
\label{sec:Formalism:Graphs}

\begin{align*}
  \sources{G} &= \SetOf{(v, p) \in \S \SuchThat{
    (u, (v, p), v') \in \inedges{G} \land u \in \U \land v' \in \V}} \\
  \inedges{G} &= \SetOf{\e \in \E \SuchThat{G(\e) = \Plus}} \\
  \outedges{v}{p} &= \SetOf{(u, (v, p), v') \in \inedges{G} \SuchThat{u \in \U \land v' \in \V }} \\
  \inedges{v} &= \SetOf{\e{=}(u, (v', p), v) \in \inedges{G} \SuchThat{u \in \U \land (v', p) \in \S}} \\
  G_v(\e{=}(u, (v, p), v')) &= \begin{cases}
    G(\e) & v' = v \\
    \bot  & v' \neq v \\
  \end{cases} \\
  \parents{v} &= \SetOf{v' \in \V \SuchThat{(u, (v', p), v) \in \inedges{G} \land u \in \U \land p \in \P}} \\
  \ancestors{v} &= \mathopen{}\left( \lfp{\ancestorsPrimeOp} \right)\mathclose{}(v) \\
  \min{\Q{v}} &= \argmin{(u, k) \in \Q{v}}{u}
\end{align*}

% % % % % % % % % % % % % % % % % % % % % % % % % % % % % % % % % % % % % % % % 

\subsubsection{Well-sortedness}

\begin{definition}
  A \emph{well sorted graph} is a graph $G : \E \rightarrow \Sigma$
  such that for all $\e \in \E$,
  if $G(\e) \in \SetOf{\Plus, \Minus}$,
  then $\e$ is a well sorted edge. 
\end{definition}

\begin{definition}
  A \emph{well sorted edge} is an edge $(u_0, v{=}(u, k), p, (u', k')) \in \E$
  such that $(v, p)$ is a well sorted source
  and $(p, \sort{k'}) \in \arity{k}$.
\end{definition}

\begin{definition}
  A \emph{well sorted source} is a source $(v{=}(u, k), p) \in \V \times \P$
  for which there exists a sort $S \in \SetOf{Exp, Pat, Var}$
  such that $(p, S) \in \arity{k}$.
\end{definition}

%%%%%%%%%%%%%%%%%%%%%%%%%%%%%%%%%%%%%%%%%%%%%%%%%%%%%%%%%%%%%%%%%%%%%%%%%%%%%%%%

\subsection{Groves}
\label{sec:Formalism:Groves}

\figureTermSyntaxContent

For conflict hole forms $\conflictHole[i < n]{t_i}$,
it must be the case that $n \geq 2$.
We identify conflict holes up to reordering.

Let $\Set$ denote a set of terms.

\begin{definition}
  A \emph{grove} $\Grove = (\Set[NP], \Set[MP], \Set[U])$
  is three sets of terms $\Set[NP], \Set[MP], \Set[U] \in Term$.
\end{definition}

\begin{definition}
  The \emph{vertex corresponding to} term $t \in Term$,
  when $t$ is not an empty hole or a conflict hole,
  is the unique vertex $v \in \V$ for which there exists an edge $\e{=}(u, v', p, v)$
  such that $\decomp{\e} = t$.
  We write $v_t$ to denote the vertex corresponding to $t$.
\end{definition}

\begin{definition}
  The \emph{ingraph of a term} $t \in Term$ is a graph $G_t : \E \rightarrow \Sigma$ such that
  \[
    G_t =
    \begin{cases}
      \bigcup_{t_i} G_{t_i} & t = \conflictHole[i < n]{t_i}, \\
      \varnothing & t = \emptyHole{v}{p}, \text{ and} \\
      G_{v_t} & \text{otherwise.} \\
    \end{cases}
  \]
\end{definition}

% % % % % % % % % % % % % % % % % % % % % % % % % % % % % % % % % % % % % % % % 

\subsubsection{Well-sortedness}

\begin{definition}
  A \emph{well sorted grove} is a grove $\gamma = (\Set[NP], \Set[MP], \Set[U])$ for which all of the following hold:
  \begin{enumerate}
    % All terms are well sorted.
    \item For each term $t \in \Set[NP] \cup \Set[MP] \cup \Set[U]$, we have $t$ is a well sorted term.
    % Any terms corresponding to edges originating from $\rootVertex$ are in $\Set[NP]$.
    \item For each term $t \in \Set[MP] \cup \Set[U]$, we have $(\rootVertex, \Root) \notin \sources{G_t}$.
    % The terms in $\Set[NP]$ correspond to edges with target vertices that have no parents.
    \item For each term $t \in \Set[NP]$, we have $\SizeOf{\inedges{G_t}} = 0$.
    % The terms in $\Set[MP]$ correspond to edges with target vertices that have multiple parents.
    \item For each term $t \in \Set[MP]$, we have $\SizeOf{\inedges{G_t}} > 1$.
    % The terms in $\Set[U]$ correspond to edges with target vertices that are unicycle roots.
    \item For each term $t \in \Set[U]$, we have $\inedges{G_t} = \SetOf{(u, v, p, v')}$ and $v' = \min{\ancestors{v'}}$.
  \end{enumerate}
\end{definition}

\begin{definition}
  A \emph{well sorted term} is a term $t \in Term$ such that one of the following holds:
  \begin{itemize}

    % An empty hole is well sorted if its source vertex can have children at the source position.
    \item $t = \emptyHole{v}{p}$ and $(v, p)$ is a well sorted source.

    \item $t = \conflictHole[i < n]{t_i}$ and all of the following hold:
      \begin{enumerate}
        % The terms are well sorted.
        \item For each term $t_i$, we have $t_i$ is a well sorted term.
        % The inedges of the terms have the same source.
        \item $\SizeOf{\sources{G_t}} = 1$.
      \end{enumerate}

    \item $t = \multiVertex{\e{=}(u, v, p, v')}$ and all of the following hold:
      \begin{enumerate}
        % The corresponding edge is well sorted ...
        \item $\e$ is a well sorted edge.
        % ... and its target vertex has multiple parents.
        \item $\SizeOf{\inedges{v'}} > 1$.
      \end{enumerate}

    \item $t = \cycleVertex{\e{=}(u, v, p, v')}$ and all of the following hold:
      \begin{enumerate}
        % The corresponding edge is well sorted ...
        \item $\e$ is a well sorted edge.
        % ... and its target vertex is the root ...
        \item $v' = \min{\ancestors{v'}}$
        % ... of a unicycle.
        \item For each vertex $v'' \in \ancestors{v'}$, we have $\SizeOf{\inedges{v''}} = 1$.
      \end{enumerate}

    \item All of the following hold:
      \begin{enumerate}
        % All incoming edges are well sorted.
        \item $G_t$ is a well sorted graph.
        % The in-graph of t is not empty.
        \item $\SizeOf{\inedges{G_t}} \geq 1$.
        % All incoming edges target the vertex corresponding to t.
        \item For each edge $(u, v, p, v') \in \inedges{G_t}$, we have $v' = v_t$.
        % The destination vertices of all incoming edges have the same id and constructor.
        % The constructor matches the term.
        \item There exists a unique identifier $u_0 \in \U$ such that,
          for each edge $(u, v, p, v') \in \inedges{G_t}$,
          we have $v' = (u_0, \constructor{t})$.
      \end{enumerate}
  \end{itemize}
\end{definition}

% % % % % % % % % % % % % % % % % % % % % % % % % % % % % % % % % % % % % % % % 

\subsubsection{Constructors}

\[
  \arraycolsep=0pt
  \begin{array}{ll}
    \multicolumn{2}{l}{\sortOp : \K \to \SetOf{Exp, Pat, Typ}} \\
    \hline
    \sort{\Root}={} & Exp \\
    \sort{\ExpVar(x)}={} & Exp \\
    \sort{\ExpLam}={} & Exp \\
    \sort{\ExpApp}={} & Exp \\
    \sort{\ExpPlus}={} & Exp \\
    \sort{\ExpTimes}={} & Exp \\
    \sort{\ExpNum(n)}={} & Exp \\
    \sort{\PatVar(x)}={} & Pat \\
    \sort{\TypArrow}={} & Typ \\
    \sort{\TypNum}={} & Typ \\
  \end{array}
\]
%
\figureArityContent
%
\[
  \arraycolsep=0pt
  \begin{array}{ll}
    \multicolumn{2}{l}{\defaultposOp : \K \to \P \times \P} \\
    \hline
    \defaultpos{\Root}={} & (\Root, \Root) \\
    \defaultpos{\ExpLam}={} & (\LamBody, \LamBody) \\
    \defaultpos{\ExpApp}={} & (\AppFun, \AppArg) \\
    \defaultpos{\ExpPlus}={} & (\PlusLeft, \PlusRight) \\
    \defaultpos{\ExpTimes}={} & (\TimesLeft, \TimesRight) \\
    \defaultpos{\TypArrow}={} & (\ArrowArg, \ArrowResult) \\
  \end{array}
\]
%
% We write $\defaultpos{k}$ undefined when there does not exist a $p$ such that $\defaultpos{k} = p$.

\noindent $\boxed{\constructor{t} = k}$
%
\begin{align*}
  \constructor{e} &= \econstructor{e} \\
  \constructor{p} &= \pconstructor{p} \\
  \constructor{\tau} &= \tconstructor{\tau}
\end{align*}

\noindent $\boxed{\econstructor{e} = k}$
%
\begin{align*}
  \econstructor{\eVar{G}{x}} &= \ExpVar(x) \\
  \econstructor{\eFun{G}{q}{\tau}{e}} &= \ExpLam \\
  \econstructor{\eApp{G}{e_1}{e_2}} &= \ExpApp \\
  \econstructor{\eNum{G}{n}} &= \ExpNum(n) \\
  \econstructor{\ePlus{G}{e_1}{e_2}} &= \ExpPlus \\
  \econstructor{\eTimes{G}{e_1}{e_2}} &= \ExpTimes
\end{align*}

\noindent $\boxed{\pconstructor{q} = k}$
%
\begin{align*}
  \pconstructor{\pVar{G}{x}} &= \PatVar(x)
\end{align*}

\noindent $\boxed{\tconstructor{\tau} = k}$
%
\begin{align*}
  \tconstructor{\tArrow{G}{\tau_1}{\tau_2}} &= \TypArrow \\
  \tconstructor{\tNum{G}} &= \TypNum
\end{align*}

%%%%%%%%%%%%%%%%%%%%%%%%%%%%%%%%%%%%%%%%%%%%%%%%%%%%%%%%%%%%%%%%%%%%%%%%%%%%%%%%

\subsection{Decomposition and Recomposition}
\label{sec:Formalism:Decomposition and Recomposition}

Let $G : \E \rightarrow \Sigma$ be a well sorted graph.

\begin{theorem}
  There exists a well sorted grove $\Grove$ such that $\decomp{G} = \Grove$ and $\recomp{\Grove} = G$.
\end{theorem}

% \begin{theorem}
%   There exists a well sorted grove $\Grove$ such that $\decomp{G} = \Grove$.
% \end{theorem}

% Proof: provide a witness that demonstrates the conclusion.

Let $\Grove = (\Set[NP], \Set[MP], \Set[U]) = \decomp{G}$.

\begin{lemma}
  $\Grove$ is a well sorted grove.
\end{lemma}

\begin{lemma}
  For each edge $\e \in \inedges{G}$,
  there exists a term $t \in \Set[NP] \cup \Set[MP] \cup \Set[U]$
  such that $\decomp{\e} = t$.
\end{lemma}

\begin{lemma}
  For each term $t \in \Set[NP] \cup \Set[MP] \cup \Set[U]$,
  there exists a subgraph $G' \subseteq G$ such that $\decomp{G'} = t$.
\end{lemma}

\begin{lemma}
  Let $G|_t$ denote the subgraph of $G$ such that $\decomp{G|_t} = t$.
  Then
  \[
    \bigcup_{t \in \Set[NP] \cup \Set[MP] \cup \Set[U]} G|_t = G.
  \]
\end{lemma}

% --

Let $\Grove = (\Set[NP], \Set[MP], \Set[U])$ be a well sorted grove,
and let $G : \E \rightarrow \Sigma = \recomp{\Grove}$.

\begin{lemma}
  $G$ is a well sorted graph.
\end{lemma}

\begin{lemma}
  For each term $t \in \Set[NP] \cup \Set[MP] \cup \Set[U]$,
  there exists a subgraph $G' \subseteq G$ such that $\recomp{t} = G'$.
\end{lemma}

\begin{lemma}
  Let $G|_t$ denote the subgraph of $G$ such that $\recomp{t} = G|_t$.
  Then
  \[
    \bigcup_{t \in \Set[NP] \cup \Set[MP] \cup \Set[U]} G|_t = G.
  \]
\end{lemma}

% \begin{theorem}
%   For any well sorted graph $G$,
%   if $\decomp{G} = \Grove$ then $\recomp{\Grove} = G$.
% \end{theorem}

% % % % % % % % % % % % % % % % % % % % % % % % % % % % % % % % % % % % % % % % 

\subsubsection{Decomposition}

\noindent $\boxed{\decomp{G} = \Grove}$
%
\figureDecompositionDefDecomp

\noindent $\boxed{\decomp{\e} = t}$
%
\figureDecompositionDefDecompTerm

\noindent $\boxed{\edecomp{\e} = e}$
%
\figureDecompositionDefEdecomp

\noindent $\boxed{\edecompPrime{\e}{p} = e}$
%
\figureDecompositionDefEdecompPrime

\noindent $\boxed{\pdecomp{\e} = q}$
%
\figureDecompositionDefPdecomp

\noindent $\boxed{\pdecompPrime{\e}{p} = q}$
%
\figureDecompositionDefPdecompPrime

\noindent $\boxed{\tdecomp{\e} = \tau}$
%
\figureDecompositionDefTdecomp

\noindent $\boxed{\tdecompPrime{\e}{p} = \tau}$
%
\figureDecompositionDefTdecompPrime%

% % % % % % % % % % % % % % % % % % % % % % % % % % % % % % % % % % % % % % % % 

\subsubsection{Recomposition}\hspace*{\fill} \\

\noindent $\boxed{\recomp{\Grove} = G}$
%
\begin{align*}
  \recomp{(\Set[NP], \Set[MP], \Set[U])} &= \bigcup_{t \in \Set[NP] \cup \Set[MP] \cup \Set[U]} \recomp{t}
\end{align*}

\noindent $\boxed{\recomp{t} = G}$
%
\begin{align*}
  \recomp{e} &= \erecomp{e} \\
  \recomp{q} &= \precomp{q} \\
  \recomp{\tau} &= \trecomp{\tau}
\end{align*}

\noindent $\boxed{\erecomp{e} = G}$
%
\begin{align*}
  \erecomp{\eVar{G}{x}} &= G
  \\
  \erecomp{\eFun{G}{q}{\tau}{e}}
    &= G \cup \precomp{q} \cup \trecomp{\tau} \cup \erecomp{e}
  \\
  \erecomp{\eApp{G}{e_1}{e_2}}
    &= G \cup \erecomp{e_1} \cup \erecomp{e_2}
  \\
  \erecomp{\eNum{G}{n}} &= G
  \\
  \erecomp{\ePlus{G}{e_1}{e_2}}
    &= G \cup \erecomp{e_1} \cup \erecomp{e_2}
  \\
  \erecomp{\eTimes{G}{e_1}{e_2}}
    &= G \cup \erecomp{e_1} \cup \erecomp{e_2}
  \\
  \erecomp{\conflictHole[i < n]{e_i}}
  &= \bigcup_{i=1}^n \erecomp{e_i}
  \\
  \erecomp{\multiVertex{\e}} &= \SetOf{\e \mapsto \Plus}
  \\
  \erecomp{\cycleVertex{\e}} &= \SetOf{\e \mapsto \Plus}
  \\
  \erecomp{\emptyHole{v}{p}} &= \SetOf{}
\end{align*}

\noindent $\boxed{\precomp{q} = G}$
%
\begin{align*}
  \precomp{\pVar{G}{x}} &= G
  \\
  \precomp{\conflictHole[i < n]{q_i}} &= \bigcup_{i=1}^n \precomp{q_i}
  \\
  \precomp{\multiVertex{\e}} &= \SetOf{\e \mapsto \Plus}
  \\
  \precomp{\cycleVertex{\e}} &= \SetOf{\e \mapsto \Plus}
  \\
  \precomp{\emptyHole{v}{p}} &= \SetOf{}
\end{align*}

\noindent $\boxed{\trecomp{\tau} = G}$
%
\begin{align*}
  \trecomp{\tArrow{G}{\tau_1}{\tau_2}}
    &= G \cup \trecomp{\tau_1} \cup \trecomp{\tau_2}
  \\
  \trecomp{\tNum{G}} &= G
  \\
  \trecomp{\conflictHole[i < n]{\tau_i}} &= \bigcup_{i=1}^n \trecomp{\tau_i}
  \\
  \trecomp{\multiVertex{\e}} &= \SetOf{\e \mapsto \Plus}
  \\
  \trecomp{\cycleVertex{\e}} &= \SetOf{\e \mapsto \Plus}
  \\
  \trecomp{\emptyHole{v}{p}} &= \SetOf{}
\end{align*}

%%%%%%%%%%%%%%%%%%%%%%%%%%%%%%%%%%%%%%%%%%%%%%%%%%%%%%%%%%%%%%%%%%%%%%%%%%%%%%%%

\subsection{Cursors}

% % % % % % % % % % % % % % % % % % % % % % % % % % % % % % % % % % % % % % % % 

\subsubsection{Zippered Terms}

\[
  \arraycolsep=0pt
  \begin{array}{lcllll}
    \Z{t} & {}\in{} & ZTerm & {}::={} &
      \Z{e}
      \mid \Z{q}
      \mid \Z{\tau}
    \\
    \Z{e} & {}\in{} & ZExp & {}::={} &
      \cursor{e}
      \mid \eFun{G}{\Z{q}}{\tau}{e}
      \mid \eFun{G}{q}{\Z{\tau}}{e}
      \mid \eFun{G}{q}{\tau}{\Z{e}}
      \mid \eApp{G}{\Z{e}}{e}
      \mid \eApp{G}{e}{\Z{e}}
      \mid \ePlus{G}{\Z{e}}{e}
      \mid \ePlus{G}{e}{\Z{e}}
      \mid \eTimes{G}{\Z{e}}{e}
      \\
    &&&&
      \mid \eTimes{G}{e}{\Z{e}}
      \mid \conflictHole{\Z{e}, \SetOf{e_i}_{i < n}}
    \\
    \Z{q} & {}\in{} & ZPat & {}::={} &
      \cursor{q}
      \mid \conflictHole{\Z{q}, \SetOf{q_i}_{i < n}}
    \\
    \Z{\tau} & {}\in{} & ZTyp & {}::={} &
      \cursor{\tau}
      \mid \tArrow{G}{\Z{\tau}}{\tau}
      \mid \tArrow{G}{\tau}{\Z{\tau}}
      \mid \conflictHole{\Z{\tau}, \SetOf{\tau_i}_{i < n}}
    \\
  \end{array}
\]

Let $\Z{\Set} = (\Set, \Z{t})$ denote a set of terms paired with a zippered term.

% % % % % % % % % % % % % % % % % % % % % % % % % % % % % % % % % % % % % % % % 

\subsubsection{Zippered Groves}

\begin{gather*}
  \arraycolsep=0pt
  \begin{array}{lll}
    \Z{\gamma} & {}::={} &
      (\Z{\Set}_{NP}, \Set[MP], \Set[U])
      \mid (\Set[NP], \Z{\Set}_{MP}, \Set[U])
      \mid (\Set[NP], \Set[MP], \Z{\Set}_U)
  \end{array}
\end{gather*}

% % % % % % % % % % % % % % % % % % % % % % % % % % % % % % % % % % % % % % % % 

\subsubsection{Cursor Erasure}\hspace*{\fill} \\

\noindent $\boxed{\erase{\Z{\gamma}} = \gamma}$
%
\begin{align*}
  \erase{(\Z{\Set}_{NP}, \Set[MP], \Set[U])} &= (\erase{\Z{\Set}_{NP}}, \Set[MP], \Set[U]) \\
  \erase{(\Set[NP], \Z{\Set}_{MP}, \Set[U])} &= (\Set[NP], \erase{\Z{\Set}_{MP}}, \Set[U]) \\
  \erase{(\Set[NP], \Set[MP], \Z{\Set}_U)} &= (\Set[NP], \Set[MP], \erase{\Z{\Set}_U})
\end{align*}

\noindent $\boxed{\erase{\Z{\Set}} = \Set}$
%
\begin{align*}
  \erase{(\Set, \Z{t})} &= \Set \cup \SetOf{t}
\end{align*}

\noindent $\boxed{\erase{\Z{e}} = e}$
%
\begin{align*}
  \erase{\cursor{e}} &= e \\
  \erase{\left(\eFun{G}{\Z{q}}{\tau}{e}\right)} &= \eFun{G}{\erase{\Z{q}}}{\tau}{e} \\
  \erase{\left(\eFun{G}{q}{\Z{\tau}}{e}\right)} &= \eFun{G}{q}{\erase{\Z{\tau}}}{e} \\
  \erase{\left(\eFun{G}{q}{\tau}{\Z{e}}\right)} &= \eFun{G}{q}{\tau}{\erase{\Z{e}}} \\
  \erase{\eApp{G}{\Z{e}_1}{e_2}} &= \eApp{G}{\erase{\Z{e}_1}}{e_2} \\
  \erase{\eApp{G}{e_1}{\Z{e}_2}} &= \eApp{G}{e_1}{\erase{\Z{e}_2}} \\
  \erase{\ePlus{G}{\Z{e}_1}{e_2}} &= \ePlus{G}{\erase{\Z{e}_1}}{e_2} \\
  \erase{\ePlus{G}{e_1}{\Z{e}_2}} &= \ePlus{G}{e_1}{\erase{\Z{e}_2}} \\
  \erase{\eTimes{G}{\Z{e}_1}{e_2}} &= \eTimes{G}{\erase{\Z{e}_1}}{e_2} \\
  \erase{\eTimes{G}{e_1}{\Z{e}_2}} &= \eTimes{G}{e_1}{\erase{\Z{e}_2}} \\
  \erase{\conflictHole{\Z{e}, \SetOf{e_i}_{i < n}}} &= \conflictHole{\erase{\Z{e}}, \SetOf{e_i}_{i < n}}
\end{align*}

\noindent $\boxed{\erase{\Z{q}} = q}$
%
\begin{align*}
  \erase{\cursor{q}} &= q \\
  \erase{\conflictHole{\Z{q}, \SetOf{q_i}_{i < n}}} &= \conflictHole{\erase{\Z{q}}, \SetOf{q_i}_{i < n}}
\end{align*}

\noindent $\boxed{\erase{\Z{\tau}} = \tau}$
%
\begin{align*}
  \erase{\cursor{\tau}} &= \tau \\
  \erase{\left(\tArrow{G}{\Z{\tau}_1}{\tau_2}\right)} &= \tArrow{G}{\erase{\Z{\tau}_1}}{\tau_2} \\
  \erase{\left(\tArrow{G}{\tau_1}{\Z{\tau}_2}\right)} &= \tArrow{G}{\tau_1}{\erase{\Z{\tau}_2}} \\
  \erase{\conflictHole{\Z{\tau}, \SetOf{\tau_i}_{i < n}}} &= \conflictHole{\erase{\Z{\tau}}, \SetOf{\tau_i}_{i < n}}
\end{align*}

%%%%%%%%%%%%%%%%%%%%%%%%%%%%%%%%%%%%%%%%%%%%%%%%%%%%%%%%%%%%%%%%%%%%%%%%%%%%%%%%

\subsection{User Actions and Graph Cursors}

\[
  \arraycolsep=0pt
  \begin{array}{lcllll}
    \alpha & {}\in{} & Act & {}::={} &
      \Construct{k}
      \mid \Delete
      \mid \Reposition{v}{p}
    \\
    c & {}\in{} & Cur & {}::={} &
      \CVertex{v}
      \mid \CEdge{\e}
      \mid \CSource{v}{p}
    \\
  \end{array}
\]

Let $\A = \SetOf{\Plus, \Minus} \times \E$ the set of all \emph{graph actions}.
We write $a$ to denote a graph action, and $\Q{a}$ to denote a set of graph actions.

\begin{definition}
  A \emph{graph cursor} is an element of $Cur$.
\end{definition}

\begin{definition}
  A \emph{well sorted graph cursor} is a graph cursor $c \in Cur$
  such that one of the following holds:
  \begin{itemize}
    \item $c = \CVertex{v}$ and $v \neq v_{root}$.
    \item $c = \CEdge{\e}$ and $\e$ is a well sorted edge.
    \item $c = \CSource{v}{p}$ and $(v, p)$ is a well sorted source.
  \end{itemize}
\end{definition}

\begin{definition}
  An \emph{almost valid graph cursor} is a well sorted graph cursor $c \in Cur$
  such that one of the following holds:
  \begin{itemize}
    \item $c = \CVertex{v}$.
    \item $c = \CEdge{\e}$ and $\e \in \inedges{G}$.
    \item $c = \CSource{v}{p}$
      and there exists an edge $\e{=}(u, v', p', v) \in \E$
      such that $G(\e) \in \SetOf{\Plus, \Minus}$.
  \end{itemize}
\end{definition}

\begin{definition}
  A \emph{valid graph cursor} is an almost valid graph cursor $c \in Cur$
  such that one of the following holds:
  \begin{itemize}
    \item $c = \CVertex{v}$.
    \item $c = \CEdge{(u, v, p, v')}$ and one of the following holds:
    \begin{itemize}
      \item $\SizeOf{\inedges{G_{v'}}} > 1$.
      \item $\SizeOf{\inedges{G_{v'}}} = 1$ and $v' = \min{\ancestors{v'}}$.
    \end{itemize}
    \item $c = \CSource{v}{p}$ and $\SizeOf{\outedges{v}{p}} \neq 1$.
  \end{itemize}
\end{definition}

Let $\Z{\Grove}$ be a zippered grove such that $\erase{\Z{\Grove}}$ is a well sorted grove,
and let $G = \recomp{\erase{\Z{\Grove}}}$ a graph,
and let $\alpha \in Act$ a user action.

\begin{theorem}[Sensibility]
  If $\applyAction{\Z{\Grove}}{\alpha}{\Q{a}}{c}$,
  then $G \action{\Q{a}} G'$ and $G'$ is well sorted.
\end{theorem}

\noindent $\boxed{\applyAction{ \Z{\Grove} }{ \alpha }{ \Q{a} }{ c }}$
%
\begin{mathpar}
  \inferrule{
    \applyAction{ \Z{\Set}_{NP} }{ \alpha }{ \Q{a} }{ c }
  }{
    \applyAction{ (\Z{\Set}_{NP}, \Set[MP], \Set[U]) }{ \alpha }{ \Q{a} }{ c }
  }

  \inferrule{
    \applyAction{ \Z{\Set}_{MP} }{ \alpha }{ \Q{a} }{ v' }{ p' }
  }{
    \applyAction{ (\Set[NP], \Z{\Set}_{MP}, \Set[U]) }{ \alpha }{ \Q{a} }{ c }
  }

  \inferrule{
    \applyAction{ \Z{\Set}_U }{ \alpha }{ \Q{a} }{ c }
  }{
    \applyAction{ (\Set[NP], \Set[MP], \Z{\Set}_U) }{ \alpha }{ \Q{a} }{ c }
  }
\end{mathpar}

\noindent $\boxed{\applyAction{ \Z{\Set} }{ \alpha }{ \Q{a} }{ c }}$
%
\begin{mathpar}
  \inferrule{
    \applyAction{ \Z{t} }{ \alpha }{ \Q{a} }{ c }
  }{
    \applyAction{ (\Set, \Z{t}) }{ \alpha }{ \Q{a} }{ c }
  }
\end{mathpar}
\noindent $\boxed{\applyAction{ \Z{t} }{ \alpha }{ \Q{a} }{ c }}$
%
\begin{mathpar}
  \inferrule{
    \applyAction{ \Z{q} }{ \alpha }{ \Q{a} }{ c }
  }{
    \applyAction{ (\eFun{G}{\Z{q}}{\tau}{e}) }{ \alpha }{ \Q{a} }{ c }
  }

  \inferrule{
    \applyAction{ \Z{\tau} }{ \alpha }{ \Q{a} }{ c }
  }{
    \applyAction{ (\eFun{G}{q}{\Z{\tau}}{e}) }{ \alpha }{ \Q{a} }{ c }
  }

  \inferrule{
    \applyAction{ \Z{e} }{ \alpha }{ \Q{a} }{ c }
  }{
    \applyAction{ (\eFun{G}{q}{\tau}{\Z{e}}) }{ \alpha }{ \Q{a} }{ c }
  }

  \inferrule{
    \applyAction{ \Z{e}_1 }{ \alpha }{ \Q{a} }{ c }
  }{
    \applyAction{ \eApp{G}{\Z{e}_1}{e_2} }{ \alpha }{ \Q{a} }{ c }
  }

  \inferrule{
    \applyAction{ \Z{e}_2 }{ \alpha }{ \Q{a} }{ c }
  }{
    \applyAction{ \eApp{G}{e_1}{\Z{e}_2} }{ \alpha }{ \Q{a} }{ c }
  }

  \inferrule{
    \applyAction{ \Z{e}_1 }{ \alpha }{ \Q{a} }{ c }
  }{
    \applyAction{ (\ePlus{G}{\Z{e}_1}{e_2}) }{ \alpha }{ \Q{a} }{ c }
  }

  \inferrule{
    \applyAction{ \Z{e}_2 }{ \alpha }{ \Q{a} }{ c }
  }{
    \applyAction{ (\ePlus{G}{e_1}{\Z{e}_2}) }{ \alpha }{ \Q{a} }{ c }
  }

  \inferrule{
    \applyAction{ \Z{e}_1 }{ \alpha }{ \Q{a} }{ c }
  }{
    \applyAction{ (\eTimes{G}{\Z{e}_1}{e_2}) }{ \alpha }{ \Q{a} }{ c }
  }

  \inferrule{
    \applyAction{ \Z{e}_2 }{ \alpha }{ \Q{a} }{ c }
  }{
    \applyAction{ (\eTimes{G}{e_1}{\Z{e}_2}) }{ \alpha }{ \Q{a} }{ c }
  }

  \inferrule{
    \applyAction{ \Z{\tau}_1 }{ \alpha }{ \Q{a} }{ c }
  }{
    \applyAction{ (\tArrow{G}{\Z{\tau}_1}{\tau_2}) }{ \alpha }{ \Q{a} }{ c }
  }

  \inferrule{
    \applyAction{ \Z{\tau}_2 }{ \alpha }{ \Q{a} }{ c }
  }{
    \applyAction{ (\tArrow{G}{\tau_1}{\Z{\tau}_2}) }{ \alpha }{ \Q{a} }{ c }
  }

  \inferrule{
    \applyAction{ \Z{t} }{ \alpha }{ \Q{a} }{ c }
  }{
    \applyAction{ \conflictHole{\Z{t}, \SetOf{t_i}_{i < n}} }{ \alpha }{
      \Q{a}
    }{ c }
  }
\end{mathpar}
%
\begin{mathpar}
  \inferrule[Construct]{
    % If:
    % 
    % 1. The term under the cursor is a hole
    % 
    % 2. The user action is well sorted
    (p_s, \sort{k}) \in \arity{k_s} \\
    % 
    % 3. The new constructor does not have default positions
    \defaultpos{k} \text{ undefined} \\
    % 
    % And we let:
    % 
    % v_k = a new vertex with the new constructor
  }{
    % Then we get the following graph actions:
    \applyAction{ \cursor{\emptyHole{v_s{=}(u_s, k_s)}{p_s}} }{ \Construct{k} }{
      % 
      % attach the new vertex to the cursor source
      \SetOf{\graphAction{\Plus}{(\fresh{u}_1, v_s, p_s, v{=}(\fresh{u}_2, k))}}
    }{ \CVertex{v} }
  }

  \inferrule[ConstructWrap]{
    % If:
    % 
    % 1. The term under the cursor is not a hole, i.e., it has an ingraph
    \inedges{G_t} = \SetOf{\e_i{=}(u_i, v_i, p_i, v_i'{=}(u_i', k_i'))}_{i < n} \\
    % 
    % 2. The new constructor has default positions
    \defaultpos{k} = (\pWrap, \pPost) \\
    % 
    % The term under the cursor has the same sort as the new constructor
    \SetOf{\sort{k_i} = \sort{k}}_{i < n} \\
    % 
    % 4. The term under the cursor is well sorted w.r.t. the wrap position
    (\pWrap, \sort{k}) \in \arity{k} \\
    % 
    % And we let:
    % 
    % v_k = a new vertex with the new constructor
    v = (\fresh{u}, k) \\
  }{
    % Then we get the following graph actions:
    \applyAction{ \cursor{t} }{ \Construct{k} }{
      \SetOf{
        % 
        % delete the vertices under the cursor
        \graphAction{\Minus}{\e_i},
        % 
        % attach the new vertex to the sources of the vertices under the cursor
        \graphAction{\Plus}{(\fresh{u}_{1i}, v_i, p_i, v)},
        % 
        % reattach the old invertex to the default position of the new vertex
        \graphAction{\Plus}{(\fresh{u}_{2i}, v, \pWrap, v_i')}
      }_{i < n}
    }{ \CSource{v}{\pPost} }
  }

  \inferrule[Delete]{
    % If:
    % 
    % 1. The term under the cursor is not a hole, i.e., it has an ingraph
    \inedges{G_t} = \SetOf{\e_i}_{i < n} \\
    % 
    % 2. The term under the cursor is not a deleted root or a multiparent root
    \sources{G_t} = \SetOf{(v, p)}
  }{
    % Then we get the following graph actions:
    \applyAction{ \cursor{t} }{ \Delete }{
      % 
      % delete the vertices under the cursor
      \SetOf{\graphAction{\Minus}{\e_i}}_{i < n}
    }{ \CSource{v}{p} }
  }

  \inferrule[DeleteMultiparent]{
    % If:
    % 
    % 1. The term under the cursor is not a hole, i.e., it has an ingraph
    \inedges{G_t} = \SetOf{\e_i{=}(u_i, v_i, p_i, v)}_{i < n} \\
    % 
    % 2. The term under the cursor is a multiparent root
    \SizeOf{\sources{G_t}} > 1
  }{
    % Then we get the following graph actions:
    \applyAction{ \cursor{t} }{ \Delete }{
      % 
      % delete the vertices under the cursor
      \SetOf{\graphAction{\Minus}{\e_i}}_{i < n}
    }{ \CVertex{v} }
  }

  \inferrule[Reposition]{
    % If:
    % 
    % 1. The term under the cursor is not a hole, i.e., it has an ingraph
    \inedges{G_t} = \SetOf{\e_i{=}(u_i, v_i, p_i, v_i'{=}(u_i', k_i'))}_{i < n} \\
    % 
    % 2. The destination is an empty hole
    \SizeOf{\outedges{v}{p}} = 0 \\
    % 
    % 3. The constructors under the cursor are well sorted w.r.t. the destination
    \SetOf{(p, \sort{k_i'}) \in \arity{k}}_{i < n}
  }{
    % Thenwe get the following graph actions:
    \applyAction{ \cursor{t} }{ \Reposition{v{=}(u, k)}{p} }{
      \SetOf{
        % 
        % delete the vertices under the cursor
        \graphAction{\Minus}{\e_i}
        % 
        % reattach them to the destination
        \graphAction{\Plus}{(\fresh{u}, v, p, v_i)}
      }_{i < n}
    }{ \CSource{v}{p} }
  }
\end{mathpar}

% TODO: point out in the narrative that leaving out |children|=0 on Reposition allows locally produced conflicts

\noindent
$\boxed{\fix{c} = c'}$
%
\begin{mathpar}
  \inferrule[FCVertex]{
  }{
    \fix{c{=}\CVertex{v}} = c
  }

  \inferrule[FCSource]{
    \SizeOf{\outedges{v}{p}} \neq 1
  }{
    \fix{c{=}\CSource{v}{p}} = c
  }

  \inferrule[FCSourceVertex]{
    \outedges{v}{p} = \SetOf{(u, v, p, v')} \\
    \SizeOf{\parents{v'}} = 1 \\
    v' \neq \min{\ancestors{v'}}
  }{
    \fix{c{=}\CSource{v}{p}} = \CVertex{v'}
  }

  \inferrule[FCSourceEdge]{
    \outedges{v}{p} = \SetOf{(u, v, p, v')} \\
    \SizeOf{\parents{v'}} > 1 \lor v' = \min{\ancestors{v'}}
  }{
    \fix{c{=}\CSource{v}{p}} = \CEdge{(\fresh{u}, v, p, v')}
  }
\end{mathpar}
