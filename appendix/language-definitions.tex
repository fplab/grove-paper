\section{Language Definitions}
\label{sec:language-definitions}

\begin{definition}
  A \emph{language representation}
  $\Delta = (\Sorts, \Constructors, \Positions, \ConstructorSorts, \PositionSorts, \PositionConstructors)$
  is a six-tuple consisting of three finite ordered sets with three binary relations:

  \begin{minipage}{.48\linewidth}
    \[
      \arraycolsep=4pt
      \begin{array}{lrcr}
        \text{Sorts}        & \Sorts        & = & \Setof{S_i}_{i < |\Sorts| }        \\
        \text{Constructors} & \Constructors & = & \Setof{C_i}_{i < |\Constructors| } \\
        \text{Positions}    & \Positions    & = & \Setof{P_i}_{i < |\Positions| }    \\
      \end{array}
    \]
  \end{minipage}
  \hfil
  \begin{minipage}{.48\linewidth}
    \[
      \arraycolsep=4pt
      \begin{array}{lrcr}
        \text{Position Constructors} & \PositionConstructors & \subseteq & \Positions \times \Constructors \\
        \text{Position Sorts}        & \PositionSorts        & \subseteq & \Positions \times \Sorts        \\
        \text{Constructor Sorts}     & \ConstructorSorts     & \subseteq & \Constructors \times \Sorts     \\
      \end{array}
    \]
  \end{minipage}
  %
  such that the relations satisfy the following properties:
  \begin{enumerate}
    \item For each position $P \in \Positions$,
          there is a unique constructor $C \in \Constructors$ such that $(P, C) \in \PositionConstructors$.
    \item For each position $P \in \Positions$,
          there is a unique sort $S \in \Sorts$ such that $(P, S) \in \PositionSorts$.
    \item For each constructor $C \in \Constructors$,
          there is a unique sort $S \in \Sorts$ such that $(C, S) \in \ConstructorSorts$.
  \end{enumerate}
\end{definition}

\begin{definition}
  In any language representation $\Delta$,
  the \emph{ports} of a constructor $C \in \Constructors$, denoted $\Ports{C}$,
  is the largest subset of $\Positions$ such that $(P, C) \in \PositionConstructors$ for all $P \in \Ports{C}$.
\end{definition}

\begin{definition}
  The \emph{arity map} of a language representation $\Delta$
  is a binary relation $\Arity \subseteq \Constructors \times \Pow{\Positions}$
  defined by $\Arity = \Setof{(C, [C]_\Positions)}_{C \in \Constructors}$.
\end{definition}

\begin{theorem}
  \label{thm:langrep-rel-funs}
  The relations of a language representation are total functions.
\end{theorem}

\begin{theorem}
  \label{thm:langrep-arity-fun}
  In any language representation, arity maps are total functions.
\end{theorem}

Theorems \ref{thm:langrep-rel-funs} and \ref{thm:langrep-arity-fun}
justify the use of function notation when working with language representations and arity maps.
For example, for any $P \in \Positions$, we may write $\PositionConstructors(P)$
to denote the unique constructor $C \in \Constructors$ such that $(P, C) \in \PositionConstructors$.

\begin{theorem}
  Any language representation $\Delta$ with arity map $\Arity$
  satisfies the property that $\PositionConstructors(P) = C$ if and only if $P \in \Arity(C)$.
\end{theorem}

\begin{proof}
  Let $\Delta$ be a language representation with arity map $\Arity$.
  For any $P \in \Positions$, if $\PositionConstructors(P) = C$ then $P \in \Ports{C}$ implies $P \in \Arity(C)$.
  Conversely, for any $C \in \Constructors$, if $P \in \Arity(C)$ then $P \in \Ports{C}$ implies $\PositionConstructors(P) = C$.
\end{proof}
