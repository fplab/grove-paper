%%%%%%%%%%%%%%%%%%%%%%%%%%%%%%%%%%%%%%%%%%%%%%%%%%%%%%%%%%%%%%%%%%%%%%%%%%%%%%%%
% Language Components

\usepackage{xstring}
\usepackage{tensor}

% \selectcolormodel{gray}

% notation
\def\_{\texttt{\textunderscore}}
\newcommand{\id}[1]{\textcolor{gray}{\ensuremath{#1}}}
\newcommand{\eid}[2]{\tensor*{#2}{^{\id{#1}}}}
\newcommand{\abs}[1]{\left\lvert#1\right\rvert}
\newcommand{\figureCode}[1]{\textbf{\texttt{#1}}\vskip1em}
\newcommand{\Z}[1]{\hat{#1}}
\newcommand{\Set}[1][]{\Theta_{#1}}
\newcommand{\SetOf}[1]{\left\{#1\right\}}
\newcommand{\SuchThat}[1]{ : #1}
\newcommand{\SequenceOf}[1]{\left(#1\right)}
\newcommand{\SizeOf}[1]{\left\lvert#1\right\rvert}
\newcommand{\userAction}[1]{\xrightarrow{#1}}
\newcommand{\graphAction}[2]{\big(#1, #2\big)}

% sets
\def\A{\mathcal{A}}
\def\E{\mathcal{E}}
\def\G{\mathcal{G}}
\def\I{\mathcal{I}}
\def\K{\mathcal{K}}
\def\P{\mathcal{P}}
\def\S{\mathcal{S}}
\def\U{\mathcal{U}}
\def\V{\mathcal{V}}
\def\e{\varepsilon}
\def\AA{\textbf{A}}
\def\EE{\textbf{E}}

% Sorts
\newcommand\Exp{\mathsf{Exp}}
\newcommand\Pat{\mathsf{Pat}}
\newcommand\Typ{\mathsf{Typ}}

% objects
\def\e{\varepsilon}
\def\Grove{\gamma}

% judgments
\newcommand{\act}[5]{#1, #2; #3 \userAction{#4} #5}

%%%%%%%%%%%%%%%%%%%%%%%%%%%%%%%%%%%%%%%%%%%%%%%%%%%%%%%%%%%%%%%%%%%%%%%%%%%%%%%%
% Graphs

\DeclareMathOperator{\sortOp}{\text{sort}}
\DeclareMathOperator{\arityOp}{\text{arity}}
\newcommand{\sort}[1]{\sortOp\mathopen{}\left(#1\right)\mathclose{}}
\newcommand{\arity}[1]{\arityOp\mathopen{}\left(#1\right)\mathclose{}}

\DeclareMathOperator{\constructorOp}{\texttt{constructor}}
\DeclareMathOperator{\econstructorOp}{\texttt{econstructor}}
\DeclareMathOperator{\pconstructorOp}{\texttt{pconstructor}}
\DeclareMathOperator{\tconstructorOp}{\texttt{tconstructor}}
\newcommand{\constructor}[1]{\constructorOp\mathopen{}\left(#1\right)\mathclose{}}
\newcommand{\econstructor}[1]{\econstructorOp\mathopen{}\left(#1\right)\mathclose{}}
\newcommand{\pconstructor}[1]{\pconstructorOp\mathopen{}\left(#1\right)\mathclose{}}
\newcommand{\tconstructor}[1]{\tconstructorOp\mathopen{}\left(#1\right)\mathclose{}}

\newcommand{\Edge}[1]{Edge~#1}
\newcommand{\Vertex}[1]{Vertex~#1}
\newcommand{\multiVertex}[1]{\textcolor{red}{\ensuremath{\curlyveedownarrow_{#1}}}}
\newcommand{\cycleVertex}[1]{\textcolor{red}{\ensuremath{\rcirclearrowleft_{#1}}}}
\newcommand{\orphanVertex}[1]{\textcolor{red}{\ensuremath{\mathbf{\ndownarrow_{#1}}}}}
\newcommand{\rootVertex}{v_\text{root}}
\newcommand{\otherVertexVskip}{\vskip0.5em}

% \DeclareMathOperator{\outvertexOp}{\text{outvertex}}
% \newcommand{\outvertex}[1]{\outvertexOp\mathopen{}\left(#1\right)\mathclose{}}

\DeclareMathOperator{\defaultposOp}{\text{defaultpos}}
\newcommand{\defaultpos}[1]{\defaultposOp\mathopen{}\left(#1\right)\mathclose{}}

\newcommand{\parens}[1]{\textcolor{gray}{(}#1\textcolor{gray}{)}}

% Constructors
\newcommand\ExpVar{\mathsf{Exp\_var}}
\newcommand\ExpLam{\mathsf{Exp\_lam}}
\newcommand\ExpApp{\mathsf{Exp\_app}}
\newcommand\ExpNum{\mathsf{Exp\_num}}
\newcommand\ExpPlus{\mathsf{Exp\_plus}}
\newcommand\ExpTimes{\mathsf{Exp\_times}}
\newcommand\PatVar{\mathsf{Pat\_var}}
\newcommand\TypNum{\mathsf{Typ\_num}}
\newcommand\TypArrow{\mathsf{Typ\_arrow}}

% Positions
\newcommand\Root{\mathsf{Root}}
\newcommand\LamParam{\mathsf{Param}}
\newcommand\LamType{\mathsf{Type}}
\newcommand\LamBody{\mathsf{Body}}
\newcommand\AppFun{\mathsf{Fun}}
\newcommand\AppArg{\mathsf{Arg}}
\newcommand\PlusLeft{\mathsf{Left}}
\newcommand\PlusRight{\mathsf{Right}}
\newcommand\TimesLeft{\mathsf{Left}}
\newcommand\TimesRight{\mathsf{Right}}
\newcommand\ArrowArg{\mathsf{Arg}}
\newcommand\ArrowResult{\mathsf{Result}}

%%%%%%%%%%%%%%%%%%%%%%%%%%%%%%%%%%%%%%%%%%%%%%%%%%%%%%%%%%%%%%%%%%%%%%%%%%%%%%%%
% Terms

\newcommand{\varExp}[2]{#1^{#2}}
\newcommand{\numExp}[2]{\underline{#1}^{#2}}
\newcommand{\lamExp}[4]{\lambda^{#1} #2 : #3.#4}
\newcommand{\appExp}[3]{\left(#1~#2\right)^{#3}}
\newcommand{\plusExp}[3]{#1 +^{#2} #3}
\newcommand{\varPat}[2]{#1^{#2}}
\newcommand{\arrowTyp}[3]{#1 \to^{#2} #3}
\newcommand{\numTyp}[1]{Num^{#1}}
\newcommand{\hole}{\ensuremath{\square}} %\textcolor{violet}{\llparenthesis}}\textcolor{violet}{\rrparenthesis}}
% TODO: finish this macro
\newcommand{\conflictHoleForm}[2]{\textcolor{red}{\textbf{\{}}#1\textcolor{red}{\textbf{\}}}_{#2}}
\newcommand{\conflictHole}[1]{%
{\noexpandarg\StrSubstitute{#1}{,}{\textcolor{red}{\;\textbf{|}\;}}[\myargs]%
{\textcolor{red}{\textbf{\{}}\myargs\textcolor{red}{\textbf{\}}}}}}%

\newcommand{\eVar}[2]{\eid{#1}{#2}}
\newcommand{\eFun}[4]{\eid{#1}{\lambda} #2 : #3 . #4}
\newcommand{\eApp}[3]{\eid{#1}{\left(#2~#3\right)}}
\newcommand{\eNum}[2]{\eid{#1}{\underline{#2}}}
\newcommand{\ePlus}[3]{#2~\eid{#1}{\texttt{+}}~#3}
\newcommand{\eTimes}[3]{#2~\eid{#1}{\texttt{*}}~#3}
\newcommand{\pVar}[2]{\eid{#1}{#2}}
\newcommand{\tArrow}[3]{#2 \eid{#1}{\rightarrow} #3}
\newcommand{\tNum}[1]{\eid{#1}{Num}}

% TODO: come up with a metavariable for zippered sets (i.e., Grove components)
%  t = e | q | \tau
%  t^ = e^ | ...
%       \Theta_{NP}, ... are sets of terms
%       \Z{\Theta} = (\Z{t}, \Theta) is a set of terms with a zippered term inside

% TODO: add conflict terms to zippered syntax and cursor erasure

% |> { e1 | ... | en } <|
% { e1^ | ... | en }
% { e1 | ... | en^ }

% |> mpc <|
% |> uc <|

% TODO: sort out which one's "a" and which one's "\alpha" in the paper body (check Hazelnut paper)

% Graph actions (a):
%   - add edge
%   - remove edge

% User actions (\alpha) (indirect):
%
% \gamma^ --\alpha--> a*
% 
% (NP^, MP, U) -- Construct(k) --> 
% (NP, MP^, U) -- Construct(k) --> 
% (NP, MP, U^) -- Construct(k) --> 

% ((|> e <|, NP), MP, U) -- Construct(k) --> 
% v, p; ((|> ?? <|, NP), MP, U) -- Construct(k) -->  (v, p, (u_fresh, k)) \mapsto +

% TODO: define well sorted grove

%%%%%%%%%%%%%%%%%%%%%%%%%%%%%%%%%%%%%%%%%%%%%%%%%%%%%%%%%%%%%%%%%%%%%%%%%%%%%%%%
% Zippered Terms

\newcommand{\cursor}[1]{{\vartriangleright}#1{\vartriangleleft}}
\newcommand{\erase}[1]{#1\mathclose{}^{\diamond}\mathclose{}}

%%%%%%%%%%%%%%%%%%%%%%%%%%%%%%%%%%%%%%%%%%%%%%%%%%%%%%%%%%%%%%%%%%%%%%%%%%%%%%%%
% Decomposition

\DeclareMathOperator{\decompOp}{\texttt{decomp}}
\DeclareMathOperator{\vertexesOp}{\texttt{vertexes}}
\newcommand{\decomp}[1]{\decompOp\mathopen{}\left(#1\right)\mathclose{}}
\newcommand{\vertexes}[1]{\vertexesOp\mathopen{}\left(#1\right)\mathclose{}}

\DeclareMathOperator{\edecompOp}{\texttt{edecomp}}
\DeclareMathOperator{\pdecompOp}{\texttt{pdecomp}}
\DeclareMathOperator{\tdecompOp}{\texttt{tdecomp}}
\DeclareMathOperator{\edecompPrimeOp}{\edecompOp^\prime}
\DeclareMathOperator{\pdecompPrimeOp}{\pdecompOp^\prime}
\DeclareMathOperator{\tdecompPrimeOp}{\tdecompOp^\prime}
\newcommand{\edecomp}[1]{\edecompOp\mathopen{}\left(#1\right)\mathclose{}}
\newcommand{\pdecomp}[1]{\pdecompOp\mathopen{}\left(#1\right)\mathclose{}}
\newcommand{\tdecomp}[1]{\tdecompOp\mathopen{}\left(#1\right)\mathclose{}}
\newcommand{\edecompPrime}[2]{\edecompPrimeOp\mathopen{}\left(#1, #2\right)\mathclose{}}
\newcommand{\pdecompPrime}[2]{\pdecompPrimeOp\mathopen{}\left(#1, #2\right)\mathclose{}}
\newcommand{\tdecompPrime}[2]{\tdecompPrimeOp\mathopen{}\left(#1, #2\right)\mathclose{}}

% helpers
\DeclareMathOperator{\outedgesOp}{\text{outedges}}
\DeclareMathOperator{\ingraphOp}{\text{ingraph}}
\DeclareMathOperator{\parentsOp}{\text{parents}}
\DeclareMathOperator{\ancestorsOp}{\text{ancestors}}
\DeclareMathOperator{\verticesOp}{\text{vertices}}
% \DeclareMathOperator{\childrenOp}{\text{children}}
\DeclareMathOperator{\lfpOp}{\text{lfp}}
\DeclareMathOperator{\ancestorsPrimeOp}{\ancestorsOp^\prime}
\newcommand{\outedges}[2]{\outedgesOp\mathopen{}\left(#1, #2\right)\mathclose{}}
\newcommand{\ingraph}[1]{\ingraphOp\mathopen{}\left(#1\right)\mathclose{}}
% \newcommand{\children}[1]{\childrenOp\mathopen{}\left(#1\right)\mathclose{}}
\newcommand{\parents}[1]{\parentsOp\mathopen{}\left(#1\right)\mathclose{}}
\newcommand{\ancestors}[1]{\ancestorsOp\mathopen{}\left(#1\right)\mathclose{}}
\newcommand{\ancestorsPrime}[1]{\ancestorsPrimeOp\mathopen{}\left(#1\right)\mathclose{}}
\newcommand{\lfp}[1]{\lfpOp\mathopen{}\left(#1\right)\mathclose{}}

\DeclareMathOperator{\minOp}{\text{min}}
\renewcommand{\min}[1]{\minOp\mathopen{}\left(#1\right)\mathclose{}}

\DeclareMathOperator{\rootsOp}{\text{roots}}
\newcommand{\roots}[1]{\rootsOp\mathopen{}\left(#1\right)\mathclose{}}

%%%%%%%%%%%%%%%%%%%%%%%%%%%%%%%%%%%%%%%%%%%%%%%%%%%%%%%%%%%%%%%%%%%%%%%%%%%%%%%%
% Recomposition

\DeclareMathOperator{\recompOp}{\texttt{recomp}}
\DeclareMathOperator{\erecompOp}{\texttt{erecomp}}
\DeclareMathOperator{\precompOp}{\texttt{precomp}}
\DeclareMathOperator{\trecompOp}{\texttt{trecomp}}
\newcommand{\recomp}[1]{\recompOp\mathopen{}\left(#1\right)\mathclose{}}
\newcommand{\erecomp}[1]{\erecompOp\mathopen{}\left(#1\right)\mathclose{}}
\newcommand{\precomp}[1]{\precompOp\mathopen{}\left(#1\right)\mathclose{}}
\newcommand{\trecomp}[1]{\trecompOp\mathopen{}\left(#1\right)\mathclose{}}

%%%%%%%%%%%%%%%%%%%%%%%%%%%%%%%%%%%%%%%%%%%%%%%%%%%%%%%%%%%%%%%%%%%%%%%%%%%%%%%%

% actions
\def\Down{\text{Down}}
\def\Enqueue{\text{Enqueue}}
\def\Left{\text{Left}}
\def\Move{\text{Move}}
\def\Num{\text{Num}}
\def\Right{\text{Right}}
\def\Select{\text{Select}}
\def\Send{\text{Send}}
\def\Up{\text{Up}}

\newcommand{\Construct}[1]{\text{Construct}\mathopen{}\left(#1\right)\mathclose{}}
\newcommand{\Delete}{\text{Delete}}
\newcommand{\Reposition}[2]{\text{Reposition}\mathopen{}\left(#1, #2\right)\mathclose{}}
\newcommand{\Wrap}[2]{\mathrm{Wrap}\mathopen{}\left(#1, #2\right)\mathclose{}}
