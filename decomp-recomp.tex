\documentclass[acmsmall,dvipsnames,10pt,review,anonymous]{acmart}\settopmatter{printfolios=true}

% \title{Decomp-Recomp}

\newcommand{\edgevar}[0]{\ensuremath{\epsilon}}
\newcommand{\set}[1]{\{#1\}}

\begin{document}
% \maketitle

\begin{definition}
    U-top(v) holds if and only if there exists a sequence w(i), $1 \leq i \leq n$, where $n > 1$, such that: 
    \begin{itemize}
        \item w(i+1) is the unique parent of w(i) for all $1\leq i < n$
        \item w(1) = v = w(n)
        \item the id of v is the min of the ids of w(i)
    \end{itemize}
\end{definition}

\begin{definition}
    X-inner(v, w) holds if and only if not U-top(v), v has unique parent v’, and either w = v’ and X-top(w), or X-inner(v’ ,w).
\end{definition}

\begin{lemma}
\label{lem:u-top}
    If U-top(v), then v has unique parent w such that either w = v or U-inner(w, v).
\end{lemma}
\begin{proof}

    $v=w(1)$ has unique parent $w=w(2)$. If $n=2$, then $w=w(2)=v$, and we have $w=v$.
    
    Otherwise, consider the minimum $1<m\leq n$ such that $v=w(m)$. 

    Helper fact: not U-top(w(i)) for all $1 < i < m$.

    \ 
    
    Prove X-inner(w(m-i), v) for all $1\leq i < m-1$, by induction on $i$. 

    Base case: w(m-1). X-inner(w(m-1), v) because not U-top(w(m-1)) by the helper, w(m-1) has unique parent w(m)=v, and U-top(v). 

    Inductive step: w(m-i-1). X-inner(w(m-i-1), v) because not U-top(w(m-i-1)) by the helper, w(m-i-1) has unique parent w(m-i), and by the induction hypothesis, X-inner(w(m-i), v). 
\end{proof}

\begin{theorem}
\label{thm:vertex-class}
    For each vertex v, one and only one of the following holds, and this classification is decidable:
\begin{itemize}
    \item NP-top(v)
    \item MP-top(v) 
    \item U-top(v)
    \item NP-inner(v,w)
    \item MP-inner(v,w)
    \item U-inner(v,w)
\end{itemize}
\end{theorem}
% \begin{proof}
%     (uses lemma \ref{lem:u-top})
% \end{proof}

\begin{corollary}
\label{cor:edge-class}
    For each edge \edgevar, one and only one of the following holds, and this classification is decidable:
\begin{itemize}
    \item NP-edge(\edgevar)
    \item MP-edge(\edgevar) 
    \item U-edge(\edgevar)
\end{itemize}
\end{corollary}

\begin{lemma}
    If X-inner(v, w), then X-top(w).
\end{lemma}

\begin{lemma}
    If X-edge(e, v), then X-top(v).
\end{lemma}

\begin{lemma}
    For each vertex v, v-of(decomp’(v)) = v.
\end{lemma}

\begin{lemma}
\label{lem:redecomp-eq}
    For each vertex v, recomp(decomp(v)) = $\bigcup_{(E\ u`\ v\ p\ v`) \in G}$ \set{(E u’ v p v’)} $\cup$ recomp(decomp’(v’)).
\end{lemma}

\begin{lemma}
\label{lem:parent-top-inner}
    If (E u’ v p v’) and X-top(v), and not MP-top(v’), and not U-top(v’), then X-inner(v’, v).
\end{lemma}

\begin{lemma}
\label{lem:parent-inner-inner}
    If (E u’ v p v’) and X-top(v), and not MP-top(v’), and not U-top(v’), then X-inner(v’, v).
\end{lemma}

\begin{lemma}
\label{lem:redecomp-side1}
    For each w s.t. X-inner(w, v), for each e in recomp(decomp(w)), X-edge(e, v).
\end{lemma}
% uses {lem:parent-inner-inner}

\begin{lemma}
\label{lem:redecomp-source}
    e is in recomp(decomp(source of e))
\end{lemma}

\begin{lemma}
\label{lem:redecomp-side2}
    If X-inner(w, v), then recomp(decomp(w)) $\subseteq$ recomp(decomp(v)). 
\end{lemma}

\begin{lemma}
\label{lem:redecomp-top}
    For each vertex v such that X-top(v), recomp(decomp(v)) = \set{\edgevar s.t. X-edge(\edgevar,v)}.
\end{lemma}
% \begin{proof}
%     (uses lemmas \ref{lem:redecomp-eq}, \ref{lem:redecomp-side1}, \ref{lem:parent-top-inner} for first direction, uses {lem:redecomp-source} and {lem:redecomp-side2} for second direction)
% \end{proof}

\begin{theorem}
\label{thm:redecomp}
    For each graph G, recomp(decomp(G)) = G. 
\end{theorem}
% \begin{proof}
%     (uses lemma \ref{lem:redecomp-top})
% \end{proof}




\end{document}