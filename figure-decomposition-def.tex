
\newcommand{\figureDecompositionDefHelpersContent}{%
\begin{align*}
  \children(v, p) &= \SetOf{v' \SuchThat{\exists \e = (u, v, p, v'), G(\e) = \Plus}} \\
  \parents(v) &= \SetOf{v' \SuchThat{\exists \e = (u, v', p, v), G(\e) = \Plus}} \\
  \ancestors(v) &= \mathopen{}\left( \lfp(\ancestors') \right)\mathclose{}(v) \\
  \ancestors'(v) &= \parents(v) \cup \ancestors'(\parents(v)) \\
  \min\mathopen{}\left(\SetOf{(u_1, k_1), \ldots, (u_n, k_n)}\right)\mathclose{} &= (u_j, k_j) \text{ s.t. } 1 \leq j \leq n \land u_j \le u_i \forall i = 1, \ldots, n
\end{align*}%
}

%%%%%%%%%%%%%%%%%%%%%%%%%%%%%%%%%%%%%%%%%%%%%%%%%%%%%%%%%%%%%%%%%%%%%%%%%%%%%%%%

\newcommand{\figureDecompositionDefDecomp}{%
\begin{align*}
  \decomp{G} &= ((NP, MP, U), D) \\
  \figureDecompositionDefDecompComponents \\
  D &= \SetOf{\e \SuchThat{G(\e) = \Minus}}
\end{align*}%
}

\newcommand{\figureDecompositionDefDecompComponents}{%
  NP &= \SetOf{\edecomp{v} \SuchThat{|\parents(v)| = 0}} \\
  MP &= \SetOf{\edecomp{v} \SuchThat{|\parents(v)| > 1}} \\
  U &= \SetOf{\edecomp{v} \SuchThat{|\parents(v)| = 1 \land v = \min(\ancestors(v))}}%
}

\newcommand{\figureDecompositionDefEdecomp}{%
\begin{align*}
  \edecomp{v=(u, Root)} &= \edecompPrime{v}{Root} \\
  \edecomp{v=(u, \ExpVar(x))} &= x^{\id{u}} \\
  \edecomp{v=(u, \ExpLam)} &= \lambda^{\id{u}} \pdecompPrime{v}{\LamParam} : \tdecompPrime{v}{\LamType} . \edecompPrime{v}{\LamBody} \\
  \edecomp{v=(u, \ExpApp)} &= (\edecompPrime{v}{\AppFun}~\edecompPrime{v}{\AppArg})^{\id{u}} \\
  \edecomp{v=(u, \ExpNum(n))} &= n^{\id{u}} \\
  \edecomp{v=(u, \ExpPlus)} &= \edecompPrime{v}{\PlusLeft}~\texttt{+}^{\id{u}}~\edecompPrime{v}{\PlusRight} \\
  \edecomp{v=(u, \ExpTimes)} &= \edecompPrime{v}{\TimesLeft}~\texttt{*}^{\id{u}}~\edecompPrime{v}{\TimesRight}
\end{align*}%
}

\newcommand{\figureDecompositionDefPdecomp}{%
\begin{align*}
  \pdecomp{v=(u, \PatVar(x))} &= x^{\id{u}}
\end{align*}%
}

\newcommand{\figureDecompositionDefTdecomp}{%
\begin{align*}
  \tdecomp{\tau=(u, \TypArrow)} &= \tdecompPrime{\tau}{\ArrowArg} \rightarrow^{\id{u}} \tdecompPrime{\tau}{\ArrowResult} \\
  \tdecomp{\tau=(u, \TypNum)} &= Num^{\id{u}}
\end{align*}%
}

\newcommand{\figureDecompositionDefEdecompPrime}{%
\begin{align*}
  \edecompPrime{v}{p} = \begin{cases}
    \hole & \children(v,p) = \varnothing \\
    \conflictHole{\edecomp{v_1}, \ldots, \edecomp{v_n}} & \children(v,p) = \SetOf{v_1, \ldots, v_n} \\
    \multiVertex{u} & \children(v,p) = \SetOf{v'} \land |\parents(v')| > 1 \\
    \cycleVertex{u} & \children(v,p) = \SetOf{v' = (u,k)} \land |\parents(v')| = 1 \\
        & \phantom{\children(v,p)} \land v' = \min(\ancestors(v')) \\
    \edecomp{v'} & \text{otherwise} \\
  \end{cases}
\end{align*}%
}

\newcommand{\figureDecompositionDefPdecompPrime}{%
\begin{align*}
  \pdecompPrime{v}{p} = \begin{cases}
    \hole & \children(v,p) = \varnothing \\
    \conflictHole{\pdecomp{v_1},\ldots,\pdecomp{v_n}} & \children(v,p) = \SetOf{v_1, \ldots, v_n} \\
    \multiVertex{u} & \children(v,p) = \SetOf{v'} \land |\parents(v')| > 1 \\
    \cycleVertex{u} & \children(v,p) = \SetOf{v'=(u,k)} \land |\parents(v')| = 1 \\
        & \phantom{\children(v,p)} \land v' = \min(\ancestors(v')) \\
    \pdecomp{v'} & \text{otherwise} \\
  \end{cases}
\end{align*}%
}

\newcommand{\figureDecompositionDefTdecompPrime}{%
\begin{align*}
  \tdecompPrime{v}{p} = \begin{cases}
    \hole & \children(v,p) = \varnothing \\
    \conflictHole{\tdecomp{v_1},\ldots,\tdecomp{v_n}} & \children(v,p) = \SetOf{v_1, \ldots, v_n} \\
    \multiVertex{u} & \children(v,p) = \SetOf{v'} \land |\parents(v')| > 1 \\
    \cycleVertex{u} & \children(v,p) = \SetOf{v' = (u,k)} \land |\parents(v')| = 1 \\
        & \phantom{\children(v,p)} \land v' = \min(\ancestors(v')) \\
    \tdecomp{v'} & \text{otherwise} \\
  \end{cases}
\end{align*}%
}

%%%%%%%%%%%%%%%%%%%%%%%%%%%%%%%%%%%%%%%%%%%%%%%%%%%%%%%%%%%%%%%%%%%%%%%%%%%%%%%%

\newcommand{\figureDecompositionDef}{%
\begin{figure}

\figureDecompositionDefEdecomp

\figureDecompositionDefPdecomp

\figureDecompositionDefTdecomp

\figureDecompositionDefEdecompPrime

\figureDecompositionDefPdecompPrime

\figureDecompositionDefTdecompPrime

\caption{Graph decomposition.}
\label{fig:Graph decomposition definition}
\end{figure}%
}

\newcommand{\figureDecompositionDefHelpers}{%
\begin{figure}
\figureDecompositionDefHelpersContent
\caption{Graph decomposition helpers.}
\label{fig:Graph decomposition definition helpers}
\end{figure}%
}