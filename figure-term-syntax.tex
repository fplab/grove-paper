
\newcommand{\figureTermSyntaxContent}{%
\[
  \arraycolsep=0pt
  \begin{array}{lrlll}
    q & {}\in Pat & {}::={} &
      x^{\id{u}}
      \mid \hole
      \mid \conflictHole{\preid{u}{q}, \ldots, \preid{u}{q}}
      \mid \multiVertex{u}
      \mid \cycleVertex{u}
      \\
    \tau & {}\in Typ & {}::={} &
      \tArrow{u}{\tau}{u}{u}{\tau}
      \mid Num^{\id{u}}
      \mid \hole
      \mid \conflictHole{\preid{u}{\tau}, \ldots, \preid{u}{\tau}}
      \mid \multiVertex{u}
      \mid \cycleVertex{u}
      \\
    e & {}\in Exp & {}::={} &
      x^{\id{u}}
      \mid \eFun{u}{u}{q}{u}{\tau}{u}{e}
      \mid \eApp{u}{e}{u}{u}{e}
      \mid n^{\id{u}}
      \mid \ePlus{u}{e}{u}{u}{e}
      \mid \eTimes{u}{e}{u}{u}{e}
      \mid \hole
      \mid \conflictHole{\preid{u}{e}, \ldots, \preid{u}{e}}
      \mid \multiVertex{u}
      \mid \cycleVertex{u}
      \\
  \end{array}
\]%
}

\newcommand{\figureTermSyntax}{%
\begin{figure}
  \figureTermSyntaxContent
  \caption{Base syntax, as a grammar, for the lambda calculus that we are considering.}
  \label{fig:Base syntax}
\end{figure}%
}