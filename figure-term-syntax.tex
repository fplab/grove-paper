
\newcommand{\figureTermSyntaxContent}{%
\[
  \arraycolsep=0pt
  \begin{array}{lcllll}
    t & {}\in{} & Term & {}::={} &
      e
      \mid \tau
      \mid q
    \\
    e & {}\in{} & Exp & {}::={} &
      \eVar{G}{x}
      \mid \eFun{G}{q}{\tau}{e}
      \mid \eApp{G}{e}{e}
      \mid \eNum{G}{n}
      \mid \ePlus{G}{e}{e}
      \mid \eTimes{G}{e}{e}
      \mid \multiVertex{\e}
      \mid \cycleVertex{\e}
      \mid \conflictHole[i \leq n]{e_i}
      \mid \emptyHole{v}{p}
    \\
    \tau & {}\in{} & Typ & {}::={} &
      \tArrow{G}{\tau}{\tau}
      \mid \tNum{G}
      \mid \multiVertex{\e}
      \mid \cycleVertex{\e}
      \mid \conflictHole[i \leq n]{\tau_i}
      \mid \emptyHole{v}{p}
    \\
    q & {}\in{} & Pat{} & {}::={} &
      \pVar{G}{x}
      \mid \multiVertex{\e}
      \mid \cycleVertex{\e}
      \mid \conflictHole[i \leq n]{q_i}
      \mid \emptyHole{v}{p}
    \\
  \end{array}
\]
}

\newcommand{\figureTermSyntax}{%
\begin{figure}
  \figureTermSyntaxContent
  \caption{Base syntax, as a grammar, for the lambda calculus that we are considering.}
  \label{fig:Base syntax}
\end{figure}%
}