%%%%%%%%%%%%%%%%%%%%%%%%%%%%%%%%
%% \figureWrap
%%%%%%%%%%%%%%%%%%%%%%%%%%%%%%%%
\newedge{WrapPlus}{Root}
\newvertex{WrapPlus}{+}
\newedge{WrapTimes}{L}
\newcommand{\figureWrapMove}{
\begin{figure}
\centering
\begin{minipage}[t]{.45\linewidth}
\hskip0.12\columnwidth
%%%%%%%%
\begin{subfigure}[t]{0.28\linewidth}
\centering
\caption{
\begin{tikzpicture}[remember picture, overlay]
\path (-7.5em,0cm) node (node/Simple/cx) [gray,anchor=base west] {{\textbf{\small Fig.~\ref*{fig:Simple:c}}}};
%%%%
\path (0cm,0cm) node (node/Wrap/a) [anchor=base] {\strut};
\path [draw,->,alice step] (node/Simple/cx) to (-2em, 0cm |- node/Wrap/a);
\end{tikzpicture}
}%
\label{fig:Wrap:a}
\figureCode{\hole{}}%
\begin{tikzpicture}
\path (0cm,0cm) graph[graph style] {
 root[root style]
};
\path (0cm,-1.7cm) graph[graph style] {
 "\vSimpleTimes" -> {
  {},
  "\vSimpleY" [> "\eSimpleY"]
 }
};
\end{tikzpicture}
\end{subfigure}
%%%%%%%%
\hfill
%%%%%%%%
\begin{subfigure}[t]{0.28\linewidth}
\centering
\caption{
\begin{tikzpicture}[remember picture, overlay]
\path (0cm,0cm) node (node/Wrap/b) [anchor=base] {\strut};
\path [draw,->,alice step] (node/Wrap/a) to (-2em, 0cm |- node/Wrap/b);
\end{tikzpicture}
}%
\label{fig:Wrap:b}
\figureCode{\hole{} + \hole{}}%
\begin{tikzpicture}
\path (0cm,0cm) graph[graph style] {
 root[root style] -> {
  "\vWrapPlus" [> "\eWrapPlus", >alice edge, alice node]
 }
};
\path (0cm,-1.7cm) graph[graph style] {
 "\vSimpleTimes" -> {
  {},
  "\vSimpleY" [> "\eSimpleY"]
 }
};
\end{tikzpicture}
\end{subfigure}
%%%%%%%%
\hfill
%%%%%%%%
\begin{subfigure}[t]{0.28\linewidth}
\centering
\caption{
\begin{tikzpicture}[remember picture, overlay]
\path (0cm,0cm) node (node/Wrap/c) [anchor=base] {\strut};
\path [draw,->,alice step] (node/Wrap/b) to (-2em, 0cm |- node/Wrap/c);
\end{tikzpicture}
}%
\label{fig:Wrap:c}
\figureCode{\hole{} * y + \hole{}}%
\begin{tikzpicture}
\path (0cm,0cm) graph[graph style] {
 root[root style] -> {
  "\vWrapPlus" [> "\eWrapPlus"] -> {
   "\vSimpleTimes" [> "\eWrapTimes"', >alice edge] -> {
    {},
    "\vSimpleY" [> "\eSimpleY"]
   },
   {}
  }
 }
};
\end{tikzpicture}
\end{subfigure}
%%%%%%%%
\hfill{}
\caption{Example of Wrapping.  (Note that this figure omits the orphaned \vSimpleX{} since it is no longer relevant to the narrative of this paper.)}%
\label{fig:Wrap}
\end{minipage}
% \end{figure}
% }
%
\hfil
%
%%%%%%%%%%%%%%%%%%%%%%%%%%%%%%%%
%% \figureMove
%%%%%%%%%%%%%%%%%%%%%%%%%%%%%%%%
\newedge{MoveTimes}{R}
% \newcommand{\figureMove}{
% \begin{figure}[H]
\begin{minipage}[t]{.45\linewidth}
\hfill
%\hskip0.12\columnwidth
%%%%%%%%
\begin{subfigure}[t]{0.43\linewidth}
\centering
\caption{
\begin{tikzpicture}[remember picture, overlay]
\path (0cm,0cm) node (node/Move/a) [anchor=base] {\strut};
\path [draw,->,alice step] (node/Wrap/c) to (-2em, 0cm |- node/Move/a);
\end{tikzpicture}
}%
\label{fig:Move:a}
\figureCode{\hole{} + \hole{}}%
\begin{tikzpicture}
\path (0cm,0cm) graph[graph style] {
 root[root style] -> {
  "\vWrapPlus" [> "\eWrapPlus"]
 }
};
\path (0cm,-1.7cm) graph [graph style] {
 "\vSimpleTimes" -> {
  {},
  "\vSimpleY" [> "\eSimpleY"]
 }
};
\end{tikzpicture}
\end{subfigure}
%%%%%%%%
\hfill
%%%%%%%%
\begin{subfigure}[t]{0.43\linewidth}
\centering
\caption{
\begin{tikzpicture}[remember picture, overlay]
\path (0cm,0cm) node (node/Move/b) [anchor=base] {\strut};
\path [draw,->,alice step] (node/Move/a) to (-2em, 0cm |- node/Move/b);
\end{tikzpicture}
}%
\label{fig:Move:b}
\figureCode{\hole{} + \hole{} * y}%
\begin{tikzpicture}
\path (0cm,0cm) graph[graph style] {
 root[root style] -> {
  "\vWrapPlus" [> "\eWrapPlus"] -> {
   {},
   "\vSimpleTimes" [> "\eMoveTimes", >alice edge] -> {
    {},
    "\vSimpleY" [> "\eSimpleY"]
   }
  }
 }
};
\end{tikzpicture}
\end{subfigure}
%%%%%%%%
\hfill{}
\caption{Example of Repositioning Code.}%
\label{fig:Move}
\end{minipage}
\end{figure}
}
