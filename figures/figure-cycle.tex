%%%%%%%%%%%%%%%%%%%%%%%%%%%%%%%%
%% \figureCycle
%%%%%%%%%%%%%%%%%%%%%%%%%%%%%%%%
\newedge{MultiCycleTimes}{L}
\newvertex{MultiCycleTimes}{*}
\newedge{MultiCyclePlus}{R}
\newvertex{MultiCyclePlus}{+}
\newedge{MultiCycleAliceTimes}{R}
\newedge{MultiCycleAlicePlus}{L}
\newedge{MultiCycleBobPlus}{R}
\newedge{MultiCycleBobTimes}{L}
\newcommand{\figureCycle}{
\begin{figure*}
\hfill
% \hskip0.07\columnwidth
%%%%%%%%
\begin{subfigure}[t]{0.25\textwidth}
\centering
\caption{
\begin{tikzpicture}[remember picture, overlay]
% \path (-9em,0cm) node (node/MultiParent/ex) [gray,anchor=base west] {{\textbf{\small Fig.~\ref*{fig:MultiParent:e}}}};
%%%%
\path (0cm,0cm) node (node/Cycle/a) [anchor=base] {\strut};
% \path [draw,->,alice step] (node/MultiParent/ex) to[star] (-2em, 0cm |- node/Cycle/a);
\end{tikzpicture}
}%
\label{fig:Cycle:a}
\figureCode{\parens{\hole{} * \hole{}} * \parens{\hole{} + \hole{}} + \hole{}%
  \otherVertexVskip\phantom{\multiVertex{30} = \multiVertex{32} * \hole{}}%
  \\\phantom{\multiVertex{32} = \multiVertex{30} * \hole{}}}%
\begin{tikzpicture}
\path (0cm,0cm) graph[graph style] {
 root[root style] -> {
  "\vWrapPlus" [> "\eWrapPlus"] -> {
   "\vSimpleTimes" [> "\eNestedPartsBob"'] -> {
    "\vMultiCycleTimes" [> "\eMultiCycleTimes"'],
    "\vMultiCyclePlus" [> "\eMultiCyclePlus"]
   },
   {}
  }
 }
};
\end{tikzpicture}
\end{subfigure}
%%%%%%%%
\hfill
%%%%%%%%
\begin{subfigure}[t]{0.25\textwidth}
\centering
\caption{
\begin{tikzpicture}[remember picture, overlay]
\path (0cm,0cm) node (node/Cycle/b) [anchor=base] {\strut};
\path [draw,->,alice step] (node/Cycle/a) to[star] (-2em, 0cm |- node/Cycle/b);
\end{tikzpicture}
}%
\label{fig:Cycle:b}
\figureCode{\hole{} * \hole{} + \parens{\hole{} + \hole{}} * \hole{}%
  \otherVertexVskip\phantom{\multiVertex{30} = \multiVertex{32} * \hole{}}%
  \\\phantom{\multiVertex{32} = \multiVertex{30} * \hole{}}}%
\begin{tikzpicture}
\path (0cm,0cm) graph[graph style] {
 root[root style] -> {
  "\vWrapPlus" [> "\eWrapPlus"] -> {
   "\vSimpleTimes" [> "\eNestedPartsBob"'] -> {
   },
   {
    "\vMultiCycleTimes" [> "\eMultiCycleAliceTimes", >alice edge, alice node] -> {
     "\vMultiCyclePlus" [> "\eMultiCycleAlicePlus"', >alice edge, alice node],
     {}
    }
   }
  }
 }
};
\end{tikzpicture}
\end{subfigure}
%%%%%%%%
\hfill
%%%%%%%%
\begin{subfigure}[t]{0.25\textwidth}
\centering
\caption{
\begin{tikzpicture}[remember picture, overlay]
\path (0cm,0cm) node (node/Cycle/c) [anchor=base] {\strut};
\path [draw,->,bob step] (node/Cycle/a) to[out=15,in=165,star] (-2em, 0cm |- node/Cycle/c);
\end{tikzpicture}
}%
\label{fig:Cycle:c}
\figureCode{\hole{} * \hole{} + \parens{\hole{} * \hole{} + \hole{}}%
  \otherVertexVskip\phantom{\multiVertex{30} = \multiVertex{32} * \hole{}}%
  \\\phantom{\multiVertex{32} = \multiVertex{30} * \hole{}}}%
\begin{tikzpicture}
\path (0cm,0cm) graph[graph style] {
 root[root style] -> {
  "\vWrapPlus" [> "\eWrapPlus"] -> {
   "\vSimpleTimes" [> "\eNestedPartsBob"'] -> {
   },
   {
    "\vMultiCyclePlus" [> "\eMultiCycleBobPlus", >bob edge, bob node] -> {
     "\vMultiCycleTimes" [> "\eMultiCycleBobTimes"', >bob edge, bob node],
     {}
    }
   }
  }
 }
};
\end{tikzpicture}
\end{subfigure}
%%%%%%%%
\hfill
%%%%%%%%
\begin{subfigure}[t]{0.18\textwidth}
\centering
\caption{
\begin{tikzpicture}[remember picture, overlay]
\path (0cm,0cm) node (node/Cycle/d) [anchor=base] {\strut};
\path [draw,->,merge step] (node/Cycle/b) to[out=15,in=165] (-2em, 0cm |- node/Cycle/d);
\path [draw,->,merge step] (node/Cycle/c) to (-2em, 0cm |- node/Cycle/d);
\end{tikzpicture}
}%
\label{fig:Cycle:d}
\figureCode{\parens{\hole{} * \hole{}} + $\binaryConflictHole{\multiVertex{\vidMultiCycleTimes}}{\multiVertex{\vidMultiCyclePlus}}$%
  \otherVertexVskip\multiVertex{\vidMultiCycleTimes} = \multiVertex{\vidMultiCyclePlus} * \hole{}%
  \\\multiVertex{\vidMultiCyclePlus} = \multiVertex{\vidMultiCycleTimes} * \hole{}}%
\begin{tikzpicture}
\path (0cm,0cm) graph[graph style] {
 root[root style] -> {
  "\vWrapPlus" [> "\eWrapPlus"] -> {
   "\vSimpleTimes" [> "\eNestedPartsBob"'anchor=-15] -> {
   },
   v1/"\vMultiCycleTimes" [> "\eMultiCycleAliceTimes"anchor=155, >merge edge, merge node],
   {},
   v2/"\vMultiCyclePlus" [> "\eMultiCycleBobPlus"anchor=-165, >merge edge, merge node]
  }
 }
};
\path[draw,-{>[bend]},merge edge] (v1) to ["\eMultiCycleAlicePlus"anchor=south,out=-60,in=-120] (v2);
\path[draw,-{>[bend]},merge edge] (v2) to ["\eMultiCycleBobTimes"anchor=north,out=-90,in=-90] (v1);
\end{tikzpicture}
\end{subfigure}
%%%%%%%%
%\hfill{}
\caption{(a) We start with a tree with a multiplication,  \vMultiCycleTimes{}, and addition, \vMultiCyclePlus{}, at the leaves. (b) Alice relocates them both, such that \vMultiCycleTimes{} is the parent of \vMultiCyclePlus{}. (c) Bob relocates them both, such that \vMultiCyclePlus{} is the parent of the \vMultiCycleTimes{}. (d) In the merged state, there is a cycle in the graph. Because the terms have a common parent, there is a local conflict. Because the cycle is connected to the rest of the graph, the cycle is broken during decomposition by relocation conflict references as shown.}
% TODO replace screenshots with text
% TODO make ref code gray
% TODO move whole figure left
% TODO remove cursor from screenshots
% TODO add multiparent box
% TODO only show parens where needed - figure 10, 11 - by associativity
% TODO repplace cycle screenshot with cycles box %
\Description{This figures shows an example of a cycle}
\label{fig:Cycle}
\end{figure*}
}
