\newvertex{DecompPlusA}{+}
\newedge{DecompTimesA}{L}
\newvertex{DecompTimesA}{*}
\newedge{DecompPlusAX}{R}
\newvertex{DecompX}{x}
\newedge{DecompTimesAX}{R}
%
\newedge{DecompPlusB}{L}
\newvertex{DecompPlusB}{+}
\newedge{DecompTimesB}{R}
\newvertex{DecompTimesB}{*}
\newedge{DecompY}{R}
\newvertex{DecompY}{y}
\newedge{DecompZ}{R}
\newvertex{DecompZ}{z}
%
\newedge{DecompTimes}{L}
%
\newcommand{\figureDecompExample}{
\begin{figure}
  \centering
  \begin{subfigure}{0.5\textwidth}
    \centering
    \begin{tikzpicture}
      \path (0cm,0cm) graph[graph style] {
        root[root style] -> 
        {
        a/{\vDecompPlusA} [> "\eSimpleTimes"] -> {
          "\vDecompTimesA" [> "\eDecompTimesA"'] -> {
          {},
          b/"\vDecompX" [> "\eDecompTimesAX"']
          },
          {}
        }
        },
      };
      \path [draw,->] (a) to ["\eDecompPlusAX",out=-60,in=60] (b);
      %
      \path (2cm,-0.85cm) graph[graph style] { plus/"\vDecompPlusB" };
      \path (3cm,-0.85cm) graph[graph style] { times/"\vDecompTimesB" };
      \path (3.25cm,-1.75cm) graph[graph style] { y/"\vDecompY" };
      \path[draw,->] (times) to ["\eDecompPlusB"'anchor=north,out=-135,in=-45] (plus);
      \path[draw,->] (plus) to ["\eDecompTimesB"anchor=south,out=45,in=135] (times);
      \path[draw,->] (times) to ["\eDecompY"anchor=west] (y);
      %
      \path (1.55cm,-2.1cm) graph[graph style] { plus/"\vDecompZ" };
    \end{tikzpicture}
    \caption{A graph}%
    \label{fig:Decomposition example graph}
  \end{subfigure}%
  \begin{subfigure}{.5\textwidth}
    \centering
    \begin{align*}
      \Theta = 
      \{ & (\emptyHole{(40,\texttt{*},L)}~\texttt{*}^{\id{40}}~(\eid{43}{}\multiVertex{43,(42,x)}))~\texttt{+}^{\id{38}}~(\eid{41}{}\multiVertex{41,(42,x)}), \\
      & \eid{52}{z}, \eid{42}{x}, \emptyHole{(46,\texttt{+},L)}~\texttt{+}^{\id{46}}~(\eid{45}{}\cycleVertex{45,(46,\texttt{+})}~\texttt{*}^{\id{48}}~(\eid{49}{}y^{\id{50}}))\} \\
    \end{align*}
    \caption{The grove corresponding to the graph}%
    \label{fig:Decomposition example grove}
  \end{subfigure}
  \begin{subfigure}{.5\textwidth}
    \centering
    \begin{align*}
      \text{t}_r & = \eid{1}{}  (\emptyHole{(40,\texttt{*},L)}~\texttt{*}^{\id{40}}~(\eid{43}{}\multiVertex{43,(42,x)}))~\texttt{+}^{\id{38}}~(\eid{41}{}\multiVertex{41,(42,x)})\\
      \text{NP} & = \{ \eid{52}{z} \} \\
      \text{MP} & = \{ \eid{42}{x} \} \\
      \text{U} & = \{ \emptyHole{(46,\texttt{+},L)}~\texttt{+}^{\id{46}}~(\eid{45}{}\cycleVertex{45,(46,\texttt{+})}~\texttt{*}^{\id{48}}~(\eid{49}{}y^{\id{50}})) \}
    \end{align*}
    \caption{The partitioned grove corresponding to the graph}%
    \label{fig:Decomposition example partitioned grove}
  \end{subfigure}
  \caption{Example of graph decomposition}%
  \Description{This figure describes the decomposition of a graph to its corresponding grove}
  \label{fig:Decomposition example}
\end{figure}
}
