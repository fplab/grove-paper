%%%%%%%%%%%%%%%%%%%%%%%%%%%%%%%%
%% \figureMultiChild
%%%%%%%%%%%%%%%%%%%%%%%%%%%%%%%%
\newedge{MultiChildAlice}{R}
\newvertex{MultiChildAlice}{x}
\newedge{MultiChildBob}{R}
\newvertex{MultiChildBob}{y}
\newcommand{\figureMultiChild}{
\begin{figure}
%\hfill
\hskip0.12\columnwidth
%%%%%%%%
\begin{subfigure}[t]{0.21\linewidth}
\centering
\caption{
\begin{tikzpicture}[remember picture, overlay]
\path (-7.5em,0cm) node (node/NestedParts/cx) [gray,anchor=base west] {{\textbf{\small \ref*{fig:NestedParts:c}}}};
%%%%
\path (0cm,0cm) node (node/MultiChild/a) [anchor=base] {\strut};
\path [draw,->,alice step] (node/NestedParts/cx) to (-2em, 0cm |- node/MultiChild/a);
\end{tikzpicture}
}%
\label{fig:MultiChild:a}
\figureCode{w * v + x}%
\begin{tikzpicture}
\path (0cm,0cm) graph[graph style] {
 root[root style] -> {
  "\vWrapPlus" [> "\eWrapPlus"] -> {
   "\vSimpleTimes" [> "\eNestedPartsBob"'] -> {
    "\vNestedPartsAlice" [> "\eNestedPartsAlice"'],
    "\vDifferentPartsBob" [> "\eDifferentPartsBob"]
   },
   {
    "\vMultiChildAlice" [> "\eMultiChildAlice", >alice edge, alice node]
   }
  }
 }
};
\end{tikzpicture}
\end{subfigure}
%%%%%%%%
\hfill
%%%%%%%%
\begin{subfigure}[t]{0.21\linewidth}
\centering
\caption{
\begin{tikzpicture}[remember picture, overlay]
\path (0cm,0cm) node (node/MultiChild/b) [anchor=base] {\strut};
\path [draw,->,bob step] (node/NestedParts/cx) to[out=15,in=165] (-2em, 0cm |- node/MultiChild/b);
\end{tikzpicture}
}%
\label{fig:MultiChild:b}
\figureCode{w * v + y}%
\begin{tikzpicture}
\path (0cm,0cm) graph[graph style] {
 root[root style] -> {
  "\vWrapPlus" [> "\eWrapPlus"] -> {
   "\vSimpleTimes" [> "\eNestedPartsBob"'] -> {
    "\vNestedPartsAlice" [> "\eNestedPartsAlice"'],
    "\vDifferentPartsBob" [> "\eDifferentPartsBob"]
   },
   "\vMultiChildBob" [> "\eMultiChildBob", >bob edge, bob node]
  }
 }
};
\end{tikzpicture}
\end{subfigure}
%%%%%%%%
\hfill
%%%%%%%%
\begin{subfigure}[t]{0.4\linewidth}
\centering
\caption{
\begin{tikzpicture}[remember picture, overlay]
\path (0cm,0cm) node (node/MultiChild/c) [anchor=base] {\strut};
\path [draw,->,merge step] (node/MultiChild/a) to[out=15,in=165] (-2em, 0cm |- node/MultiChild/c);
\path [draw,->,merge step] (node/MultiChild/b) to (-2em, 0cm |- node/MultiChild/c);
\end{tikzpicture}
}%
\label{fig:MultiChild:c}
\figureCode{w * v + $\binaryConflictHole{x}{y}$}%
\begin{tikzpicture}
\path (0cm,0cm) graph[graph style] {
 root[root style] -> {
  "\vWrapPlus" [> "\eWrapPlus"] -> {
   "\vSimpleTimes" [> "\eNestedPartsBob"'anchor=-15] -> {
    "\vNestedPartsAlice" [> "\eNestedPartsAlice"'],
    "\vDifferentPartsBob" [> "\eDifferentPartsBob"]
   },
   "\vMultiChildAlice" [> "\eMultiChildAlice"anchor=153, >merge edge, merge node],
   {},
   "\vMultiChildBob" [> "\eMultiChildBob"anchor=195, >merge edge, merge node]
  }
 }
};
\end{tikzpicture}
\end{subfigure}
%%%%%%%%
%\hfill{}
\caption{(a) Alice fills the hole at the \li{R} position of the addition with \li{x}. (b) Bob fills the same hole with \li{y}. (c) When the corresponding patches are merged, there are two edges in the \li{R} position of the addition. Decomposition turns these into a \emph{local conflict}, leaving it to the users to resolve the problem by performing normal edits.}%
\Description{This figure shows an example of multi-child conflicts}
\label{fig:MultiChild}
\end{figure}
}
