%%%%%%%%%%%%%%%%%%%%%%%%%%%%%%%%
%% \figureMultiParent
%%%%%%%%%%%%%%%%%%%%%%%%%%%%%%%%
\newedge{MultiParentAlice}{R}
\newedge{MultiParentBob}{R}
\newcommand{\figureMultiParent}{
\begin{figure*}
\hfill
% \hskip0.1\columnwidth
%%%%%%%%
\begin{subfigure}[t]{0.12\linewidth}
\centering
\caption{
\begin{tikzpicture}[remember picture, overlay]
% \path (-7.5em,0cm) node (node/MultiChild/cx) [gray,anchor=base west] {{\textbf{\small Fig.~\ref*{fig:MultiChild:c}}}};
%%%%
\path (0cm,0cm) node (node/MultiParent/a) [anchor=base] {\strut};
% \path [draw,->,alice step] (node/MultiChild/cx) to[star] (-2em, 0cm |- node/MultiParent/a);
\end{tikzpicture}
}%
\label{fig:MultiParent:a}
\figureCode{w * \hole{} + \hole{}\otherVertexVskip\phantom{\multiVertex{58} = w}}%
\begin{tikzpicture}
\path (0cm,0cm) graph[graph style] {
 root[root style] -> {
  "\vWrapPlus" [> "\eWrapPlus"] -> {
   "\vSimpleTimes" [> "\eNestedPartsBob"'] -> {
    "\vNestedPartsAlice" [> "\eNestedPartsAlice"'],
    {}
   },
   {}
  }
 }
};
\end{tikzpicture}
\end{subfigure}
%%%%%%%%
\hfill
%%%%%%%%
\begin{subfigure}[t]{0.14\linewidth}
\centering
\caption{
\begin{tikzpicture}[remember picture, overlay]
\path (0cm,0cm) node (node/MultiParent/b) [anchor=base] {\strut};
\path [draw,->,alice step] (node/MultiParent/a) to[star] (-2em, 0cm |- node/MultiParent/b);
\end{tikzpicture}
}%
\label{fig:MultiParent:b}
\figureCode{\hole{} * w + \hole{}\otherVertexVskip\phantom{\multiVertex{58} = w}}%
\begin{tikzpicture}
\path (0cm,0cm) graph[graph style] {
 root[root style] -> {
  "\vWrapPlus" [> "\eWrapPlus"] -> {
   "\vSimpleTimes" [> "\eNestedPartsBob"'] -> {
    {},
    "\vNestedPartsAlice" [> "\eMultiParentAlice"', >alice edge]
   },
   {}
  }
 }
};
\end{tikzpicture}
\end{subfigure}
%%%%%%%%
\hfill
%%%%%%%%
\begin{subfigure}[t]{0.14\linewidth}
\centering
\caption{
\begin{tikzpicture}[remember picture, overlay]
\path (0cm,0cm) node (node/MultiParent/c) [anchor=base] {\strut};
\path [draw,->,bob step] (node/MultiParent/a) to[out=15,in=165,star] (-2em, 0cm |- node/MultiParent/c);
\end{tikzpicture}
}%
\label{fig:MultiParent:c}
\figureCode{\hole{} * \hole{} + w\otherVertexVskip\phantom{\multiVertex{58} = w}}%
\begin{tikzpicture}
\path (0cm,0cm) graph[graph style] {
 root[root style] -> {
  "\vWrapPlus" [> "\eWrapPlus"] -> {
   "\vSimpleTimes" [> "\eNestedPartsBob"'],
   "\vNestedPartsAlice" [> "\eMultiParentBob", >bob edge]
  }
 }
};
\end{tikzpicture}
\end{subfigure}
%%%%%%%%
\hfill
%%%%%%%%
\begin{subfigure}[t]{0.18\linewidth}
\centering
\caption{
\begin{tikzpicture}[remember picture, overlay]
\path (0cm,0cm) node (node/MultiParent/d) [anchor=base] {\strut};
\path [draw,->,merge step] (node/MultiParent/b) to[out=15,in=165] (-2em, 0cm |- node/MultiParent/d);
\path [draw,->,merge step] (node/MultiParent/c) to[out=15,in=165] (-2em, 0cm |- node/MultiParent/d);
\end{tikzpicture}
}%
\label{fig:MultiParent:d}
\figureCode{\hole{} * \multiVertex{58} + \multiVertex{58}\otherVertexVskip\multiVertex{58} = w}%
\begin{tikzpicture}
\path (0cm,0cm) graph[graph style] {
 root[root style] -> {
  a/{\vWrapPlus} [> "\eWrapPlus"] -> {
   "\vSimpleTimes" [> "\eNestedPartsBob"'] -> {
    {},
    b/"\vNestedPartsAlice" [> "\eMultiParentAlice"', >merge edge]
   },
   {}
  }
 },
};
\path [draw,->,merge edge] (a) to ["\eMultiParentBob",out=-60,in=60] (b);
\end{tikzpicture}
\end{subfigure}
%%%%%%%%
\hfill
%%%%%%%%
\begin{subfigure}[t]{0.13\linewidth}
\centering
\caption{
\begin{tikzpicture}[remember picture, overlay]
\path (0cm,0cm) node (node/MultiParent/e) [anchor=base] {\strut};
\path [draw,->,alice step] (node/MultiParent/d) to (-2em, 0cm |- node/MultiParent/e);
\end{tikzpicture}
}%
\label{fig:MultiParent:e}
\figureCode{\hole{} * w + \hole{}\otherVertexVskip\phantom{\multiVertex{58} = w}}%
\begin{tikzpicture}
\path (0cm,0cm) graph[graph style] {
 root[root style] -> {
  v8/"\vWrapPlus" [> "\eWrapPlus"] -> {
   v2/"\vSimpleTimes" [> "\eNestedPartsBob"'] -> {
    {},
    "\vNestedPartsAlice" [> "\eMultiParentAlice"']
   },
   {}
  }
 }
};
\end{tikzpicture}
\end{subfigure}
%%%%%%%%
\hfill{}
\caption{(a) We start in a state with a variable, \li{w}, and two holes. (b) Alice relocates \li{w} to the left hole. (c) Bob relocates \li{w} to the right hole. (d) After merging, vertex \vNestedPartsAlice{} has two incoming edges, i.e. it has a \emph{relocation conflict}. The corresponding decomposition leaves a \emph{relocation conflict reference} at both locations, partially addressing the \textbf{relocation conflict problem}. Terms that have a relocation conflict are tracked separately by decomposition. (e) The relocation conflict can be resolved by deleting all but one reference.}%
\Description{This figure depicts an example of relocation conflicts}
\label{fig:MultiParent}
\end{figure*}
}
