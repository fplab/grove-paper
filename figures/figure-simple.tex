%%%%%%%%%%%%%%%%%%%%%%%%%%%%%%%%
%% \figureSimple
%%%%%%%%%%%%%%%%%%%%%%%%%%%%%%%%
\newedge{SimpleTimes}{Root}%
\newvertex{SimpleTimes}{*}%
\newedge{SimpleX}{L}%
\newvertex{SimpleX}{x}%
\newedge{SimpleY}{R}%
\newvertex{SimpleY}{y}%
\newcommand{\figureSimple}{
\begin{figure}
\hfill
%%%%%%%%
\begin{subfigure}[t]{0.32\linewidth}
\centering
\caption{
\begin{tikzpicture}[remember picture, overlay]
\path (0cm,0cm) node (node/Simple/a) [anchor=base] {\strut};
\end{tikzpicture}
}%
\label{fig:Simple:a}
\figureCode{x * \hole{}}%
\begin{tikzpicture}
\path (0cm,0cm) graph[graph style] {
 root[root style, color=black] -> {
  "\vSimpleTimes" [> "\eSimpleTimes", >color=black, color=black] -> {
   "\vSimpleX" [> "\eSimpleX"', >color=black, color=black],
   {}
  }
 }
};
\end{tikzpicture}
\end{subfigure}
%%%%%%%%
\begin{subfigure}[t]{0.32\linewidth}
\centering
\caption{
\begin{tikzpicture}[remember picture, overlay]
\path (0cm,0cm) node (node/Simple/b) [anchor=base] {\strut};
\path [draw,->,alice step] (node/Simple/a) to (-2em, 0cm |- node/Simple/b);
\end{tikzpicture}
}%
\label{fig:Simple:b}
\figureCode{x * y}%
\begin{tikzpicture}
\path (0cm,0cm) graph[graph style] {
  root[root style] -> {
  "\vSimpleTimes" [> "\eSimpleTimes"] -> {
    "\vSimpleX" [> "\eSimpleX"'],
    "\vSimpleY" [> "\eSimpleY", >alice edge, alice node]
  }
  }
};
\end{tikzpicture}
\end{subfigure}
%%%%%%%%
\hfill
%%%%%%%%
\begin{subfigure}[t]{0.32\linewidth}
\centering
\caption{
\begin{tikzpicture}[remember picture, overlay]
\path (0cm,0cm) node (node/Simple/c) [anchor=base] {\strut};
\path [draw,->,alice step] (node/Simple/b) to (-2em, 0cm |- node/Simple/c);
\end{tikzpicture}
}%
\label{fig:Simple:c}
\figureCode{\hole{} * y}%
\begin{tikzpicture}
\path (0cm,0cm) graph[graph style] {
  root[root style] -> {
  "\vSimpleTimes" [> "\eSimpleTimes"] -> {
    {},
    "\vSimpleY" [> "\eSimpleY"]
  }
  }
};
\path (-0.25cm,-1.7cm) graph[graph style] {
  "\vSimpleX"
};
\end{tikzpicture}
\end{subfigure}
%%%%%%%%
\hfill{}
\caption{(a) We represent collaborative program sketches as graphs. (b) Hole filling translates to edge insertion. (c) Term deletion (or cut) deletes an edge, but the vertex persists. (We omit it in subsequent figures.)}%
\Description{This figure shows some simple edits with holes}
\label{fig:Simple}
\vspace{-10px}
\end{figure}
}
