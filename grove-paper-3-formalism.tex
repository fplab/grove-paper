
\section{Formalism}%
\label{sec:Formalism}

We will now make the intuitions developed in the previous section precise
by defining a collaborative structure editor calculus called Grove.
We begin in Sec.~\ref{sub:Syntax} with the syntax, then discuss
the internal graph representation in Sec.~\ref{sub:Convergent Graphs},
...


TODO: point of this section is commutativity theorem

TODO: (though as we show in \autoref{sec:Formalism} creating a vertex is implicit in edge creation)

The formalism is rather simple, \autoref{sec:Grove By Example} shows
that it is sufficient to handle many common situations,
and this simplicity aids in the predictability.

\figureTermSyntax

\subsection{Syntax}%
\label{sub:Syntax}

The language we are modeling is shown in \autoref{fig:Base syntax}, standard for a simply-typed lambda calculus except that all sorts have empty holes (represented by $\hole$) and conflict holes (of the form $\conflictHole{q,\ldots,q}$).
In particular, the argument of a lambda can be either a bound variable or a hole.
Additionally, all non-hole forms have a superscripted identifier $u \in \U$.

% Turn that into a constructor set K + position set P + arity function (has
% type: K -> $\wp$(P)).

% TODO: replace positions with macros

\figureArity

\subsection{Convergent Graphs}%
\label{sub:Convergent Graphs}

\subsubsection{Graphs}%
\label{sub:Graphs}

A graph $G : \E \rightarrow \Sigma$ is a function from edges to edge states,
where $\E = \U \times \V \times \P \times \V$,
unique IDs are drawn from some suitable set $\U$ equipped with a total ordering $\leq$,
vertices are drawn from $\V$,
positions are drawn from $\P$,
and edge states are drawn from $\Sigma$, all defined below.

Each edge $\e = (u, v, p, v^\prime)$ has a unique name, $u \in \U$, and connects
the child position $p$ of originating vertex $v$ to destination vertex
$v^\prime$. The state of each edge, $\e$, is determined by the corresponding
edge state, $G(\e) \in \Sigma = \{ \bot, \Plus, \Minus \}$. When $G(\e) = \bot$,
$\e$ has not been created or deleted. When $G(\e)$ is $\Plus$, the edge has been
created. When $G(\e)$ is $\Minus$, the edge has been deleted. The total
ordering $\bot \sqsubset \Plus \sqsubset \Minus$ forms a lattice over $\Sigma$.

A vertex $v = (u, k)$ identifies a distinct instance of some constructor $k \in
\K$, corresponding to a non-hole form defined in \autoref{fig:Base syntax}. Each
contructor has an arity, which we define as a set of positions. A position $p
\in \P$ identifies a location at some originating vertex from which edges to
destination vertices may originate. The $\arity$ function in
\autoref{fig:Constructors, Indexes and Arity} gives the complete mapping from
constructors to potential edge positions with their corresponding sorts. Every
graph contains a distinguished root vertex, root constructor, and root position,
respectively named $v_0$, $k_0$, and $p_0$.

\vspace*{\baselineskip}

% TODO: add a lead-in sentence here

%%%%%%%%%%%%%%%%%%%%%%%%%%%%%%%%%%%%%%%%%%%%%%%%%%%%%%%%%%%%%%%%%%%%%%%%%%%%%%%%
% TODO: put this in a definition box

A graph $G$ is well sorted if
for all $\e = (u, (u_1, k_1), p, (u_2, k_2))$
such that $G(\e) \neq \bot$,
we have $(p, \sort(k_2)) \in \arity(k_1)$.

%%%%%%%%%%%%%%%%%%%%%%%%%%%%%%%%%%%%%%%%%%%%%%%%%%%%%%%%%%%%%%%%%%%%%%%%%%%%%%%%


% In decomp section:
% - If a graph is well sorted, then there exists a grove. (i.e., exhaustiveness check for expr)
%   - want to show that expr is a total function on Exp's
% - If decomp(G) = Grove, then vertices(Grove) = vertices(edges(G))
%   - every vertex should be part of the decomposition
% - correctness of grove components:
%   - e_r: rootVertex(e_r) = v_r
%   - MP: for every expression in MP, |parents(the root vertex)| > 1
%   - NP: for every expression in NP, |parents(the root vertex)| = 0
%   - U: for every expression in U, |parents(the root vertex)| = 1 and the root vertex = min(ancestors(the root vertex))
%     - every other vertex notin U has exactly 1 parent and is not its own min-ancestor
%   - we've accounted for every edge:
%     - want to say edges(Grove) = edges(G), but we don't have a clean way to say that
%     - there is a choice from (an infinite family of graphs) recomposition(Grove) (which are all structurally equivalent to G, modulo edge ids)

% correctness covers:
% - invertibility (w/ some identity assignment)
% - parentage (things are multiparented correctly)

\vspace*{\baselineskip}

% TODO: should graph "recomposition" have an algorithm, too?
% TODO: update this to match our current understanding
A graph can be derived from an expression in the calculus by starting with an
empty graph, selecting unique ids $u, u_k$ and constructor $k$ for the outermost
expression, and creating edge $\e = (u, v_0, p_0, (u_k, k))$. Then, for each
sub-term at some position $p \in \arity(k)$, select unique id $u'$ and
constructor $k'$ and create edge $e' = (u', v, p, v')$ with $v = (u_k, k)$, $v'
= (u', k')$. Continuing in this manner, treat each sub-term as if it were the
outermost, starting with the ones at positions in $\arity(k')$, and create edges
to the vertices representing their immediate sub-terms until only empty holes
remain. Note that empty holes are not modeled explicitly in the graph. Wherever
an empty hole appears in an expression, the corresponding graph position has no
edges originating from it.

% You can map from syntax to graph by selecting
% unique IDs $u$ (for both edges and vertexes).
% - holes are not explicit in the graph

\subsubsection{Graph Edits}%
\label{sub:Graph Edits}

We define a graph edit $\alpha \in \A = \left\{\Plus, \Minus\right\} \times \E$
as a pair of an edge state, excluding $\bot$, and an edge. As a shorthand, we
define $+\e$ to be $(\Plus, \e)$ and $-\e$ to be $(\Minus, \e)$.

We define the join operation $s_1 \sqcup s_2$ to be the least upper bound of
$s_1$ and $s_2$ with respect to the $\sqsubset$ ordering. Accordingly, $\Plus
\sqcup \bot = \Plus$ means that edges can only be created if they do not already
exist, and $\Plus \sqcup \Minus = \Minus = \Minus \sqcup \Minus$ means that once
an edge is deleted it can never be restored.

We define the semantics of graph actions via the following transition relation
between graphs.
\[
  G \overset{(s,\e)}{\longrightarrow} G\left[ \e \mapsto s \sqcup G(\e) \right\}
\]
Applying graph action $\left(s, \e\right)$ to graph $G$ results in the
updated graph $G' = G\left[\e \mapsto s \sqcup G(\e) \right]$, wherein the edge
state associated with edge $\e$ in $G$ becomes the join of $s$ with the state of
$\e$ in $G$, and $G'(\e') = G(\e')$ for all other $\e' \ne \e$ in the domain of
$G$.

% Graph action semantics is a transition system between graphs.
% (join with whatever it was before)

% Intuitively, joining any edge state with $\bot$ is always the non-$\bot$ edge state (once an edge exists, it always exists).
% Joining anything with $\Minus$ is always $\Minus$ (once an edge is marked as deleted, it is permanently deleted).

\subsection{Commutativity}%
\label{sub:Commutativity:formal}

In order to define commutativity, we first define a lattice over $G$.
Then we show that for any graphs $G$ and $G'$ and graph action $\alpha$,
$G \overset{\alpha}\rightarrow G'$
iff $G \sqcup \llbracket\alpha\rrbracket = G'$.
Thus the commutativity of graph actions is established
by the commutativity of the join operations defining those graph actions.

Discuss uniqueness of uuid

\begin{lemma}[Join Commutativity]
  \label{lem:Join Commutativity}
\end{lemma}

\begin{lemma}[Joining]
  \label{lem:Joining}
  For all graphs $G$ and $G'$ and all edit actions $\alpha$,
  $G \overset{\alpha}{\longrightarrow} G'$
  iff $G \sqcup \llbracket\alpha_1\rrbracket = G'$.
\end{lemma}
\begin{proof}
  If $G \overset{\alpha_1\alpha_2}{\longrightarrow} G'$
\end{proof}


\begin{theorem}[Commutativity]
  \label{thm:Commutativity}
  For all graphs $G$ and $G'$ and all edit actions $\alpha_1$ and $\alpha_2$,
  if $G \overset{\alpha_1\alpha_2}{\longrightarrow} G'$,
  then $G \overset{\alpha_2\alpha_1}{\longrightarrow} G'$.
\end{theorem}
\begin{proof}
  Using \autoref{lem:Joining} to unfold $\rightarrow$ into $\sqcup$,
  $G \sqcup \llbracket\alpha_1 \sqcup \llbracket \sqcup \alpha_2 \rrbracket = G'$.
  Then by \autoref{lem:Join Commutativity},
  $G \sqcup \llbracket\alpha_2 \sqcup \llbracket \sqcup \alpha_1 \rrbracket = G'$.
  Finally using \autoref{lem:Joining} to fold $\sqcup$ into $\rightarrow$,
  $G \overset{\alpha_2\alpha_1}{\longrightarrow} G'$.
\end{proof}

\subsubsection{Agda Mechanization}%
\label{sub:Agda Mechanization}

\subsection{Graph Decomposition}%
\label{sub:Graph Decomposition}

% NOTE base grammar has holes. rooted expression grammar has refs, conflicts,
% and multiparents

% NOTE: this whole section is for describing the 4-tuple

Trees are often easier to work with than arbitrary graphs directly, for example to present the graph to users or for type checking.
A graph can be decomposed into a collection of expressions governed by the grammar in \autoref{fig:Base syntax}, which we organize into a structure called a grove.

\figureDecompExample

% TODO: make sure we're not claiming to invent the term "unicycular graph"
% TODO: maybe add a forward reference at end here to Implementation section
% TODO: make ID of y lower than the unicycle root, to show what we're minimizing

A grove is a 3-tuple of the following form:
\[
  (NP, MP, U, D)
\]
where
\begin{align*}
  % e_r &= \expr(v_0) \\
  NP &= \{\edecomp{v} : |\parents(v)| = 0\} \\
  MP &= \{\edecomp{v} : |\parents(v)| > 1\} \\
  U  &= \{\edecomp{v} : |\parents(v)| = 1 \wedge v = \min(\ancestors(v))\} \\
  D  &= \{\e : G(\e) = \Minus\}
\end{align*}
For example, the grove corresponding to the graph in \autoref{fig:Decomposition example graph} is
given in \autoref{fig:Decomposition example grove}.

Let us now examine each of the four components in more detail.
%
The first component $NP$ is the set of expressions rooted at a vertex with no parents, including the main root vertex.
% TODO: change vRoot to dot w/ subscript
In the example, $e_r$ is the expression corresponding to the main root, $\vRoot$.
% TODO: change z_... to macro use w/ same appearance as in the graph
In addition, $z_{\id{52}}$ has no parents because it was deleted, so it also appears in $NP$.

% The first component $e_r$ is the expression derived from the root vertex $v_0$.
% \autoref{fig:Decomposition expr} defines $\expr(v)$ which determines the expression derived from $v$.
% There is one rule for each term constructor.
% For example, the expression derived from ${+}_{\id{38}} = (38, \ExpPlus)$ is $\expr'({+}_{\id{38}}, \PlusLeft) \text{+}^{\id{38}} \expr'({+}_{\id{38}}, \PlusRight)$.
% In each child position, $\expr'(v, p)$ determines the subterm for the children of $v$ at position $p$.
% An empty hole $\hole$ appears if there are no children at the given position.
% A conflict hole $\conflictHole{e_1,\ldots,e_n}$ appears if there are multiple children.
% If there is one child, the expression is determined recursively unless that child appears in $MP$ or $U$, discussed below, in which case we leave a reference, $\multiVertex{u}$ or $\cycleVertex{u}$, respectively.
% For the $\ExpLam$ case, we define $\patt(q)$ and $\type(\tau)$ similarly.

The second component $MP$ is the set of expressions derived from a vertex with multiple parents.
Wherever an edge to such a vertex appears, we place a reference to it $\multiVertex{u}$.
In the example, $x_{\id{42}}$ is referenced twice in $e_r$.

The third component $U$ is the set of expressions derived from the remaining vertices which necessarily have exactly one parent.
These vertices form structures formally called unicycular graphs~\citep{DBLP:journals/algorithmica/KruskalRS90}, or \emph{unicycles} for short.
Every unicycle consists of a single cycle where each vertex in the cycle may have additional children that are not part of the cycle.
Breaking the cycle turns a unicyclic graph into a tree.
% TODO: try clearing up "the root" as e.g., "the component root" or "the root of this components of the grove"
% TODO: make sure we say "the graph root" when talking about the actual root, and "the component root" everywhere else.
To break the cycle, we arbitrarily choose the vertex on the cycle with the smallest id, $u$, as the root of that tree.
Formally, we choose the vertex $v$ such that $v = \min(\ancestors(v))$ because
only a vertex on the cycle can be its own ancestor.
When an edge to this vertex appears, we place a reference to it $\cycleVertex{u}$.
In our example, ${+}_{\id{46}}$ is referenced by the $LEFT$ edge of ${*}_{\id{48}}$.

% XXX

decomposition produces sets of trees. As long as everything is well sorted, we
can use them to produce expressions in the main grammar. (alt: define wrt the
graph instead, once we establish the graph can be turned into expressions)

*** well-sortedness judgment on trees: the tree only has valid positions for the
constructors, and the child at each position has the sort corresponding to its
position, and edges in position maps all start at the given constructor.

Theorem: well-sorted graphs produce decompositions of well-sorted trees

``A graph has this property iff these conditions hold''

then we can define a total mapping from well-sorted trees to expressions
(because the tree structure is very generic, so this ties them back to the
earlier grammar.)


forest = 4-tuple

forest validity: each of the trees in the forest are well sorted, and all Refs
point to roots of other trees of the correct sort.

every valid forest corresponds to a well behaved grove (i.e., ``expression
forest'')

soundness: if you start with a valid graph, you get a well behaved grove that
connects every vertex (and only those vertices) in the graph. (i.e., if you turn
it back into a graph, it's the same graph (modulo edge IDs unless we add $\e$ to
Refs??? OR make Ref nullary and pull edge/vertex info from the position map and
add Root sort

OR:

use edge IDs instead of edges and build a mapping from vertices to
sorts---change the $\e$s to $u$s)

% TODO: maybe add a forward reference here to Implementation section

\subsection{User Actions}%
\label{sub:User Actions}

% TODO: define a grammar for user actions --> mapping to sequences of underlying graph edits
% TODO: define well-sorted graph & theorem: graph is well sorted => exists a decomposition

move = delete + add

other composite edits

remember commutativity diagram: we want edits on trees to be equivalent to edits on graphs

given a semantics for user actions, we can show that operating on the grove is equivalent to operating on the graph.
This way is more efficient (less annoying?) because then we don't have to deal with decomposition for every edit.
(we probably want to do it this way in Hazel)

OR

We could go the indirect route and avoid having to prove the theorem.
(this way is easier to start with, and it's how GRV works)

% Mapping to b actions

% \[
%   \begin{array}{l}
%     \textrm{User Actions}: \grave{e} \in Act ::= e \mid \#u \mid \conflict{\grave{e}}{\grave{e}}                                                       \\
%     \s : \G \to Act \times \V^{\ast{}} \times \V^{\ast{}} \times \V^{\ast{}}                                                                                          \\
%     \s(G) = (\grave{e}, \MP(G), \CC(G), \D(G))                                                                                                         \\
%     \MP(G) = \left\{ v \in \vertexes(G) \mid \exists e_1,e_2 \in \liveEdges(G), e_1= \right.                                                           \\
%     \qquad \left. (v,p_1,v_1), e_2=(v,p_2,v_2), e_1 \ne e_2 \right\}                                                                                   \\
%     \CC(G) = \ldots                                                                                                                                    \\
%     \D(G) = \orphans(G) \setminus \left\{ v_0 \right\}                                                                                                 \\
%     \orphans(G) = \left\{ v \in \vertexes(G) \mid \forall e=(v^\prime,p,v^{\prime\prime}) \in \right.                                                  \\
%     \qquad \left. \liveEdges(G), v^{\prime\prime} \ne v \right\}                                                                                       \\
%     \vertexes(G) = \left\{ v \mid \exists e = (v^\prime, p, v^{\prime\prime}) \in \edges(G), v \in \left\{ v^\prime, v^{\prime\prime}\right\} \right\} \\
%     \edges(G) = \left\{ e \mid G(e) \ne \bot \right\}                                                                                                  \\
%     \liveEdges(G) = \left\{ e \mid G(e) = + \right\}                                                                                                   \\
%   \end{array}
% \]
