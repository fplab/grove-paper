
\section{Implementation}%
\label{sec:Implementation}

% TODO: talk about code generation somewhere in here

% TODO: talk about:
% - decomp vertex set generation and unicycle traversal
% - how to choose which branch of expr' to take
% - lfp(ancestors') and min of it

% TODO: can we save any of this in a README or code comments?

% TODO: talk about performance: O(decomp) given "size of the graph" (probably linear in # edges)

% GRV -- how it is implemented, how it connects to the formalism, describe the graph,

We implemented the core Grove calculus of \autoref{sec:Formalism} as an OCaml library.
Our code is parameterized by a syntax specification for expressions that we generate automatically, which makes the implemented language of expressions easier to modify or replace.
We also built a Web browser-based Grove editor as a proof of concept.

Since we cannot implement an infinite mapping directly, the graph is a data structure, \texttt{Graph.t}, that maps from edges to the edge states $\left\{\Plus, \Minus\right\}$.
We do not represent edges that map to $\bot$.

% TODO: rename Decomposition.ml to Grove.ml

There's a function \verb!decompose! of type \verb!Graph.t) : t * Edge.Set.t Vertex.Map.t! that 

From this graph, we compute the set of all edges and the set of live edges with a linear scan.

% TODO: make this into a listing

\begin{verbatim}
  Graph.edges : Graph.t -> Edge.Set.t
  Graph.live_edges : Graph.t -> Edge.Set.t
\end{verbatim}

% TODO: keep connecting the code to the formalism like this

These vertices are partitioned into three sets: multi-parented, single-parented, or orphaned.
A vertex $v$ is considered a parent of another vertex $v'$ if there is a live edge from $v$ to $v'$.

For each vertex, we take the parent and child edge sets.

% TODO: fix the formalism for parents (to work with edges instead of vertices) and then check that it matches the implementation. Try renaming parents to vparents and adding parents for edges.

% TODO: Talk about how the graph is implemented (not as an actual function, but as a map from edges to edge states) so that we can denote the set of created + deleted edges clearly, as well as all of the vertices that have ever been created.

For a graph $G : \E \to \Sigma$ where $\E = \U \times \V \times \P \times \V$, our graph decomposition algorithm runs in $O(\abs{\V} \log \abs{\V} + \abs{\E} \log \abs{\V})$.
It begins with a scan of all edges that have been created or deleted $O(\abs{\E})$.
Their vertices at both ends are partitioned into three sets: multi-parented, single-parented, or orphaned.
Their relationships are recorded in maps for $O(\log \abs{\V})$ lookups of parent and child edge sets.
After the vertices have been partitioned, we traverse the various single-parented components and produce equivalent expressions $O(\abs{\V})$.
For unicycles, we traverse backwards until a vertex is seen twice $O(\abs{\V})$, then proceed forward to find the least vertex on the cycle $O(\abs{\V})$.

% TODO: is this still a thing?
% Compared to the core language, our implementation lacks support for lists and case expressions.

Optimizations: least fixed point


We determine $e$ by starting at the root of the graph and including every connected
vertex that has a single parent. In positions where there are no children, we
leave an empty hole $\hole$. In positions where there are multiple children, we
leave a conflict hole of the form $\conflictHole{e_1,\ldots,e_n}{v}{p}$. Whenever we
encounter a vertex with multiple parents, we add its corresponding expression to
the second component $MP$ and leave an indirect reference $\multiVertex{u}$, where $u$ is
the unique identifier of the referenced vertex

% here are some properties that show how to get one from the other, and to/from
% graph

algorithmic section: if we want to implement these conversions, here are the
steps (without introducing new concepts):

- linear scan
- breaking out roots
- ...

Graph decomposition occurs in three steps.



% TODO: use code generation instead: have Lang.ml generate expr / expr' / ... for us; then we don't need this figure.
% TODO: maybe show this figure and say that it's possible to handle generically without code generation

% \begin{figure}
%   \[
%     \arraycolsep=0pt
%     \begin{array}{lrlll}
%       \textrm{Trees:\qquad}  & t & {}\in Tree & {}::={} & \textrm{Vertex}(v, m) \mid \textrm{Ref}(v) \\
%       \textrm{Maps:}         & m & {}\in Map & {}::={} & \varnothing \mid p \mapsto [(\e, t), \ldots, (\e, t)]; m \\
%     \end{array}
%   \]
%   \caption{Syntax of graph decomposition trees as a grammar.}
%   \label{fig:Syntax of graph decomposition trees}
% \end{figure}


lemma: a unicycle is a cycle with trees hanging off of it

lemma: unicycle traversal always produces a valid tree

Unicycular graph traversal occurs in two directions. A vertex is chosen as an arbitrary
starting point and its parents are traversed and marked as seen. Any vertex that
has been seen twice must be on the cycle. We take the first one as the root of
the unicycle and traverse its children to produce a tree that covers the unicycular graph.
Any remaining single-parented vertices must be parts of other unicycular graphs and are
unicycle-traversed until there are no more single-parented vertices left.
