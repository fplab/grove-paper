
\section{Introduction}%
\label{sec:Introduction}

TODO: fix the following warning which are reported by LaTeX but not Overleaf: Class acmart Warning: A possible image without description on input line 321.

Motivation:

- collaborative editing (both synchronous ala Google Docs and asynchronous version control)
is good and important as computing grows

- semantic structure editing is good because it solves the gap problem (semantic editor services
are always available) -- cite Hazelnut papers (talk about holes)

- previous approaches to collaborative editing have limitations

- diff/merge based approaches (trying to solve the inverse problem based on final states --
you lose the actual actions that were performed, and have to reconstruct them or an approx.
of them i.e. add line/delete line actions -- would need to adapt this to structure editing,
some papers have started to look at that, but fundamentally we don't want to throw away the
knowledge we have about the edits!)

- operational transforms (complexity, you have to patch previous actions based on new actions)

- CRDT-based collaborative editing (that's all been on text, not PL semantics) -- this is good
because it is relatively simple: you just send all the edits to all the replicas and they are
convergent by design

- we want to have the same convergence for a CRDT-based collaborative structure editor that maintains
the sensibility invariant of Hazelnut, i.e. every editor state has meaning. mention that maintaining sensibility
allows scaling of semantic editor services in the presence of large number of collaborators (in contrast,
using VS Code or other collaborative text editors with large numbers of collaborators means that almost always
the semantic editor services will be disabled because the program is going to be broken in multiple places
transiently)

this is tricky because:

- some edits might be conflicting -- solve this with "conflict holes"

- adding cut/paste or delete/restore allows for degenerate programs (cycles, multiple parents, etc.)

- since we are commutative, we solve both synchronus and async collaborative editing

- and this resolves issues around merges and conflicts

- contribution of this paper is to solve these problems from type-theoretic first principles:

- ...

- Hazel

\subsection{Contributions and Paper Organization}%
\label{sec:Contributions and Paper Organization}
