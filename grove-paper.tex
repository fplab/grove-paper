% !TEX program = lualatex

%% For double-blind review submission, w/o CCS and ACM Reference (max submission space)
%\documentclass[acmsmall,review,anonymous,nonacm]{acmart}\settopmatter{printfolios=true,printccs=false,printacmref=false}
\documentclass[acmsmall,10pt,review,anonymous]{acmart}\settopmatter{printfolios=true,printccs=false,printacmref=false}
%% For double-blind review submission, w/ CCS and ACM Reference
%\documentclass[acmsmall,review,anonymous]{acmart}\settopmatter{printfolios=true}
%% For single-blind review submission, w/o CCS and ACM Reference (max submission space)
%\documentclass[acmsmall,review]{acmart}\settopmatter{printfolios=true,printccs=false,printacmref=false}
%% For single-blind review submission, w/ CCS and ACM Reference
%\documentclass[acmsmall,review]{acmart}\settopmatter{printfolios=true}
%% For final camera-ready submission, w/ required CCS and ACM Reference
%\documentclass[acmsmall]{acmart}\settopmatter{}


%% Journal information
%% Supplied to authors by publisher for camera-ready submission;
%% use defaults for review submission.
\acmJournal{PACMPL}
\acmVolume{1}
\acmNumber{POPL} % CONF = POPL or ICFP or OOPSLA
\acmArticle{1}
\acmYear{2021}
\acmMonth{1}
\acmDOI{} % \acmDOI{10.1145/nnnnnnn.nnnnnnn}
\startPage{1}

%% Copyright information
%% Supplied to authors (based on authors' rights management selection;
%% see authors.acm.org) by publisher for camera-ready submission;
%% use 'none' for review submission.
\setcopyright{none}
%\setcopyright{acmcopyright}
%\setcopyright{acmlicensed}
%\setcopyright{rightsretained}
%\copyrightyear{2018}           %% If different from \acmYear

\bibliographystyle{ACM-Reference-Format}
%\citestyle{acmauthoryear}
\citestyle{acmnumeric}

% TODO: consolidate all the macros into here
% Generic Notations
\newcommand{\Set}[1]{\overline{#1}}
\newcommand{\Setof}[1]{\left\{#1\right\}}
\newcommand{\suchthat}{\mid}
\newcommand{\Pow}[1]{\mathscr{P}\mathopen{}\left(#1\right)\mathclose{}}

% Language Definitions
\newcommand{\Sorts}{\mathcal{S}}
\newcommand{\Constructors}{\mathcal{C}}
\newcommand{\Positions}{\mathcal{P}}
\newcommand{\ConstructorSorts}{\Sorts_\Constructors}
\newcommand{\PositionSorts}{\Sorts_\Positions}
\newcommand{\PositionConstructors}{\Constructors_\Positions}
\newcommand{\Arity}{\mathcal{A}}
\newcommand{\Ports}[1]{[#1]_\Positions}

% Identifier Sets
\newcommand{\IdSet}{\mathcal{U}}
\newcommand{\IdxSet}[1]{\IdSet_{#1}}

\DeclareMathOperator{\join}{\sqcup}

%%%%%%%%

\renewcommand{\topfraction}{1} % Allow floats to take up the page
\renewcommand{\textfraction}{0}

%%%%%%%%
% \autoref from hyperref
\renewcommand{\AMSautorefname}          {Equation}
\renewcommand{\appendixautorefname}     {Appendix}
\renewcommand{\chapterautorefname}      {Chapter}
\renewcommand{\equationautorefname}     {Equation}
\renewcommand{\FancyVerbLineautorefname}{Line}
\renewcommand{\figureautorefname}       {Figure}
\renewcommand{\footnoteautorefname}     {Footnote}
\renewcommand{\Hfootnoteautorefname}    {Footnote}
\renewcommand{\itemautorefname}         {Item}
\renewcommand{\Itemautorefname}         {Item}
\renewcommand{\pageautorefname}         {Page}
\renewcommand{\paragraphautorefname}    {Section}
\renewcommand{\partautorefname}         {Part}
\renewcommand{\sectionautorefname}      {Section}
\renewcommand{\subparagraphautorefname} {Section}
\renewcommand{\subsectionautorefname}   {Section}
\renewcommand{\subsubsectionautorefname}{Section}
\renewcommand{\tableautorefname}        {Table}
\renewcommand{\theoremautorefname}      {Theorem}

%% Packages
\usepackage{scalerel}
\usepackage[most]{tcolorbox}
\usepackage{booktabs}
\usepackage[rule=false]{subcaption}
\usepackage{xspace}
\usepackage{graphicx}
\usepackage{relsize}
% \usepackage{centernot}
\usepackage{stmaryrd}
% \usepackage{semantic}

\let\colonapprox\undefined % Avoid redefinition error in `colonequals`
\let\colonsim\undefined % Avoid redefinition error in `colonequals`


% \usepackage{todonotes}
% \usepackage[ruled]{algorithm2e}



% Load MnSymbol without clobbering \ast
% See https://tex.stackexchange.com/a/269691
\usepackage{amsmath}% needed before mathabx
\let\amsast=\ast
\usepackage[matha]{mathabx}% needed to prevent \ast getting clobbered
\let\abxast=\ast
% \usepackage{MnSymbol}
\let\mnast=\ast
\let\ast=\abxast

\usepackage{bm}
\usepackage{thmtools}
\usepackage{colonequals}
\usepackage{mathpartir}
\usepackage{stmaryrd}
\usepackage{fontawesome}
\usepackage{array}
\usepackage{mathtools}
\usepackage{centernot}

\newtcolorbox{mybox}[2][]
{
  on line,
  hbox,
  boxsep=0pt,
  left=1pt,
  right=1pt,
  top=1pt,
  bottom=1pt,
  colframe=white,
  colback=#2
  #1,
}
\newcommand\goodcolor[2]{%
  \protect\leavevmode
  \begingroup
    \color{#1}%
    #2%
  \endgroup
}
%%%%%%%%
% TikZ Stuff
%\usepackage{etex} % Fix "No room for new \dimen" error
\usepackage{shellesc} % Fix bug that breaks the tikz 'external' library
\usepackage{tikz}
\usetikzlibrary{babel} % Ensure compatibility the 'babel' package

\usetikzlibrary{external} % Needs to be separately enabled
%\tikzexternalize % Enable externalization
%\usepackage{lua-visual-debug}

\usetikzlibrary{arrows.meta} % Arrow Tips
\tikzset{>=Stealth}
%\tikzset{<=stealth}
%\tikzset{arrows={-Stealth[scale=50]}}
%\tikzset{edge from parent/.style={draw,->,line width=0.6pt}}
%\tikzset{wideline/.style={line width=0.7pt}}
%\tikzset{boldline/.style={color=black,line width=1.0pt}}

\usetikzlibrary{
  backgrounds,  % Provides "framed" and "gridded"
  bending,      % bending arrow tips
  decorations.pathmorphing,   % Provides wavy edges
  graphs,       % Graph *notation*
  graphdrawing, % Graph *layout*
  quotes,       % Quote syntax (e.g., "foo")
}

\usegdlibrary{
  trees,
}

\tikzset{
  %every picture/.style={framed, background rectangle/.style={draw=gray!50}},
}
\tikzset{edge style/.style={
  draw,
  %color=gray,
  font={\small\ttfamily},
  /tikz/every edge quotes/.style={
    %draw=gray!20,
    anchor=west,
    swap/.append code={
      \ifpgfarrowswap
        \pgfkeysalso{anchor=west}
      \else
        \pgfkeysalso{anchor=east}
      \fi}},
}}
\tikzset{graphs/graph style/.style={
  tree layout,
  level distance=0.5cm,
  level sep=0.5cm,
  sibling distance=0.5cm,
  sibling sep=0.1cm,
  part distance=0.1cm,
  part sep=0.1cm,
  component distance=0.1cm,
  component sep=0.1cm,
  nodes={
    draw,
    %color=gray,
    inner sep=2pt,
    rounded corners=1mm},
  edges={edge style},
}}
\tikzset{graphs/root style/.style={
 %draw=none,
 as={\textbullet$_{\id{0}}$}
}}
\tikzset{alice/.style={
  color=red!80!black,
  font={\bfseries\small},
  thick,
}}
\tikzset{bob/.style={
  color=green!60!black,
  font={\bfseries\small},
  thick,
}}
\tikzset{merge/.style={
  color=blue,
  font={\bfseries\small},
  thick,
}}
\tikzset{alice edge/.style={alice, edge style, font={\bfseries\small}}}
\tikzset{alice node/.style={alice}}
\tikzset{alice step/.style={alice}}
\tikzset{bob edge/.style={bob, edge style}, font={\bfseries\small}}
\tikzset{bob node/.style={bob}}
\tikzset{bob step/.style={bob}}
\tikzset{merge edge/.style={merge, edge style}, font={\bfseries\small}}
\tikzset{merge node/.style={merge}}
\tikzset{merge step/.style={merge,decorate,decoration={coil,amplitude=1.0pt,segment length=7.0pt,aspect=0}}}
\tikzset{star/.style={edge node={node[inner sep=0pt,at end,sloped] {\textbf{\huge${}^{\ast}$}}}}}

% Define outline versions of + and -
\def\outlinepad{0.4pt}
\def\outlinestroke{0.4pt}
\newcommand{\Plus}{\mathord{
\begin{tikzpicture}[anchor=base, baseline]
%\node at (0,0) {+};
\path[draw, line width=\outlinestroke]
   ( 0.333em+\outlinestroke/2+\outlinepad,  0.270em+\outlinestroke/2+\outlinepad)
 --( 0.333em+\outlinestroke/2+\outlinepad,  0.229em-\outlinestroke/2-\outlinepad)
 --( 0.021em+\outlinestroke/2+\outlinepad,  0.229em-\outlinestroke/2-\outlinepad)
 --( 0.021em+\outlinestroke/2+\outlinepad, -0.084em-\outlinestroke/2-\outlinepad)
 --(-0.020em-\outlinestroke/2-\outlinepad, -0.084em-\outlinestroke/2-\outlinepad)
 --(-0.020em-\outlinestroke/2-\outlinepad,  0.229em-\outlinestroke/2-\outlinepad)
 --(-0.333em-\outlinestroke/2-\outlinepad,  0.229em-\outlinestroke/2-\outlinepad)
 --(-0.333em-\outlinestroke/2-\outlinepad,  0.270em+\outlinestroke/2+\outlinepad)
 --(-0.020em-\outlinestroke/2-\outlinepad,  0.270em+\outlinestroke/2+\outlinepad)
 --(-0.020em-\outlinestroke/2-\outlinepad,  0.583em+\outlinestroke/2+\outlinepad)
 --( 0.021em+\outlinestroke/2+\outlinepad,  0.583em+\outlinestroke/2+\outlinepad)
 --( 0.021em+\outlinestroke/2+\outlinepad,  0.270em+\outlinestroke/2+\outlinepad)
 --cycle
 ;
\end{tikzpicture}
}}

\newcommand{\Minus}{\mathord{
\begin{tikzpicture}[anchor=base, baseline]
%\node at (0,0) {$-$};
\path[draw, line width=\outlinestroke]
   ( 0.306em+\outlinestroke/2+\outlinepad,  0.270em+\outlinestroke/2+\outlinepad)
 --( 0.306em+\outlinestroke/2+\outlinepad,  0.229em-\outlinestroke/2-\outlinepad)
 --(-0.306em-\outlinestroke/2-\outlinepad,  0.229em-\outlinestroke/2-\outlinepad)
 --(-0.306em-\outlinestroke/2-\outlinepad,  0.270em+\outlinestroke/2+\outlinepad)
 --cycle
 ;
\end{tikzpicture}
}}

%%%%%%%%%%%%%%%%%%%%%%%%%%%%%%%%%%%%%%%%%%%%%%%%%%%%%%%%%%%%%%%%%%%%%%%%%%%%%%%%
% Language Components

\usepackage{xstring}
\usepackage{tensor}

% \selectcolormodel{gray}

% notation
\def\_{\texttt{\textunderscore}}
\newcommand{\id}[1]{\textcolor{gray}{\ensuremath{#1}}}
\newcommand{\eid}[2]{\tensor*{#2}{^{\id{#1}}}}
\newcommand{\abs}[1]{\left\lvert#1\right\rvert}
\newcommand{\figureCode}[1]{\textbf{\texttt{#1}}\vskip1em}
\newcommand{\Z}[1]{\hat{#1}}
\newcommand{\Set}[1][]{\Theta_{#1}}
\newcommand{\SetOf}[1]{\left\{#1\right\}}
\newcommand{\SuchThat}[1]{ : #1}
\newcommand{\SequenceOf}[1]{\left(#1\right)}
\newcommand{\SizeOf}[1]{\left\lvert#1\right\rvert}
\newcommand{\userAction}[1]{\xrightarrow{#1}}
\newcommand{\graphAction}[2]{\big(#1, #2\big)}

% sets
\def\A{\mathcal{A}}
\def\E{\mathcal{E}}
\def\G{\mathcal{G}}
\def\I{\mathcal{I}}
\def\K{\mathcal{K}}
\def\P{\mathcal{P}}
\def\S{\mathcal{S}}
\def\U{\mathcal{U}}
\def\V{\mathcal{V}}
\def\e{\varepsilon}
\def\AA{\textbf{A}}
\def\EE{\textbf{E}}

% Sorts
\newcommand\Exp{\mathsf{Exp}}
\newcommand\Pat{\mathsf{Pat}}
\newcommand\Typ{\mathsf{Typ}}

% objects
\def\e{\varepsilon}
\def\Grove{\gamma}

% judgments
\newcommand{\act}[5]{#1, #2; #3 \userAction{#4} #5}

%%%%%%%%%%%%%%%%%%%%%%%%%%%%%%%%%%%%%%%%%%%%%%%%%%%%%%%%%%%%%%%%%%%%%%%%%%%%%%%%
% Graphs

\DeclareMathOperator{\sortOp}{\text{sort}}
\DeclareMathOperator{\arityOp}{\text{arity}}
\newcommand{\sort}[1]{\sortOp\mathopen{}\left(#1\right)\mathclose{}}
\newcommand{\arity}[1]{\arityOp\mathopen{}\left(#1\right)\mathclose{}}

\DeclareMathOperator{\constructorOp}{\texttt{constructor}}
\DeclareMathOperator{\econstructorOp}{\texttt{econstructor}}
\DeclareMathOperator{\pconstructorOp}{\texttt{pconstructor}}
\DeclareMathOperator{\tconstructorOp}{\texttt{tconstructor}}
\newcommand{\constructor}[1]{\constructorOp\mathopen{}\left(#1\right)\mathclose{}}
\newcommand{\econstructor}[1]{\econstructorOp\mathopen{}\left(#1\right)\mathclose{}}
\newcommand{\pconstructor}[1]{\pconstructorOp\mathopen{}\left(#1\right)\mathclose{}}
\newcommand{\tconstructor}[1]{\tconstructorOp\mathopen{}\left(#1\right)\mathclose{}}

\newcommand{\Edge}[1]{Edge~#1}
\newcommand{\Vertex}[1]{Vertex~#1}
\newcommand{\multiVertex}[1]{\textcolor{red}{\ensuremath{\curlyveedownarrow_{#1}}}}
\newcommand{\cycleVertex}[1]{\textcolor{red}{\ensuremath{\rcirclearrowleft_{#1}}}}
\newcommand{\orphanVertex}[1]{\textcolor{red}{\ensuremath{\mathbf{\ndownarrow_{#1}}}}}
\newcommand{\rootVertex}{v_\text{root}}
\newcommand{\otherVertexVskip}{\vskip0.5em}

% \DeclareMathOperator{\outvertexOp}{\text{outvertex}}
% \newcommand{\outvertex}[1]{\outvertexOp\mathopen{}\left(#1\right)\mathclose{}}

\DeclareMathOperator{\defaultposOp}{\text{defaultpos}}
\newcommand{\defaultpos}[1]{\defaultposOp\mathopen{}\left(#1\right)\mathclose{}}

\newcommand{\parens}[1]{\textcolor{gray}{(}#1\textcolor{gray}{)}}

% Constructors
\newcommand\ExpVar{\mathsf{Exp\_var}}
\newcommand\ExpLam{\mathsf{Exp\_lam}}
\newcommand\ExpApp{\mathsf{Exp\_app}}
\newcommand\ExpNum{\mathsf{Exp\_num}}
\newcommand\ExpPlus{\mathsf{Exp\_plus}}
\newcommand\ExpTimes{\mathsf{Exp\_times}}
\newcommand\PatVar{\mathsf{Pat\_var}}
\newcommand\TypNum{\mathsf{Typ\_num}}
\newcommand\TypArrow{\mathsf{Typ\_arrow}}

% Positions
\newcommand\Root{\mathsf{Root}}
\newcommand\LamParam{\mathsf{Param}}
\newcommand\LamType{\mathsf{Type}}
\newcommand\LamBody{\mathsf{Body}}
\newcommand\AppFun{\mathsf{Fun}}
\newcommand\AppArg{\mathsf{Arg}}
\newcommand\PlusLeft{\mathsf{Left}}
\newcommand\PlusRight{\mathsf{Right}}
\newcommand\TimesLeft{\mathsf{Left}}
\newcommand\TimesRight{\mathsf{Right}}
\newcommand\ArrowArg{\mathsf{Arg}}
\newcommand\ArrowResult{\mathsf{Result}}

%%%%%%%%%%%%%%%%%%%%%%%%%%%%%%%%%%%%%%%%%%%%%%%%%%%%%%%%%%%%%%%%%%%%%%%%%%%%%%%%
% Terms

\newcommand{\varExp}[2]{#1^{#2}}
\newcommand{\numExp}[2]{\underline{#1}^{#2}}
\newcommand{\lamExp}[4]{\lambda^{#1} #2 : #3.#4}
\newcommand{\appExp}[3]{\left(#1~#2\right)^{#3}}
\newcommand{\plusExp}[3]{#1 +^{#2} #3}
\newcommand{\varPat}[2]{#1^{#2}}
\newcommand{\arrowTyp}[3]{#1 \to^{#2} #3}
\newcommand{\numTyp}[1]{Num^{#1}}
\newcommand{\hole}{\ensuremath{\square}} %\textcolor{violet}{\llparenthesis}}\textcolor{violet}{\rrparenthesis}}
% TODO: finish this macro
\newcommand{\conflictHoleForm}[2]{\textcolor{red}{\textbf{\{}}#1\textcolor{red}{\textbf{\}}}_{#2}}
\newcommand{\conflictHole}[1]{%
{\noexpandarg\StrSubstitute{#1}{,}{\textcolor{red}{\;\textbf{|}\;}}[\myargs]%
{\textcolor{red}{\textbf{\{}}\myargs\textcolor{red}{\textbf{\}}}}}}%

\newcommand{\eVar}[2]{\eid{#1}{#2}}
\newcommand{\eFun}[4]{\eid{#1}{\lambda} #2 : #3 . #4}
\newcommand{\eApp}[3]{\eid{#1}{\left(#2~#3\right)}}
\newcommand{\eNum}[2]{\eid{#1}{\underline{#2}}}
\newcommand{\ePlus}[3]{#2~\eid{#1}{\texttt{+}}~#3}
\newcommand{\eTimes}[3]{#2~\eid{#1}{\texttt{*}}~#3}
\newcommand{\pVar}[2]{\eid{#1}{#2}}
\newcommand{\tArrow}[3]{#2 \eid{#1}{\rightarrow} #3}
\newcommand{\tNum}[1]{\eid{#1}{Num}}

% TODO: come up with a metavariable for zippered sets (i.e., Grove components)
%  t = e | q | \tau
%  t^ = e^ | ...
%       \Theta_{NP}, ... are sets of terms
%       \Z{\Theta} = (\Z{t}, \Theta) is a set of terms with a zippered term inside

% TODO: add conflict terms to zippered syntax and cursor erasure

% |> { e1 | ... | en } <|
% { e1^ | ... | en }
% { e1 | ... | en^ }

% |> mpc <|
% |> uc <|

% TODO: sort out which one's "a" and which one's "\alpha" in the paper body (check Hazelnut paper)

% Graph actions (a):
%   - add edge
%   - remove edge

% User actions (\alpha) (indirect):
%
% \gamma^ --\alpha--> a*
% 
% (NP^, MP, U) -- Construct(k) --> 
% (NP, MP^, U) -- Construct(k) --> 
% (NP, MP, U^) -- Construct(k) --> 

% ((|> e <|, NP), MP, U) -- Construct(k) --> 
% v, p; ((|> ?? <|, NP), MP, U) -- Construct(k) -->  (v, p, (u_fresh, k)) \mapsto +

% TODO: define well sorted grove

%%%%%%%%%%%%%%%%%%%%%%%%%%%%%%%%%%%%%%%%%%%%%%%%%%%%%%%%%%%%%%%%%%%%%%%%%%%%%%%%
% Zippered Terms

\newcommand{\cursor}[1]{{\vartriangleright}#1{\vartriangleleft}}
\newcommand{\erase}[1]{#1\mathclose{}^{\diamond}\mathclose{}}

%%%%%%%%%%%%%%%%%%%%%%%%%%%%%%%%%%%%%%%%%%%%%%%%%%%%%%%%%%%%%%%%%%%%%%%%%%%%%%%%
% Decomposition

\DeclareMathOperator{\decompOp}{\texttt{decomp}}
\DeclareMathOperator{\vertexesOp}{\texttt{vertexes}}
\newcommand{\decomp}[1]{\decompOp\mathopen{}\left(#1\right)\mathclose{}}
\newcommand{\vertexes}[1]{\vertexesOp\mathopen{}\left(#1\right)\mathclose{}}

\DeclareMathOperator{\edecompOp}{\texttt{edecomp}}
\DeclareMathOperator{\pdecompOp}{\texttt{pdecomp}}
\DeclareMathOperator{\tdecompOp}{\texttt{tdecomp}}
\DeclareMathOperator{\edecompPrimeOp}{\edecompOp^\prime}
\DeclareMathOperator{\pdecompPrimeOp}{\pdecompOp^\prime}
\DeclareMathOperator{\tdecompPrimeOp}{\tdecompOp^\prime}
\newcommand{\edecomp}[1]{\edecompOp\mathopen{}\left(#1\right)\mathclose{}}
\newcommand{\pdecomp}[1]{\pdecompOp\mathopen{}\left(#1\right)\mathclose{}}
\newcommand{\tdecomp}[1]{\tdecompOp\mathopen{}\left(#1\right)\mathclose{}}
\newcommand{\edecompPrime}[2]{\edecompPrimeOp\mathopen{}\left(#1, #2\right)\mathclose{}}
\newcommand{\pdecompPrime}[2]{\pdecompPrimeOp\mathopen{}\left(#1, #2\right)\mathclose{}}
\newcommand{\tdecompPrime}[2]{\tdecompPrimeOp\mathopen{}\left(#1, #2\right)\mathclose{}}

% helpers
\DeclareMathOperator{\outedgesOp}{\text{outedges}}
\DeclareMathOperator{\ingraphOp}{\text{ingraph}}
\DeclareMathOperator{\parentsOp}{\text{parents}}
\DeclareMathOperator{\ancestorsOp}{\text{ancestors}}
\DeclareMathOperator{\verticesOp}{\text{vertices}}
% \DeclareMathOperator{\childrenOp}{\text{children}}
\DeclareMathOperator{\lfpOp}{\text{lfp}}
\DeclareMathOperator{\ancestorsPrimeOp}{\ancestorsOp^\prime}
\newcommand{\outedges}[2]{\outedgesOp\mathopen{}\left(#1, #2\right)\mathclose{}}
\newcommand{\ingraph}[1]{\ingraphOp\mathopen{}\left(#1\right)\mathclose{}}
% \newcommand{\children}[1]{\childrenOp\mathopen{}\left(#1\right)\mathclose{}}
\newcommand{\parents}[1]{\parentsOp\mathopen{}\left(#1\right)\mathclose{}}
\newcommand{\ancestors}[1]{\ancestorsOp\mathopen{}\left(#1\right)\mathclose{}}
\newcommand{\ancestorsPrime}[1]{\ancestorsPrimeOp\mathopen{}\left(#1\right)\mathclose{}}
\newcommand{\lfp}[1]{\lfpOp\mathopen{}\left(#1\right)\mathclose{}}

\DeclareMathOperator{\minOp}{\text{min}}
\renewcommand{\min}[1]{\minOp\mathopen{}\left(#1\right)\mathclose{}}

\DeclareMathOperator{\rootsOp}{\text{roots}}
\newcommand{\roots}[1]{\rootsOp\mathopen{}\left(#1\right)\mathclose{}}

%%%%%%%%%%%%%%%%%%%%%%%%%%%%%%%%%%%%%%%%%%%%%%%%%%%%%%%%%%%%%%%%%%%%%%%%%%%%%%%%
% Recomposition

\DeclareMathOperator{\recompOp}{\texttt{recomp}}
\DeclareMathOperator{\erecompOp}{\texttt{erecomp}}
\DeclareMathOperator{\precompOp}{\texttt{precomp}}
\DeclareMathOperator{\trecompOp}{\texttt{trecomp}}
\newcommand{\recomp}[1]{\recompOp\mathopen{}\left(#1\right)\mathclose{}}
\newcommand{\erecomp}[1]{\erecompOp\mathopen{}\left(#1\right)\mathclose{}}
\newcommand{\precomp}[1]{\precompOp\mathopen{}\left(#1\right)\mathclose{}}
\newcommand{\trecomp}[1]{\trecompOp\mathopen{}\left(#1\right)\mathclose{}}

%%%%%%%%%%%%%%%%%%%%%%%%%%%%%%%%%%%%%%%%%%%%%%%%%%%%%%%%%%%%%%%%%%%%%%%%%%%%%%%%

% actions
\def\Down{\text{Down}}
\def\Enqueue{\text{Enqueue}}
\def\Left{\text{Left}}
\def\Move{\text{Move}}
\def\Num{\text{Num}}
\def\Right{\text{Right}}
\def\Select{\text{Select}}
\def\Send{\text{Send}}
\def\Up{\text{Up}}

\newcommand{\Construct}[1]{\text{Construct}\mathopen{}\left(#1\right)\mathclose{}}
\newcommand{\Delete}{\text{Delete}}
\newcommand{\Reposition}[2]{\text{Reposition}\mathopen{}\left(#1, #2\right)\mathclose{}}
\newcommand{\Wrap}[2]{\mathrm{Wrap}\mathopen{}\left(#1, #2\right)\mathclose{}}

% Calling \newvertex{Foo}{bar} defines
%   \vidFoo to be a new id number, and
%   \vFoo to be \texttt{bar}\ensuremath{_{\vidFoo}}
\newcounter{NodeVertexCounter}
\newcommand{\newvertex}[2]{%
\ifodd\theNodeVertexCounter
  \addtocounter{NodeVertexCounter}{1}%
\else
  \addtocounter{NodeVertexCounter}{2}%
\fi
\expandafter\newcommand\csname vid#1\endcsname{}% This is just to check if this is a redefinition
\expandafter\global\expandafter\edef\csname vid#1\endcsname{\theNodeVertexCounter}%
\expandafter\newcommand\csname v#1\endcsname{}% This is just to check if this is a redefinition
\expandafter\gdef\csname v#1\endcsname{\texttt{#2}\ensuremath{_{\id{\csname vid#1\endcsname}}}}%
}
\newcommand{\newedge}[2]{%
\ifodd\theNodeVertexCounter
  \addtocounter{NodeVertexCounter}{2}%
\else
  \addtocounter{NodeVertexCounter}{1}%
\fi
\expandafter\newcommand\csname eid#1\endcsname{}% This is just to check if this is a redefinition
\expandafter\global\expandafter\edef\csname eid#1\endcsname{\theNodeVertexCounter}%
\expandafter\newcommand\csname e#1\endcsname{}% This is just to check if this is a redefinition
\expandafter\gdef\csname e#1\endcsname{\texttt{#2}\ensuremath{_{\id{\csname eid#1\endcsname}}}}%
}
\setcounter{NodeVertexCounter}{-1}
\newvertex{Root}{Root}

% Support \includegraphics of .dot files
\DeclareGraphicsRule{.dot}{pdf}{.pdf}{`dot -Tpdf #1 -o \noexpand\OutputFile}


%%%%%%%%%%%%%%%%%%%%%%%%%%%%%%%%
%% \figureSimple
%%%%%%%%%%%%%%%%%%%%%%%%%%%%%%%%
\newedge{SimpleTimes}{Root}%
\newvertex{SimpleTimes}{*}%
\newedge{SimpleX}{L}%
\newvertex{SimpleX}{x}%
\newedge{SimpleY}{R}%
\newvertex{SimpleY}{y}%
\newcommand{\figureSimple}{
\begin{figure}
\hfill
%%%%%%%%
\begin{subfigure}[t]{0.32\linewidth}
\centering
\caption{
\begin{tikzpicture}[remember picture, overlay]
\path (0cm,0cm) node (node/Simple/a) [anchor=base] {\strut};
\end{tikzpicture}
}%
\label{fig:Simple:a}
\figureCode{x * \hole{}}%
\begin{tikzpicture}
\path (0cm,0cm) graph[graph style] {
 root[root style, color=black] -> {
  "\vSimpleTimes" [> "\eSimpleTimes", >color=black, color=black] -> {
   "\vSimpleX" [> "\eSimpleX"', >color=black, color=black],
   {}
  }
 }
};
\end{tikzpicture}
\end{subfigure}
%%%%%%%%
\begin{subfigure}[t]{0.32\linewidth}
\centering
\caption{
\begin{tikzpicture}[remember picture, overlay]
\path (0cm,0cm) node (node/Simple/b) [anchor=base] {\strut};
\path [draw,->,alice step] (node/Simple/a) to (-2em, 0cm |- node/Simple/b);
\end{tikzpicture}
}%
\label{fig:Simple:b}
\figureCode{x * y}%
\begin{tikzpicture}
\path (0cm,0cm) graph[graph style] {
  root[root style] -> {
  "\vSimpleTimes" [> "\eSimpleTimes"] -> {
    "\vSimpleX" [> "\eSimpleX"'],
    "\vSimpleY" [> "\eSimpleY", >alice edge, alice node]
  }
  }
};
\end{tikzpicture}
\end{subfigure}
%%%%%%%%
\hfill
%%%%%%%%
\begin{subfigure}[t]{0.32\linewidth}
\centering
\caption{
\begin{tikzpicture}[remember picture, overlay]
\path (0cm,0cm) node (node/Simple/c) [anchor=base] {\strut};
\path [draw,->,alice step] (node/Simple/b) to (-2em, 0cm |- node/Simple/c);
\end{tikzpicture}
}%
\label{fig:Simple:c}
\figureCode{\hole{} * y}%
\begin{tikzpicture}
\path (0cm,0cm) graph[graph style] {
  root[root style] -> {
  "\vSimpleTimes" [> "\eSimpleTimes"] -> {
    {},
    "\vSimpleY" [> "\eSimpleY"]
  }
  }
};
\path (-0.25cm,-1.7cm) graph[graph style] {
  "\vSimpleX"
};
\end{tikzpicture}
\end{subfigure}
%%%%%%%%
\hfill{}
\caption{(a) We represent collaborative program sketches as graphs. (b) Hole filling translates to edge insertion. (c) Term deletion (or cut) deletes an edge, but the vertex persists. (We omit it in subsequent figures.)}%
\Description{This figure shows some simple edits with holes}
\label{fig:Simple}
\vspace{-10px}
\end{figure}
}

%%%%%%%%%%%%%%%%%%%%%%%%%%%%%%%%
%% \figureWrap
%%%%%%%%%%%%%%%%%%%%%%%%%%%%%%%%
\newedge{WrapPlus}{Root}
\newvertex{WrapPlus}{+}
\newedge{WrapTimes}{L}
\newcommand{\figureWrapMove}{
\begin{figure}
\centering
\begin{minipage}[t]{.45\linewidth}
\hskip0.12\columnwidth
%%%%%%%%
\begin{subfigure}[t]{0.28\linewidth}
\centering
\caption{
\begin{tikzpicture}[remember picture, overlay]
\path (-7.5em,0cm) node (node/Simple/cx) [gray,anchor=base west] {{\textbf{\small Fig.~\ref*{fig:Simple:c}}}};
%%%%
\path (0cm,0cm) node (node/Wrap/a) [anchor=base] {\strut};
\path [draw,->,alice step] (node/Simple/cx) to (-2em, 0cm |- node/Wrap/a);
\end{tikzpicture}
}%
\label{fig:Wrap:a}
\figureCode{\hole{}}%
\begin{tikzpicture}
\path (0cm,0cm) graph[graph style] {
 root[root style]
};
\path (0cm,-1.7cm) graph[graph style] {
 "\vSimpleTimes" -> {
  {},
  "\vSimpleY" [> "\eSimpleY"]
 }
};
\end{tikzpicture}
\end{subfigure}
%%%%%%%%
\hfill
%%%%%%%%
\begin{subfigure}[t]{0.28\linewidth}
\centering
\caption{
\begin{tikzpicture}[remember picture, overlay]
\path (0cm,0cm) node (node/Wrap/b) [anchor=base] {\strut};
\path [draw,->,alice step] (node/Wrap/a) to (-2em, 0cm |- node/Wrap/b);
\end{tikzpicture}
}%
\label{fig:Wrap:b}
\figureCode{\hole{} + \hole{}}%
\begin{tikzpicture}
\path (0cm,0cm) graph[graph style] {
 root[root style] -> {
  "\vWrapPlus" [> "\eWrapPlus", >alice edge, alice node]
 }
};
\path (0cm,-1.7cm) graph[graph style] {
 "\vSimpleTimes" -> {
  {},
  "\vSimpleY" [> "\eSimpleY"]
 }
};
\end{tikzpicture}
\end{subfigure}
%%%%%%%%
\hfill
%%%%%%%%
\begin{subfigure}[t]{0.28\linewidth}
\centering
\caption{
\begin{tikzpicture}[remember picture, overlay]
\path (0cm,0cm) node (node/Wrap/c) [anchor=base] {\strut};
\path [draw,->,alice step] (node/Wrap/b) to (-2em, 0cm |- node/Wrap/c);
\end{tikzpicture}
}%
\label{fig:Wrap:c}
\figureCode{\hole{} * y + \hole{}}%
\begin{tikzpicture}
\path (0cm,0cm) graph[graph style] {
 root[root style] -> {
  "\vWrapPlus" [> "\eWrapPlus"] -> {
   "\vSimpleTimes" [> "\eWrapTimes"', >alice edge] -> {
    {},
    "\vSimpleY" [> "\eSimpleY"]
   },
   {}
  }
 }
};
\end{tikzpicture}
\end{subfigure}
%%%%%%%%
\hfill{}
\caption{Example of Wrapping.  (Note that this figure omits the orphaned \vSimpleX{} since it is no longer relevant to the narrative of this paper.)}%
\label{fig:Wrap}
\end{minipage}
% \end{figure}
% }
%
\hfil
%
%%%%%%%%%%%%%%%%%%%%%%%%%%%%%%%%
%% \figureMove
%%%%%%%%%%%%%%%%%%%%%%%%%%%%%%%%
\newedge{MoveTimes}{R}
% \newcommand{\figureMove}{
% \begin{figure}[H]
\begin{minipage}[t]{.45\linewidth}
\hfill
%\hskip0.12\columnwidth
%%%%%%%%
\begin{subfigure}[t]{0.43\linewidth}
\centering
\caption{
\begin{tikzpicture}[remember picture, overlay]
\path (0cm,0cm) node (node/Move/a) [anchor=base] {\strut};
\path [draw,->,alice step] (node/Wrap/c) to (-2em, 0cm |- node/Move/a);
\end{tikzpicture}
}%
\label{fig:Move:a}
\figureCode{\hole{} + \hole{}}%
\begin{tikzpicture}
\path (0cm,0cm) graph[graph style] {
 root[root style] -> {
  "\vWrapPlus" [> "\eWrapPlus"]
 }
};
\path (0cm,-1.7cm) graph [graph style] {
 "\vSimpleTimes" -> {
  {},
  "\vSimpleY" [> "\eSimpleY"]
 }
};
\end{tikzpicture}
\end{subfigure}
%%%%%%%%
\hfill
%%%%%%%%
\begin{subfigure}[t]{0.43\linewidth}
\centering
\caption{
\begin{tikzpicture}[remember picture, overlay]
\path (0cm,0cm) node (node/Move/b) [anchor=base] {\strut};
\path [draw,->,alice step] (node/Move/a) to (-2em, 0cm |- node/Move/b);
\end{tikzpicture}
}%
\label{fig:Move:b}
\figureCode{\hole{} + \hole{} * y}%
\begin{tikzpicture}
\path (0cm,0cm) graph[graph style] {
 root[root style] -> {
  "\vWrapPlus" [> "\eWrapPlus"] -> {
   {},
   "\vSimpleTimes" [> "\eMoveTimes", >alice edge] -> {
    {},
    "\vSimpleY" [> "\eSimpleY"]
   }
  }
 }
};
\end{tikzpicture}
\end{subfigure}
%%%%%%%%
\hfill{}
\caption{Example of Repositioning Code.}%
\label{fig:Move}
\end{minipage}
\end{figure}
}

%%%%%%%%%%%%%%%%%%%%%%%%%%%%%%%%
%% \figureDifferentParts
%%%%%%%%%%%%%%%%%%%%%%%%%%%%%%%%
\newedge{DifferentPartsAlice}{L}
\newvertex{DifferentPartsAlice}{u}
\newedge{DifferentPartsBob}{R}
\newvertex{DifferentPartsBob}{v}
% \newcommand{\figureDifferentParts}{
\newcommand{\figureDifferentPartsNestedParts}{
\begin{figure}
\begin{minipage}[t]{0.45\linewidth}
%\hfill
\hskip0.12\columnwidth
%%%%%%%%
\begin{subfigure}[t]{0.28\linewidth}
\centering
\caption{
\begin{tikzpicture}[remember picture, overlay]
\path (-7.5em,0cm) node (node/Move/bx) [gray,anchor=base west] {{\textbf{\small Fig.~\ref*{fig:Move:b}}}};
\path (0cm,0cm) node (node/DifferentParts/a) [anchor=base] {\strut};
\path [draw,->,alice step] (node/Move/bx) to (-2em, 0cm |- node/DifferentParts/a);
\end{tikzpicture}
}%
\label{fig:DifferentParts:a}
\figureCode{\hole{} + u * y}%
\begin{tikzpicture}
\path (0cm,0cm) graph[graph style] {
 root[root style] -> {
  "\vWrapPlus" [> "\eWrapPlus"] -> {
  {},
  "\vSimpleTimes" [> "\eMoveTimes"] -> {
    "\vDifferentPartsAlice" [> "\eDifferentPartsAlice"', >alice edge, alice node],
    "\vSimpleY" [> "\eSimpleY"]
  }
  }
 }
};
\end{tikzpicture}
\end{subfigure}
%%%%%%%%
\hfill
%%%%%%%%
\begin{subfigure}[t]{0.28\linewidth}
\centering
\caption{
\begin{tikzpicture}[remember picture, overlay]
\path (0cm,0cm) node (node/DifferentParts/b) [anchor=base] {\strut};
\path [draw,->,bob step] (node/Move/bx) to [out=15,in=165,star] (-2em, 0cm |- node/DifferentParts/b);
\end{tikzpicture}
}%
\label{fig:DifferentParts:b}
\figureCode{\hole{} + \hole{} * v}%
\begin{tikzpicture}
\path (0cm,0cm) graph[graph style] {
 root[root style] -> {
  "\vWrapPlus" [> "\eWrapPlus"] -> {
  {},
  "\vSimpleTimes" [> "\eMoveTimes"] -> {
    {},
    "\vDifferentPartsBob" [> "\eDifferentPartsBob", >bob edge, bob node]
  }
  }
 }
};
\end{tikzpicture}
\end{subfigure}
%%%%%%%%
\hfill
%%%%%%%%
\begin{subfigure}[t]{0.28\linewidth}
\centering
\caption{
\begin{tikzpicture}[remember picture, overlay]
\path (0cm,0cm) node (node/DifferentParts/c) [anchor=base] {\strut};
\path [draw,->,merge step] (node/DifferentParts/a) to [out=15,in=165] (-2em, 0cm |- node/DifferentParts/c);
\path [draw,->,merge step] (node/DifferentParts/b) to (-2em, 0cm |- node/DifferentParts/c);
\end{tikzpicture}
}%
\label{fig:DifferentParts:c}
\figureCode{\hole{} + u * v}%
\begin{tikzpicture}
\path (0cm,0cm) graph[graph style] {
 root[root style] -> {
  "\vWrapPlus" [> "\eWrapPlus"] -> {
  {},
  "\vSimpleTimes" [> "\eMoveTimes"] -> {
    "\vDifferentPartsAlice" [> "\eDifferentPartsAlice"', >merge edge, merge node],
    "\vDifferentPartsBob" [> "\eDifferentPartsBob", >merge edge, merge node]
  }
  }
 }
};
\end{tikzpicture}
\end{subfigure}
%%%%%%%%
\hfill{}
\caption{Example of Users Editing Different Parts of the Code.}%
\label{fig:DifferentParts}
\end{minipage}
% \end{figure}
% }
%
\hfil
%
%%%%%%%%%%%%%%%%%%%%%%%%%%%%%%%%
%% \figureCommutativity
%%%%%%%%%%%%%%%%%%%%%%%%%%%%%%%%
% \newcommand{\figureCommutativity}{
% \begin{figure}
% \centering
% \begin{figure}
% \centering
% \begin{tikzpicture}
% \path (-3cm, 0cm) node (a) [align=center]       {Original \\ Version};
% \path ( 0cm, 1cm) node (b) [align=center,alice node] {Alice's  \\ Version};
% \path ( 0cm,-1cm) node (c) [align=center,bob node]   {Bob's    \\ Version};
% \path ( 3cm, 0cm) node (d) [align=center]       {Combined \\ Version};
% \path [draw,->,alice step] (a) -- node [pos=0.7,auto] {Alice's Edits} (b);
% \path [draw,->,bob step]   (a) -- node [pos=0.7,auto,swap]      {Bob's Edits}   (c);
% \path [draw,->,merge step] (b) -- node [pos=0.3,auto] {Sync} (d);
% \path [draw,->,merge step] (c) -- node [pos=0.3,auto,swap]      {Sync} (d);
% \end{tikzpicture}
% \caption{TODO:Commutativity}
% \label{fig:Commutativity}
% \end{figure}
% }
%
%%%%%%%%%%%%%%%%%%%%%%%%%%%%%%%%
%% \figureNestedParts
%%%%%%%%%%%%%%%%%%%%%%%%%%%%%%%%
\newedge{NestedPartsAlice}{L}
\newvertex{NestedPartsAlice}{w}
\newedge{NestedPartsBob}{L}
% \newcommand{\figureNestedParts}{
% \begin{figure}
\begin{minipage}[t]{0.45\linewidth}
\hfill
%\hskip0.12\columnwidth
%%%%%%%%
\begin{subfigure}[t]{0.25\linewidth}
\centering
\caption{
\begin{tikzpicture}[remember picture, overlay]
\path (0cm,0cm) node (node/NestedParts/a) [anchor=base] {\strut};
\path [draw,->,alice step] (node/DifferentParts/c) to[star] (-2em, 0cm |- node/NestedParts/a);
\end{tikzpicture}
}%
\label{fig:NestedParts:a}
\figureCode{\hole{} + w * v}%
\begin{tikzpicture}
\path (0cm,0cm) graph[graph style] {
 root[root style] -> {
  "\vWrapPlus" [> "\eWrapPlus"] -> {
   {},
   "\vSimpleTimes" [> "\eMoveTimes"] -> {
    "\vNestedPartsAlice" [> "\eNestedPartsAlice"', >alice edge, alice node],
    "\vDifferentPartsBob" [> "\eDifferentPartsBob"]
   }
  }
 }
};
\end{tikzpicture}
\end{subfigure}
%%%%%%%%
\hfill
%%%%%%%%
\begin{subfigure}[t]{0.25\linewidth}
\centering
\caption{
\begin{tikzpicture}[remember picture, overlay]
\path (0cm,0cm) node (node/NestedParts/b) [anchor=base] {\strut};
\path [draw,->,bob step] (node/DifferentParts/c) to [out=15,in=165,star] (-2em, 0cm |- node/NestedParts/b);
\end{tikzpicture}
}%
\label{fig:NestedParts:b}
\figureCode{u * v + \hole{}}%
\begin{tikzpicture}
\path (0cm,0cm) graph[graph style] {
 root[root style] -> {
  "\vWrapPlus" [> "\eWrapPlus"] -> {
   "\vSimpleTimes" [> "\eNestedPartsBob"', >bob edge] -> {
    "\vDifferentPartsAlice" [> "\eDifferentPartsAlice"'],
    "\vDifferentPartsBob" [> "\eDifferentPartsBob"]
   },
   {}
  }
 }
};
\end{tikzpicture}
\end{subfigure}
%%%%%%%%
\hfill
%%%%%%%%
\begin{subfigure}[t]{0.25\linewidth}
\centering
\caption{
\begin{tikzpicture}[remember picture, overlay]
\path (0cm,0cm) node (node/NestedParts/c) [anchor=base] {\strut};
\path [draw,->,merge step] (node/NestedParts/a) to [out=15,in=165] (-2em, 0cm |- node/NestedParts/c);
\path [draw,->,merge step] (node/NestedParts/b) to [out=15,in=165] (-2em, 0cm |- node/NestedParts/c);
\end{tikzpicture}
}%
\label{fig:NestedParts:c}
\figureCode{w * v + \hole{}}%
\begin{tikzpicture}
\path (0cm,0cm) graph[graph style] {
 root[root style] -> {
  "\vWrapPlus" [> "\eWrapPlus"] -> {
   "\vSimpleTimes" [> "\eNestedPartsBob"', >merge edge] -> {
    "\vNestedPartsAlice" [> "\eNestedPartsAlice"', >merge edge, merge node],
    "\vDifferentPartsBob" [> "\eDifferentPartsBob"]
   },
   {}
  }
 }
};
\end{tikzpicture}
\end{subfigure}
%%%%%%%%
\hfill{}
\caption{Example of Users Editing Nested Parts of the Code.}%
\label{fig:NestedParts}
\end{minipage}
\end{figure}
}

%%%%%%%%%%%%%%%%%%%%%%%%%%%%%%%%
%% \figureMultiChild
%%%%%%%%%%%%%%%%%%%%%%%%%%%%%%%%
\newedge{MultiChildAlice}{R}
\newvertex{MultiChildAlice}{x}
\newedge{MultiChildBob}{R}
\newvertex{MultiChildBob}{y}
\newcommand{\figureMultiChild}{
\begin{figure}
%\hfill
\hskip0.12\columnwidth
%%%%%%%%
\begin{subfigure}[t]{0.21\linewidth}
\centering
\caption{
\begin{tikzpicture}[remember picture, overlay]
\path (-7.5em,0cm) node (node/NestedParts/cx) [gray,anchor=base west] {{\textbf{\small Fig.~\ref*{fig:NestedParts:c}}}};
%%%%
\path (0cm,0cm) node (node/MultiChild/a) [anchor=base] {\strut};
\path [draw,->,alice step] (node/NestedParts/cx) to (-2em, 0cm |- node/MultiChild/a);
\end{tikzpicture}
}%
\label{fig:MultiChild:a}
\figureCode{w * v + x}%
\begin{tikzpicture}
\path (0cm,0cm) graph[graph style] {
 root[root style] -> {
  "\vWrapPlus" [> "\eWrapPlus"] -> {
   "\vSimpleTimes" [> "\eNestedPartsBob"'] -> {
    "\vNestedPartsAlice" [> "\eNestedPartsAlice"'],
    "\vDifferentPartsBob" [> "\eDifferentPartsBob"]
   },
   {
    "\vMultiChildAlice" [> "\eMultiChildAlice", >alice edge, alice node]
   }
  }
 }
};
\end{tikzpicture}
\end{subfigure}
%%%%%%%%
\hfill
%%%%%%%%
\begin{subfigure}[t]{0.21\linewidth}
\centering
\caption{
\begin{tikzpicture}[remember picture, overlay]
\path (0cm,0cm) node (node/MultiChild/b) [anchor=base] {\strut};
\path [draw,->,bob step] (node/NestedParts/cx) to[out=15,in=165] (-2em, 0cm |- node/MultiChild/b);
\end{tikzpicture}
}%
\label{fig:MultiChild:b}
\figureCode{w * v + y}%
\begin{tikzpicture}
\path (0cm,0cm) graph[graph style] {
 root[root style] -> {
  "\vWrapPlus" [> "\eWrapPlus"] -> {
   "\vSimpleTimes" [> "\eNestedPartsBob"'] -> {
    "\vNestedPartsAlice" [> "\eNestedPartsAlice"'],
    "\vDifferentPartsBob" [> "\eDifferentPartsBob"]
   },
   "\vMultiChildBob" [> "\eMultiChildBob", >bob edge, bob node]
  }
 }
};
\end{tikzpicture}
\end{subfigure}
%%%%%%%%
\hfill
%%%%%%%%
\begin{subfigure}[t]{0.4\linewidth}
\centering
\caption{
\begin{tikzpicture}[remember picture, overlay]
\path (0cm,0cm) node (node/MultiChild/c) [anchor=base] {\strut};
\path [draw,->,merge step] (node/MultiChild/a) to[out=15,in=165] (-2em, 0cm |- node/MultiChild/c);
\path [draw,->,merge step] (node/MultiChild/b) to (-2em, 0cm |- node/MultiChild/c);
\end{tikzpicture}
}%
\label{fig:MultiChild:c}
\figureCode{w * v + $\conflictHole{x,y}$}%
\begin{tikzpicture}
\path (0cm,0cm) graph[graph style] {
 root[root style] -> {
  "\vWrapPlus" [> "\eWrapPlus"] -> {
   "\vSimpleTimes" [> "\eNestedPartsBob"'anchor=-15] -> {
    "\vNestedPartsAlice" [> "\eNestedPartsAlice"'],
    "\vDifferentPartsBob" [> "\eDifferentPartsBob"]
   },
   "\vMultiChildAlice" [> "\eMultiChildAlice"anchor=153, >merge edge, merge node],
   {},
   "\vMultiChildBob" [> "\eMultiChildBob"anchor=195, >merge edge, merge node]
  }
 }
};
\end{tikzpicture}
\end{subfigure}
%%%%%%%%
%\hfill{}
\caption{Example of Multi-Child Conflicts.}%
\label{fig:MultiChild}
\end{figure}
}

%%%%%%%%%%%%%%%%%%%%%%%%%%%%%%%%
%% \figureMultiParent
%%%%%%%%%%%%%%%%%%%%%%%%%%%%%%%%
\newedge{MultiParentAlice}{R}
\newedge{MultiParentBob}{R}
\newcommand{\figureMultiParent}{
\begin{figure*}
\hfill
% \hskip0.1\columnwidth
%%%%%%%%
\begin{subfigure}[t]{0.12\linewidth}
\centering
\caption{
\begin{tikzpicture}[remember picture, overlay]
% \path (-7.5em,0cm) node (node/MultiChild/cx) [gray,anchor=base west] {{\textbf{\small Fig.~\ref*{fig:MultiChild:c}}}};
%%%%
\path (0cm,0cm) node (node/MultiParent/a) [anchor=base] {\strut};
% \path [draw,->,alice step] (node/MultiChild/cx) to[star] (-2em, 0cm |- node/MultiParent/a);
\end{tikzpicture}
}%
\label{fig:MultiParent:a}
\figureCode{w * \hole{} + \hole{}\otherVertexVskip\phantom{\multiVertex{58} = w}}%
\begin{tikzpicture}
\path (0cm,0cm) graph[graph style] {
 root[root style] -> {
  "\vWrapPlus" [> "\eWrapPlus"] -> {
   "\vSimpleTimes" [> "\eNestedPartsBob"'] -> {
    "\vNestedPartsAlice" [> "\eNestedPartsAlice"'],
    {}
   },
   {}
  }
 }
};
\end{tikzpicture}
\end{subfigure}
%%%%%%%%
\hfill
%%%%%%%%
\begin{subfigure}[t]{0.14\linewidth}
\centering
\caption{
\begin{tikzpicture}[remember picture, overlay]
\path (0cm,0cm) node (node/MultiParent/b) [anchor=base] {\strut};
\path [draw,->,alice step] (node/MultiParent/a) to[star] (-2em, 0cm |- node/MultiParent/b);
\end{tikzpicture}
}%
\label{fig:MultiParent:b}
\figureCode{\hole{} * w + \hole{}\otherVertexVskip\phantom{\multiVertex{58} = w}}%
\begin{tikzpicture}
\path (0cm,0cm) graph[graph style] {
 root[root style] -> {
  "\vWrapPlus" [> "\eWrapPlus"] -> {
   "\vSimpleTimes" [> "\eNestedPartsBob"'] -> {
    {},
    "\vNestedPartsAlice" [> "\eMultiParentAlice"', >alice edge]
   },
   {}
  }
 }
};
\end{tikzpicture}
\end{subfigure}
%%%%%%%%
\hfill
%%%%%%%%
\begin{subfigure}[t]{0.14\linewidth}
\centering
\caption{
\begin{tikzpicture}[remember picture, overlay]
\path (0cm,0cm) node (node/MultiParent/c) [anchor=base] {\strut};
\path [draw,->,bob step] (node/MultiParent/a) to[out=15,in=165,star] (-2em, 0cm |- node/MultiParent/c);
\end{tikzpicture}
}%
\label{fig:MultiParent:c}
\figureCode{\hole{} * \hole{} + w\otherVertexVskip\phantom{\multiVertex{58} = w}}%
\begin{tikzpicture}
\path (0cm,0cm) graph[graph style] {
 root[root style] -> {
  "\vWrapPlus" [> "\eWrapPlus"] -> {
   "\vSimpleTimes" [> "\eNestedPartsBob"'],
   "\vNestedPartsAlice" [> "\eMultiParentBob", >bob edge]
  }
 }
};
\end{tikzpicture}
\end{subfigure}
%%%%%%%%
\hfill
%%%%%%%%
\begin{subfigure}[t]{0.18\linewidth}
\centering
\caption{
\begin{tikzpicture}[remember picture, overlay]
\path (0cm,0cm) node (node/MultiParent/d) [anchor=base] {\strut};
\path [draw,->,merge step] (node/MultiParent/b) to[out=15,in=165] (-2em, 0cm |- node/MultiParent/d);
\path [draw,->,merge step] (node/MultiParent/c) to[out=15,in=165] (-2em, 0cm |- node/MultiParent/d);
\end{tikzpicture}
}%
\label{fig:MultiParent:d}
\figureCode{\hole{} * \multiVertex{58} + \multiVertex{58}\otherVertexVskip\multiVertex{58} = w}%
\begin{tikzpicture}
\path (0cm,0cm) graph[graph style] {
 root[root style] -> {
  a/{\vWrapPlus} [> "\eWrapPlus"] -> {
   "\vSimpleTimes" [> "\eNestedPartsBob"'] -> {
    {},
    b/"\vNestedPartsAlice" [> "\eMultiParentAlice"', >merge edge]
   },
   {}
  }
 },
};
\path [draw,->,merge edge] (a) to ["\eMultiParentBob",out=-60,in=60] (b);
\end{tikzpicture}
\end{subfigure}
%%%%%%%%
\hfill
%%%%%%%%
\begin{subfigure}[t]{0.13\linewidth}
\centering
\caption{
\begin{tikzpicture}[remember picture, overlay]
\path (0cm,0cm) node (node/MultiParent/e) [anchor=base] {\strut};
\path [draw,->,alice step] (node/MultiParent/d) to (-2em, 0cm |- node/MultiParent/e);
\end{tikzpicture}
}%
\label{fig:MultiParent:e}
\figureCode{\hole{} * w + \hole{}\otherVertexVskip\phantom{\multiVertex{58} = w}}%
\begin{tikzpicture}
\path (0cm,0cm) graph[graph style] {
 root[root style] -> {
  v8/"\vWrapPlus" [> "\eWrapPlus"] -> {
   v2/"\vSimpleTimes" [> "\eNestedPartsBob"'] -> {
    {},
    "\vNestedPartsAlice" [> "\eMultiParentAlice"']
   },
   {}
  }
 }
};
\end{tikzpicture}
\end{subfigure}
%%%%%%%%
\hfill{}
\caption{(a) We start in a state with a variable, \li{w}, and two holes. (b) Alice relocates \li{w} to the left hole. (c) Bob relocates \li{w} to the right hole. (d) After merging, vertex \vNestedPartsAlice{} has two incoming edges, i.e. it has a \emph{relocation conflict}. The corresponding decomposition leaves a \emph{relocation conflict reference} at both locations, partially addressing the \textbf{relocation conflict problem}. Terms that have a relocation conflict are tracked separately by decomposition. (e) The relocation conflict can be resolved by deleting all but one reference.}%
\Description{This figure depicts an example of relocation conflicts}
\label{fig:MultiParent}
\end{figure*}
}

%%%%%%%%%%%%%%%%%%%%%%%%%%%%%%%%
%% \figureCycle
%%%%%%%%%%%%%%%%%%%%%%%%%%%%%%%%
\newedge{MultiCycleTimes}{L}
\newvertex{MultiCycleTimes}{*}
\newedge{MultiCyclePlus}{R}
\newvertex{MultiCyclePlus}{+}
\newedge{MultiCycleAliceTimes}{R}
\newedge{MultiCycleAlicePlus}{L}
\newedge{MultiCycleBobPlus}{R}
\newedge{MultiCycleBobTimes}{L}
\newcommand{\figureCycle}{
\begin{figure*}
%\hfill
\hskip0.07\columnwidth
%%%%%%%%
\begin{subfigure}[t]{0.2\textwidth}
\centering
\caption{
\begin{tikzpicture}[remember picture, overlay]
\path (-9em,0cm) node (node/MultiParent/ex) [gray,anchor=base west] {{\textbf{\small Fig.~\ref*{fig:MultiParent:e}}}};
%%%%
\path (0cm,0cm) node (node/Cycle/a) [anchor=base] {\strut};
\path [draw,->,alice step] (node/MultiParent/ex) to[star] (-2em, 0cm |- node/Cycle/a);
\end{tikzpicture}
}%
\label{fig:Cycle:a}
\figureCode{\parens{\hole{} * \hole{}} * \parens{\hole{} + \hole{}} + \hole{}%
  \otherVertexVskip\phantom{\multiVertex{30} = \multiVertex{32} * \hole{}}%
  \\\phantom{\multiVertex{32} = \multiVertex{30} * \hole{}}}%
\begin{tikzpicture}
\path (0cm,0cm) graph[graph style] {
 root[root style] -> {
  "\vWrapPlus" [> "\eWrapPlus"] -> {
   "\vSimpleTimes" [> "\eNestedPartsBob"'] -> {
    "\vMultiCycleTimes" [> "\eMultiCycleTimes"', >alice edge, alice node],
    "\vMultiCyclePlus" [> "\eMultiCyclePlus", >alice edge, alice node]
   },
   {}
  }
 }
};
\end{tikzpicture}
\end{subfigure}
%%%%%%%%
\hfill
%%%%%%%%
\begin{subfigure}[t]{0.2\textwidth}
\centering
\caption{
\begin{tikzpicture}[remember picture, overlay]
\path (0cm,0cm) node (node/Cycle/b) [anchor=base] {\strut};
\path [draw,->,alice step] (node/Cycle/a) to[star] (-2em, 0cm |- node/Cycle/b);
\end{tikzpicture}
}%
\label{fig:Cycle:b}
\figureCode{\hole{} * \hole{} + \parens{\hole{} + \hole{}} * \hole{}%
  \otherVertexVskip\phantom{\multiVertex{30} = \multiVertex{32} * \hole{}}%
  \\\phantom{\multiVertex{32} = \multiVertex{30} * \hole{}}}%
\begin{tikzpicture}
\path (0cm,0cm) graph[graph style] {
 root[root style] -> {
  "\vWrapPlus" [> "\eWrapPlus"] -> {
   "\vSimpleTimes" [> "\eNestedPartsBob"'] -> {
   },
   {
    "\vMultiCycleTimes" [> "\eMultiCycleAliceTimes", >alice edge, alice node] -> {
     "\vMultiCyclePlus" [> "\eMultiCycleAlicePlus"', >alice edge, alice node],
     {}
    }
   }
  }
 }
};
\end{tikzpicture}
\end{subfigure}
%%%%%%%%
\hfill
%%%%%%%%
\begin{subfigure}[t]{0.2\textwidth}
\centering
\caption{
\begin{tikzpicture}[remember picture, overlay]
\path (0cm,0cm) node (node/Cycle/c) [anchor=base] {\strut};
\path [draw,->,bob step] (node/Cycle/a) to[out=15,in=165,star] (-2em, 0cm |- node/Cycle/c);
\end{tikzpicture}
}%
\label{fig:Cycle:c}
\figureCode{\hole{} * \hole{} + \parens{\hole{} * \hole{} + \hole{}}%
  \otherVertexVskip\phantom{\multiVertex{30} = \multiVertex{32} * \hole{}}%
  \\\phantom{\multiVertex{32} = \multiVertex{30} * \hole{}}}%
\begin{tikzpicture}
\path (0cm,0cm) graph[graph style] {
 root[root style] -> {
  "\vWrapPlus" [> "\eWrapPlus"] -> {
   "\vSimpleTimes" [> "\eNestedPartsBob"'] -> {
   },
   {
    "\vMultiCyclePlus" [> "\eMultiCycleBobPlus", >bob edge, bob node] -> {
     "\vMultiCycleTimes" [> "\eMultiCycleBobTimes"', >bob edge, bob node],
     {}
    }
   }
  }
 }
};
\end{tikzpicture}
\end{subfigure}
%%%%%%%%
\hfill
%%%%%%%%
\begin{subfigure}[t]{0.2\textwidth}
\centering
\caption{
\begin{tikzpicture}[remember picture, overlay]
\path (0cm,0cm) node (node/Cycle/d) [anchor=base] {\strut};
\path [draw,->,merge step] (node/Cycle/b) to[out=15,in=165] (-2em, 0cm |- node/Cycle/d);
\path [draw,->,merge step] (node/Cycle/c) to (-2em, 0cm |- node/Cycle/d);
\end{tikzpicture}
}%
\label{fig:Cycle:d}
\figureCode{\parens{\hole{} * \hole{}} + $\conflictHole{\multiVertex{\vidMultiCycleTimes},\multiVertex{\vidMultiCyclePlus}}{\vidWrapPlus}{R}$%
  \otherVertexVskip\multiVertex{\vidMultiCycleTimes} = \multiVertex{\vidMultiCyclePlus} * \hole{}%
  \\\multiVertex{\vidMultiCyclePlus} = \multiVertex{\vidMultiCycleTimes} * \hole{}}%
\begin{tikzpicture}
\path (0cm,0cm) graph[graph style] {
 root[root style] -> {
  "\vWrapPlus" [> "\eWrapPlus"] -> {
   "\vSimpleTimes" [> "\eNestedPartsBob"'anchor=-15] -> {
   },
   v1/"\vMultiCycleTimes" [> "\eMultiCycleAliceTimes"anchor=155, >merge edge, merge node],
   {},
   v2/"\vMultiCyclePlus" [> "\eMultiCycleBobPlus"anchor=-165, >merge edge, merge node]
  }
 }
};
\path[draw,-{>[bend]},merge edge] (v1) to ["\eMultiCycleAlicePlus"anchor=south,out=-60,in=-120] (v2);
\path[draw,-{>[bend]},merge edge] (v2) to ["\eMultiCycleBobTimes"anchor=north,out=-90,in=-90] (v1);
\end{tikzpicture}
\end{subfigure}
%%%%%%%%
%\hfill{}
\caption{Example of Cycles.}
% TODO replace screenshots with text
% TODO make ref code gray
% TODO move whole figure left
% TODO remove cursor from screenshots
% TODO add multiparent box
% TODO only show parens where needed - figure 10, 11 - by associativity
% TODO repplace cycle screenshot with cycles box %
\label{fig:Cycle}
\end{figure*}
}

%%%%%%%%%%%%%%%%%%%%%%%%%%%%%%%%
%% \figureDisconnect
%%%%%%%%%%%%%%%%%%%%%%%%%%%%%%%%
\newedge{DisconnectAlice}{R}
\newedge{DisconnectBob}{L}
\newcommand{\figureDisconnect}{
\begin{figure*}
%\hfill
\hskip0.12\columnwidth
%%%%%%%%
\begin{subfigure}[t]{0.21\textwidth}
\centering
\caption{
\begin{tikzpicture}[remember picture, overlay]
\path (-7.5em,0cm) node (node/Cycle/dx) [gray,anchor=base west] {{\textbf{\small Fig.~\ref*{fig:Cycle:d}}}};
%%%%
\path (0cm,0cm) node (node/Disconnect/a) [anchor=base] {\strut};
\path [draw,->,alice step] (node/Cycle/dx) to[star] (-2em, 0cm |- node/Disconnect/a);
\end{tikzpicture}
}%
\label{fig:Disconnect:a}
\figureCode{\hole{} * \hole{} + \hole{} * \hole{}\otherVertexVskip\strut}%
\begin{tikzpicture}
\path (0cm,0cm) graph[graph style] {
 root[root style] -> {
  "\vWrapPlus" [> "\eWrapPlus"] -> {
   "\vSimpleTimes" [> "\eNestedPartsBob"'],
   "\vMultiCycleTimes" [> "\eMultiCycleAliceTimes"]
  }
 }
};
\end{tikzpicture}
\end{subfigure}
%%%%%%%%
\hfill
%%%%%%%%
\begin{subfigure}[t]{0.21\textwidth}
\centering
\caption{
\begin{tikzpicture}[remember picture, overlay]
\path (0cm,0cm) node (node/Disconnect/b) [anchor=base] {\strut};
\path [draw,->,alice step] (node/Disconnect/a) to[star] (-2em, 0cm |- node/Disconnect/b);
\end{tikzpicture}
}%
\label{fig:Disconnect:b}
\figureCode{\hole{} * \parens{\hole{} * \hole{}} + \hole{}\otherVertexVskip\strut}%
\begin{tikzpicture}
\path (0cm,0cm) graph[graph style] {
 root[root style] -> {
  "\vWrapPlus" [> "\eWrapPlus"] -> {
   "\vSimpleTimes" [> "\eNestedPartsBob"'] -> {
    {},
    "\vMultiCycleTimes" [> "\eDisconnectAlice", >alice edge]
   },
   {}
  }
 }
};
\end{tikzpicture}
\end{subfigure}
%%%%%%%%
\hfill
%%%%%%%%
\begin{subfigure}[t]{0.21\textwidth}
\centering
\caption{
\begin{tikzpicture}[remember picture, overlay]
\path (0cm,0cm) node (node/Disconnect/c) [anchor=base] {\strut};
\path [draw,->,bob step] (node/Disconnect/a) to[out=15,in=165,star] (-2em, 0cm |- node/Disconnect/c);
\end{tikzpicture}
}%
\label{fig:Disconnect:c}
\figureCode{\hole{} + \parens{\hole{} * \hole{}} * \hole{}\otherVertexVskip\strut}%
\begin{tikzpicture}
\path (0cm,0cm) graph[graph style] {
 root[root style] -> {
  "\vWrapPlus" [> "\eWrapPlus"] -> {
   {},
   "\vMultiCycleTimes" [> "\eMultiCycleAliceTimes"] -> {
    "\vSimpleTimes" [> "\eDisconnectBob"', >bob edge],
    {}
   }
  }
 }
};
\end{tikzpicture}
\end{subfigure}
%%%%%%%%
\hfill
%%%%%%%%
\begin{subfigure}[t]{0.21\textwidth}
\centering
\caption{
\begin{tikzpicture}[remember picture, overlay]
\path (0cm,0cm) node (node/Disconnect/d) [anchor=base] {\strut};
\path [draw,->,merge step] (node/Disconnect/b) to[out=15,in=165] (-2em, 0cm |- node/Disconnect/d);
\path [draw,->,merge step] (node/Disconnect/c) to (-2em, 0cm |- node/Disconnect/d);
\end{tikzpicture}
}%
\label{fig:Disconnect:d}
\figureCode{\hole{} + \hole{}\otherVertexVskip\cycleVertex{2} = \hole{} * \parens{\cycleVertex{2} * \hole{}}}%
\begin{tikzpicture}
\path (0cm,0cm) graph[graph style] {
 root[root style] -> {
  "\vWrapPlus" [> "\eWrapPlus"]
 }
};
\path (-0.5cm,-2cm) graph[graph style] { plus/"\vSimpleTimes" };
\path ( 0.5cm,-2cm) graph[graph style] { times/"\vMultiCycleTimes" };
\path[draw,->,merge edge] (times) to ["\eMultiCycleBobTimes"'anchor=north,out=-135,in=-45] (plus);
\path[draw,->,merge edge] (plus) to ["\eMultiCycleBobPlus"anchor=south,out=45,in=135] (times);
\end{tikzpicture}
\end{subfigure}
%%%%%%%%
\hfill{}
\caption{Example of Disconnection.}%
\label{fig:Disconnection}
\end{figure*}
}

\newvertex{DecompPlusA}{+}
\newedge{DecompTimesA}{L}
\newvertex{DecompTimesA}{*}
\newedge{DecompPlusAX}{R}
\newvertex{DecompX}{x}
\newedge{DecompTimesAX}{R}
%
\newedge{DecompPlusB}{L}
\newvertex{DecompPlusB}{+}
\newedge{DecompTimesB}{R}
\newvertex{DecompTimesB}{*}
\newedge{DecompY}{R}
\newvertex{DecompY}{y}
\newedge{DecompZ}{R}
\newvertex{DecompZ}{z}
%
\newedge{DecompTimes}{L}
%
\newcommand{\figureDecompExample}{
\begin{figure}
  \centering
  \begin{subfigure}{0.5\textwidth}
    \centering
    \begin{tikzpicture}
      \path (0cm,0cm) graph[graph style] {
        root[root style] -> 
        {
        a/{\vDecompPlusA} [> "\eSimpleTimes"] -> {
          "\vDecompTimesA" [> "\eDecompTimesA"'] -> {
          {},
          b/"\vDecompX" [> "\eDecompTimesAX"']
          },
          {}
        }
        },
      };
      \path [draw,->] (a) to ["\eDecompPlusAX",out=-60,in=60] (b);
      %
      \path (2cm,-0.85cm) graph[graph style] { plus/"\vDecompPlusB" };
      \path (3cm,-0.85cm) graph[graph style] { times/"\vDecompTimesB" };
      \path (3.25cm,-1.75cm) graph[graph style] { y/"\vDecompY" };
      \path[draw,->] (times) to ["\eDecompPlusB"'anchor=north,out=-135,in=-45] (plus);
      \path[draw,->] (plus) to ["\eDecompTimesB"anchor=south,out=45,in=135] (times);
      \path[draw,->] (times) to ["\eDecompY"anchor=west] (y);
      %
      \path (1.55cm,-2.1cm) graph[graph style] { plus/"\vDecompZ" };
    \end{tikzpicture}
    \caption{A graph}%
    \label{fig:Decomposition example graph}
  \end{subfigure}%
  \begin{subfigure}{.5\textwidth}
    \centering
    \begin{align*}
      \Theta = 
      \{ & (\emptyHole{(40,\texttt{*},L)}~\texttt{*}^{\id{40}}~(\eid{43}{}\multiVertex{43,(42,x)}))~\texttt{+}^{\id{38}}~(\eid{41}{}\multiVertex{41,(42,x)}), \\
      & \eid{52}{z}, \eid{42}{x}, \emptyHole{(46,\texttt{+},L)}~\texttt{+}^{\id{46}}~(\eid{45}{}\cycleVertex{45,(46,\texttt{+})}~\texttt{*}^{\id{48}}~(\eid{49}{}y^{\id{50}}))\} \\
    \end{align*}
    \caption{The grove corresponding to the graph}%
    \label{fig:Decomposition example grove}
  \end{subfigure}
  \begin{subfigure}{.5\textwidth}
    \centering
    \begin{align*}
      \text{t}_r & = \eid{1}{}  (\emptyHole{(40,\texttt{*},L)}~\texttt{*}^{\id{40}}~(\eid{43}{}\multiVertex{43,(42,x)}))~\texttt{+}^{\id{38}}~(\eid{41}{}\multiVertex{41,(42,x)})\\
      \text{NP} & = \{ \eid{52}{z} \} \\
      \text{MP} & = \{ \eid{42}{x} \} \\
      \text{U} & = \{ \emptyHole{(46,\texttt{+},L)}~\texttt{+}^{\id{46}}~(\eid{45}{}\cycleVertex{45,(46,\texttt{+})}~\texttt{*}^{\id{48}}~(\eid{49}{}y^{\id{50}})) \}
    \end{align*}
    \caption{The partitioned grove corresponding to the graph}%
    \label{fig:Decomposition example partitioned grove}
  \end{subfigure}
  \caption{Example of graph decomposition}%
  \Description{This figure describes the decomposition of a graph to its corresponding grove}
  \label{fig:Decomposition example}
\end{figure}
}

\newcommand{\termMV}{t}
\newcommand{\subtermMV}{\overline{t}}

\newcommand{\figureTermSyntaxContent}{%

\[
\begin{array}{lclll}
     \termMV \in & \textit{Term} & \coloneqq & 
        % \eid{u}{k}\ {\textbf{\{} \subtermMV_p \textbf{\}}}_{p \in arity(k)}
        \genericTerm{u}{\subtermMV_p}
        \mid \multiref
        \mid \uniref \\
     \subtermMV \in & \textit{ChildTerm} & \coloneqq & 
        \ehole
        \mid \lexp{\termMV} 
        \mid \conflict{\termMV} \\  
\end{array}
\]
}

\newcommand{\figureTermSyntax}{%
\begin{figure}
  \figureTermSyntaxContent
  \caption{Syntax of terms}
  \Description{This figure describes the grammar for the syntax of generic terms}
  \label{fig:Syntax}
\end{figure}%
}
\newcommand{\figureArityContent}{%
\[
  \arraycolsep=0pt
  \begin{array}{ll}
    \multicolumn{2}{l}{\arityOp : \K \rightarrow \wp(\P \times \{Exp, Pat, Typ\})} \\
    \hline
    \arity{\Root}={} & \left\{ (\Root, Exp) \right\} \\
    \arity{\PatVar(x)}={} & \left\{ \right\} \\
    \arity{\ExpVar(x)}={} & \left\{ \right\} \\
    \arity{\ExpLam}={} & \left\{ (\LamParam, Pat), (\mathtt{Type}, Typ), (\mathtt{Body}, Exp) \right\} \\
    \arity{\ExpApp}={} & \left\{ (\AppFun, Exp), (\AppArg, Exp) \right\} \\
    \arity{\ExpPlus}={} & \left\{ (\PlusLeft, Exp), (\PlusRight, Exp) \right\} \\
    \arity{\ExpTimes}={} & \left\{ (\PlusLeft, Exp), (\PlusRight, Exp) \right\} \\
    \arity{\ExpNum(n)}={} & \left\{ \right\} \\
    \arity{\TypArrow}={} & \left\{ (\ArrowArg, Typ), (\ArrowResult, Typ) \right\} \\
    \arity{\TypNum}={} & \left\{ \right\} \\
  \end{array}
\]%
}

\newcommand{\figureArity}{%
\begin{figure}
\figureArityContent
\caption{Constructors, Indexes and Arity}
\label{fig:Constructors, Indexes and Arity}
\end{figure}%
}

\newcommand{\figureDecompositionDefDecomp}{%
\begin{align*}
  \decomp{G} &= (\Set[NP], \Set[MP], \Set[U]) \\
  \figureDecompositionDefDecompComponents
\end{align*}%
}

\newcommand{\figureDecompositionDefDecompComponents}{%
  \Set[NP] &= \SetOf{\decomp{(u, v, p, v')} \SuchThat{|\parents{v'}| = 0}} \\
  \Set[MP] &= \SetOf{\decomp{(u, v, p, v')} \SuchThat{|\parents{v'}| > 1}} \\
  \Set[U] &= \SetOf{\decomp{(u, v, p, v')} \SuchThat{|\parents{v'}| = 1 \land v' = \min{\ancestors{v'}}}}%
}

\newcommand{\figureDecompositionDefDecompTerm}{%
\begin{align*}
  \decomp{\e{=}(u, v, p, (u', k))} &= \begin{cases}
    \edecomp{\e} &\quad \sort{k} = Exp \\
    \pdecomp{\e} &\quad \sort{k} = Pat \\
    \tdecomp{\e} &\quad \sort{k} = Typ \\
  \end{cases}
\end{align*}
}

\newcommand{\figureDecompositionDefEdecomp}{%
\begin{align*}
  \edecomp{\e{=}(u, v, p, v'{=}(u', \ExpVar(x)))} &= \eVar{\ingraph{v'}}{x} \\
  \edecomp{\e{=}(u, v, p, v'{=}(u', \ExpLam))} &= \eFun{\ingraph{v'}}{q}{\tau}{e} \\
  & q = \pdecompPrime{\e}{\LamParam} \\
  & \tau = \tdecompPrime{\e}{\LamType} \\
  & e = \edecompPrime{\e}{\LamBody} \\
  \edecomp{\e{=}(u, v, p, v'{=}(u', \ExpApp))} &= \eApp{\ingraph{v'}}{e_\AppFun}{e_\AppArg} \\
  & e_\AppFun = \edecompPrime{\e}{\AppFun} \\
  & e_\AppArg = \edecompPrime{\e}{\AppArg} \\
  \edecomp{\e{=}(u, v, p, v'{=}(u', \ExpNum(n)))} &= \eNum{\ingraph{v'}}{n} \\
  \edecomp{\e{=}(u, v, p, v'{=}(u', \ExpPlus))} &= \ePlus{\ingraph{v'}}{e_\PlusLeft}{e_\PlusRight} \\
  & e_\PlusLeft = \edecompPrime{\e}{\PlusLeft} \\
  & e_\PlusRight = \edecompPrime{\e}{\PlusRight} \\
  \edecomp{\e{=}(u, v, p, v'{=}(u', \ExpTimes))} &= \eTimes{\ingraph{v'}}{e_\TimesLeft}{e_\TimesRight} \\
  & e_\TimesLeft = \edecompPrime{\e}{\TimesLeft} \\
  & e_\TimesRight = \edecompPrime{\e}{\TimesRight}
\end{align*}%
}

\newcommand{\figureDecompositionDefPdecomp}{%
\begin{align*}
  \pdecomp{\e{=}(u, v, p, v'{=}(u', \PatVar(x)))} &= \pVar{\ingraph{v'}}{x}
\end{align*}%
}

\newcommand{\figureDecompositionDefTdecomp}{%
\begin{align*}
  \tdecomp{\e{=}(u, v, p, v'{=}(u', \TypArrow))} &= \tArrow{\ingraph{v'}}{\tau_\ArrowArg}{\tau_\ArrowResult} \\
  & \tau_\ArrowArg = \tdecompPrime{\e}{\ArrowArg} \\
  & \tau_\ArrowResult = \tdecompPrime{\e}{\ArrowResult} \\
  \tdecomp{\e{=}(u, v, p, v'{=}(u', \TypNum))} &= \tNum{\ingraph{v'}}
\end{align*}%
}

\newcommand{\figureDecompositionDefEdecompPrime}{%
\[
  \begin{array}{l}
    \edecompPrime{\e{=}(u, v, p, v')}{p'} = \\
    \left\{
      \begin{array}{ll}
        \conflictHole[i \leq n]{\edecomp{\e_i}} &
          \children{v'}{p'} = \SetOf{\e_1, \ldots, \e_n}
        \\
        \multiVertex{\e'} &
          \children{v'}{p'} = \SetOf{\e'{=}(u_1, v_1, p_1, v_1')}
          \land |\parents{v_1'}| > 1
        \\
        \cycleVertex{\e'} &
          \children{v'}{p'} = \SetOf{\e'{=}(u_1, v_1, p_1, v_1')}
          \land |\parents{v_1'}| = 1
          \\ & \quad
          \land v_1' = \min{\ancestors{v_1'}}
        \\
        \edecomp{\e'} &
          \children{v'}{p'} = \SetOf{\e'{=}(u_1, v_1, p_1, v_1')}
          \land |\parents{v_1'}| = 1
          \\ & \quad
          \land v_1' \neq \min{\ancestors{v_1'}}
        \\
        \emptyHole{v'}{p'} & \children{v'}{p'} = \varnothing \\
      \end{array}
    \right.
  \end{array}
\]%
}

\newcommand{\figureDecompositionDefPdecompPrime}{%
\[
  \begin{array}{l}
    \pdecompPrime{\e{=}(u, v, p, v')}{p'} = \\
    \left\{
      \begin{array}{ll}
        \conflictHole[i \leq n]{\pdecomp{\e_i}} &
          \children{v'}{p'} = \SetOf{\e_1, \ldots, \e_n}
        \\
        \multiVertex{\e'} &
          \children{v'}{p'} = \SetOf{\e'{=}(u_1, v_1, p_1, v_1')}
          \land |\parents{v_1'}| > 1
        \\
        \cycleVertex{\e'} &
          \children{v'}{p'} = \SetOf{\e'{=}(u_1, v_1, p_1, v_1')}
          \land |\parents{v_1'}| = 1
          \\ & \quad
          \land v_1' = \min{\ancestors{v_1'}}
        \\
        \pdecomp{\e'} &
          \children{v'}{p'} = \SetOf{\e'{=}(u_1, v_1, p_1, v_1')}
          \land |\parents{v_1'}| = 1
          \\ & \quad
          \land v_1' \neq \min{\ancestors{v_1'}}
        \\
        \emptyHole{v'}{p'} & \children{v'}{p'} = \varnothing \\
      \end{array}
    \right.
  \end{array}
\]%
}

\newcommand{\figureDecompositionDefTdecompPrime}{%
\[
  \begin{array}{l}
    \tdecompPrime{\e{=}(u, v, p, v')}{p'} = \\
    \left\{
      \begin{array}{ll}
        \conflictHole[i \leq n]{\tdecomp{\e_i}} &
          \children{v'}{p'} = \SetOf{\e_1, \ldots, \e_n}
        \\
        \multiVertex{\e'} &
          \children{v'}{p'} = \SetOf{\e'{=}(u_1, v_1, p_1, v_1')}
          \land |\parents{v_1'}| > 1
        \\
        \cycleVertex{\e'} &
          \children{v'}{p'} = \SetOf{\e'{=}(u_1, v_1, p_1, v_1')}
          \land |\parents{v_1'}| = 1
          \\ & \quad
          \land v_1' = \min{\ancestors{v_1'}}
        \\
        \tdecomp{\e'} &
          \children{v'}{p'} = \SetOf{\e'{=}(u_1, v_1, p_1, v_1')}
          \land |\parents{v_1'}| = 1
          \\ & \quad
          \land v_1' \neq \min{\ancestors{v_1'}}
        \\
        \emptyHole{v'}{p'} & \children{v'}{p'} = \varnothing \\
      \end{array}
    \right.
  \end{array}
\]%
}

%%%%%%%%%%%%%%%%%%%%%%%%%%%%%%%%%%%%%%%%%%%%%%%%%%%%%%%%%%%%%%%%%%%%%%%%%%%%%%%%

\newcommand{\figureDecompositionDef}{%
\begin{figure}

\figureDecompositionDefEdecomp

\figureDecompositionDefPdecomp

\figureDecompositionDefTdecomp

\figureDecompositionDefEdecompPrime

\figureDecompositionDefPdecompPrime

\figureDecompositionDefTdecompPrime

\caption{Graph decomposition.}
\label{fig:Graph decomposition definition}
\end{figure}%
}


\begin{document}

%% Title information
% \title[Grove]{Convergent Collaborative Structure Editing}
\title[Grove]{Grove: A Convergent Collaborative Structure-Editor Calculus}
%\subtitle{Subtitle}

%% Author information
\author{Michael D. Adams}
\orcid{0000-0003-3160-6972}

\author{Eric Griffis}
\orcid{0000-0003-1693-6172}

\author{Cyrus Omar}
\orcid{0000-0003-4502-7971}
\affiliation{
  %\position{Assistant Research Scientist}
  \department[0]{Computer Science and Engineering}
  \department[1]{Electrical Engineering and Computer Science}
  \department[2]{College of Engineering}
  \institution{University of Michigan}
  \streetaddress{Bob and Betty Beyster Building, 2260 Hayward Street}
  \city{Ann Arbor}
  \state{MI}
  \postcode{48109-2121}
  \country{USA}
}


%% Abstract
%% Note: \begin{abstract}...\end{abstract} environment must come
%% before \maketitle command
\begin{abstract}
  Text of abstract \ldots.
\end{abstract}


%% 2012 ACM Computing Classification System (CSS) concepts
%% Generate at 'http://dl.acm.org/ccs/ccs.cfm'.
%\begin{CCSXML}
%<ccs2012>
%<concept>
%<concept_id>10011007.10011006.10011008</concept_id>
%<concept_desc>Software and its engineering~General programming languages</concept_desc>
%<concept_significance>500</concept_significance>
%</concept>
%<concept>
%<concept_id>10003456.10003457.10003521.10003525</concept_id>
%<concept_desc>Social and professional topics~History of programming languages</concept_desc>
%<concept_significance>300</concept_significance>
%</concept>
%</ccs2012>
%\end{CCSXML}
%\ccsdesc[500]{Software and its engineering~General programming languages}
%\ccsdesc[300]{Social and professional topics~History of programming languages}
%% End of generated code


%% Keywords
%% comma separated list
% \keywords{keyword1, keyword2, keyword3}  %% \keywords are mandatory in final camera-ready submission


%% \maketitle
%% Note: \maketitle command must come after title commands, author
%% commands, abstract environment, Computing Classification System
%% environment and commands, and keywords command.
\maketitle


\section{Introduction}%
\label{sec:Introduction}

% Software development is a highly collaborative activity. 
Development teams typically collaborate with the aid of a version control system (VCS) like Git, Subversion, or Darcs~\cite{DBLP:conf/haskell/Roundy05}. 
These systems maintain a branching history of \emph{commits} to a source code repository, each consisting of a \emph{patch} together with various metadata, e.g. a human-readable commit message. Patches are imperative programs written in a 
\emph{patch language} defining a set of primitive editing commands.
The standard POSIX \li{patch} language, for example, specifies commands for inserting and deleting specified lines of text at specified line numbers within a file. 

Developers do not typically express program edits using the patch language directly, nor in any case do version control systems typically have access to a log of the developer's edits. Instead, version control systems must \emph{synthesize} patches from the file system state using heuristic algorithms, such as the classic \li{diff} algorithm that synthesizes a patch that minimizes the edit distance between two file system states \cite{DiffAlgorithm}.

When two patches, developed concurrently in branches based on a common ancestral commit, must be merged, version control systems deploy a \emph{three-way merge algorithm}. The standard approach is to apply the \emph{local patch} first, then modify the \emph{remote patch} by shifting its line numbers to account for the local patch's line insertions and deletions. This algorithm is an \emph{operational transform}~\cite{DBLP:conf/sigmod/EllisG89}. 
Character-level operational transforms are similarly used for real-time collaborative editing, e.g. in tools like Google Docs and in Visual Studio Code's Live Share feature. 

\paragraph{Problem 1: Merging Sensibly} 
When merging patches, conflicts (e.g. due to different modifications to the same location) are unavoidable. However, 
standard three-way merge algorithms for line-based patch languages commonly identify spurious conflicts, fail to identify legitimate conflicts, or silently duplicate or misplace code. Let us briefly review some classic problems with these systems. The supplemental material includes Git repositories that demonstrate each of these problems.
%Appendix~\ref{appendix:merge-problems} shows examples of the problematic default behavior of Git in each of these scenarios.\todo{write this, can move to supplement if needed}

The \textbf{granularity problem} 
arises when both commits make edits to different locations within a single line of code. For example, one patch might rename a function argument while the other adds a new argument. 
Similarly, if one patch renames a type while another 
makes unrelated changes to code that references that type,
there will be conflicts at every line of code shared between the two patches. 
A notable special case is the \textbf{nesting problem}, which arises when one patch changes the nesting of code structures, e.g. by wrapping a code block within a new control flow construct, thereby changing the indentation of every line in the block. These changes can cause conflicts.
 
The \textbf{relocation conflict problem} arises when two patches relocate a code block to different locations. 
Standard patch languages operationalize code relocation as simply a deletion paired with an insertion. The merge will therefore fail to identify this legitimate location conflict and instead silently duplicate the code block at both locations.

The \textbf{relocation modification problem} arises when one commit relocates a code block that another commit modifies. 
A naive approach would silently leave the modifications at the original location. A more sophisticated block-based approach, like that deployed by \li{git}, might indicate a conflict when an insertion occurs within the bounds of a code block that has been deleted.

These classic problems have motivated research into richer patch languages, more sophisticated patch synthesis algorithms, and corresponding improvements to three-way merge algorithms. 
For example, systems like Git address the nesting problem by allowing indentation changes from one patch to be merged with other changes that do not modify indentation. 
More sophisticated systems deploy parsers and tree differencing algorithms~\cite{DBLP:conf/sigmod/ChawatheG97, DBLP:journals/tse/FluriWPG07,DBLP:conf/kbse/FalleriMBMM14,DBLP:conf/doceng/Lindholm04,DBLP:conf/fase/NguyenNPN10,DBLP:journals/scp/SchwagerlUW15} to better
address the granularity problem.

Addressing the relocation-related problems is more difficult.
A common approach is to enrich the patch language to make code relocation a primitive command.
However, correctly synthesizing code relocation commands given only the initial and final states of the repository requires heuristics. 
The typical approach is to assume that a matching deletion and insertion is due to code relocation. However, relocated code is often also modified. 
In these cases, it is difficult to determine whether a deletion and a similar but not identical insertion are related by relocation, rather than coincidental code similarity. 
Developers often intentionally copy and then modify code, so there may be multiple partially matching insertions for a given deletion and there is no clear way to decide which, if any, are related by relocation. 
The developer's actual actions, e.g. cuts, copies, and pastes, are not persisted into the file system, nor are there persistent identifiers associated with code structures represented as text, so text-file-based systems have no choice but to deploy imperfect heuristics.

\paragraph{Contribution 1: Grove: A Collaborative Structure Editor Calculus} 
This paper considers the problem of collaborative editing for \emph{structure editors},
which eschew text editing. Instead, developers code by applying tree edit actions directly to a continuously evolving \emph{program sketch}, i.e. a syntax tree with holes, shown projected visually in various ways to the developer. 
Structure editing has been studied since the 1980s with the Cornell Program Synthesizer~\cite{DBLP:journals/cacm/TeitelbaumR81}
and research continues to this day, with numerous active projects including Scratch~\cite{maloney2010scratch} and other block-based editors (which are widely used in educational and end-user programming settings), Jetbrains MPS (which has been deployed in industry)~\cite{voelter2011language}, and Hazel (a live functional programming environment rooted in a structure editor calculus called Hazelnut, which serves as 
an active research platform)~\cite{DBLP:conf/popl/OmarVHAH17}. 

Inspired by Hazelnut, this paper introduces Grove, a \emph{collaborative structure editor calculus} for arbitrary syntax trees that does not suffer from the problems just outlined. 
This is in large part due to a substantial simplification of the overall collaborative
editing architecture. In particular, we \emph{eliminate patch synthesis (i.e. diff algorithms) entirely}, 
instead deriving patches directly from the log of edits performed by the developer. 
We also \emph{eliminate the need for three-way merge algorithms (i.e. operational transforms) entirely}.
Instead, we define the patch language such that all edits commute, 
so remote patches can be applied without transformation. 
We prove a \emph{convergence theorem} that ensures that branches of a repository will converge to the same state 
when the same set of patches are applied, regardless of the order in which they are applied.
The patch language forms what is known as a \emph{commutative replicated data type (CmRDT)}~\cite{preguicca2018conflict,shapiro2011conflict}.

Defining a structure editor calculus that supports code insertion, deletion, and relocation using only commutative edits, and avoiding the problems outlined above, is not trivial. The Hazelnut action language is neither commutative nor does it support relocation. 
Relocation is particularly challenging because of the potential for relocation conflicts,
including the potential for cyclic relocation (when one commit relocates node $A$ beneath node $B$, and the other \emph{vice versa}). 

This need to represent conflicting states means that we cannot use a single syntax tree as Grove's 
core data structure. Instead, we use a directed labeled multi-graph. A vertex corresponds to a syntax tree node and is labeled with a \emph{unique identifier (UID)} and a \emph{constructor}, e.g. \li{Plus}. An edge is also labeled with a UID and establishes a parent-child relationship at a labeled \emph{position} for the parent vertex's constructor, e.g. at the \li{L} or \li{R} position of the \li{Plus} constructor. We refer to a parent vertex and position collectively as a \emph{location}.

The \emph{Grove patch language} is quite simple: a patch can insert or delete an edge (which might cause the creation of a mentioned vertex if it did not already exist). 
To ensure commutativity, deletion of an edge is permanent, i.e.
the edges form a two-phase set (2P-Set) CmRDT~\cite{shapiro2011conflict}. The edit actions that users perform are given meaning by a straightforward translation to a graph patch. 
Relocation simply translates to 
edge deletion and insertion. Critically, vertices are \emph{not} deleted during relocation,
so we do not need to deploy heuristics to identify relocated nodes.

Structure editors are designed to provide editing affordances for trees, not graphs where there may be any number of children at a given location, vertices may have any number of parents, and where there may be cycles. 
To support conventional tree-based structure editing, 
we define a \emph{decomposition} of our program graph into a \emph{grove}, which is a set of 
programs with holes, local conflicts, and conflict references between them to account for motifs that arise when editing collaboratively. 
In particular, a location with no out-edges decomposes to a \emph{hole}. 
More than one out-edge at a given location decomposes to an explicitly represented \emph{local conflict}. 
More than one in-edge indicates that a vertex has a \emph{relocation conflict}, so we leave a \emph{relocation conflict reference} at each of the conflicting locations. 
Finally, cycles are broken during decomposition by leaving a \emph{unicycle conflict reference} at an arbitrary (but deterministically chosen) edge. 
Resolving these various conflicts simply requires 
manipulating these constructs like any other syntactic construct, e.g. deleting all but one relocation conflict reference to determine a unique location for a node. 

\paragraph{Problem 2: Semantic Gaps During Conflict Resolution}
When working with traditional version control systems,
resolving conflicts can take time and require reasoning about 
syntax, types, and program behavior.
Traditionally, however, conflicts are indicated by inserting extra-linguistic markers
into files. 
These markers are not typically understood by the parser, and 
because they include conflicting alternatives, they cannot generally be removed 
or concatenated to result in a sensible program. 
Consequently, language services that require a well-formed, meaningful program
(e.g. type error localization, go-to-definition, live evaluation and so on) 
either fail to operate or exhibit gaps in service, e.g. because they are relying on 
data from a compile prior to the merge attempt. Developers are left to reason without the aid of much of their tooling during conflict resolution. This is an instance of the more general \textbf{semantic gap problem} when programming tools encounter incomplete programs~\cite{DBLP:conf/snapl/OmarVHSGAH17}. 

The previous work on the Hazelnut structure editor calculus addresses the semantic gap problem in the single-user setting by 
defining a type system and type error localization system (the \emph{marked lambda calculus}) for incomplete programs, i.e. programs with holes~\cite{DBLP:journals/pacmpl/ZhaoMDBPO24}. 
Notably, type error localization is proven to be \emph{total}, i.e. the system is able to assign static meaning to \emph{every} syntactically well-formed expression by inserting \emph{marks} to localize errors. Marking employs local type inference as codified by a \emph{bidirectional type system}~\cite{10.1145/3450952} as well as \emph{gradual typing}~\cite{siek2015}, i.e. the theory of \emph{type holes}, to recover from situations where type errors make it impossible to 
determine a known type. 
A separate type hole filling phase deploys unification-based (i.e. non-local) type inference to fill type holes when possible, 
or allows the user to interactively select from hole fillings that partially satisfy generated constraints when there are type conflicts.


\paragraph{Contribution 2: Total Type Error Localization and Recovery for Groves}

This paper extends this prior work on the marked lambda calculus to develop a total type error localization system for groves, introduced in Contribution 1 above as the result of decomposing a commutatively edited graph into a set of terms with empty holes, local conflicts, relocation conflict references, and unicycle conflict references. This paper develops a type (and type error localization) discipline for handling these novel constructs. We follow the marked lambda calculus in rooting our \emph{marked grove calculus} in bidirectional type checking, deploying gradual typing when conflicts do not allow a single type to be inferred, and then layering on a unification-based type inference system to opportunistically fill holes or suggest partial solutions when there are conflicting types due to conflicting syntax.

\paragraph{Paper Outline}

\autoref{sec:Grove By Example} introduces Grove by example, demonstrating its behavior in each of the problematic scenarios named above. 
\autoref{sec:Formalism} then formally defines Grove's graph structure, commutative patch language, grove decomposition procedure, and edit action language. We establish key metatheoretic properties using the Agda proof assistant. 
\autoref{sec:Implementation} describes our implementation of the Grove Workbench, which is defined modularly to allow it to be instantiated with arbitrary syntax trees. 
\autoref{sec:Marking System} instantiates Grove with a simply typed lambda calculus, then defines a bidirectional type and type localization system for groves and proves totality and other key metatheoretic properties using the Agda proof assistant. 
\autoref{sec:Related Work} reviews related work in more detail. 
\autoref{sec:Discussion and Conclusion} concludes with a discussion and directions for future work on collaborative editing.


% A lot of collaborative editing of other sorts (not necessarily programming) requires a tree model,
% But that problem doesn't seem to have been studied directly (maybe except JSON CRDT).

% A \emph{collaborative structure editor} is an editor that
% (1) allows multiple concurrent users to work on a shared document while also
% (2) providing structure-aware editor services such as projectional editing, syntax highlighting, or automatic code folding.
% %
% Collaborative editing research focuses on the design and implementation of real-time, multi-user, character-based communication systems,
% whereas structure editors typically presume a more complex document schema and then focus on some other aspect of the user experience.
% In both settings, preservation of user intent is a core technical challenge.
% %
% Although collaborative editors and structure editors have overlapping goals (optimal user experience)
% and complementary design challenges (subject-subject versus subject-object harmony),
% To our knowledge, there is no comprehensive, principled account of their combined use.

% Since collaborative editors are essentially distributed systems, existing work tends to focus on extensions to distribution protocols.
% Lots of examples using OT. (Google Docs)
% OT is complex and largely textual.
% OT can make sense for real-time systems: users typically change one character at a time, and instant feedback can help to prevent conflicts.
% On the other hand, OT system designs can be difficult to extend.

% Alternatively, there's CRDT. (Peritext?)
% CRDT is easier to implement but harder to design for.
% There's an All-CRDT editor---it turned out to be not so realistic. (what's it called again?)

% Structure editing has been a recurring theme in computer science literature since at least Engelbart's ``Mother of All Demos.''
% Provides automation for domain experts and reduces the barrier to entry for everyone else.
% Popular for editing programs, i.e., for programming language-specific editors.

% However, modern program editors typically disable editor services (like what?) when the document is not in a consistent state,
% a phenomenon called the ``gap problem.''
% Of course, in the presence of multiple concurrent users, the problem gets worse.

% In a collaborative setting,
% Hazel is a structure editor with support for advanced editing services, (e.g., semantic actions?).





% \newpage




% Motivation:

% - collaborative editing (both synchronous ala Google Docs and asynchronous version control)
% is good and important as computing grows

% - semantic structure editing is good because it solves the gap problem (semantic editor services
% are always available) -- cite Hazelnut papers (talk about holes)

% - previous approaches to collaborative editing have limitations

% - diff/merge-based approaches (trying to solve the inverse problem based on final states --
% you lose the actual actions that were performed and have to reconstruct them or an approx.
% of them i.e., add line/delete line actions -- would need to adapt this to structure editing,
% Some papers have started to look at that, but fundamentally, we don't want to throw away the
% knowledge we have about the edits!)

% - operational transforms (complexity, you have to patch previous actions based on new actions)

% - CRDT-based collaborative editing (that's all been on text, not PL semantics) -- this is good
% because it is relatively simple: you just send all the edits to all the replicas and they are
% convergent by design

% - we want to have the same convergence for a CRDT-based collaborative structure editor that maintains
% the sensibility invariant of Hazelnut, i.e., every editor state has meaning. mention that maintaining sensibility
% allows scaling of semantic editor services in the presence of a large number of collaborators (in contrast,
% using VS Code or other collaborative text editors with large numbers of collaborators means that almost always
% The semantic editor services will be disabled because the program is going to be broken in multiple places
% transiently)

% This is tricky because:

% - some edits might be conflicting -- solve this with "conflict holes"

% - adding cut/paste or delete/restore allows for degenerate programs (cycles, multiple parents, etc.)

% - since we are commutative, we solve both synchronous and async collaborative editing

% - and this resolves issues around merges and conflicts

% - contribution of this paper is to solve these problems from type-theoretic first principles:

% - ...

% - Hazel

% \subsection{Contributions and Paper Organization}%
% \label{sec: Contributions and Paper Organization}


\section{Grove By Example}%
\label{sec:Grove By Example}

This section introduces collaborative structure editing in Grove by example.
\autoref{sub:Program Representation} describes how we use graphs to represent collaborative program sketches. 
\autoref{sub:Single-User Actions} then shows examples of edits being performed by a single user, Alice. 
\autoref{sub:Multi-user Interactions}-\ref{sub:Merge Conflicts} then describes a collaboration between two users, Alice and Bob,
as they edit their own branches of a program and periodically merge in each other's edits, starting with examples without conflicts, then considering the various kinds of conflicts that might arise.

For simplicity and concision, all of the examples in this section will be for a 
language of standard arithmetic operations, 
but our formalism in \autoref{sec:Formalism} and our implementation in \autoref{sec:Implementation} are parameterized by an arbitrary abstract syntax.

\figureSimple


Grove can form the basis for both a conventional version control workflow,
where edits are batched into commits, or real-time collaborative editing, 
where edits are communicated as they occur. This paper makes no assumptions about which batching mode is in use, nor do we consider the well-studied problem of reliably and efficiently communicating patches over a network.
%\autoref{sub:Cursors} discusses representing cursor locations when in real-time mode. 

\subsection{Representing Collaborative Program Sketches as Graphs}%
\label{sub:Program Representation}

The \textit{edit state} of a Grove branch is a directed multi-graph representing a \emph{collaborative program sketch}, meaning an incomplete program, i.e. one that may have \emph{holes} and (as we will return to) conflicts. 
For example, \autoref{fig:Simple:a} gives one such graph and its corresponding \emph{decomposition} into, in this case, a single syntax tree,~\texttt{x * \hole},
whose missing right operand is a hole, denoted $\hole$.

Each vertex represents a term in the specified language, except for a distinguished root vertex, 
and is labeled with a unique identifier (UID) and a \emph{constructor}. 
In \autoref{fig:Simple:a}, the root vertex has UID 0 and constructor~\textbullet. 
Vertices \vSimpleTimes{} and \vSimpleX{} have UIDs 2 and 4 and constructors~\texttt{*} and \texttt{var(}$x$\texttt{)}, respectively.
For clarity, we abbreviate \texttt{var(}$x$\texttt{)} as simply \texttt{x}; here, $x$ is a constructor parameter. We treat identifiers and literals as indivisible, but we discuss character-level editing in \autoref{sec:Discussion and Conclusion}.

An edge indicates that the destination vertex is a child of the origin vertex. 
Each edge is labeled with a UID~(e.g.,~1 and~3 in \autoref{fig:Simple:a})
and a \emph{position} (e.g., \texttt{Root} and \texttt{L} in \autoref{fig:Simple:a}). 
The parent vertex's constructor determines a set of valid {positions}. 
For instance, the~\texttt{*} constructor defines positions
~\texttt{L}~(for the left operand)
and~\texttt{R}~(for the right operand).
The~\texttt{var} constructor is a leaf so it defines no positions.
The root vertex constructor~\textbullet~has a single child position, \texttt{Root}.

%Visually, we indicate the position of an edge by the location of its origin.

Holes arise in the decomposition by the absence of a child at a valid position.
For example, in \autoref{fig:Simple:a} the absence of an \texttt{R} child under \vSimpleTimes{}
corresponds to the hole in the right operand of~\texttt{x * \hole}.

For clarity, we use odd numbers for vertex UIDs and even numbers for edge UIDs. In practice, UIDs would be generated by a mechanism
that effectively ensures that collaborators always generate distinct UIDs, e.g. by generating
universally unique IDs (UUIDs)~\cite{paskin1999toward}.

% \figureMove{}

\subsection{Structure Editing}%
\label{sub:Single-User Actions}
\figureWrapMove

Individual users perform \emph{edits} to evolve the edit state. We consider several standard edits, including insertion, deletion, cut-and-paste (relocation), copy-and-paste, and undo/redo. This paper abstracts over the user interface aspects of structure editors and makes no usability-related claims; these edits could be performed through, for example, drag-and-drop interactions (as in block-based editors like Scratch) or keyboard interactions (as in MPS and Hazel).  % User interface design for structure editors remains an active research area.
% The common thread is that structure editors maintain a program sketch throughout the editing process.

Each edit translates directly to a \emph{graph patch}, which consists of a sequence of \emph{patch commands}. The Grove patch language requires only two patch commands:  \emph{edge insertion} and \emph{edge deletion}. A vertex is inserted when it is included in an edge insertion command.

To illustrate the Grove patch language, let us consider a sequence of standard edits, found across structure editors, performed by a single user, Alice.

\subsubsection{Hole Filling}%
\label{sub:Construction}

First, Alice fills the hole in the right position of \texttt{x * \hole} from \autoref{fig:Simple:a} with the variable~\texttt{y}. The resulting edit state is shown in \autoref{fig:Simple:b}.
The patch corresponding to this hole filling action inserts an edge, labeled \eSimpleY{}, from the vertex corresponding to the parent term, \vSimpleTimes{}, to the newly constructed variable's vertex, \vSimpleY{}. 
The resulting graph decomposes to the term \texttt{x * y}.

\subsubsection{Deletion}%
\label{sub:Deletion}

% \figureMove{}

Next, Alice moves the cursor to~\texttt{x} in \autoref{fig:Simple:b} and deletes it, 
causing the deletion of edge~\eSimpleX{} and  resulting in the decomposition~\texttt{\hole{} * y} as shown in \autoref{fig:Simple:c}.

Once an edge with a particular identifier is deleted, it cannot be re-inserted.
For instance, if Alice performed an ``undo'' on this deletion,
a fresh edge between \vSimpleTimes{} and \vSimpleX{} would be created. (We can allow simple undo only if the patch has not yet been communicated to a collaborator). 

Notice that vertex \vSimpleX{} continues to exist (and if it had any children, they would remain connected to it; see below for the implications in the collaborative setting). In the remaining figures, we omit such orphaned vertices if they are not relevant to the exposition.

% The same state would arise if Alice cut \li{x}: we assume each user has an individually managed clipboard, so until a corresponding {paste} edit is performed the vertex is deleted but it can be restored by edge insertion.

\subsubsection{Wrapping}%
\label{sub:Wrapping}

Next, Alice moves the cursor to the parent term \vSimpleTimes{} in \autoref{fig:Simple:c}
with the intention of wrapping it in a binary addition expression with constructor~\texttt{+}.

Many structure editors define a primitive wrapping edit, choosing a position heuristically (e.g. favoring the left). Others 
require the user to cut the original term, construct the new outer term, then paste
the original term in the intended position. 

In either case, the corresponding sequence of patch commands would produce the edit states shown in \autoref{fig:Wrap}: the edge connecting the root to the original term is deleted (effectively cutting the original term), leaving \vSimpleTimes{} temporarily orphaned, then an edge to the new outer term is inserted, followed by an edge reconnecting the original term (effectively pasting the original term).

\subsubsection{Relocation}%
\label{sub:Repositioning}

Alice changes her mind and decides to relocate the multiplication from the left to the right position of the addition. A structure editor might support this using drag-and-drop or cut-and-paste affordances. In either case, the resulting patch commands will proceed through the two states shown in \autoref{fig:Move}: deleting the original incoming edge and then inserting an edge at the new location. Notice that the sub-graph corresponding to the relocated term itself is never deleted nor re-inserted, in contrast to conventional line-based patch languages.
See below for the implications in the collaborative setting.

\subsubsection{Copying} 
\label{sub:Copy}
A copy-and-paste, or a cut followed by multiple pastes, would of course involve copying the graph structure of the original term but generating fresh UIDs (not shown).


% \figureDifferentParts{}

%\figureCommutativity{}
\subsection{Collaboration}%
\label{sub:Multi-user Interactions}

We now turn our attention to how Grove handles collaboration.
The examples in this section generalize to collaborations between any number of users,
but for simplicity we consider only two: Alice and Bob.
Alice and Bob are each concurrently editing their own branches of the repository (or their own instance of a real-time collaborative editor), 
performing edits that translate to patches as described above. 
They periodically communicate these patches to one another. \autoref{fig:Commutativity} diagrams the Grove collaboration model.



\begin{figure}[h]
  \centering
  \begin{tikzpicture}
    \path (-3cm, 0cm) node (a) [align=center]            {Base \\ Branch};
    \path ( 0cm, 1cm) node (b) [align=center,alice node] {Alice's  \\ Branch};
    \path ( 0cm,-1cm) node (c) [align=center,bob node]   {Bob's    \\ Branch};
    \path ( 3cm, 0cm) node (d) [align=center]            {Merged \\ Branch};
    \path [draw,->,alice step] (a) -- node [pos=0.7,align=center,auto]      {Alice's \\ Edits} (b);
    \path [draw,->,bob step]   (a) -- node [pos=0.7,align=center,auto,swap] {Bob's \\ Edits}   (c);
    \path [draw,->,merge step] (b) -- node [pos=0.3,align=center,auto]      {Apply \\Bob's Patch} (d);
    \path [draw,->,merge step] (c) -- node [pos=0.3,align=center,auto,swap] {Apply \\Alice's Patch} (d);
  \end{tikzpicture}
  \caption{Collaboration in Grove is simple due to the commutativity of Grove's patch language.}
  \Description{This figures describes the commutativity of edits}
  \label{fig:Commutativity}
\end{figure}



\subsubsection{Commutativity}%
\figureDifferentPartsNestedParts

\label{sub:Commutativity:informal}
The Grove patch language is commutative, meaning that there is no need for a 
complex three-way merge algorithm (i.e. operational transform). Instead, each 
user can simply apply incoming patches to their own edit state as they arrive, no matter the order in 
which they arrive. If two users have received the same set of patches, their edit state will converge.

The key properties that make the Grove patch language commutative is that 
edge deletion is permanent and vertex insertion is permanent.
We establish commutativity formally in \autoref{sec:Formalism}.
For now, let us consider several example scenarios that demonstrate 
how Grove handles different collaborative editing scenarios, particularly 
the problematic situations outlined in \autoref{sec:Introduction}. 

\subsubsection{Solving the Granularity Problem}%
\label{sub:Editing Different Parts of the Code}

Alice and Bob start where Alice left off in \autoref{fig:Move:b} 
with the term \texttt{\hole{} + \hole{} * y}.
Alice then adds~\vDifferentPartsAlice{} as the left child of \vSimpleTimes{}.
Concurrently, Bob changes \vSimpleY{} to \vDifferentPartsBob{}.
Before sharing their patches,
Alice and Bob have the edit states \autoref{fig:DifferentParts:a} and \autoref{fig:DifferentParts:b}, respectively.
Note that the transition from \autoref{fig:Move:b} to \autoref{fig:DifferentParts:b}
represents multiple graph updates,
i.e., deleting~\eSimpleY{} and adding~\eDifferentPartsBob{} along with its child~\vDifferentPartsBob{}.
We thus mark the transition with a star.
Once Alice and Bob share their patches and apply each other's patch to their own edit state,
both edit states converge to the graph in \autoref{fig:DifferentParts:c}. 
Because Grove's patch language is structural rather than line-based, the fact that these edits happened to be close to one another (i.e. in the same arithmetic expression) does not run afoul of the \textbf{granularity problem} described in \autoref{sec:Introduction}.

% \figureNestedParts{}

\subsubsection{Solving the Relocation Modification Problem}%%
\label{sub:Editing Nested Parts of the Code}

After converging on \texttt{\hole{} + u * y} in \autoref{fig:DifferentParts:c}, 
Alice changes~\vDifferentPartsAlice{} to~\vNestedPartsAlice{},
producing the edit state in \autoref{fig:NestedParts:a}.
Meanwhile, Bob relocates~\vSimpleTimes{} (bringing along its children)
from the~\texttt{R} position of~\vWrapPlus{} to the~\texttt{L} position. As discussed above, this involves deleting~\eMoveTimes{} and adding~\eNestedPartsBob{}.
Bob's resulting edit state is shown in \autoref{fig:NestedParts:b}.

Although Alice has modified a term that Bob has concurrently relocated, the edits commute: Alice's modifications are relocated to the new location chosen by Bob.
This addresses the \textbf{relocation modification problem} described in \autoref{sec:Introduction}.
In a line-based setting, this kind of edit can lead to silent code duplication or spurious conflicts (the threat of which, in the author's experience, can inhibit development teams from performing useful code reorganizations).

% TODO: \todo{TODO}She can restore that code using a restoration action and add a fresh edge to it.

\subsubsection{Warning of Edits under Disconnected Terms}

If instead of relocating the multiplication in \autoref{fig:NestedParts}, Bob had deleted it (i.e. disconnected it from the root), then Alice's edits would still commute, but her edits would be applied under a deleted term. 

This situation could also arise in a real-time collaborative editor, where each individual edit might arrive at any time (rather than in atomic commits). If Alice, say, receives Bob's deletion of~\eMoveTimes{}, then makes her edits 
before receiving Bob's subsequent insertion of~\eNestedPartsBob{} to complete the relocation, 
Alice's edits would then temporarily be under a disconnected term. 

This does not present a formal problem or conflict. A subsequent edit might reconnect a disconnected term, so it is sensible for edits to these terms to be recorded. 
However, heuristically, a system might warn users, perhaps after a period of quiescence in a real-time setting, that Alice's edits were effectively deleted and provide affordances for interacting with disconnected terms.

\subsection{Conflicts}%
\label{sub:Merge Conflicts}

The collaborative edits discussed so far merge cleanly,
but in general, merging patches can lead to graphs that do not map cleanly to a conventional syntax tree. We identify several different motifs that might arise, all of which give rise to different kinds of conflicts in the graph decomposition: \emph{local conflicts}, \emph{relocation conflicts}, and \emph{unicyclic relocation conflicts}. 
As with merge conflicts in version-control systems such as git, these all require user intervention to resolve. 
%As we will return to in \autoref{sec:Type System}\todo{sec}, conflicts need not disable type-based editor services (the \textbf{semantic gap problem}). Instead, the system can type check around these conflict (and in some cases, infer a type for the conflict itself).

\subsubsection{Local Conflicts}%
\label{sub:Multi-child conflicts}

\figureMultiChild

Suppose Alice and Bob both start with the edit state \texttt{w * v + \hole} from \autoref{fig:NestedParts:c}.
Alice moves the cursor to the hole and constructs~\vMultiChildAlice{} as the~\texttt{R} child of~\vWrapPlus{}.
At the same time, Bob constructs~\vMultiChildBob{} at the same location.
Now Alice and Bob have the graphs in \autoref{fig:MultiChild:a} and \autoref{fig:MultiChild:b}, respectively.

When these patches are merged in \autoref{fig:MultiChild:c}, \emph{both} \eMultiChildAlice{} and \eMultiChildBob{}
appear in the merged graph. 
When decomposing this graph to a syntax tree, 
we resolve this conflict in the~\texttt{R} position of~\vWrapPlus~by decomposing to a \emph{local conflict}, {\binaryConflictHole{x}{y}}.

% TODO: "delete both"

Local conflicts can be resolved simply by deleting or relocating all but one of the conflicting terms (and editing the remaining term into the correctly merged value, if needed), which would remove the corresponding edge.
For example, Alice could resolve the problem by wrapping \li{x} and \li{y} with a multiplication, effectively moving them to non-conflicting locations.
No special edit actions are needed for conflict resolution.

% To ensure users can continue editing while in a conflicted state,
% all conflicts must be resolved before a program can be run.\footnote{It
%   may be possible to develop evaluation models that allow these sorts of conflicts,
%   but that is beyond the scope of this paper.})
% % TODO: need example of fully-automatic resolution and semi-automatic resolution.
% Note that as a convenience to the user, certain simple conflicts might be automatically resolved,
% but we consider this a higher-level, user-interface consideration.

It is worth noting one special situation: if Alice and Bob independently filled the hole with structurally identical terms, e.g. \li{x}, Grove would \emph{still} formally identify a conflict, because the terms have distinct UIDs. In this situation, it would be reasonable for the system to resolve the conflict without further coordination by deterministically choosing one of the two terms, e.g. the smallest.

When the conflicted terms are similar up to UID differences but not identical, it might be helpful  to give the developer the option to ``push down'' the conflicts as deeply as possible, using a tree differencing algorithm. However, this would increase the number of conflicts overall, so it may not always be preferable. Tree differencing is not fundamental to the collaboration model of Grove.


\subsubsection{Relocation Conflicts}%
\label{sub:Multi-parent conflicts}

\figureMultiParent

Grove's support for repositioning creates the possibility for \emph{relocation conflicts}. These occur when a merge causes a vertex to have multiple incoming edges, indicating that it does not have a uniquely determined location (as opposed to local conflicts, which occur when there are multiple outgoing edges at a specified location). 

For example, \autoref{fig:MultiParent} shows an example where Alice and Bob relocate a term, \li{w}, to two different locations (the two holes in \autoref{fig:MultiParent:a}). In both cases, the edits are modeled as one edge deletion followed by one edge addition.
Alice deletes~\eNestedPartsAlice{} and adds~\eMultiParentAlice{} in \autoref{fig:MultiParent:b}.
At the same time, Bob deletes~\eNestedPartsAlice{} and adds~\eMultiParentBob{} in \autoref{fig:MultiParent:c}.
Edge deletions is idempotent,
so the fact that Alice and Bob both deleted~\eNestedPartsAlice{} will not lead to a conflict. 
However, both~\eMultiParentAlice{} (added by Alice) and~\eMultiParentBob{} (added by Bob)
point to the same vertex.
Once these patches are merged, the resulting edit state is given in \autoref{fig:MultiParent:d}.
Notice~\vNestedPartsAlice{} has two edges pointing to it. 
When decomposing the graph, we leave a \emph{relocation conflict reference}, $\multiVertex{18}$, at each conflicting location. The conflicted term is separately tracked in the result of decomposition, which, due to conflicts like these, is formally a set of terms with references like these between them. We call this set a \emph{grove}. 
This approach partially addresses the \textbf{relocation conflict problem} from \autoref{sec:Introduction} (we also need to handle unicycles, see below, to fully address the problem).

To resolve relocation conflicts, a user can simply delete all but one of the relocation conflict references. This will cause the corresponding edges to be deleted,
and when only one edge remains, there will no longer be a conflict. An editor might provide a more convenient way of deleting all but a selected relocation conflict reference, and provide affordances for displaying these terms, e.g. by transcluding them inline at each location or showing them in a separate sidebar.

% \subsubsection{Single-User Cycles}
% \label{sub:Cycles}

% TODO: cut this section

% \figureSingleUserCycles{}

% Another case we must consider is when cycles appear in the graph.
% We categorize these into cycles caused by the action of a single user
% and cycles caused by the interaction of the actions of multiple users.

% In a single user context,
% normal insertion and deletion of code by the user cannot create cycles.
% However, it is possible with certain kinds of copy and paste.
% For example, suppose Alice is editing the code in \autoref{fig:Single-User Cycles:a}
% and uses copy and paste to copy \Vertex{TODO} to the right child of \Vertex{TODO}.
% There are two ways to interpret the paste action.
% The first interpretation is to create a deep copy of \Vertex{TODO}.
% This results in \autoref{fig:Single-User Cycles:b} and
% does not cause a cycle.
% The second interpretation is to simply add an edge to \Vertex{TODO}.
% This results in \autoref{fig:Single-User Cycles:b}
% and causes a cycle.

% Note that not all pastes should be deep copies.
% For example, Alice may have accomplished code move in \autoref{sub:Editing Nested Parts of the Code}
% by a cutting from the old position and pasting to the new position.
% Preserving Bob's nested edits requires that the paste be by reference instead of by copy.
% Distinguishing when a paste should be by reference versus by copy
% is ultimately a user interface question.
% Cycles caused by the local user's edits can be detected as soon as a user enters them
% by noting either that the graph would contain a cycle or the vertex
% already has a parent somewhere in the graph.

% Thus, as a user interface consideration, it might be best to either
% disallow such edits to at least warn users when their
% edits would create a cycle.

\subsubsection{Cycles}%
\label{sub:Multi-User Cycles}

\figureCycle
\figureDisconnect

Relocation in a collaborative setting can also lead to cycles in the graph. 
For example, consider the situation in \autoref{fig:Cycle}.
Starting in \autoref{fig:Cycle:a}, 
Alice relocates~\vMultiCycleTimes{} to the \texttt{R} child of~\vWrapPlus{}
and then~\vMultiCyclePlus{} underneath that in \autoref{fig:Cycle:b}.
Bob does the opposite, putting~\vMultiCycleTimes{} under~\vMultiCyclePlus{} in \autoref{fig:Cycle:c}. On their own, neither of these edits creates a cycle.
However, merging the two patches results in the graph
in \autoref{fig:Cycle:d}, which has a cycle
between~\vMultiCycleTimes{} and~\vMultiCyclePlus{}.

The main difficulty with cycles has to do with decomposition back to a term, i.e. a syntax tree. We do not want decomposition to traverse endlessly attempting to create an infinite tree, so we need to break the cycle somewhere.

If the cycle is connected to a larger term, then there will necessarily be at least one vertex along the cycle that has multiple incoming edges. In \autoref{fig:Cycle}, both \vMultiCycleTimes{} and \vMultiCyclePlus{} have multiple incoming edges. As described above, decomposition will leave relocation conflict references in these positions, thereby breaking the cycle. In this example, these references appear within a local conflict as well, because both vertices were relocated under a common parent vertex. As before, this cycle can be broken by deleting or otherwise modifying the terms until there are no longer any such conflicts.

It is also possible to merge patches such that a \emph{disconnected unicycle} emerges in the graph, even when neither patch disconnects any vertex from the root. \autoref{fig:Disconnection} shows a simple example of when this could occur: Alice relocates~\vMultiCycleTimes{} under~\vSimpleTimes{},
while Bob relocates~\vSimpleTimes{} under~\vMultiCycleTimes{}. 
This results in \autoref{fig:Disconnect:b} and \autoref{fig:Disconnect:c}, respectively.
Merging these causes both vertices to become disconnected, because they were both relocated. The inserted edges form a \emph{unicycle}, meaning a cycle where every vertex has in-degree $1$. In this case, we cannot rely on relocation references to break the cycle. Instead, we break the cycle by arbitrarily but deterministically choosing an edge along the unicycle, e.g. the edge with the smallest UID, and leaving a \emph{unicycle conflict reference} at that location, as shown in \autoref{fig:Disconnect:d}. A user can be notified of this situation when merging and resolve these conflicts again by relocating or deleting terms until the cycle no longer exists in the graph.

Relocation conflict references and unicycle conflict references together fully address the \textbf{relocation conflict problem} from \autoref{sec:Introduction}.


% TODO: technically a cycle

% TODO: in general detect when there are or were edits to something now deleted

% \todo{UI would show orphans only once edits orphans occur}

% \subsection{Cursors}%
% \label{sub:Cursors}

% \paragraph{Representation}

% For most editing a cursor is represented by a position within a vertex.
% For example, in TODO, Bob's cursor might be at the TODO position of the TODO:vertex.
% Representing it this way means that if Bob or some other user deletes TODO:vertex
% and replaces it with a different vertex, Bob's cursor is still valid.
% We specify a position within a vertex instead of a vertex because TODO.

% However, there is one case where we need a more precise cursor.
% That is when a vertex has a multi-child conflict (e.g. TODO in FIG:TODO).
% We want users to be able to put their cursor on either the conflict as a whole
% (e.g., TODO: \texttt{\{ x | y \}})
% or on an individual element (e.g., \texttt{x} or \texttt{y}).
% In the former case, a pair of a vertex and a child position suffices.
% For the latter case, we represent cursors by an edge identifier (e.g., TODO in TODO:FIG).

% \paragraph{Communication}

% In our system all edits use edge and vertex identifiers, and user
% cursors do not affect the interpretation of graph edits.
% Thus, cursors need not be communicated to other users.
% However, in a collaborative setting, seeing the cursors of other users
% can be useful.
% For this purpose, editors can announce their cursor position to other editors.
% At any point in time, the announcement
% with the most recent timestamp
% for a particular user
% is used.

% Note that this cursor position may refer to edges or verticies that
% do not yet exist on the receivers machine.
% In these cases, we can either not display the other user's cursor (and perhaps
% have a visual display flagging this fact), or we can display the most recent
% cursor that represents a valid position in the local graph (perhaps
% shown in a fadded color to show that this cursor is known to not be up to date).
% (Or perhaps, all cursors always fade/decay over time like on radar blips.)

\documentclass[acmsmall,10pt,review,anonymous]{acmart}\settopmatter{printfolios=true,printccs=false,printacmref=false}

\acmJournal{PACMPL}
\acmVolume{1}
\acmNumber{POPL} % CONF = POPL or ICFP or OOPSLA
\acmArticle{1}
\acmYear{2021}
\acmMonth{1}
\acmDOI{} % \acmDOI{10.1145/nnnnnnn.nnnnnnn}
\startPage{1}

\setcopyright{none}

%%%%%%%%

\renewcommand{\topfraction}{1} % Allow floats to take up the page
\renewcommand{\textfraction}{0}

%%%%%%%%
% \autoref from hyperref
\renewcommand{\AMSautorefname}          {Equation}
\renewcommand{\appendixautorefname}     {Appendix}
\renewcommand{\chapterautorefname}      {Chapter}
\renewcommand{\equationautorefname}     {Equation}
\renewcommand{\FancyVerbLineautorefname}{Line}
\renewcommand{\figureautorefname}       {Figure}
\renewcommand{\footnoteautorefname}     {Footnote}
\renewcommand{\Hfootnoteautorefname}    {Footnote}
\renewcommand{\itemautorefname}         {Item}
\renewcommand{\Itemautorefname}         {Item}
\renewcommand{\pageautorefname}         {Page}
\renewcommand{\paragraphautorefname}    {Section}
\renewcommand{\partautorefname}         {Part}
\renewcommand{\sectionautorefname}      {Section}
\renewcommand{\subparagraphautorefname} {Section}
\renewcommand{\subsectionautorefname}   {Section}
\renewcommand{\subsubsectionautorefname}{Section}
\renewcommand{\tableautorefname}        {Table}
\renewcommand{\theoremautorefname}      {Theorem}

%% Packages
\usepackage{scalerel}
\usepackage[most]{tcolorbox}
\usepackage{booktabs}
\usepackage[rule=false]{subcaption}
\usepackage{xspace}
\usepackage{graphicx}
\usepackage{relsize}
% \usepackage{centernot}
\usepackage{stmaryrd}
% \usepackage{semantic}

\let\colonapprox\undefined % Avoid redefinition error in `colonequals`
\let\colonsim\undefined % Avoid redefinition error in `colonequals`


% \usepackage{todonotes}
% \usepackage[ruled]{algorithm2e}



% Load MnSymbol without clobbering \ast
% See https://tex.stackexchange.com/a/269691
\usepackage{amsmath}% needed before mathabx
\let\amsast=\ast
\usepackage[matha]{mathabx}% needed to prevent \ast getting clobbered
\let\abxast=\ast
% \usepackage{MnSymbol}
\let\mnast=\ast
\let\ast=\abxast

\usepackage{bm}
\usepackage{thmtools}
\usepackage{colonequals}
\usepackage{mathpartir}
\usepackage{stmaryrd}
\usepackage{fontawesome}
\usepackage{array}
\usepackage{mathtools}
\usepackage{centernot}

\newtcolorbox{mybox}[2][]
{
  on line,
  hbox,
  boxsep=0pt,
  left=1pt,
  right=1pt,
  top=1pt,
  bottom=1pt,
  colframe=white,
  colback=#2
  #1,
}
\newcommand\goodcolor[2]{%
  \protect\leavevmode
  \begingroup
    \color{#1}%
    #2%
  \endgroup
}
%%%%%%%%
% TikZ Stuff
%\usepackage{etex} % Fix "No room for new \dimen" error
\usepackage{shellesc} % Fix bug that breaks the tikz 'external' library
\usepackage{tikz}
\usetikzlibrary{babel} % Ensure compatibility the 'babel' package

\usetikzlibrary{external} % Needs to be separately enabled
%\tikzexternalize % Enable externalization
%\usepackage{lua-visual-debug}

\usetikzlibrary{arrows.meta} % Arrow Tips
\tikzset{>=Stealth}
%\tikzset{<=stealth}
%\tikzset{arrows={-Stealth[scale=50]}}
%\tikzset{edge from parent/.style={draw,->,line width=0.6pt}}
%\tikzset{wideline/.style={line width=0.7pt}}
%\tikzset{boldline/.style={color=black,line width=1.0pt}}

\usetikzlibrary{
  backgrounds,  % Provides "framed" and "gridded"
  bending,      % bending arrow tips
  decorations.pathmorphing,   % Provides wavy edges
  graphs,       % Graph *notation*
  graphdrawing, % Graph *layout*
  quotes,       % Quote syntax (e.g., "foo")
}

\usegdlibrary{
  trees,
}

\tikzset{
  %every picture/.style={framed, background rectangle/.style={draw=gray!50}},
}
\tikzset{edge style/.style={
  draw,
  %color=gray,
  font={\small\ttfamily},
  /tikz/every edge quotes/.style={
    %draw=gray!20,
    anchor=west,
    swap/.append code={
      \ifpgfarrowswap
        \pgfkeysalso{anchor=west}
      \else
        \pgfkeysalso{anchor=east}
      \fi}},
}}
\tikzset{graphs/graph style/.style={
  tree layout,
  level distance=0.5cm,
  level sep=0.5cm,
  sibling distance=0.5cm,
  sibling sep=0.1cm,
  part distance=0.1cm,
  part sep=0.1cm,
  component distance=0.1cm,
  component sep=0.1cm,
  nodes={
    draw,
    %color=gray,
    inner sep=2pt,
    rounded corners=1mm},
  edges={edge style},
}}
\tikzset{graphs/root style/.style={
 %draw=none,
 as={\textbullet$_{\id{0}}$}
}}
\tikzset{alice/.style={
  color=red!80!black,
  font={\bfseries\small},
  thick,
}}
\tikzset{bob/.style={
  color=green!60!black,
  font={\bfseries\small},
  thick,
}}
\tikzset{merge/.style={
  color=blue,
  font={\bfseries\small},
  thick,
}}
\tikzset{alice edge/.style={alice, edge style, font={\bfseries\small}}}
\tikzset{alice node/.style={alice}}
\tikzset{alice step/.style={alice}}
\tikzset{bob edge/.style={bob, edge style}, font={\bfseries\small}}
\tikzset{bob node/.style={bob}}
\tikzset{bob step/.style={bob}}
\tikzset{merge edge/.style={merge, edge style}, font={\bfseries\small}}
\tikzset{merge node/.style={merge}}
\tikzset{merge step/.style={merge,decorate,decoration={coil,amplitude=1.0pt,segment length=7.0pt,aspect=0}}}
\tikzset{star/.style={edge node={node[inner sep=0pt,at end,sloped] {\textbf{\huge${}^{\ast}$}}}}}

% Define outline versions of + and -
\def\outlinepad{0.4pt}
\def\outlinestroke{0.4pt}
\newcommand{\Plus}{\mathord{
\begin{tikzpicture}[anchor=base, baseline]
%\node at (0,0) {+};
\path[draw, line width=\outlinestroke]
   ( 0.333em+\outlinestroke/2+\outlinepad,  0.270em+\outlinestroke/2+\outlinepad)
 --( 0.333em+\outlinestroke/2+\outlinepad,  0.229em-\outlinestroke/2-\outlinepad)
 --( 0.021em+\outlinestroke/2+\outlinepad,  0.229em-\outlinestroke/2-\outlinepad)
 --( 0.021em+\outlinestroke/2+\outlinepad, -0.084em-\outlinestroke/2-\outlinepad)
 --(-0.020em-\outlinestroke/2-\outlinepad, -0.084em-\outlinestroke/2-\outlinepad)
 --(-0.020em-\outlinestroke/2-\outlinepad,  0.229em-\outlinestroke/2-\outlinepad)
 --(-0.333em-\outlinestroke/2-\outlinepad,  0.229em-\outlinestroke/2-\outlinepad)
 --(-0.333em-\outlinestroke/2-\outlinepad,  0.270em+\outlinestroke/2+\outlinepad)
 --(-0.020em-\outlinestroke/2-\outlinepad,  0.270em+\outlinestroke/2+\outlinepad)
 --(-0.020em-\outlinestroke/2-\outlinepad,  0.583em+\outlinestroke/2+\outlinepad)
 --( 0.021em+\outlinestroke/2+\outlinepad,  0.583em+\outlinestroke/2+\outlinepad)
 --( 0.021em+\outlinestroke/2+\outlinepad,  0.270em+\outlinestroke/2+\outlinepad)
 --cycle
 ;
\end{tikzpicture}
}}

\newcommand{\Minus}{\mathord{
\begin{tikzpicture}[anchor=base, baseline]
%\node at (0,0) {$-$};
\path[draw, line width=\outlinestroke]
   ( 0.306em+\outlinestroke/2+\outlinepad,  0.270em+\outlinestroke/2+\outlinepad)
 --( 0.306em+\outlinestroke/2+\outlinepad,  0.229em-\outlinestroke/2-\outlinepad)
 --(-0.306em-\outlinestroke/2-\outlinepad,  0.229em-\outlinestroke/2-\outlinepad)
 --(-0.306em-\outlinestroke/2-\outlinepad,  0.270em+\outlinestroke/2+\outlinepad)
 --cycle
 ;
\end{tikzpicture}
}}

%%%%%%%%%%%%%%%%%%%%%%%%%%%%%%%%%%%%%%%%%%%%%%%%%%%%%%%%%%%%%%%%%%%%%%%%%%%%%%%%
% Language Components

\usepackage{xstring}
\usepackage{tensor}

% \selectcolormodel{gray}

% notation
\def\_{\texttt{\textunderscore}}
\newcommand{\id}[1]{\textcolor{gray}{\ensuremath{#1}}}
\newcommand{\eid}[2]{\tensor*{#2}{^{\id{#1}}}}
\newcommand{\abs}[1]{\left\lvert#1\right\rvert}
\newcommand{\figureCode}[1]{\textbf{\texttt{#1}}\vskip1em}
\newcommand{\Z}[1]{\hat{#1}}
\newcommand{\Set}[1][]{\Theta_{#1}}
\newcommand{\SetOf}[1]{\left\{#1\right\}}
\newcommand{\SuchThat}[1]{ : #1}
\newcommand{\SequenceOf}[1]{\left(#1\right)}
\newcommand{\SizeOf}[1]{\left\lvert#1\right\rvert}
\newcommand{\userAction}[1]{\xrightarrow{#1}}
\newcommand{\graphAction}[2]{\big(#1, #2\big)}

% sets
\def\A{\mathcal{A}}
\def\E{\mathcal{E}}
\def\G{\mathcal{G}}
\def\I{\mathcal{I}}
\def\K{\mathcal{K}}
\def\P{\mathcal{P}}
\def\S{\mathcal{S}}
\def\U{\mathcal{U}}
\def\V{\mathcal{V}}
\def\e{\varepsilon}
\def\AA{\textbf{A}}
\def\EE{\textbf{E}}

% Sorts
\newcommand\Exp{\mathsf{Exp}}
\newcommand\Pat{\mathsf{Pat}}
\newcommand\Typ{\mathsf{Typ}}

% objects
\def\e{\varepsilon}
\def\Grove{\gamma}

% judgments
\newcommand{\act}[5]{#1, #2; #3 \userAction{#4} #5}

%%%%%%%%%%%%%%%%%%%%%%%%%%%%%%%%%%%%%%%%%%%%%%%%%%%%%%%%%%%%%%%%%%%%%%%%%%%%%%%%
% Graphs

\DeclareMathOperator{\sortOp}{\text{sort}}
\DeclareMathOperator{\arityOp}{\text{arity}}
\newcommand{\sort}[1]{\sortOp\mathopen{}\left(#1\right)\mathclose{}}
\newcommand{\arity}[1]{\arityOp\mathopen{}\left(#1\right)\mathclose{}}

\DeclareMathOperator{\constructorOp}{\texttt{constructor}}
\DeclareMathOperator{\econstructorOp}{\texttt{econstructor}}
\DeclareMathOperator{\pconstructorOp}{\texttt{pconstructor}}
\DeclareMathOperator{\tconstructorOp}{\texttt{tconstructor}}
\newcommand{\constructor}[1]{\constructorOp\mathopen{}\left(#1\right)\mathclose{}}
\newcommand{\econstructor}[1]{\econstructorOp\mathopen{}\left(#1\right)\mathclose{}}
\newcommand{\pconstructor}[1]{\pconstructorOp\mathopen{}\left(#1\right)\mathclose{}}
\newcommand{\tconstructor}[1]{\tconstructorOp\mathopen{}\left(#1\right)\mathclose{}}

\newcommand{\Edge}[1]{Edge~#1}
\newcommand{\Vertex}[1]{Vertex~#1}
\newcommand{\multiVertex}[1]{\textcolor{red}{\ensuremath{\curlyveedownarrow_{#1}}}}
\newcommand{\cycleVertex}[1]{\textcolor{red}{\ensuremath{\rcirclearrowleft_{#1}}}}
\newcommand{\orphanVertex}[1]{\textcolor{red}{\ensuremath{\mathbf{\ndownarrow_{#1}}}}}
\newcommand{\rootVertex}{v_\text{root}}
\newcommand{\otherVertexVskip}{\vskip0.5em}

% \DeclareMathOperator{\outvertexOp}{\text{outvertex}}
% \newcommand{\outvertex}[1]{\outvertexOp\mathopen{}\left(#1\right)\mathclose{}}

\DeclareMathOperator{\defaultposOp}{\text{defaultpos}}
\newcommand{\defaultpos}[1]{\defaultposOp\mathopen{}\left(#1\right)\mathclose{}}

\newcommand{\parens}[1]{\textcolor{gray}{(}#1\textcolor{gray}{)}}

% Constructors
\newcommand\ExpVar{\mathsf{Exp\_var}}
\newcommand\ExpLam{\mathsf{Exp\_lam}}
\newcommand\ExpApp{\mathsf{Exp\_app}}
\newcommand\ExpNum{\mathsf{Exp\_num}}
\newcommand\ExpPlus{\mathsf{Exp\_plus}}
\newcommand\ExpTimes{\mathsf{Exp\_times}}
\newcommand\PatVar{\mathsf{Pat\_var}}
\newcommand\TypNum{\mathsf{Typ\_num}}
\newcommand\TypArrow{\mathsf{Typ\_arrow}}

% Positions
\newcommand\Root{\mathsf{Root}}
\newcommand\LamParam{\mathsf{Param}}
\newcommand\LamType{\mathsf{Type}}
\newcommand\LamBody{\mathsf{Body}}
\newcommand\AppFun{\mathsf{Fun}}
\newcommand\AppArg{\mathsf{Arg}}
\newcommand\PlusLeft{\mathsf{Left}}
\newcommand\PlusRight{\mathsf{Right}}
\newcommand\TimesLeft{\mathsf{Left}}
\newcommand\TimesRight{\mathsf{Right}}
\newcommand\ArrowArg{\mathsf{Arg}}
\newcommand\ArrowResult{\mathsf{Result}}

%%%%%%%%%%%%%%%%%%%%%%%%%%%%%%%%%%%%%%%%%%%%%%%%%%%%%%%%%%%%%%%%%%%%%%%%%%%%%%%%
% Terms

\newcommand{\varExp}[2]{#1^{#2}}
\newcommand{\numExp}[2]{\underline{#1}^{#2}}
\newcommand{\lamExp}[4]{\lambda^{#1} #2 : #3.#4}
\newcommand{\appExp}[3]{\left(#1~#2\right)^{#3}}
\newcommand{\plusExp}[3]{#1 +^{#2} #3}
\newcommand{\varPat}[2]{#1^{#2}}
\newcommand{\arrowTyp}[3]{#1 \to^{#2} #3}
\newcommand{\numTyp}[1]{Num^{#1}}
\newcommand{\hole}{\ensuremath{\square}} %\textcolor{violet}{\llparenthesis}}\textcolor{violet}{\rrparenthesis}}
% TODO: finish this macro
\newcommand{\conflictHoleForm}[2]{\textcolor{red}{\textbf{\{}}#1\textcolor{red}{\textbf{\}}}_{#2}}
\newcommand{\conflictHole}[1]{%
{\noexpandarg\StrSubstitute{#1}{,}{\textcolor{red}{\;\textbf{|}\;}}[\myargs]%
{\textcolor{red}{\textbf{\{}}\myargs\textcolor{red}{\textbf{\}}}}}}%

\newcommand{\eVar}[2]{\eid{#1}{#2}}
\newcommand{\eFun}[4]{\eid{#1}{\lambda} #2 : #3 . #4}
\newcommand{\eApp}[3]{\eid{#1}{\left(#2~#3\right)}}
\newcommand{\eNum}[2]{\eid{#1}{\underline{#2}}}
\newcommand{\ePlus}[3]{#2~\eid{#1}{\texttt{+}}~#3}
\newcommand{\eTimes}[3]{#2~\eid{#1}{\texttt{*}}~#3}
\newcommand{\pVar}[2]{\eid{#1}{#2}}
\newcommand{\tArrow}[3]{#2 \eid{#1}{\rightarrow} #3}
\newcommand{\tNum}[1]{\eid{#1}{Num}}

% TODO: come up with a metavariable for zippered sets (i.e., Grove components)
%  t = e | q | \tau
%  t^ = e^ | ...
%       \Theta_{NP}, ... are sets of terms
%       \Z{\Theta} = (\Z{t}, \Theta) is a set of terms with a zippered term inside

% TODO: add conflict terms to zippered syntax and cursor erasure

% |> { e1 | ... | en } <|
% { e1^ | ... | en }
% { e1 | ... | en^ }

% |> mpc <|
% |> uc <|

% TODO: sort out which one's "a" and which one's "\alpha" in the paper body (check Hazelnut paper)

% Graph actions (a):
%   - add edge
%   - remove edge

% User actions (\alpha) (indirect):
%
% \gamma^ --\alpha--> a*
% 
% (NP^, MP, U) -- Construct(k) --> 
% (NP, MP^, U) -- Construct(k) --> 
% (NP, MP, U^) -- Construct(k) --> 

% ((|> e <|, NP), MP, U) -- Construct(k) --> 
% v, p; ((|> ?? <|, NP), MP, U) -- Construct(k) -->  (v, p, (u_fresh, k)) \mapsto +

% TODO: define well sorted grove

%%%%%%%%%%%%%%%%%%%%%%%%%%%%%%%%%%%%%%%%%%%%%%%%%%%%%%%%%%%%%%%%%%%%%%%%%%%%%%%%
% Zippered Terms

\newcommand{\cursor}[1]{{\vartriangleright}#1{\vartriangleleft}}
\newcommand{\erase}[1]{#1\mathclose{}^{\diamond}\mathclose{}}

%%%%%%%%%%%%%%%%%%%%%%%%%%%%%%%%%%%%%%%%%%%%%%%%%%%%%%%%%%%%%%%%%%%%%%%%%%%%%%%%
% Decomposition

\DeclareMathOperator{\decompOp}{\texttt{decomp}}
\DeclareMathOperator{\vertexesOp}{\texttt{vertexes}}
\newcommand{\decomp}[1]{\decompOp\mathopen{}\left(#1\right)\mathclose{}}
\newcommand{\vertexes}[1]{\vertexesOp\mathopen{}\left(#1\right)\mathclose{}}

\DeclareMathOperator{\edecompOp}{\texttt{edecomp}}
\DeclareMathOperator{\pdecompOp}{\texttt{pdecomp}}
\DeclareMathOperator{\tdecompOp}{\texttt{tdecomp}}
\DeclareMathOperator{\edecompPrimeOp}{\edecompOp^\prime}
\DeclareMathOperator{\pdecompPrimeOp}{\pdecompOp^\prime}
\DeclareMathOperator{\tdecompPrimeOp}{\tdecompOp^\prime}
\newcommand{\edecomp}[1]{\edecompOp\mathopen{}\left(#1\right)\mathclose{}}
\newcommand{\pdecomp}[1]{\pdecompOp\mathopen{}\left(#1\right)\mathclose{}}
\newcommand{\tdecomp}[1]{\tdecompOp\mathopen{}\left(#1\right)\mathclose{}}
\newcommand{\edecompPrime}[2]{\edecompPrimeOp\mathopen{}\left(#1, #2\right)\mathclose{}}
\newcommand{\pdecompPrime}[2]{\pdecompPrimeOp\mathopen{}\left(#1, #2\right)\mathclose{}}
\newcommand{\tdecompPrime}[2]{\tdecompPrimeOp\mathopen{}\left(#1, #2\right)\mathclose{}}

% helpers
\DeclareMathOperator{\outedgesOp}{\text{outedges}}
\DeclareMathOperator{\ingraphOp}{\text{ingraph}}
\DeclareMathOperator{\parentsOp}{\text{parents}}
\DeclareMathOperator{\ancestorsOp}{\text{ancestors}}
\DeclareMathOperator{\verticesOp}{\text{vertices}}
% \DeclareMathOperator{\childrenOp}{\text{children}}
\DeclareMathOperator{\lfpOp}{\text{lfp}}
\DeclareMathOperator{\ancestorsPrimeOp}{\ancestorsOp^\prime}
\newcommand{\outedges}[2]{\outedgesOp\mathopen{}\left(#1, #2\right)\mathclose{}}
\newcommand{\ingraph}[1]{\ingraphOp\mathopen{}\left(#1\right)\mathclose{}}
% \newcommand{\children}[1]{\childrenOp\mathopen{}\left(#1\right)\mathclose{}}
\newcommand{\parents}[1]{\parentsOp\mathopen{}\left(#1\right)\mathclose{}}
\newcommand{\ancestors}[1]{\ancestorsOp\mathopen{}\left(#1\right)\mathclose{}}
\newcommand{\ancestorsPrime}[1]{\ancestorsPrimeOp\mathopen{}\left(#1\right)\mathclose{}}
\newcommand{\lfp}[1]{\lfpOp\mathopen{}\left(#1\right)\mathclose{}}

\DeclareMathOperator{\minOp}{\text{min}}
\renewcommand{\min}[1]{\minOp\mathopen{}\left(#1\right)\mathclose{}}

\DeclareMathOperator{\rootsOp}{\text{roots}}
\newcommand{\roots}[1]{\rootsOp\mathopen{}\left(#1\right)\mathclose{}}

%%%%%%%%%%%%%%%%%%%%%%%%%%%%%%%%%%%%%%%%%%%%%%%%%%%%%%%%%%%%%%%%%%%%%%%%%%%%%%%%
% Recomposition

\DeclareMathOperator{\recompOp}{\texttt{recomp}}
\DeclareMathOperator{\erecompOp}{\texttt{erecomp}}
\DeclareMathOperator{\precompOp}{\texttt{precomp}}
\DeclareMathOperator{\trecompOp}{\texttt{trecomp}}
\newcommand{\recomp}[1]{\recompOp\mathopen{}\left(#1\right)\mathclose{}}
\newcommand{\erecomp}[1]{\erecompOp\mathopen{}\left(#1\right)\mathclose{}}
\newcommand{\precomp}[1]{\precompOp\mathopen{}\left(#1\right)\mathclose{}}
\newcommand{\trecomp}[1]{\trecompOp\mathopen{}\left(#1\right)\mathclose{}}

%%%%%%%%%%%%%%%%%%%%%%%%%%%%%%%%%%%%%%%%%%%%%%%%%%%%%%%%%%%%%%%%%%%%%%%%%%%%%%%%

% actions
\def\Down{\text{Down}}
\def\Enqueue{\text{Enqueue}}
\def\Left{\text{Left}}
\def\Move{\text{Move}}
\def\Num{\text{Num}}
\def\Right{\text{Right}}
\def\Select{\text{Select}}
\def\Send{\text{Send}}
\def\Up{\text{Up}}

\newcommand{\Construct}[1]{\text{Construct}\mathopen{}\left(#1\right)\mathclose{}}
\newcommand{\Delete}{\text{Delete}}
\newcommand{\Reposition}[2]{\text{Reposition}\mathopen{}\left(#1, #2\right)\mathclose{}}
\newcommand{\Wrap}[2]{\mathrm{Wrap}\mathopen{}\left(#1, #2\right)\mathclose{}}

% Calling \newvertex{Foo}{bar} defines
%   \vidFoo to be a new id number, and
%   \vFoo to be \texttt{bar}\ensuremath{_{\vidFoo}}
\newcounter{NodeVertexCounter}
\newcommand{\newvertex}[2]{%
\ifodd\theNodeVertexCounter
  \addtocounter{NodeVertexCounter}{1}%
\else
  \addtocounter{NodeVertexCounter}{2}%
\fi
\expandafter\newcommand\csname vid#1\endcsname{}% This is just to check if this is a redefinition
\expandafter\global\expandafter\edef\csname vid#1\endcsname{\theNodeVertexCounter}%
\expandafter\newcommand\csname v#1\endcsname{}% This is just to check if this is a redefinition
\expandafter\gdef\csname v#1\endcsname{\texttt{#2}\ensuremath{_{\id{\csname vid#1\endcsname}}}}%
}
\newcommand{\newedge}[2]{%
\ifodd\theNodeVertexCounter
  \addtocounter{NodeVertexCounter}{2}%
\else
  \addtocounter{NodeVertexCounter}{1}%
\fi
\expandafter\newcommand\csname eid#1\endcsname{}% This is just to check if this is a redefinition
\expandafter\global\expandafter\edef\csname eid#1\endcsname{\theNodeVertexCounter}%
\expandafter\newcommand\csname e#1\endcsname{}% This is just to check if this is a redefinition
\expandafter\gdef\csname e#1\endcsname{\texttt{#2}\ensuremath{_{\id{\csname eid#1\endcsname}}}}%
}
\setcounter{NodeVertexCounter}{-1}
\newvertex{Root}{Root}

% Support \includegraphics of .dot files
\DeclareGraphicsRule{.dot}{pdf}{.pdf}{`dot -Tpdf #1 -o \noexpand\OutputFile}


\begin{document}

\title[Grove]{Excerpt: Recomposition}

\maketitle

\section{Formalism}

\begin{definition}
  Let $\e = (u, (u_1, k_1), p, (u_2, k_2))$.
  A graph $G$ is well sorted if $(p, \sort(k_2)) \in \arity(k_1)$
  for all $\e$ such that $G(\e) \in \{\Plus, \Minus\}$.
\end{definition}

\begin{theorem}
  Let $G$ be a well sorted graph.
  There exists a grove $\Grove$ such that $\decomp{G} = \Grove$.
\end{theorem}

Proof: provide a witness that demonstrates the conclusion.

% \begin{theorem}
%   Let $G$ be a well sorted graph.
%   If $\decomp{G} = \Grove$ then $\vertexes{G} = \vertexes{\Grove}$.
% \end{theorem}

% \begin{theorem}
%   Let $G$ be a well sorted graph.
%   If $\recomp{\Grove} = G$ then $\vertexes{\Grove} = \vertexes{G}$.
% \end{theorem}

\begin{theorem}
  Let $G$ be a well sorted graph.
  if $\decomp{G} = \Grove$ then $\recomp{\Grove} \cong G$.
\end{theorem}

%%%%%%%%%%%%%%%%%%%%%%%%%%%%%%%%%%%%%%%%%%%%%%%%%%%%%%%%%%%%%%%%%%%%%%%%%%%%%%%%

\subsection{Decomposition}

\begin{align*}
  \children(v, p) &= \SetOf{v' \SuchThat{\exists \e = (u, v, p, v'), G(\e) = \Plus}} \\
  \parents(v) &= \SetOf{v' \SuchThat{\exists \e = (u, v', p, v), G(\e) = \Plus}} \\
  \ancestors(v) &= \mathopen{}\left( \lfp(\ancestors') \right)\mathclose{}(v) \\
  \ancestors'(v) &= \parents(v) \cup \ancestors'(\parents(v)) \\
  \min\mathopen{}\left(\SetOf{(u_1, k_1), \ldots, (u_n, k_n)}\right)\mathclose{} &= (u_j, k_j) \text{ s.t. } 1 \leq j \leq n \land u_j \le u_i \forall i = 1, \ldots, n
\end{align*}

\noindent $\boxed{\decomp{G} = \Grove}$
%
\begin{align*}
  \decomp{G} &= (NP, MP, U, D) \\
  NP &= \SetOf{\expr(v) \SuchThat{|\parents(v)| = 0}} \\
  MP &= \SetOf{\expr(v) \SuchThat{|\parents(v)| > 1}} \\
  U &= \SetOf{\expr(v) \SuchThat{|\parents(v)| = 1 \land v = \min(\ancestors(v))}} \\
  D &= \SetOf{\e \SuchThat{G(\e) = \Minus}}
\end{align*}

\noindent $\boxed{\expr(v) = e}$
%
\begin{align*}
  \expr(v=(u, Root)) &= expr'(v, Root) \\
  \expr(v=(u, \ExpVar(x))) &= x^{\id{u}} \\
  \expr(v=(u, \ExpLam)) &= \lambda^{\id{u}} \patt'(v, \LamParam) : \type'(v, \LamType) . \expr'(v, \LamBody) \\
  \expr(v=(u, \ExpApp)) &= (\expr'(v, \AppFun)~\expr'(v, \AppArg))^{\id{u}} \\
  \expr(v=(u, \ExpNum(n))) &= n^{\id{u}} \\
  \expr(v=(u, \ExpPlus)) &= \expr'(v, \PlusLeft)~\texttt{+}^{\id{u}}~\expr'(v, \PlusRight) \\
  \expr(v=(u, \ExpTimes)) &= \expr'(v, \TimesLeft)~\texttt{*}^{\id{u}}~\expr'(v, \TimesRight)
\end{align*}

\noindent $\boxed{\patt(v) = q}$
%
\begin{align*}
  \patt(v=(u, \PatVar(x))) &= x^{\id{u}}
\end{align*}

\noindent $\boxed{\type(v) = \tau}$
%
\begin{align*}
  \type(\tau=(u, \TypArrow)) &= \type'(\tau, \ArrowArg) \rightarrow^{\id{u}} \type'(\tau, \ArrowResult) \\
  \type(\tau=(u, \TypNum) &= Num^{\id{u}}
\end{align*}

\noindent $\boxed{\expr'(v, p) = e}$
%
\begin{align*}
  \expr'(v,p) = \begin{cases}
    \hole & \children(v,p) = \varnothing \\
    \conflictHole{\expr(v_1),\ldots,\expr(v_n)} & \children(v,p) = \SetOf{v_1, \ldots, v_n} \\
    \multiVertex{u} & \children(v,p) = \SetOf{v'} \land |\parents(v')| > 1 \\
    \cycleVertex{u} & \children(v,p) = \SetOf{v' = (u,k)} \land |\parents(v')| = 1 \\
        & \phantom{\children(v,p)} \land v' = \min(\ancestors(v')) \\
    \expr(v') & \text{otherwise} \\
  \end{cases}
\end{align*}

\noindent $\boxed{\patt'(v, p) = q}$
%
\begin{align*}
  \patt'(v,p) = \begin{cases}
    \hole & \children(v,p) = \varnothing \\
    \conflictHole{\patt(v_1),\ldots,\patt(v_n)} & \children(v,p) = \SetOf{v_1, \ldots, v_n} \\
    \multiVertex{u} & \children(v,p) = \SetOf{v'} \land |\parents(v')| > 1 \\
    \cycleVertex{u} & \children(v,p) = \SetOf{v'=(u,k)} \land |\parents(v')| = 1 \\
        & \phantom{\children(v,p)} \land v' = \min(\ancestors(v')) \\
    \patt(v') & \text{otherwise} \\
  \end{cases}
\end{align*}

\noindent $\boxed{\type'(v, p) = \tau}$
%
\begin{align*}
  \type'(v,p) = \begin{cases}
    \hole & \children(v,p) = \varnothing \\
    \conflictHole{\type(v_1),\ldots,\type(v_n)} & \children(v,p) = \SetOf{v_1, \ldots, v_n} \\
    \multiVertex{u} & \children(v,p) = \SetOf{v'} \land |\parents(v')| > 1 \\
    \cycleVertex{u} & \children(v,p) = \SetOf{v' = (u,k)} \land |\parents(v')| = 1 \\
        & \phantom{\children(v,p)} \land v' = \min(\ancestors(v')) \\
    \type(v') & \text{otherwise} \\
  \end{cases}
\end{align*}

%%%%%%%%%%%%%%%%%%%%%%%%%%%%%%%%%%%%%%%%%%%%%%%%%%%%%%%%%%%%%%%%%%%%%%%%%%%%%%%%

\subsection{Recomposition}

\noindent $\boxed{\recomp{\Grove} \cong G}$
%
\begin{align*}
  \recomp{(MP, NP, U, D)} &\cong \recompPlus{MP, NP, U} \cup \SetOf{\e \mapsto \Minus \SuchThat{\e \in D}}
\end{align*}

\noindent $\boxed{\recompPlus{MP, NP, U} \cong G}$
%
\begin{align*}
  \recompPlus{MP, NP, U} &= \SetOf{\e \mapsto \Plus \SuchThat{\e \in \bigcup_{e \in MP \cup NP \cup U} \erecomp{e}}}
\end{align*}

\noindent $\boxed{\erecomp{e} \cong \E}$
%
\begin{align*}
  \erecomp{\varExp{x}{u}} &= \SetOf{}
  \\
  \erecomp{\numExp{n}{u}} &= \SetOf{}
  \\
  \erecomp{\appExp{e_\AppFun}{e_\AppArg}{u}}
  &= \erecomp{e_\AppFun}
  \cup \erecomp{e_\AppArg} \\
  &\cup \erecompPrime{v}{\AppFun}{e_\AppFun}
  \cup \erecompPrime{v}{\AppArg}{e_\AppArg} \\
  &\quad\text{where } v = (u, \ExpApp)
  \\
  \erecomp{\plusExp{e_\PlusLeft}{u}{e_\PlusRight}}
  &= \erecomp{e_\PlusLeft}
  \cup \erecomp{e_\PlusRight} \\
  &\cup \erecompPrime{v}{\PlusLeft}{e_\PlusLeft}
  \cup \erecompPrime{v}{\PlusRight}{e_\PlusRight} \\
  &\quad\text{where } v = (u, \ExpPlus)
  \\
  \erecomp{\lamExp{u}{q_\LamParam}{\tau_\LamType}{e_\LamBody}}
  &= \precomp{q_\LamParam}
  \cup \trecomp{\tau_\LamType}
  \cup \erecomp{e_\LamBody} \\
  &\cup \precompPrime{v}{\LamParam}{q_\LamParam}
  \cup \trecompPrime{v}{\LamType}{\tau_\LamType} \\
  &\cup \erecompPrime{v}{\LamBody}{e_\LamBody} \\
  &\quad\text{where } v = (u, \ExpLam)
  \\
  \erecomp{\hole} &= \SetOf{}
  \\
  \erecomp{\conflictHole{e_1, \cdots, e_n}}
  &= \bigcup_{k=1}^n \erecomp{e_k}
  \\
  \erecomp{\multiVertex{v}} &= \SetOf{}
  \\
  \erecomp{\cycleVertex{v}} &= \SetOf{}
\end{align*}
%
% \vskip\baselineskip
$\boxed{\erecompPrime{v}{p}{e} \cong \E}$
%
\begin{align*}
  \erecompPrime{v}{p}{\varExp{x}{u}}
  &= \SetOf{(u', v, p, (u, \ExpVar(x)))}
  \\
  \erecompPrime{v}{p}{\numExp{n}{u}}
  &= \SetOf{(u', v, p, (u, \ExpNum(n)))}
  \\
  \erecompPrime{v}{p}{\appExp{e_\AppFun}{e_\AppArg}{u}}
  &= \SetOf{(u', v, p, (u, \ExpApp))}
  \\
  \erecompPrime{v}{p}{\plusExp{e_\PlusLeft}{u} {e_\PlusRight}}
  &= \SetOf{(u', v, p, (u, \ExpPlus))}
  \\
  \erecompPrime{v}{p}{\lamExp{u}{q_\LamParam}{\tau_\LamType}{e_\LamBody}}
  &= \SetOf{(u', v, p, (u, \ExpLam))}
  \\
  \erecompPrime{v}{p}{\hole} &= \SetOf{}
  \\
  \erecompPrime{v}{p}{\conflictHole{e_1, \cdots, e_n}}
  &= \bigcup_{i=1}^n \erecompPrime{v}{p}{e_i}
  \\
  \erecompPrime{v}{p}{\multiVertex{v'}}
  &= \SetOf{(u', v, p, v')}
  \\
  \erecompPrime{v}{p}{\cycleVertex{v'}}
  &= \SetOf{(u', v, p, v')}
\end{align*}
%
% \vskip\baselineskip
$\boxed{\precomp{q} \cong \E}$
%
\begin{align*}
  \precomp{\varPat{x}{u}} &= \SetOf{}
  \\
  \precomp{\hole} &= \SetOf{}
  \\
  \precomp{\conflictHole{q_1, \cdots, q_n}}
  &= \bigcup_{i=1}^n \precomp{q_i}
  \\
  \precomp{\multiVertex{v}} &= \SetOf{}
  \\
  \precomp{\cycleVertex{v}} &= \SetOf{}
\end{align*}
%
% \vskip\baselineskip
$\boxed{\precompPrime{v}{p}{q} \cong \E}$
%
\begin{align*}
  \precompPrime{v}{p}{\varPat{x}{u}}
  &= \SetOf{(u', v, p, (u, \PatVar(x)))}
  \\
  \precompPrime{v}{p}{\hole} &= \SetOf{}
  \\
  \precompPrime{v}{p}{\conflictHole{e_1, \cdots, e_n}}
  &= \bigcup_{i=1}^n \precompPrime{v}{p}{e_i}
  \\
  \precompPrime{v}{p}{\multiVertex{v'}}
  &= \SetOf{(u', v, p, v')}
  \\
  \precompPrime{v}{p}{\cycleVertex{v'}}
  &= \SetOf{(u', v, p, v')}
\end{align*}
%
% \vskip\baselineskip
$\boxed{\trecomp{\tau} \cong \E}$
%
\begin{align*}
  \trecomp{\arrowTyp{\tau_\ArrowArg}{u}{\tau_\ArrowResult}}
  &= \trecomp{\tau_\ArrowArg}
  \cup \trecomp{\tau_\ArrowResult} \\
  &\cup \trecompPrime{v}{\ArrowArg}{\tau_\ArrowArg}
  \cup \trecompPrime{v}{\ArrowResult}{\tau_\ArrowResult} \\
  &\quad\text{where } v = (u, \TypArrow)
  \\
  \trecomp{\numTyp{u}} &= \SetOf{}
  \\
  \trecomp{\hole} &= \SetOf{}
  \\
  \trecomp{\conflictHole{\tau_1, \cdots, \tau_n}}
  &= \bigcup_{i=1}^n \trecomp{\tau_i}
  \\
  \trecomp{\multiVertex{u}} &= \SetOf{}
  \\
  \trecomp{\cycleVertex{u}} &= \SetOf{}
\end{align*}
%
% \vskip\baselineskip
$\boxed{\trecompPrime{v}{p}{\tau} \cong \E}$
%
\begin{align*}
  \trecompPrime{v}{p}{\arrowTyp{\tau_\ArrowArg}{u}{\tau_\ArrowResult}}
  &= \SetOf{(u', v, p, (u, \TypArrow)))}
  \\
  \trecompPrime{v}{p}{\numTyp{u}}
  &= \SetOf{(u', v, p, (u, \TypNum)))}
  \\
  \trecompPrime{v}{p}{\hole} &= \SetOf{}
  \\
  \trecompPrime{v}{p}{\conflictHole{\tau_1, \cdots, \tau_n}}
  &= \bigcup_{i=1}^n \trecompPrime{v}{p}{\tau_i}
  \\
  \trecompPrime{v}{p}{\multiVertex{v'}}
  &= \SetOf{(u', v, p, v')}
  \\
  \trecompPrime{v}{p}{\cycleVertex{v'}}
  &= \SetOf{(u', v, p, v')}
\end{align*}

%%%%%%%%%%%%%%%%%%%%%%%%%%%%%%%%%%%%%%%%%%%%%%%%%%%%%%%%%%%%%%%%%%%%%%%%%%%%%%%%

\subsection{User Actions}

\[
  \arraycolsep=0pt
  \begin{array}{llll}
    \alpha & {}::={} & \Create{k} \mid \Delete \mid \Restore{v} \\
    %  \mid \Drop{\e}
  \end{array}
\]

\begin{align*}
  \Wrap{k, q} &= \Delete; v=\Create{k}; \Restore{v, q} \\
  \Reposition{v, q} &= \Delete; \Restore{v, q} \\
\end{align*}
%
where $v$ is the destination vertex implied by the cursor, when it exists.
When no such vertex exists, $\Restore{v}$ is a no-op.

\end{document}


\section{The Grove Workbench}%
\label{sec:Implementation}

We implemented the core Grove calculus of the previous section as an \texttt{OCaml} library called the Grove Workbench and a corresponding web-based collaborative structure editor written using \texttt{js\_of\_ocaml}~\cite{DBLP:journals/spe/VouillonB14} primarily intended to demonstrate the collaborative features of the workbench and serve as a companion to the formal developments in this paper. The library is parameterized by a syntax specification for expressions, with the necessary data structures generated automatically.

\subsection{The Grove Workbench}
\label{sub:impl-grv}

On opening up the \emph{Grove Workbench}, the user is met with two almost identical panels side-by-side, emulating a collaborative editor environment between two users. Additional collaborators can be generated on command. In each case, a cursor is placed on an empty hole at the root of the displayed term decomposition, which is displayed as a graph visualization of the same graph structure so that the UI resembles the figures in \autoref{sec:Grove By Example}. Below this are buttons for various user edit actions, and commands to send queued commands to specific other users. In addition, we have separate panels for multi-parented, deleted, and unicycle panels, corresponding to the partitioned grove datastructure in the previous section. We will now examine them in correspondence to the formalism described in \autoref{sec:Formalism}. 

% As the graph forms a CmRDT, sharing edits across instances is a matter of record and replay. When the user performs some edits, which correspond to patches formally, the \emph{actions} panel displays the local history of edit actions. These patches can be shared across editor instances, and the local history panel is flushed upon synchronization. The user edits the state of the graph, which is displayed in the graph visualization, and upon these edits, the graph gets \emph{decomposed} into a grove. Under the hood, the model is able to \emph{classify} the vertices as one of $NP$, $MP$, or $U$ roots or not a root. The terms in $NP$, $MP$, and $U$ root cases correspond to the deleted, multiparented, and unicycle panels in the workbench, where the latter two are instances of location-based conflicts. The deleted panel also allows the user to select a deleted term and restore it to the cursor location in the editor, which is attached using a fresh edge, as edge deletion is permanent. In addition to performing edits and sharing them, the user can also \emph{clone} and \emph{drop} editor instances to add more collaborative users into the mix. This is a brief overview of the features of the workbench.


% patches -> history of edit actions -> actions panel
% user interacts with tree structure editor backed by the graph data-structure and recomp-decomp
% generates dsa in Ocaml given a syntax tree spec

\subsection{Graph Implementation}
\label{sec: Graph Implementation}
The graph data structure is implemented as \texttt{OCaml Map} data structure with insertion and selection operations whose asymptotic worst-case complexity is logarithmic with respect to the size of the map. Since we cannot implement an infinite mapping directly, the graph only maps live or deleted edges to edge states $\left\{\Plus, \Minus\right\}$. We do not represent edges that map to $\bot$.

For a graph $G : (\E = \U \times \V \times \P \times \V) \to \Sigma$, our graph decomposition algorithm runs in $O(\abs{\V} \log \abs{\V} + \abs{\E} \log \abs{\V})$.
It begins with a scan of all edges that have been created or deleted $O(\abs{\E})$.
Their vertices at both ends are partitioned into three sets: multi-parented, single-parented, or orphaned.
Their relationships are recorded in maps for $O(\log \abs{\V})$ lookups of parent and child edge sets.
After the vertices have been partitioned, we traverse the various single-parented components and produce equivalent expressions $O(\abs{\V})$.
For unicycles, we traverse backward until a vertex is seen twice $O(\abs{\V})$, then proceed forward to find the least vertex on the cycle $O(\abs{\V})$. Once the least vertex is found, decomposition of unicycle begins with it, thus ensuring any edges to it become references to a unicycle root.

% \subsection{Grove Implementation}
% \label{sec: Grove Implementation}


% \subsection{Parameterized Code Generation}
% \label{sec: Parameterized Code Generation}

% How do you give it the corresponding language definition to generate child positions, arities


% % TODO: rename Decomposition.ml to Grove.ml

% There's a function \verb!decompose! of type \verb!Graph.t) : t * Edge.Set.t Vertex.Map.t! that 

% From this graph, we compute the set of all edges and the set of live edges with a linear scan.

% % TODO: make this into a listing

% \begin{verbatim}
%   Graph.edges : Graph.t -> Edge.Set.t
%   Graph.live_edges : Graph.t -> Edge.Set.t
% \end{verbatim}

% % TODO: keep connecting the code to the formalism like this

% These vertices are partitioned into three sets: multi-parented, single-parented, or orphaned.
% A vertex $v$ is considered a parent of another vertex $v'$ if there is a live edge from $v$ to $v'$.

% For each vertex, we take the parent and child edge sets.

% % TODO: fix the formalism for parents (to work with edges instead of vertices) and then check that it matches the implementation. Try renaming parents to vparents and adding parents for edges.

% % TODO: Talk about how the graph is implemented (not as an actual function, but as a map from edges to edge states) so that we can denote the set of created + deleted edges clearly, as well as all of the vertices that have ever been created.



% TODO: is this still a thing?
% Compared to the core language, our implementation lacks support for lists and case expressions.

% Optimizations: least fixed point


% We determine $e$ by starting at the root of the graph and including every connected
% vertex that has a single parent. In positions where there are no children, we
% leave an empty hole $\hole$. In positions where there are multiple children, we
% leave a conflict hole of the form $\conflictHole{e_1,\ldots,e_n}$. Whenever we
% encounter a vertex with multiple parents, we add its corresponding expression to
% the second component $MP$ and leave an indirect reference $\multiVertex{u}$, where $u$ is
% the unique identifier of the referenced vertex

% here are some properties that show how to get one from the other, and to/from
% graph

% algorithmic section: if we want to implement these conversions, here are the
% steps (without introducing new concepts):

% - linear scan
% - breaking out roots
% - ...

% Graph decomposition occurs in three steps.



% TODO: use code generation instead: have Lang.ml generate expr / expr' / ... for us; then we don't need this figure.
% TODO: maybe show this figure and say that it's possible to handle generically without code generation

% \begin{figure}
%   \[
%     \arraycolsep=0pt
%     \begin{array}{lrlll}
%       \textrm{Trees:\qquad}  & t & {}\in Tree & {}::={} & \textrm{Vertex}(v, m) \mid \textrm{Ref}(v) \\
%       \textrm{Maps:}         & m & {}\in Map & {}::={} & \varnothing \mid p \mapsto [(\e, t), \ldots, (\e, t)]; m \\
%     \end{array}
%   \]
%   \caption{Syntax of graph decomposition trees as a grammar.}
%   \Description{This figure describes the graph decomposition grammar}
%   \label{fig:Syntax of graph decomposition trees}
% \end{figure}


% lemma: a unicycle is a cycle with trees hanging off of it

% lemma: unicycle traversal always produces a valid tree

% Unicycular graph traversal occurs in two directions. A vertex is chosen as an arbitrary
% starting point and its parents are traversed and marked as seen. Any vertex that
% has been seen twice must be on the cycle. We take the first one as the root of
% the unicycle and traverse its children to produce a tree that covers the unicycular graph.
% Any remaining single-parented vertices must be parts of other unicycular graphs and are
% unicycle-traversed until there are no more single-parented vertices left.



% TODO: Talk about Uuid Generation
% First, we set up \texttt{Uuid.ml}, a module to generate well-founded monotonically increasing unique IDs for all the parts of our code that require unique identifiers.This module has features that allow it to both generate and check for the well-foundedness of a given ID.. Our programmatically generated target language syntax, explained further in \autoref{sec: Parameterized Code Generation}, provides a \texttt{Lang.Constructor.t} corresponding to $k \in \K$ from \autoref{sec:Grove Formalism}.Further, we have a \texttt{Vertex.ml} module that \emph{wraps} a given \texttt{Lang.Constructor.t} with a UUID, such that, $\V = \K \times \U$. This also allows for the vertices to be ordered by their UUIDs. Internally, we represent edge-sources ($s = (v,p)$) as a record containing \texttt{Vertex.t}\textit{s} and \texttt{Lang.Position.t}\textit{s}, where the latter encodes the child positions and arities of the various constructors, in \texttt{Cursor.ml}. That is, a cursor (or a source) serves as a point of focus for graph actions and it references a position relative to some vertex. 

% Then we have edges corresponding to $\E = \U \times \S \times \V$, implemented in \texttt{Egde.ml} UUID wrapped \emph{source} from a source to a target. Edge states ($\Sigma$), are expressed as variant type with two variants \texttt{Created} and \texttt{Deleted} representing $\Plus$ and $\Minus$ respectively.Since we do not directly implement infinite mappings in \texttt{Graph}, $\bot$ does not have a corresponding variant. Further, trees are defined in \texttt{Tree.ml} to represent disjointed graph components. A tree is a vertex with a (possibly empty) set of children or is a reference to one. The children are a list of \emph{child}s, where a child is a tree paired with an edge UUID. We also have a helper \texttt{child\_map : (Lang.Position.t * (Uuid.Id.t $*$ t) list) list $->$ children Position\_map.t} that takes in pairs of target language positions and lists of pairs Uuid and trees to return a map of the children. Here, a position\_map is a structure that \emph{maps} \texttt{Lang.position.t}s to their appropriate child representations.

% Finally, the logic and infrastructure for a grove that a given graph decomposes into using the decomposition algorithm is implemented in \texttt{Grove.ml}.

% TODO: Tree.ml, Position_map.mk, Grove.ml, Graph_action.ml

% TODO: talk about code generation somewhere in here

% TODO: talk about:
% - decomp vertex set generation and unicycle traversal
% - how to choose which branch of expr' to take
% - lfp(ancestors') and min of it

% TODO: can we save any of this in a README or code comments?

% TODO: talk about performance: O(decomp) given "size of the graph" (probably linear in # edges)

% GRV -- how it is implemented, how it connects to the formalism, describe the graph,


\section{Related Work}%
\label{sec:Related Work}

The core components of a version control system are a patch language, a method for synthesizing patches from user actions, and an approach for merging patches. 
% Existing patch based systems such as Darcs~\cite{DBLP:conf/haskell/Roundy05} and Pijul achieve commutativity between patches in many cases, but Grove achieves total commutativity. 

% Grove stands out from existing systems in the precision of its patch language and the simplicity of its patch synthesis and merge processes, as well as its intrinsic guarantee of commutativity by employing a graph CmRDT as its repository data structure. 

% Patch languages vary in their native data structure and level of precision, with Grove's patch language being exceptionally fine-grained and molded to the language. Traditional patch synthesis algorithms reconstruct patches from version differences, while Grove uniquely synthesizes patches directly from edit actions. Traditional merge algorithms, such as those used by Git, employ operational transforms to 

% Traditional patch synthesis and merge algorithms, such as those used by Git, are complex and indirect, while Grove joins a family of new systems employing 

\subsection{Patch Languages}
There have been many different designs for patch languages and indeed many imperative data structures and their associated operations can be construed as patch languages in the most general sense. In the context of collaborative coding, patch languages can differ in the data structures they operate on (e.g. line-based text~\cite{DBLP:conf/haskell/Roundy05}, character-based text~\cite{DBLP:conf/sigmod/EllisG89,DBLP:journals/pacmhci/LittLKH22}, tree-structured data). 
They can also differ in how they identify locations within the data (e.g. by using numeric offsets~\cite{DBLP:conf/sigmod/EllisG89}, unique identifiers, or paths through a tree).
Finally, patch languages differ in which specific actions are supported explicitly. Insertion and deletion are common, while code relocation, copying, undo, and other operations are variously also included.

In this paper, our focus was on syntax trees with holes (i.e. program sketches~\cite{DBLP:conf/aplas/Solar-Lezama09, DBLP:conf/popl/OmarVHAH17}) and explicit conflicts, which we represented as directed graphs. We identify locations using unique IDs. We have a two-level patch language, with a low-level graph patch language supporting only edge insertion and deletion and a higher-level user edit action language focused on insertion, deletion, and relocation, with some additional narrative consideration of copying and undo (a fuller account of which we leave to future work).  Our user edit language therefore forms a structure editor calculus, inspired closely by recent work on the Hazelnut structure editor calculus (which did not support relocation)~\cite{DBLP:conf/popl/OmarVHAH17} and patch languages for other tree-based data structures~\cite{DBLP:conf/sigmod/ChawatheG97, DBLP:journals/tse/FluriWPG07,DBLP:conf/kbse/FalleriMBMM14,DBLP:conf/doceng/Lindholm04,DBLP:conf/fase/NguyenNPN10,DBLP:journals/scp/SchwagerlUW15}.

\subsection{Patch Synthesis}
The most common approach to patch synthesis is to use a differencing algorithm to compare two states, e.g. from the file system, to generate a patch. 
The classic \li{diff} algorithm~\cite{DiffAlgorithm}, for example, minimizes edit distance for a patch language involving line insertions and deletions.

In the setting of tree-based editing, there have been a number of tree differencing algorithms described in the literature \cite{DBLP:conf/esa/Klein98,DBLP:journals/tcs/Bille05,DBLP:journals/talg/DemaineMRW09,DBLP:journals/fuin/AratsuHK10,DBLP:conf/sigmod/ChawatheG97, DBLP:journals/tse/FluriWPG07,DBLP:conf/kbse/FalleriMBMM14,DBLP:conf/doceng/Lindholm04,DBLP:conf/fase/NguyenNPN10,DBLP:journals/scp/SchwagerlUW15}. 
As described in \autoref{sec:Introduction}, synthesizing insertions and deletions is well-understood, but synthesizing relocations is more complex and requires heuristics.

The approach we explore in this paper is far simpler: we directly translate from the log of user edit actions to graph patches, without needing a differencing algorithm at all. This is only possible with a structure editor integrated with the version control system, but the benefit of direct visibility into the edit action log is that we do not need heuristics to synthesize relocations.

\subsection{Merging}
\subsubsection{Operational Transforms}
The most common approach to merging concurrently developed patches is to deploy an operational transform~\cite{DBLP:conf/sigmod/EllisG89} whereby locations in a remote patch are modified based on the action of a local patch. Standard three-way merge algorithms in version control systems deploy this approach, as do real-time collaborative editors. There have been a number of papers studying the algebraic properties of merging patches in this style. 

For example, the Darcs~\cite{DBLP:conf/haskell/Roundy05} version control system, like Grove, represents repositories using sets of patches. Using operational transforms, Darcs achieves commutativity in many cases, but not between conflicting patches. The theory of Darcs defines and algebraically characterizes when operational transforms do satisfy the properties of associativity and commutativity. Recent work on homotopical patch theory \cite{DBLP:journals/jfp/AngiuliMLH16} has similarly developed an abstract algebraic framework for distinguishing sensible merges. 

\subsubsection{CRDTs}

The observation that commutativity is a particularly helpful property when dealing with concurrent systems, including version control systems, has led to the development of a number of data structures centered on commutativity. These are known as CRDTs, which stands variously for \emph{conflict-free}, \emph{convergent} (CvRDT), or \emph{commutative replicated datatypes} (CmRDT) depending on particular details about the operations and the state representation~\cite{shapiro2011conflict}.
Our approach draws directly from this line of work: the Grove patch language forms a CmRDT on directed graphs, from which we observe that we can derive a CmRDT for trees with explicit conflicts.

Other examples of CRDT-based editors include and Peritext~\cite{DBLP:journals/pacmhci/LittLKH22} and Zed~\cite{zed-blog}. These systems differ from Grove by targeting synchronous editing rather than version control and by operating on sequential text rather than tree structures.

There have been other recent efforts to develop CmRDTs for tree data structures~\cite{DBLP:conf/icdcs/PreguicaMSL09, Highly-Available}. However, the focus in this work has been on avoiding conflicts and cycles entirely by applying \emph{ad hoc} heuristics for conflict resolution at merge-time, e.g. using reported timestamps or favoring particular directions in the tree. Our approach instead embraces manual conflict resolution, as is common practice in software projects where arbitrarily losing code is not acceptable. 

% Zed is a Integrated Development Environment (IDE) that reasons text edits using CRDTs and differs in approach by using Lamport timestamps to order events across replicas. Peritext, on the other hand, is a character based CRDT editor but is limited to the realm of rich-text editing.

Pijul iterates on the patch based system of Darcs, obtaining commutativity between conflicting patches using a CRDT graph data structure~\cite{pijul-talk}. Pijul's graph data structure is very similar to that of Grove in its treatment of edges, but, unlike Grove, requires vertices to be created before they can be referred to. This imposes a dependency relation between patches, causing Pijul to fall short of the full commutativity enjoyed by Grove.
Pijul is also language agnostic and models only the linear structure of text. However, Pijul goes beyond this version of Grove by extending the data structure to support history and branches, and Grove may be similarly extended in future work.

% \url{https://www.waitingforcode.com/big-data-algorithms/conflict-free-replicated-data-types-flags-graphs-maps/read}



% We are using action based instead of tree diff

% We have to deal with merge

% We have to deal with edits from multiple people

% J. W. Hunt and M. D. McIlroy. 1976. An Algorithm for Differential File Comparison. Technical Report CSTR 41. Bell Laboratories, Murray Hill, NJ.

% Type-directed diffing of structured data \url{https://dl.acm.org/doi/10.1145/3122975.3122976}

%   Approximating Tree Edit Distance through String Edit Distance \url{https://dl.acm.org/doi/10.5555/3118232.3118518}

%   Meaningful change detection in structured data \url{https://dl.acm.org/doi/10.1145/253260.253266}

%   An optimal decomposition algorithm for tree edit distance \url{https://dl.acm.org/doi/10.5555/2394539.2394560}

%   Diff/TS: A Tool for Fine-Grained Structural Change Analysis \url{https://dl.acm.org/doi/10.1109/WCRE.2008.44}

% An efficient algorithm for type-safe structural diffing \url{https://dl.acm.org/doi/10.1145/3341717}

%   Precise Version Control of Trees with Line-Based Version Control Systems \url{https://dl.acm.org/doi/10.1007/978-3-662-54494-5_9}

%   A survey on tree edit distance and related problems \url{https://dl.acm.org/doi/10.1016/j.tcs.2004.12.030}

%   Cycle-aware minimization of acyclic deterministic finite-state automata \url{https://dl.acm.org/doi/10.1016/j.dam.2013.08.003}

%   Computing the Edit-Distance between Unrooted Ordered Trees \url{https://dl.acm.org/doi/10.5555/647908.740125}

%   Type-safe diff for families of datatypes \url{https://dl.acm.org/doi/10.1145/1596614.1596624}

%   A Categorical Theory of Patches \url{https://dl.acm.org/doi/10.1016/j.entcs.2013.09.018}

%   Type-directed diffing of structured data \url{https://dl.acm.org/doi/10.1145/3122975.3122976}

%   The Semantics of Version Control \url{https://dl.acm.org/doi/10.1145/2661136.2661137}

%   The Tree-to-Tree Correction Problem \url{https://dl.acm.org/doi/10.1145/322139.322143}

%   Generic Diff3 for algebraic datatypes \url{https://dl.acm.org/doi/10.1145/2976022.2976026}

% \subsection{Version Control}

% Git \url{https://git-scm.com/}

% Darcs \url{https://darcs.net/}

%   Darcs: distributed version management in haskell \url{https://dl.acm.org/doi/10.1145/1088348.1088349}

% Hg? \url{https://www.mercurial-scm.org/}

% SVN \url{https://subversion.apache.org/}

% Pijul and (Anu is a rewrite of Pijul and seems to have been subsumed into Pijul)

%   \url{https://pijul.org/}
%   \url{https://pijul.org/manual/theory.html}

%   \url{https://tahoe-lafs.org/~zooko/badmerge/simple.html}

% \subsection{Collaborative Editing}

% Collaborative Structure Editing

% SmallTalk collaboration with images

% TouchDevelop papers

% Lots of list-of-chars or list-of-list-of-chars (we ignore these except to discuss them here)

% \subsection{CRDTs}
% (Are we a known CRDT?)

% List of CRTD papers: \url{https://crdt.tech/papers.html}

% Bottom = Tombstone

% \url{https://www.waitingforcode.com/big-data-algorithms/conflict-free-replicated-data-types-flags-graphs-maps/read}
%  - Add-Remove Partial Order data type
%  - 2P2P-Sets
%  - Replicated Growable Array

% \url{https://github.com/PsychoLlama/graph-crdt}
%  - Graph CRDT
%  - Uses a LWW-E-Set

% \url{https://martin.kleppmann.com/2020/07/06/crdt-hard-parts-hydra.html} (overview talk)
%  We don't have interleaving problems because
%    - everything is relative to a specific ID not a position
%    - we don't try to auto resolve
%    - we are tree not list
%    Part three: moving sub tree
%    - Last parent writter wins (prevents cycles)
%      Equivalent to us if we filter multiparent edges, different for cycles, no way to delete?

% A commutative replicated data type for cooperative editing
%   \url{https://hal.inria.fr/inria-00445975/document}
%   Describes TreeDoc but this uses a document model that is a list (tree is just how it is implemented)

% Logoot : a Scalable Optimistic Replication Algorithm for Collaborative Editing on P2P Networks
%   \url{http://pagesperso.lina.univ-nantes.fr/~molli-p/pmwiki/uploads/Main/weiss09.pdf}

% Specification and Complexity of Collaborative Text Editing
%   \url{https://www.microsoft.com/en-us/research/wp-content/uploads/2016/07/podc16-complete.pdf}

% LSEQ: an Adaptive Structure for Sequences in Distributed Collaborative Editing,
%   \url{https://hal.archives-ouvertes.fr/file/index/docid/921633/filename/fp025-nedelec.pdf}

% Data consistency for P2P collaborative editing
%   \url{https://hal.archives-ouvertes.fr/file/index/docid/108523/filename/OsterCSCW06.pdf}

% Interleaving anomalies in collaborative text editors
%   \url{https://martin.kleppmann.com/papers/interleaving-papoc19.pdf}

% Moving Elements in List CRDTs
%   \url{https://martin.kleppmann.com/papers/list-move-papoc20.pdf}

% A highly-available move operation for replicated trees and distributed filesystems
%   \url{https://martin.kleppmann.com/papers/move-op.pdf}

% ...
%  commutative replicated data types CmRDT
%  convergent replicated data types, or CvRDTs
%  Delta state CRDTs[12][13] (or simply Delta CRDTs

% The Causal Graph CRDT for Complex Document Structure
%   \url{https://dl.acm.org/doi/10.1145/3209280.3229110}

% \url{https://en.wikipedia.org/wiki/Conflict-free_replicated_data_type}

% G-Set
% PN-Set
% 2P-Set
% LWW-element-Set (Last-Write-Wins)
% OR-Set
% MV-Register: Multi-Value Register
% U-Set (this is what we are for edges, not OR-set due to duplicates)
% Add-Remove Partial Order data type
%   2P-Set for vertices, and a G-Set for edges.

% A comprehensive study of Convergent and Commutative Replicated Data Types (2011)
%   \url{https://hal.inria.fr/inria-00555588/document}

% CRDTs: Consistency without concurrency control
%   \url{https://arxiv.org/abs/0907.0929}

% https://medium.com/@amberovsky/crdt-conflict-free-replicated-data-types-b4bfc8459d26
%  removeVertex() has priority, all incident edges are removed
%  addEdge() has priority, all removed vertices are re-added
%  Delay removeVertex() execution till all concurrent removeVertex() are executed.
% First one is 2P2P-Set

% https://crdt.tech/papers.html

% Mahsa Najafzadeh, Marc Shapiro, and Patrick Eugster. Co-design and verification of an available file system. In 19th International Conference on Verification, Model Checking, and Abstract Interpretation, VMCAI 2018, pages 358--381. Springer LNCS volume 10747, January 2018. [ bib | DOI | .pdf ]
%   \url{http://dx.doi.org/10.1007/978-3-319-73721-8_17}
%   https://pages.lip6.fr/Marc.Shapiro/papers/VMCAI-2018-filesys.pdf

% Martin Kleppmann and Alastair R Beresford. A conflict-free replicated JSON datatype. IEEE Transactions on Parallel and Distributed Systems, 28(10):2733--2746, April 2017. [ bib | DOI | arXiv ]
%   http://dx.doi.org/10.1109/TPDS.2017.2697382
%   http://arxiv.org/abs/1608.03960

% Vinh Tao, Marc Shapiro, and Vianney Rancurel. Merging semantics for conflict updates in geo-distributed file systems. In 8th ACM International Systems and Storage Conference, SYSTOR 2015. ACM, May 2015. [ bib | DOI | .pdf ]
%   http://dx.doi.org/10.1145/2757667.2757683
%   https://pages.lip6.fr/Marc.Shapiro/papers/geodistr-FS-Systor-2015.pdf

% Mehdi Ahmed-Nacer, Stéphane Martin, and Pascal Urso. File system on CRDT. Research Report RR-8027, INRIA, July 2012. [ bib | arXiv | http ]
%   http://arxiv.org/abs/1207.5990
%   https://hal.inria.fr/hal-00720681/

% Stéphane Martin, Pascal Urso, and Stéphane Weiss. Scalable XML collaborative editing with undo. In On the Move to Meaningful Internet Systems (OTM), pages 507--514. Springer LNCS volume 6426, October 2010. [ bib | DOI | arXiv ]
%   \url{http://dx.doi.org/10.1007/978-3-642-16934-2_37}
%   http://arxiv.org/abs/1010.3615



% 2P-Sets
% Anomaly: Creating a lone deleted vertex requires create and delete of otherwise unneeded edge

% Operational Transforms

% Etherpad

% Live Share

% \subsection{Synchronization}

% Unison
% \url{https://www.cis.upenn.edu/~bcpierce/unison/}
% \url{https://www.cis.upenn.edu/%7Ebcpierce/papers/index.shtml#File%20Synchronization}

% \subsection{Homotopical Patch Theory}

% Homotopical Patch Theory: \url{https://www.cambridge.org/core/journals/journal-of-functional-programming/article/homotopical-patch-theory/42AD8BB8A91688BCAC16FD4D6A2C3FE7}
% Homotopical patch theory: \url{https://dl.acm.org/doi/10.1145/2628136.2628158}

% - diff
%     - A file comparison program 1985
%     - myers an O(ND) difference algorithm and its variations
% - merging
%     - a state-of-the-art survey on software merging
% - homotopy patch theory
%     - 7 10 15 16 28 31
%     - general algebraic framework for discussing patch languages
% - darcs
%     - unordered series of patches (but dependencies between them)
%     - we don't focus on inversion in this paper
%     - some properties of darcs patch theory
%         - http://urchin.earth.li/darcs/ganesh/darcs-patch-theory/theory/formal.pdf
%     - A formalization of darcs patch theory using inverse
% semigroups
%         - 2, 3, 5, 11, 18
%         - 15 = selective undo https://ww3.math.ucla.edu/camreport/cam09-83.pdf
%         - patch commutation requires an operational transform! https://ww3.math.ucla.edu/camreport/cam09-83.pdf
%         - (we also define it as a semigroup)
% - dagit - type correct changes
%     - Type-correct changes—a safe approach to version control
% implementation
%     - MS thesis
%     - Darcs: Distributed version management in haskell.
%     - "patches that modify different parts of the code are considered, by default, independent"
%     - patience diff
% - A categorical theory of patches
% - OT = "Concurrency control in groupware systems"
%     - Formal design and
% verification of operational transformation algorithms for copies convergence
% - The Semantics of Version Control
%     - https://dl.acm.org/doi/abs/10.1145/2661136.2661137?casa_token=hKqrcG9g7_UAAAAA:vN0l9AJhdnjc_2wnAIBl3TumxJBgy_4JYYSsG7522fYnbdfzh7Mpc0PIwUKcsX2VbwA0M67Cy_Y
%     - cite for this is a version control system
%     - CACM article by o'sullivan
%         - https://dl.acm.org/doi/pdf/10.1145/1562164.1562183
%         - "We are not by any means near the end
% of the road in the evolution of revisioncontrol systems. The field has received
% only fitful attention from academia.
% Much work could be done on its formal foundations, which could lead to
% more powerful and safer ways for developers to work together. Alas, I know
% of only one notable publication on the
% topic in the past decade.1
%  As a simple
% example, when merging potentially
% conflicting changes, almost everybody
% uses either three-way merging, which
% is decades old, or unpublished ad hoc
% approaches in which there is little reason to be confident."
% - touchdevelop
%     - https://dl.acm.org/doi/pdf/10.1145/2846661.2846672
% - tree CRDTs
%     - A Highly-Available Move Operation for Replicated Trees
%         - "last writer wins" conflict resolution not explicit conflict representation
%         - using timestamps
%         - Martin et al 28, 29 CRDT for XML data + Kleppman and Beresford CRDT for JSON but no moves
%         - [31] outline approaches to handling conflicts on trees but no algor
%         - treedoc 34 also no move
%         - 37, 38, 39 Ts for trees
%         - 40 defines OT with a move operation but need a total order broadcast (not commutative)
%     - Maram
%         - The price to pay is that some move operations “lose”, i.e., have no
% effect; achieving the same end result as previous correct approaches but at a lower cost
%         - https://arxiv.org/pdf/2103.04828
%     - UDR tree -- ordering
%     - Najafzadeh et al Subtree_1 requires locks
%     - Tao et al [17]
%         - . Merging semantics for conflict updates in
% geo-distributed file systems
%         - Can lead to duplication
%         - 
% - tree diff based approaches
%     - https://dl.acm.org/doi/10.1007/978-3-662-54494-5_9
%         - 14, 15, 21, 24, 25
%     - Change distilling: tree differencing
% for fine-grained source code change extraction.
%     - Fine-grained
% and accurate source code differencing.
%     - Tree-based Version Control in Envision
%     - A three-way merge for XML documents
%     - Operation-Based, Fine-Grained Version Control Model for Tree-Based Representation
%     - https://www.sciencedirect.com/science/article/pii/S0167642315000532
% - do not guess!
%     - https://dl.acm.org/doi/pdf/10.1145/2508075.2508092 (brief note)
%         - Cedalion: a language for language-oriented programming
% - variability aware execution
%     - https://dl.acm.org/doi/abs/10.1145/2786805.2803208
% - text CRDTs?


\section{Discussion and Conclusion}%
\label{sec:Discussion and Conclusion}
\begin{quote}
    \textit{``The fact that commutation can fail [in Darcs] makes a huge difference in the whole patch formalism. It may be possible to create a formalism in which commutation always succeeds, with the result of what would otherwise be a commutation that fails being something like a virtual particle ... and it may be that such a formalism would allow strict mathematical proofs ... However, I’m not sure how you’d deal with a request to delete a file that has not yet been created, for example. Obviously you’d need to create some kind of antifile, which would annihilate with the file when that file finally got created ...''} 
    
    -- David Roundy, Theory of patches~\cite{old-darcs-manual}
\end{quote}



This paper proposes a radically simpler, albeit practically ambitious, rearchitecture of collaborative editing. Our contributions together result in a typed collaborative structure calculus called Grove 
where, uniquely, all edits, including code relocations that stymie existing approaches, commute and where there are no semantic gaps: all possible editor states, including editor states with various kinds of unresolved conflicts, are semantically meaningful. 

This paper focuses on the core theoretical underpinnings of this approach,
developing mechanized metatheory for both the patch language and the type system. 
A number of research problems on algorithmic, networking, and user interface aspects of the problem open up given these foundations. For example, we point out several situations where presenting non-conflicted but heuristically attention-worthy merges may be worthwhile, and we leave to future work the user experience design of this process. We intend to use the Grove workbench to integrate these efforts into the Hazel programming environment, though its editor component has been evolving so rapidly as to prevent experimentation in this direction so far.

Although our focus was on tree editing, some aspects of a program are more naturally linearly structured, e.g. string literals. We also leave to future work the problem of combining existing work on sequence CRDTs~\cite{ahmed2011evaluating,kleppmann2020moving} with our work on tree/graph CRDTs.

% In practice, it may be helpful to garbage collect orphaned vertices once there is consensus across collaborators that the deletion is permanent, but we do not consider this consensus protocol formally in this paper. In an open-ended collaboration scenario (where the set of collaborators is not known, e.g. on GitHub), we simply retain all vertices.

% TODO: talk about linear sequences in code


% TODO: leaves

% TODO: our model supports treating these as a cons-list of characters

% TODO: (Place somewhere) This move semantics gives us a richer structure than
% when treating code as a list of lines.  We exploit
% this in Section~REF:TODO in order merge edits that involve moving code.

% TODO: Traditional diff: no relation (and indent might change)

% TODO: memory usage

% TODO: cache eviction algorithm

% TODO: Finally, note that through we present several complex scenarios, this is merely for presentation.
% In practice, these complex scenarios occur less frequently than portrayed here.

% \subsection{Variable names, strings, and numbers}%
% \label{sub:Variable names, strings, and numbers}

% NOTE: we could implement each digit as a separate characters

% GUI for string conflicts: use popups


%% Acknowledgments
\begin{acks}                            %% acks environment is optional
  %% contents suppressed with 'anonymous'
  %% Commands \grantsponsor{<sponsorID>}{<name>}{<url>} and
  %% \grantnum[<url>]{<sponsorID>}{<number>} should be used to
  %% acknowledge financial support and will be used by metadata
  %% extraction tools.
  This material is based upon work supported by the
  \grantsponsor{GS100000001}{National Science
    Foundation}{http://dx.doi.org/10.13039/100000001} under Grant
  No.~\grantnum{GS100000001}{nnnnnnn} and Grant
  No.~\grantnum{GS100000001}{mmmmmmm}.  Any opinions, findings, and
  conclusions or recommendations expressed in this material are those
  of the author and do not necessarily reflect the views of the
  National Science Foundation.
\end{acks}

%% Appendix
\appendix

%% Appendixes that will be published  go **BEFORE** the bibliography and count towards the page count

%% Bibliography
\bibliography{grove-paper}

%% Temporary appendixes that will not be published go **AFTER** the bibliography and do not count towards the page count

\clearpage % Put these appendixes on separate pages so we can easily remove them from the document.  Also ensure all figures have been placed.


\section{Supplemental Material: Complete Graph Sequences for All Figures}%
\label{apx:Supplemental Material: Complete Graph Sequences for All Figures}

TODO: put all intermediate states for all graphs along with the edge actions

\section{Formalism}

%%%%%%%%%%%%%%%%%%%%%%%%%%%%%%%%%%%%%%%%%%%%%%%%%%%%%%%%%%%%%%%%%%%%%%%%%%%%%%%%

\subsection{Terms}

\figureTermSyntaxContent

\noindent $\boxed{\constructor{t} = k}$
%
\begin{align*}
  \constructor{e} &= \econstructor{e} \\
  \constructor{p} &= \pconstructor{p} \\
  \constructor{\tau} &= \tconstructor{\tau}
\end{align*}

\noindent $\boxed{\econstructor{e} = k}$
%
\begin{align*}
  \econstructor{\eVar{G}{x}} &= \ExpVar(x) \\
  \econstructor{\eFun{G}{q}{\tau}{e}} &= \ExpLam \\
  \econstructor{\eApp{G}{e_1}{e_2}} &= \ExpApp \\
  \econstructor{\eNum{G}{n}} &= \ExpNum(n) \\
  \econstructor{\ePlus{G}{e_1}{e_2}} &= \ExpPlus \\
  \econstructor{\eTimes{G}{e_1}{e_2}} &= \ExpTimes
\end{align*}

\noindent $\boxed{\pconstructor{q} = k}$
%
\begin{align*}
  \pconstructor{\pVar{G}{x}} &= \PatVar(x)
\end{align*}

\noindent $\boxed{\tconstructor{\tau} = k}$
%
\begin{align*}
  \tconstructor{\tArrow{G}{\tau_1}{\tau_2}} &= \TypArrow \\
  \tconstructor{\tNum{G}} &= \TypNum
\end{align*}

%%%%%%%%%%%%%%%%%%%%%%%%%%%%%%%%%%%%%%%%%%%%%%%%%%%%%%%%%%%%%%%%%%%%%%%%%%%%%%%%

\subsection{Graphs}

Let $t$ denote a term and $\Set$ a set of terms.

\begin{definition}
  A \emph{graph} $G : \E \rightarrow \Sigma$ is a function from edges to edge states,
  where $\E = \U \times \V \times \P \times \V$,
  unique IDs are drawn from some suitable set $\U$ equipped with a total ordering $\leq$,
  vertices are drawn from $\V$,
  positions are drawn from $\P$,
  and edge states are drawn from $\Sigma$.
\end{definition}

Let $G_t$, called the \emph{in-graph} of term $t$,
denote the outermost graph produced by $\ingraphOp$ in the construction of term $t$,
when such a graph exists.

% % % % % % % % % % % % % % % % % % % % % % % % % % % % % % % % % % % % % % % % 

\subsubsection{Well-sortedness}

\figureArityContent

\begin{definition}
  A graph $G$ is well sorted if $\e$ is well sorted
  for all edges $\e$ such that $G(\e) \in \{\Plus, \Minus\}$.
\end{definition}

\begin{definition}
  An edge $\e = (u, (u_1, k_1), p, (u_2, k_2))$ is well sorted
  if $(p, \sort(k_2)) \in \arity(k_1)$.
\end{definition}

\begin{definition}
  A grove $\Grove = (\Set[NP], \Set[MP], \Set[U])$ is well sorted if all of the following hold:
  \begin{enumerate}
    \item
      The terms of $\Set[NP], \Set[MP],$ and $\Set[U]$ are well sorted.
      % ($\Set[NP]$, $\Set[MP]$, and $\Set[U]$ contain only well sorted terms.)
    \item
      Either $\Set[NP] = \varnothing$
      or there exists an $e \in \Set[NP]$ such that
      $\SizeOf{\SetOf{\e{=}(\rootVertex, \Root, v) \SuchThat{G_e(\e) = \Plus}}} = 1$ and
      $\SizeOf{\SetOf{\e'{=}(\rootVertex, \Root, v) \neq e \SuchThat{G_{e'} = \Plus}}} = 0$.
      % (Exactly one of the terms in $\Set[NP]$, if any,
      % corresponds to an edge originating from $\rootVertex$.)
    \item
      For all $t \in \Set[NP]$,
        $\SizeOf{\SetOf{\e \SuchThat{G_t(\e) = \Plus}}} = 0$.
      % (The terms in $\Set[NP]$ correspond to vertices with no parents.)
    \item
      For all $t \in \Set[MP]$,
        $\SizeOf{\SetOf{\e \SuchThat{G_t(\e) = \Plus}}} > 1$ and
        $\SizeOf{\SetOf{\e \SuchThat{G_t(\e) = \Minus}}} = 0$.
      % (The terms in $\Set[MP]$ correspond to vertices with multiple parents.)
    \item
      For all $t \in \Set[U]$,
      and all $\e{=}(v, p, v')$ such that $G_t(\e) \in \SetOf{\Plus, \Minus}$,
        $\SizeOf{\SetOf{\e \SuchThat{G_t(\e) = \Plus}}} = 1$ and
        $\SizeOf{\SetOf{\e \SuchThat{G_t(\e) = \Minus}}} = 0$ and
        $v' = \min{\ancestors{v'}}$.
      % (The terms in $\Set[U]$ correspond to vertices that are unicycle roots.)
  \end{enumerate}
\end{definition}

\begin{definition}
  A term $t$ is well sorted if one of the following holds:
  \begin{enumerate}
    \item $t = \hole$.
    \item $t = \conflictHole{t_1, \ldots, t_n}$ and $t_1, \ldots, t_n$ are well sorted.
    \item $t = \multiVertex{\e}$ and all of the following hold:
      \begin{enumerate}
        \item $\e$ is well sorted.
        \item
          $\SizeOf{\SetOf{\e \SuchThat{G_t(\e) = \Plus}}} > 1$.
          % all multiparent references refer to a multiparent root
          % (TODO: do we need this? be careful with wording)
      \end{enumerate}
    \item $t = \cycleVertex{(v, p, v')}$ and all of the following hold:
      \begin{enumerate}
        \item $(v, p, v')$ is well sorted.
        \item $v' = \min{\ancestors{v'}}$
        \item
          For all $\e{=}(v_{\e}, p_{\e}, v_{\e}')$
          such that $v_{\e}' \in \ancestors{v'}$,
          $\SizeOf{\parents{v_{\e}'}} = 1$.
          % all unicycle references refer to a unicycle root
          % (TODO: do we need this? be careful with wording.
          % Could try to reconstruct the path from root (referenced) to ref (referencer).)
      \end{enumerate}
    \item Otherwise, $G_t$ exists and all of the following hold:
    \begin{enumerate}
      \item
      $G_t$ is well sorted.
      % all incoming edges are well sorted
      \item
      $\SizeOf{\SetOf{\e \SuchThat{G_t(\e) \in \SetOf{\Plus, \Minus}}}} \geq 1$.
      % The in-graph of t is not empty.
    \item
      For all $\e{=}(v, p, v')$ such that $G_t(\e) \in \SetOf{\Plus, \Minus}$,
      $v' = (u, \constructor{t})$ fixed.
      % All term constructors match their corresponding target vertex constructors.
      % The ids of the target vertices of all incoming edges are the same.
    \end{enumerate}
  \end{enumerate}
\end{definition}

%%%%%%%%%%%%%%%%%%%%%%%%%%%%%%%%%%%%%%%%%%%%%%%%%%%%%%%%%%%%%%%%%%%%%%%%%%%%%%%%

\subsection{Decomposition and Recomposition}

\begin{theorem}
  For any well sorted graph $G$,
  there exists a (well sorted) grove $\Grove$
  such that $\decomp{G} = \Grove$.
\end{theorem}

Proof: provide a witness that demonstrates the conclusion.

% \begin{theorem}
%   Let $G$ be a well sorted graph.
%   If $\decomp{G} = \Grove$ then $\vertexes{G} = \vertexes{\Grove}$.
% \end{theorem}

% \begin{theorem}
%   Let $G$ be a well sorted graph.
%   If $\recomp{\Grove} = G$ then $\vertexes{\Grove} = \vertexes{G}$.
% \end{theorem}

\figureDecompositionDefHelpersContent

% % % % % % % % % % % % % % % % % % % % % % % % % % % % % % % % % % % % % % % % 

\subsubsection{Decomposition}\hspace*{\fill} \\

\noindent $\boxed{\decomp{G} = \Grove}$
%
\figureDecompositionDefDecomp

\noindent $\boxed{\edecomp{\e} = e}$
%
\figureDecompositionDefEdecomp

\noindent $\boxed{\pdecomp{\e} = q}$
%
\figureDecompositionDefPdecomp

\noindent $\boxed{\tdecomp{\e} = \tau}$
%
\figureDecompositionDefTdecomp

\noindent $\boxed{\edecompPrime{\e}{p} = e}$
%
\figureDecompositionDefEdecompPrime

\noindent $\boxed{\pdecompPrime{\e}{p} = q}$
%
\figureDecompositionDefPdecompPrime

\noindent $\boxed{\tdecompPrime{\e}{p} = \tau}$
%
\figureDecompositionDefTdecompPrime%

% % % % % % % % % % % % % % % % % % % % % % % % % % % % % % % % % % % % % % % % 

\subsubsection{Recomposition}\hspace*{\fill} \\

\begin{theorem}
  If $G$ is a well sorted graph such that $\decomp{G} = \Grove$,
  then $\recomp{\Grove} = G$.
\end{theorem}

\noindent $\boxed{\recomp{\Grove} = G}$
%
\begin{align*}
  \recomp{(\Set[NP], \Set[MP], \Set[U])} &= \bigcup_{e \in \Set[NP] \cup \Set[MP] \cup \Set[U]} \erecomp{e}
\end{align*}

\noindent $\boxed{\erecomp{e} = G}$
%
\begin{align*}
  \erecomp{\eVar{G}{x}} &= G
  \\
  \erecomp{\eFun{G}{q}{\tau}{e}}
    &= G \cup \precomp{q} \cup \trecomp{\tau} \cup \erecomp{e}
  \\
  \erecomp{\eApp{G}{e_1}{e_2}}
    &= G \cup \erecomp{e_1} \cup \erecomp{e_2}
  \\
  \erecomp{\eNum{G}{n}} &= G
  \\
  \erecomp{\ePlus{G}{e_1}{e_2}}
    &= G \cup \erecomp{e_1} \cup \erecomp{e_2}
  \\
  \erecomp{\eTimes{G}{e_1}{e_2}}
    &= G \cup \erecomp{e_1} \cup \erecomp{e_2}
  \\
  \erecomp{\conflictHole{e_1, \cdots, e_n}}
  &= \bigcup_{i=1}^n \erecomp{e_i}
  \\
  \erecomp{\multiVertex{\e}} &= \SetOf{\e \mapsto \Plus}
  \\
  \erecomp{\cycleVertex{\e}} &= \SetOf{\e \mapsto \Plus}
  \\
  \erecomp{\hole} &= \SetOf{}
\end{align*}

\noindent $\boxed{\precomp{q} = G}$
%
\begin{align*}
  \precomp{\pVar{G}{x}} &= G
  \\
  \precomp{\conflictHole{q_1, \cdots, q_n}} &= \bigcup_{i=1}^n \precomp{q_i}
  \\
  \precomp{\multiVertex{\e}} &= \SetOf{\e \mapsto \Plus}
  \\
  \precomp{\cycleVertex{\e}} &= \SetOf{\e \mapsto \Plus}
  \\
  \precomp{\hole} &= \SetOf{}
\end{align*}

\noindent $\boxed{\trecomp{\tau} = G}$
%
\begin{align*}
  \trecomp{\tArrow{G}{\tau_1}{\tau_2}}
    &= G \cup \trecomp{\tau_1} \cup \trecomp{\tau_2}
  \\
  \trecomp{\tNum{G}} &= G
  \\
  \trecomp{\conflictHole{\tau_1, \cdots, \tau_n}} &= \bigcup_{i=1}^n \trecomp{\tau_i}
  \\
  \trecomp{\multiVertex{\e}} &= \SetOf{\e \mapsto \Plus}
  \\
  \trecomp{\cycleVertex{\e}} &= \SetOf{\e \mapsto \Plus}
  \\
  \trecomp{\hole} &= \SetOf{}
\end{align*}

%%%%%%%%%%%%%%%%%%%%%%%%%%%%%%%%%%%%%%%%%%%%%%%%%%%%%%%%%%%%%%%%%%%%%%%%%%%%%%%%

\subsection{Cursors}

% % % % % % % % % % % % % % % % % % % % % % % % % % % % % % % % % % % % % % % % 

\subsubsection{Zippered Terms}

\[
  \arraycolsep=0pt
  \begin{array}{lrlll}
    \hat{e} & {}\in ZExp & {}::={} &
      \cursor{e}
      \mid \eFun{G}{\hat{q}}{\tau}{e}
      \mid \eFun{G}{q}{\hat{\tau}}{e}
      \mid \eFun{G}{q}{\tau}{\hat{e}}
      \mid \eApp{G}{\hat{e}}{e}
      \mid \eApp{G}{e}{\hat{e}}
      \mid \ePlus{G}{\hat{e}}{e}
      \mid \ePlus{G}{e}{\hat{e}}
      \mid \eTimes{G}{\hat{e}}{e}
      \mid \eTimes{G}{e}{\hat{e}}
    \\
    \hat{\tau} & {}\in ZTyp & {}::={} &
      \cursor{\tau}
      \mid \tArrow{G}{\hat{\tau}}{\tau}
      \mid \tArrow{G}{\tau}{\hat{\tau}}
    \\
    \hat{q} & {}\in ZPat & {}::={} &
      \cursor{q}
    \\
  \end{array}
\]

% % % % % % % % % % % % % % % % % % % % % % % % % % % % % % % % % % % % % % % % 

\subsubsection{Zippered Groves}

\begin{gather*}
  \arraycolsep=0pt
  \begin{array}{lll}
    \hat{\gamma} & {}::={} &
      (\hat{NP}, MP, U)
      \mid (NP, \hat{MP}, U)
      \mid (NP, MP, \hat{U})
  \end{array} \\
  \hat{NP}, \hat{MP}, \hat{U} = \SetOf{e_1, \ldots, e_{k-1}, \hat{e_k}, e_{k+1}, \ldots, e_n} \\
  \qquad 1 \leq k \leq n
\end{gather*}

(Exactly one of $NP, MP, U$ may contain a single zippered term.)

% % % % % % % % % % % % % % % % % % % % % % % % % % % % % % % % % % % % % % % % 

\subsubsection{Cursor Erasure}\hspace*{\fill} \\

\noindent $\boxed{\erase{\hat{\gamma}} = \gamma}$
%
\begin{align*}
  \erase{(\hat{NP}, MP, U)} &= (\erase{\hat{NP}}, MP, U) \\
  \erase{(NP, \hat{MP}, U)} &= (NP, \erase{\hat{MP}}, U) \\
  \erase{(NP, MP, \hat{U})} &= (NP, MP, \erase{\hat{U}})
\end{align*}

\noindent $\boxed{\erase{\left(Exp \cup ZExp\right)} = Exp}$
%
\begin{align*}
  \erase{\SetOf{e_1, \ldots, e_{k-1}, \hat{e_k}, e_{k+1}, \ldots, e_n}} &= \SetOf{e_1, \ldots, e_{k-1}, \erase{\hat{e_k}}, e_{k+1}, \ldots, e_n}
\end{align*}

\noindent $\boxed{\erase{\hat{e}} = e}$
%
\begin{align*}
  \erase{\cursor{e}} &= e \\
  \erase{\left(\eFun{G}{\hat{q}}{\tau}{e}\right)} &= \eFun{G}{\erase{\hat{q}}}{\tau}{e} \\
  \erase{\left(\eFun{G}{q}{\hat{\tau}}{e}\right)} &= \eFun{G}{q}{\erase{\hat{\tau}}}{e} \\
  \erase{\left(\eFun{G}{q}{\tau}{\hat{e}}\right)} &= \eFun{G}{q}{\tau}{\erase{\hat{e}}} \\
  \erase{\eApp{G}{\hat{e_1}}{e_2}} &= \eApp{G}{\erase{\hat{e_1}}}{e_2} \\
  \erase{\eApp{G}{e_1}{\hat{e_2}}} &= \eApp{G}{e_1}{\erase{\hat{e_2}}} \\
  \erase{\ePlus{G}{\hat{e_1}}{e_2}} &= \ePlus{G}{\erase{\hat{e_1}}}{e_2} \\
  \erase{\ePlus{G}{e_1}{\hat{e_2}}} &= \ePlus{G}{e_1}{\erase{\hat{e_2}}} \\
  \erase{\eTimes{G}{\hat{e_1}}{e_2}} &= \eTimes{G}{\erase{\hat{e_1}}}{e_2} \\
  \erase{\eTimes{G}{e_1}{\hat{e_2}}} &= \eTimes{G}{e_1}{\erase{\hat{e_2}}}
\end{align*}

\noindent $\boxed{\erase{\hat{\tau}} = \tau}$
%
\begin{align*}
  \erase{\cursor{\tau}} &= \tau \\
  \erase{\left(\tArrow{G}{\hat{\tau_1}}{\tau_2}\right)} &= \tArrow{G}{\erase{\hat{\tau_1}}}{\tau_2} \\
  \erase{\left(\tArrow{G}{\tau_1}{\hat{\tau_2}}\right)} &= \tArrow{G}{\tau_1}{\erase{\hat{\tau_2}}}
\end{align*}

\noindent $\boxed{\erase{\hat{q}} = q}$
%
\begin{align*}
  \erase{\cursor{q}} &= q
\end{align*}

%%%%%%%%%%%%%%%%%%%%%%%%%%%%%%%%%%%%%%%%%%%%%%%%%%%%%%%%%%%%%%%%%%%%%%%%%%%%%%%%

\subsection{User Actions}

\[
  \arraycolsep=0pt
  \begin{array}{llll}
    \alpha & {}::={} &
      \Construct{k}
      \mid \Delete
      \mid \Reposition{v, p}
    \\
  \end{array}
\]

% \noindent $\boxed{\hat{e} \aArrow{\alpha} \hat{e}}}$
% %
% \begin{align*}
%   \
% \end{align*}

% Cases:
% \begin{itemize}
%   \item construct @ hole (create current edge)
%   \item construct @ conflict (create new edge into conflict ??)
%   \item construct @ non-hole (wrap current edge)
%   \item delete @ hole (no-op)
%   \item delete @ conflict (destroy all edges ??)
%   \item delete @ non-hole (destroy current edge)
%   \item reposition @ hole (no-op)
%   \item reposition @ conflict (???)
%   \item reposition @ non-hole (destroy current edge and create new edge with same destination vertex)
% \end{itemize}


\end{document}
