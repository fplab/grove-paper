% !TEX program = lualatex

%% For double-blind review submission, w/o CCS and ACM Reference (max submission space)
%\documentclass[acmsmall,review,anonymous,nonacm]{acmart}\settopmatter{printfolios=true,printccs=false,printacmref=false}
\documentclass[acmsmall,10pt,review,anonymous]{acmart}\settopmatter{printfolios=true,printccs=false,printacmref=false}
%% For double-blind review submission, w/ CCS and ACM Reference
%\documentclass[acmsmall,review,anonymous]{acmart}\settopmatter{printfolios=true}
%% For single-blind review submission, w/o CCS and ACM Reference (max submission space)
%\documentclass[acmsmall,review]{acmart}\settopmatter{printfolios=true,printccs=false,printacmref=false}
%% For single-blind review submission, w/ CCS and ACM Reference
%\documentclass[acmsmall,review]{acmart}\settopmatter{printfolios=true}
%% For final camera-ready submission, w/ required CCS and ACM Reference
%\documentclass[acmsmall]{acmart}\settopmatter{}


%% Journal information
%% Supplied to authors by publisher for camera-ready submission;
%% use defaults for review submission.
\acmJournal{PACMPL}
\acmVolume{1}
\acmNumber{POPL} % CONF = POPL or ICFP or OOPSLA
\acmArticle{1}
\acmYear{2021}
\acmMonth{1}
\acmDOI{} % \acmDOI{10.1145/nnnnnnn.nnnnnnn}
\startPage{1}

%% Copyright information
%% Supplied to authors (based on authors' rights management selection;
%% see authors.acm.org) by publisher for camera-ready submission;
%% use 'none' for review submission.
\setcopyright{none}
%\setcopyright{acmcopyright}
%\setcopyright{acmlicensed}
%\setcopyright{rightsretained}
%\copyrightyear{2018}           %% If different from \acmYear

\bibliographystyle{ACM-Reference-Format}
%\citestyle{acmauthoryear}
\citestyle{acmnumeric}

% TODO: consolidate all the macros into here
% Generic Notations
\newcommand{\EmptySet}{\emptyset}
\newcommand{\Set}[1]{\overline{#1}}
\newcommand{\Setof}[1]{\left\{#1\right\}}
\newcommand{\suchthat}{\mid}
\newcommand{\Pow}[1]{\mathscr{P}\mathopen{}\left(#1\right)\mathclose{}}

% judgments
\newcommand{\proves}{\vdash}
\DeclareMathOperator{\sorting}{\mathrm{sorting}}

\newcommand{\Sorting}{\hat{\Gamma}}
% calculus of well-sorted syntax
% \newcommand{\fun}[2]{\lambda #1.#2}

% Language Syntax
\newcommand{\Sig}{\mathcal{L}}
\newcommand{\Sorts}{\mathcal{S}}
\newcommand{\Constructors}{\mathcal{C}}
\newcommand{\Positions}{\mathcal{P}}
\newcommand{\Sc}{\Sorts_\Constructors}
\newcommand{\Sp}{\Sorts_\Positions}
\newcommand{\Arity}{\mathcal{A}}
\newcommand{\ArityInv}{\Arity^{-1}}
\newcommand{\Ports}[1]{\Positions/#1}

% Language Format
\newcommand{\Format}{\Phi}

% Identifier Sets
\newcommand{\IdSet}{U}
\newcommand{\IdxSet}[1]{\IdSet_{#1}}

% Terms
\newcommand{\e}{\varepsilon}

% Graphs
\newcommand{\States}{\Sigma}

% Graph Helpers
\newcommand{\inedges}{\mathrm{inedges}}
\newcommand{\outedges}{\mathrm{outedges}}
\newcommand{\sources}{\mathrm{sources}}
\newcommand{\parents}{\mathrm{parents}}
\newcommand{\ancestors}{\mathrm{ancestors}}
\newcommand{\lfp}{\mathrm{lfp}}
\newcommand{\argmin}{\mathrm{arg}\,\mathrm{min}}

% Trees
\newcommand{\sort}{\sigma}
\newcommand{\term}{t}
\newcommand{\ptree}{\breve{\term}}
\newcommand{\tree}{\hat{\term}}
\newcommand{\multiparent}[2]{\textcolor{red}{{\curlyveedownarrow_{#1}^{#2}}}}
\newcommand{\cycle}[2]{\textcolor{red}{\rcirclearrowleft_{#1}^{#2}}}
\newcommand{\Hole}{\square} %\textcolor{violet}{\llparenthesis}}\textcolor{violet}{\rrparenthesis}}
\newcommand{\multichild}[2][]{%
{\noexpandarg\StrSubstitute{#2}{,}{\textcolor{red}{\,\textbf{|}\,}}[\myargs]%
{\textcolor{red}{\textbf{\{}}\myargs\textcolor{red}{\textbf{\}}}_{#1}}}}%
% \newcommand{\Hole}[2]{\square_{\id{(#1, #2)}}}

% Category Theory
\renewcommand{\L}{\mathbb{L}}
\renewcommand{\S}{\mathbb{S}}
\newcommand{\ev}{\varepsilon}
\newcommand{\T}[1]{\tilde{#1}}
\newcommand{\id}{\textrm{id}}



%%%%%%%%

\renewcommand{\topfraction}{1} % Allow floats to take up the page
\renewcommand{\textfraction}{0}

%%%%%%%%
% \autoref from hyperref
\renewcommand{\AMSautorefname}          {Equation}
\renewcommand{\appendixautorefname}     {Appendix}
\renewcommand{\chapterautorefname}      {Chapter}
\renewcommand{\equationautorefname}     {Equation}
\renewcommand{\FancyVerbLineautorefname}{Line}
\renewcommand{\figureautorefname}       {Figure}
\renewcommand{\footnoteautorefname}     {Footnote}
\renewcommand{\Hfootnoteautorefname}    {Footnote}
\renewcommand{\itemautorefname}         {Item}
\renewcommand{\Itemautorefname}         {Item}
\renewcommand{\pageautorefname}         {Page}
\renewcommand{\paragraphautorefname}    {Section}
\renewcommand{\partautorefname}         {Part}
\renewcommand{\sectionautorefname}      {Section}
\renewcommand{\subparagraphautorefname} {Section}
\renewcommand{\subsectionautorefname}   {Section}
\renewcommand{\subsubsectionautorefname}{Section}
\renewcommand{\tableautorefname}        {Table}
\renewcommand{\theoremautorefname}      {Theorem}

%% Packages
\usepackage{booktabs}
\usepackage[rule=false]{subcaption}
\usepackage{graphicx}
% \usepackage{semantic}
\let\colonapprox\undefined % Avoid redefinition error in `colonequals`
\let\colonsim\undefined % Avoid redefinition error in `colonequals`
\usepackage{colonequals}
\usepackage{mathpartir}
\usepackage{stmaryrd}
\usepackage{fontawesome}
\usepackage{array}
\usepackage{todonotes}
% \usepackage[ruled]{algorithm2e}

% Load MnSymbol without clobbering \ast
% See https://tex.stackexchange.com/a/269691
\usepackage{amsmath}% needed before mathabx
\let\amsast=\ast
\usepackage[matha]{mathabx}% needed to prevent \ast getting clobbered
\let\abxast=\ast
\usepackage{MnSymbol}
\let\mnast=\ast
\let\ast=\abxast

%%%%%%%%
% TikZ Stuff
%\usepackage{etex} % Fix "No room for new \dimen" error
\usepackage{shellesc} % Fix bug that breaks the tikz 'external' library
\usepackage{tikz}
\usetikzlibrary{babel} % Ensure compatibility the 'babel' package

\usetikzlibrary{external} % Needs to be separately enabled
%\tikzexternalize % Enable externalization
%\usepackage{lua-visual-debug}

\usetikzlibrary{arrows.meta} % Arrow Tips
\tikzset{>=Stealth}
%\tikzset{<=stealth}
%\tikzset{arrows={-Stealth[scale=50]}}
%\tikzset{edge from parent/.style={draw,->,line width=0.6pt}}
%\tikzset{wideline/.style={line width=0.7pt}}
%\tikzset{boldline/.style={color=black,line width=1.0pt}}

\usetikzlibrary{
  backgrounds,  % Provides "framed" and "gridded"
  bending,      % bending arrow tips
  decorations.pathmorphing,   % Provides wavy edges
  graphs,       % Graph *notation*
  graphdrawing, % Graph *layout*
  quotes,       % Quote syntax (e.g., "foo")
}

\usegdlibrary{
  trees,
}

\tikzset{
  %every picture/.style={framed, background rectangle/.style={draw=gray!50}},
}
\tikzset{edge style/.style={
  draw,
  %color=gray,
  font={\small\ttfamily},
  /tikz/every edge quotes/.style={
    %draw=gray!20,
    anchor=west,
    swap/.append code={
      \ifpgfarrowswap
        \pgfkeysalso{anchor=west}
      \else
        \pgfkeysalso{anchor=east}
      \fi}},
}}
\tikzset{graphs/graph style/.style={
  tree layout,
  level distance=0.5cm,
  level sep=0.5cm,
  sibling distance=0.5cm,
  sibling sep=0.1cm,
  part distance=0.1cm,
  part sep=0.1cm,
  component distance=0.1cm,
  component sep=0.1cm,
  nodes={
    draw,
    %color=gray,
    inner sep=2pt,
    rounded corners=1mm},
  edges={edge style},
}}
\tikzset{graphs/root style/.style={
 %draw=none,
 as={\textbullet$_{\id{0}}$}
}}
\tikzset{alice/.style={
  color=red!80!black,
  font={\bfseries\small},
  thick,
}}
\tikzset{bob/.style={
  color=green!60!black,
  font={\bfseries\small},
  thick,
}}
\tikzset{merge/.style={
  color=blue,
  font={\bfseries\small},
  thick,
}}
\tikzset{alice edge/.style={alice, edge style, font={\bfseries\small}}}
\tikzset{alice node/.style={alice}}
\tikzset{alice step/.style={alice}}
\tikzset{bob edge/.style={bob, edge style}, font={\bfseries\small}}
\tikzset{bob node/.style={bob}}
\tikzset{bob step/.style={bob}}
\tikzset{merge edge/.style={merge, edge style}, font={\bfseries\small}}
\tikzset{merge node/.style={merge}}
\tikzset{merge step/.style={merge,decorate,decoration={coil,amplitude=1.0pt,segment length=7.0pt,aspect=0}}}
\tikzset{star/.style={edge node={node[inner sep=0pt,at end,sloped] {\textbf{\huge${}^{\ast}$}}}}}

% Define outline versions of + and -
\def\outlinepad{0.4pt}
\def\outlinestroke{0.4pt}
\newcommand{\Plus}{\mathord{
\begin{tikzpicture}[anchor=base, baseline]
%\node at (0,0) {+};
\path[draw, line width=\outlinestroke]
   ( 0.333em+\outlinestroke/2+\outlinepad,  0.270em+\outlinestroke/2+\outlinepad)
 --( 0.333em+\outlinestroke/2+\outlinepad,  0.229em-\outlinestroke/2-\outlinepad)
 --( 0.021em+\outlinestroke/2+\outlinepad,  0.229em-\outlinestroke/2-\outlinepad)
 --( 0.021em+\outlinestroke/2+\outlinepad, -0.084em-\outlinestroke/2-\outlinepad)
 --(-0.020em-\outlinestroke/2-\outlinepad, -0.084em-\outlinestroke/2-\outlinepad)
 --(-0.020em-\outlinestroke/2-\outlinepad,  0.229em-\outlinestroke/2-\outlinepad)
 --(-0.333em-\outlinestroke/2-\outlinepad,  0.229em-\outlinestroke/2-\outlinepad)
 --(-0.333em-\outlinestroke/2-\outlinepad,  0.270em+\outlinestroke/2+\outlinepad)
 --(-0.020em-\outlinestroke/2-\outlinepad,  0.270em+\outlinestroke/2+\outlinepad)
 --(-0.020em-\outlinestroke/2-\outlinepad,  0.583em+\outlinestroke/2+\outlinepad)
 --( 0.021em+\outlinestroke/2+\outlinepad,  0.583em+\outlinestroke/2+\outlinepad)
 --( 0.021em+\outlinestroke/2+\outlinepad,  0.270em+\outlinestroke/2+\outlinepad)
 --cycle
 ;
\end{tikzpicture}
}}

\newcommand{\Minus}{\mathord{
\begin{tikzpicture}[anchor=base, baseline]
%\node at (0,0) {$-$};
\path[draw, line width=\outlinestroke]
   ( 0.306em+\outlinestroke/2+\outlinepad,  0.270em+\outlinestroke/2+\outlinepad)
 --( 0.306em+\outlinestroke/2+\outlinepad,  0.229em-\outlinestroke/2-\outlinepad)
 --(-0.306em-\outlinestroke/2-\outlinepad,  0.229em-\outlinestroke/2-\outlinepad)
 --(-0.306em-\outlinestroke/2-\outlinepad,  0.270em+\outlinestroke/2+\outlinepad)
 --cycle
 ;
\end{tikzpicture}
}}

%%%%%%%%%%%%%%%%%%%%%%%%%%%%%%%%%%%%%%%%%%%%%%%%%%%%%%%%%%%%%%%%%%%%%%%%%%%%%%%%
% Language Components

\usepackage{xstring}
\usepackage{tensor}

% \selectcolormodel{gray}

% notation
\def\_{\texttt{\textunderscore}}
\newcommand{\id}[1]{\textcolor{gray}{\ensuremath{#1}}}
\newcommand{\eid}[2]{\tensor*[_{#1}]{#2}{}}
\newcommand{\vid}[2]{\tensor*{#1}{^{#2}}}
\newcommand{\evid}[3]{\tensor*[_{\id{#1}}]{#3}{^{\id{#2}}}}
\newcommand{\backrefs}{\overline{\e}}
\newcommand{\abs}[1]{\left\lvert#1\right\rvert}
\newcommand{\figureCode}[1]{\textbf{\texttt{#1}}\vskip1em}
\newcommand{\SetOf}[1]{\left\{#1\right\}}
\newcommand{\SuchThat}[1]{ : #1}

% sets
\def\A{\mathcal{A}}
\def\E{\mathcal{E}}
\def\G{\mathcal{G}}
\def\I{\mathcal{I}}
\def\K{\mathcal{K}}
\def\P{\mathcal{P}}
\def\S{\mathcal{S}}
\def\U{\mathcal{U}}
\def\V{\mathcal{V}}
\def\e{\varepsilon}
\def\AA{\textbf{A}}
\def\EE{\textbf{E}}

% Sorts
\newcommand\Exp{\mathsf{Exp}}
\newcommand\Pat{\mathsf{Pat}}
\newcommand\Typ{\mathsf{Typ}}

% objects
\def\e{\varepsilon}
\def\Grove{\gamma}

%%%%%%%%%%%%%%%%%%%%%%%%%%%%%%%%%%%%%%%%%%%%%%%%%%%%%%%%%%%%%%%%%%%%%%%%%%%%%%%%
% Graphs

\newcommand{\Edge}[1]{Edge~#1}
\newcommand{\Vertex}[1]{Vertex~#1}
\newcommand{\multiVertex}[1]{\textcolor{red}{\ensuremath{\curlyveedownarrow_{#1}}}}
\newcommand{\cycleVertex}[1]{\textcolor{red}{\ensuremath{\rcirclearrowleft_{#1}}}}
\newcommand{\orphanVertex}[1]{\textcolor{red}{\ensuremath{\mathbf{\ndownarrow_{#1}}}}}
\newcommand{\parens}[1]{\textcolor{gray}{(}#1\textcolor{gray}{)}}
\newcommand{\otherVertexVskip}{\vskip0.5em}

% Constructors
\newcommand\ExpVar{\mathsf{Exp\_var}}
\newcommand\ExpLam{\mathsf{Exp\_lam}}
\newcommand\ExpApp{\mathsf{Exp\_app}}
\newcommand\ExpNum{\mathsf{Exp\_num}}
\newcommand\ExpPlus{\mathsf{Exp\_plus}}
\newcommand\ExpTimes{\mathsf{Exp\_times}}
\newcommand\PatVar{\mathsf{Pat\_var}}
\newcommand\TypNum{\mathsf{Typ\_num}}
\newcommand\TypArrow{\mathsf{Typ\_arrow}}

% Positions
\newcommand\Root{\mathsf{Root}}
\newcommand\LamParam{\mathsf{Param}}
\newcommand\LamType{\mathsf{Type}}
\newcommand\LamBody{\mathsf{Body}}
\newcommand\AppFun{\mathsf{Fun}}
\newcommand\AppArg{\mathsf{Arg}}
\newcommand\PlusLeft{\mathsf{Left}}
\newcommand\PlusRight{\mathsf{Right}}
\newcommand\TimesLeft{\mathsf{Left}}
\newcommand\TimesRight{\mathsf{Right}}
\newcommand\ArrowArg{\mathsf{Arg}}
\newcommand\ArrowResult{\mathsf{Result}}

%%%%%%%%%%%%%%%%%%%%%%%%%%%%%%%%%%%%%%%%%%%%%%%%%%%%%%%%%%%%%%%%%%%%%%%%%%%%%%%%
% Terms

\newcommand{\varExp}[2]{#1^{#2}}
\newcommand{\numExp}[2]{\underline{#1}^{#2}}
\newcommand{\lamExp}[4]{\lambda^{#1} #2 : #3.#4}
\newcommand{\appExp}[3]{\left(#1~#2\right)^{#3}}
\newcommand{\plusExp}[3]{#1 +^{#2} #3}
\newcommand{\varPat}[2]{#1^{#2}}
\newcommand{\arrowTyp}[3]{#1 \to^{#2} #3}
\newcommand{\numTyp}[1]{Num^{#1}}
\newcommand{\hole}{\ensuremath{\square}} %\textcolor{violet}{\llparenthesis}}\textcolor{violet}{\rrparenthesis}}
\newcommand{\conflictHole}[1]{%
{\noexpandarg\StrSubstitute{#1}{,}{\textcolor{red}{\;\textbf{|}\;}}[\myargs]%
{\textcolor{red}{\textbf{\{}}\myargs\textcolor{red}{\textbf{\}}}}}}%

\newcommand{\eVar}[3]{\evid{#1}{#2}{#3}}
\newcommand{\eFun}[5]{\evid{#1}{#2}{\lambda} #3 : #4 . #5}
\newcommand{\eApp}[4]{\evid{#1}{#2}{\left(#3~#4\right)}}
\newcommand{\eNum}[3]{\evid{#1}{#2}{\underline{#3}}}
\newcommand{\ePlus}[4]{#3~\evid{#1}{#2}{\texttt{+}}~#4}
\newcommand{\eTimes}[4]{#3~\evid{#1}{#2}{\texttt{*}}~#4}
\newcommand{\pVar}[3]{\evid{#1}{#2}{#3}}
\newcommand{\tArrow}[4]{#3 \evid{#1}{#2}{\rightarrow} #4}
\newcommand{\tNum}[2]{\evid{#1}{#2}{Num}}

%%%%%%%%%%%%%%%%%%%%%%%%%%%%%%%%%%%%%%%%%%%%%%%%%%%%%%%%%%%%%%%%%%%%%%%%%%%%%%%%
% Decomposition

\DeclareMathOperator{\decompOp}{\texttt{decomp}}
\DeclareMathOperator{\vertexesOp}{\texttt{vertexes}}
\newcommand{\decomp}[1]{\decompOp\mathopen{}\left(#1\right)\mathclose{}}
\newcommand{\vertexes}[1]{\vertexesOp\mathopen{}\left(#1\right)\mathclose{}}

\DeclareMathOperator{\edecompOp}{\texttt{edecomp}}
\DeclareMathOperator{\pdecompOp}{\texttt{pdecomp}}
\DeclareMathOperator{\tdecompOp}{\texttt{tdecomp}}
\DeclareMathOperator{\edecompPrimeOp}{\edecompOp^\prime}
\DeclareMathOperator{\pdecompPrimeOp}{\pdecompOp^\prime}
\DeclareMathOperator{\tdecompPrimeOp}{\tdecompOp^\prime}
\newcommand{\edecomp}[1]{\edecompOp\mathopen{}\left(#1\right)\mathclose{}}
\newcommand{\pdecomp}[1]{\pdecompOp\mathopen{}\left(#1\right)\mathclose{}}
\newcommand{\tdecomp}[1]{\tdecompOp\mathopen{}\left(#1\right)\mathclose{}}
\newcommand{\edecompPrime}[2]{\edecompPrimeOp\mathopen{}\left(#1, #2\right)\mathclose{}}
\newcommand{\pdecompPrime}[2]{\pdecompPrimeOp\mathopen{}\left(#1, #2\right)\mathclose{}}
\newcommand{\tdecompPrime}[2]{\tdecompPrimeOp\mathopen{}\left(#1, #2\right)\mathclose{}}

% helpers
\DeclareMathOperator{\adjacent}{\text{adjacent}}
\DeclareMathOperator{\parents}{\text{parents}}
\DeclareMathOperator{\ancestors}{\text{ancestors}}
\DeclareMathOperator{\vertices}{\text{vertices}}
\DeclareMathOperator{\children}{\text{children}}
\DeclareMathOperator{\lfp}{\text{lfp}}

\DeclareMathOperator{\minOp}{\text{min}}
\renewcommand{\min}[1]{\minOp\mathopen{}\left(#1\right)\mathclose{}}

%%%%%%%%%%%%%%%%%%%%%%%%%%%%%%%%%%%%%%%%%%%%%%%%%%%%%%%%%%%%%%%%%%%%%%%%%%%%%%%%
% Recomposition

\DeclareMathOperator{\recompPlusOp}{\texttt{recomp}^{+}}
\newcommand{\recompPlus}[1]{\recompPlusOp\mathopen{}\left(#1\right)\mathclose{}}

\DeclareMathOperator{\recompOp}{\texttt{recomp}}
\DeclareMathOperator{\erecompOp}{\texttt{erecomp}}
\DeclareMathOperator{\precompOp}{\texttt{precomp}}
\DeclareMathOperator{\trecompOp}{\texttt{trecomp}}
\DeclareMathOperator{\erecompPrimeOp}{\erecompOp^\prime}
\DeclareMathOperator{\precompPrimeOp}{\precompOp^\prime}
\DeclareMathOperator{\trecompPrimeOp}{\trecompOp^\prime}
\newcommand{\recomp}[2]{\recompOp\mathopen{}\left(#1, #2\right)\mathclose{}}
\newcommand{\erecomp}[1]{\erecompOp\mathopen{}\left(#1\right)\mathclose{}}
\newcommand{\precomp}[1]{\precompOp\mathopen{}\left(#1\right)\mathclose{}}
\newcommand{\trecomp}[1]{\trecompOp\mathopen{}\left(#1\right)\mathclose{}}
\newcommand{\erecompPrime}[3]{\erecompPrimeOp\mathopen{}\left(#1, #2, #3\right)\mathclose{}}
\newcommand{\precompPrime}[3]{\precompPrimeOp\mathopen{}\left(#1, #2, #3\right)\mathclose{}}
\newcommand{\trecompPrime}[3]{\trecompPrimeOp\mathopen{}\left(#1, #2, #3\right)\mathclose{}}

%%%%%%%%%%%%%%%%%%%%%%%%%%%%%%%%%%%%%%%%%%%%%%%%%%%%%%%%%%%%%%%%%%%%%%%%%%%%%%%%

% actions
% \def\Create{\text{Create}}
% \def\Destroy{\text{Destroy}}
\def\Down{\text{Down}}
\def\Enqueue{\text{Enqueue}}
\def\Left{\text{Left}}
\def\Move{\text{Move}}
\def\Num{\text{Num}}
% \def\Restore{\text{Restore}}
\def\Right{\text{Right}}
\def\Select{\text{Select}}
\def\Send{\text{Send}}
\def\Up{\text{Up}}

\newcommand\Delete{\mathrm{Delete}}
\newcommand{\Create}[1]{\mathrm{Create}\mathopen{}\left(#1\right)\mathclose{}}
% \newcommand{\Drop}[1]{\mathrm{Drop}\mathopen{}\left(#1\right)\mathclose{}}
\newcommand{\Restore}[1]{\mathrm{Restore}\mathopen{}\left(#1\right)\mathclose{}}
\newcommand{\Wrap}[1]{\mathrm{Wrap}\mathopen{}\left(#1\right)\mathclose{}}
\newcommand{\Reposition}[1]{\mathrm{Reposition}\mathopen{}\left(#1\right)\mathclose{}}

% \Create{k} \mid \Delete \mid \Drop{\e} \mid \Restore{v}

% relations
\DeclareMathOperator{\sort}{\text{sort}}
\DeclareMathOperator{\arity}{\text{arity}}

% Calling \newvertex{Foo}{bar} defines
%   \vidFoo to be a new id number, and
%   \vFoo to be \texttt{bar}\ensuremath{_{\vidFoo}}
\newcounter{NodeVertexCounter}
\newcommand{\newvertex}[2]{%
\ifodd\theNodeVertexCounter
  \addtocounter{NodeVertexCounter}{1}%
\else
  \addtocounter{NodeVertexCounter}{2}%
\fi
\expandafter\newcommand\csname vid#1\endcsname{}% This is just to check if this is a redefinition
\expandafter\global\expandafter\edef\csname vid#1\endcsname{\theNodeVertexCounter}%
\expandafter\newcommand\csname v#1\endcsname{}% This is just to check if this is a redefinition
\expandafter\gdef\csname v#1\endcsname{\texttt{#2}\ensuremath{_{\id{\csname vid#1\endcsname}}}}%
}
\newcommand{\newedge}[2]{%
\ifodd\theNodeVertexCounter
  \addtocounter{NodeVertexCounter}{2}%
\else
  \addtocounter{NodeVertexCounter}{1}%
\fi
\expandafter\newcommand\csname eid#1\endcsname{}% This is just to check if this is a redefinition
\expandafter\global\expandafter\edef\csname eid#1\endcsname{\theNodeVertexCounter}%
\expandafter\newcommand\csname e#1\endcsname{}% This is just to check if this is a redefinition
\expandafter\gdef\csname e#1\endcsname{\texttt{#2}\ensuremath{_{\id{\csname eid#1\endcsname}}}}%
}
\setcounter{NodeVertexCounter}{-1}
\newvertex{Root}{Root}

% Support \includegraphics of .dot files
\DeclareGraphicsRule{.dot}{pdf}{.pdf}{`dot -Tpdf #1 -o \noexpand\OutputFile}


%%%%%%%%%%%%%%%%%%%%%%%%%%%%%%%%
%% \figureSimple
%%%%%%%%%%%%%%%%%%%%%%%%%%%%%%%%
\newedge{SimpleTimes}{Root}%
\newvertex{SimpleTimes}{*}%
\newedge{SimpleX}{L}%
\newvertex{SimpleX}{x}%
\newedge{SimpleY}{R}%
\newvertex{SimpleY}{y}%
\newcommand{\figureSimple}{
\begin{figure}
\hfill
%%%%%%%%
\begin{subfigure}[t]{0.32\linewidth}
\centering
\caption{
\begin{tikzpicture}[remember picture, overlay]
\path (0cm,0cm) node (node/Simple/a) [anchor=base] {\strut};
\end{tikzpicture}
}%
\label{fig:Simple:a}
\figureCode{x * \hole{}}%
\begin{tikzpicture}
\path (0cm,0cm) graph[graph style] {
 root[root style, color=black] -> {
  "\vSimpleTimes" [> "\eSimpleTimes", >color=black, color=black] -> {
   "\vSimpleX" [> "\eSimpleX"', >color=black, color=black],
   {}
  }
 }
};
\end{tikzpicture}
\end{subfigure}
%%%%%%%%
\begin{subfigure}[t]{0.32\linewidth}
\centering
\caption{
\begin{tikzpicture}[remember picture, overlay]
\path (0cm,0cm) node (node/Simple/b) [anchor=base] {\strut};
\path [draw,->,alice step] (node/Simple/a) to (-2em, 0cm |- node/Simple/b);
\end{tikzpicture}
}%
\label{fig:Simple:b}
\figureCode{x * y}%
\begin{tikzpicture}
\path (0cm,0cm) graph[graph style] {
  root[root style] -> {
  "\vSimpleTimes" [> "\eSimpleTimes"] -> {
    "\vSimpleX" [> "\eSimpleX"'],
    "\vSimpleY" [> "\eSimpleY", >alice edge, alice node]
  }
  }
};
\end{tikzpicture}
\end{subfigure}
%%%%%%%%
\hfill
%%%%%%%%
\begin{subfigure}[t]{0.32\linewidth}
\centering
\caption{
\begin{tikzpicture}[remember picture, overlay]
\path (0cm,0cm) node (node/Simple/c) [anchor=base] {\strut};
\path [draw,->,alice step] (node/Simple/b) to (-2em, 0cm |- node/Simple/c);
\end{tikzpicture}
}%
\label{fig:Simple:c}
\figureCode{\hole{} * y}%
\begin{tikzpicture}
\path (0cm,0cm) graph[graph style] {
  root[root style] -> {
  "\vSimpleTimes" [> "\eSimpleTimes"] -> {
    {},
    "\vSimpleY" [> "\eSimpleY"]
  }
  }
};
\path (-0.25cm,-1.7cm) graph[graph style] {
  "\vSimpleX"
};
\end{tikzpicture}
\end{subfigure}
%%%%%%%%
\hfill{}
\caption{Code that Includes a Hole and Some Simple Edits.}%
\label{fig:Simple}
\end{figure}
}

%%%%%%%%%%%%%%%%%%%%%%%%%%%%%%%%
%% \figureWrap
%%%%%%%%%%%%%%%%%%%%%%%%%%%%%%%%
\newedge{WrapPlus}{Root}
\newvertex{WrapPlus}{+}
\newedge{WrapTimes}{L}
\newcommand{\figureWrapMove}{
\begin{figure}
\centering
\begin{minipage}[t]{.45\linewidth}
\hskip0.12\columnwidth
%%%%%%%%
\begin{subfigure}[t]{0.28\linewidth}
\centering
\caption{
\begin{tikzpicture}[remember picture, overlay]
\path (-7.5em,0cm) node (node/Simple/cx) [gray,anchor=base west] {{\textbf{\small Fig.~\ref*{fig:Simple:c}}}};
%%%%
\path (0cm,0cm) node (node/Wrap/a) [anchor=base] {\strut};
\path [draw,->,alice step] (node/Simple/cx) to (-2em, 0cm |- node/Wrap/a)
\end{tikzpicture}
}%
\label{fig:Wrap:a}
\figureCode{\hole{}}%
\begin{tikzpicture}
\path (0cm,0cm) graph[graph style] {
 root[root style]
};
\path (0cm,-1.7cm) graph[graph style] {
 "\vSimpleTimes" -> {
  {},
  "\vSimpleY" [> "\eSimpleY"]
 }
};
\end{tikzpicture}
\end{subfigure}
%%%%%%%%
\hfill
%%%%%%%%
\begin{subfigure}[t]{0.28\linewidth}
\centering
\caption{
\begin{tikzpicture}[remember picture, overlay]
\path (0cm,0cm) node (node/Wrap/b) [anchor=base] {\strut};
\path [draw,->,alice step] (node/Wrap/a) to (-2em, 0cm |- node/Wrap/b);
\end{tikzpicture}
}%
\label{fig:Wrap:b}
\figureCode{\hole{} + \hole{}}%
\begin{tikzpicture}
\path (0cm,0cm) graph[graph style] {
 root[root style] -> {
  "\vWrapPlus" [> "\eWrapPlus", >alice edge, alice node]
 }
};
\path (0cm,-1.7cm) graph[graph style] {
 "\vSimpleTimes" -> {
  {},
  "\vSimpleY" [> "\eSimpleY"]
 }
};
\end{tikzpicture}
\end{subfigure}
%%%%%%%%
\hfill
%%%%%%%%
\begin{subfigure}[t]{0.28\linewidth}
\centering
\caption{
\begin{tikzpicture}[remember picture, overlay]
\path (0cm,0cm) node (node/Wrap/c) [anchor=base] {\strut};
\path [draw,->,alice step] (node/Wrap/b) to (-2em, 0cm |- node/Wrap/c);
\end{tikzpicture}
}%
\label{fig:Wrap:c}
\figureCode{\hole{} * y + \hole{}}%
\begin{tikzpicture}
\path (0cm,0cm) graph[graph style] {
 root[root style] -> {
  "\vWrapPlus" [> "\eWrapPlus"] -> {
   "\vSimpleTimes" [> "\eWrapTimes"', >alice edge] -> {
    {},
    "\vSimpleY" [> "\eSimpleY"]
   },
   {}
  }
 }
};
\end{tikzpicture}
\end{subfigure}
%%%%%%%%
\hfill{}
\caption{Wrapping results in a patch which (a) cuts the original term (edge deletion), (b) creates the outer term at the same location (edge insertion), then (c) pastes the original term (edge insertion).}%
\label{fig:Wrap}
\end{minipage}
% \end{figure}
% }
%
\hfil
%
%%%%%%%%%%%%%%%%%%%%%%%%%%%%%%%%
%% \figureMove
%%%%%%%%%%%%%%%%%%%%%%%%%%%%%%%%
\newedge{MoveTimes}{R}
% \newcommand{\figureMove}{
% \begin{figure}[H]
\begin{minipage}[t]{.45\linewidth}
\hfill
%\hskip0.12\columnwidth
%%%%%%%%
\begin{subfigure}[t]{0.43\linewidth}
\centering
\caption{
\begin{tikzpicture}[remember picture, overlay]
\path (0cm,0cm) node (node/Move/a) [anchor=base] {\strut};
\path [draw,->,alice step] (node/Wrap/c) to (-2em, 0cm |- node/Move/a);
\end{tikzpicture}
}%
\label{fig:Move:a}
\figureCode{\hole{} + \hole{}}%
\begin{tikzpicture}
\path (0cm,0cm) graph[graph style] {
 root[root style] -> {
  "\vWrapPlus" [> "\eWrapPlus"]
 }
};
\path (0cm,-1.7cm) graph [graph style] {
 "\vSimpleTimes" -> {
  {},
  "\vSimpleY" [> "\eSimpleY"]
 }
};
\end{tikzpicture}
\end{subfigure}
%%%%%%%%
\hfill
%%%%%%%%
\begin{subfigure}[t]{0.43\linewidth}
\centering
\caption{
\begin{tikzpicture}[remember picture, overlay]
\path (0cm,0cm) node (node/Move/b) [anchor=base] {\strut};
\path [draw,->,alice step] (node/Move/a) to (-2em, 0cm |- node/Move/b);
\end{tikzpicture}
}%
\label{fig:Move:b}
\figureCode{\hole{} + \hole{} * y}%
\begin{tikzpicture}
\path (0cm,0cm) graph[graph style] {
 root[root style] -> {
  "\vWrapPlus" [> "\eWrapPlus"] -> {
   {},
   "\vSimpleTimes" [> "\eMoveTimes", >alice edge] -> {
    {},
    "\vSimpleY" [> "\eSimpleY"]
   }
  }
 }
};
\end{tikzpicture}
\end{subfigure}
%%%%%%%%
\hfill{}
\caption{Term relocation results in a patch which (a) cuts the term (edge deletion) then (b) pastes it in its new location (edge insertion). Vertex identities and downstream edges are conserved.}%
\Description{This figures shows an example of the repositioning action}
\label{fig:Move}
\end{minipage}
\vspace{-10px}
\end{figure}
}

%%%%%%%%%%%%%%%%%%%%%%%%%%%%%%%%
%% \figureDifferentParts
%%%%%%%%%%%%%%%%%%%%%%%%%%%%%%%%
\newedge{DifferentPartsAlice}{L}
\newvertex{DifferentPartsAlice}{u}
\newedge{DifferentPartsBob}{R}
\newvertex{DifferentPartsBob}{v}
% \newcommand{\figureDifferentParts}{
\newcommand{\figureDifferentPartsNestedParts}{
\begin{figure}
\begin{minipage}[t]{0.45\linewidth}
%\hfill
\hskip0.12\columnwidth
%%%%%%%%
\begin{subfigure}[t]{0.28\linewidth}
\centering
\caption{
\begin{tikzpicture}[remember picture, overlay]
\path (-7.5em,0cm) node (node/Move/bx) [gray,anchor=base west] {{\textbf{\small Fig.~\ref*{fig:Move:b}}}};
\path (0cm,0cm) node (node/DifferentParts/a) [anchor=base] {\strut};
\path [draw,->,alice step] (node/Move/bx) to (-2em, 0cm |- node/DifferentParts/a);
\end{tikzpicture}
}%
\label{fig:DifferentParts:a}
\figureCode{\hole{} + u * y}%
\begin{tikzpicture}
\path (0cm,0cm) graph[graph style] {
 root[root style] -> {
  "\vWrapPlus" [> "\eWrapPlus"] -> {
  {},
  "\vSimpleTimes" [> "\eMoveTimes"] -> {
    "\vDifferentPartsAlice" [> "\eDifferentPartsAlice"', >alice edge, alice node],
    "\vSimpleY" [> "\eSimpleY"]
  }
  }
 }
};
\end{tikzpicture}
\end{subfigure}
%%%%%%%%
\hfill
%%%%%%%%
\begin{subfigure}[t]{0.28\linewidth}
\centering
\caption{
\begin{tikzpicture}[remember picture, overlay]
\path (0cm,0cm) node (node/DifferentParts/b) [anchor=base] {\strut};
\path [draw,->,bob step] (node/Move/bx) to [out=15,in=165,star] (-2em, 0cm |- node/DifferentParts/b);
\end{tikzpicture}
}%
\label{fig:DifferentParts:b}
\figureCode{\hole{} + \hole{} * v}%
\begin{tikzpicture}
\path (0cm,0cm) graph[graph style] {
 root[root style] -> {
  "\vWrapPlus" [> "\eWrapPlus"] -> {
  {},
  "\vSimpleTimes" [> "\eMoveTimes"] -> {
    {},
    "\vDifferentPartsBob" [> "\eDifferentPartsBob", >bob edge, bob node]
  }
  }
 }
};
\end{tikzpicture}
\end{subfigure}
%%%%%%%%
\hfill
%%%%%%%%
\begin{subfigure}[t]{0.28\linewidth}
\centering
\caption{
\begin{tikzpicture}[remember picture, overlay]
\path (0cm,0cm) node (node/DifferentParts/c) [anchor=base] {\strut};
\path [draw,->,merge step] (node/DifferentParts/a) to [out=15,in=165] (-2em, 0cm |- node/DifferentParts/c);
\path [draw,->,merge step] (node/DifferentParts/b) to (-2em, 0cm |- node/DifferentParts/c);
\end{tikzpicture}
}%
\label{fig:DifferentParts:c}
\figureCode{\hole{} + u * v}%
\begin{tikzpicture}
\path (0cm,0cm) graph[graph style] {
 root[root style] -> {
  "\vWrapPlus" [> "\eWrapPlus"] -> {
  {},
  "\vSimpleTimes" [> "\eMoveTimes"] -> {
    "\vDifferentPartsAlice" [> "\eDifferentPartsAlice"', >merge edge, merge node],
    "\vDifferentPartsBob" [> "\eDifferentPartsBob", >merge edge, merge node]
  }
  }
 }
};
\end{tikzpicture}
\end{subfigure}
%%%%%%%%
\hfill{}
\caption{Example of Users Editing Different Parts of the Code.}%
\label{fig:DifferentParts}
\end{minipage}
% \end{figure}
% }
%
\hfil
%
%%%%%%%%%%%%%%%%%%%%%%%%%%%%%%%%
%% \figureCommutativity
%%%%%%%%%%%%%%%%%%%%%%%%%%%%%%%%
% \newcommand{\figureCommutativity}{
% \begin{figure}
% \centering
% \begin{figure}
% \centering
% \begin{tikzpicture}
% \path (-3cm, 0cm) node (a) [align=center]       {Original \\ Version};
% \path ( 0cm, 1cm) node (b) [align=center,alice node] {Alice's  \\ Version};
% \path ( 0cm,-1cm) node (c) [align=center,bob node]   {Bob's    \\ Version};
% \path ( 3cm, 0cm) node (d) [align=center]       {Combined \\ Version};
% \path [draw,->,alice step] (a) -- node [pos=0.7,auto] {Alice's Edits} (b);
% \path [draw,->,bob step]   (a) -- node [pos=0.7,auto,swap]      {Bob's Edits}   (c);
% \path [draw,->,merge step] (b) -- node [pos=0.3,auto] {Sync} (d);
% \path [draw,->,merge step] (c) -- node [pos=0.3,auto,swap]      {Sync} (d);
% \end{tikzpicture}
% \caption{TODO:Commutativity}
% \label{fig:Commutativity}
% \end{figure}
% }
%
%%%%%%%%%%%%%%%%%%%%%%%%%%%%%%%%
%% \figureNestedParts
%%%%%%%%%%%%%%%%%%%%%%%%%%%%%%%%
\newedge{NestedPartsAlice}{L}
\newvertex{NestedPartsAlice}{w}
\newedge{NestedPartsBob}{L}
% \newcommand{\figureNestedParts}{
% \begin{figure}
\begin{minipage}[t]{0.45\linewidth}
\hfill
%\hskip0.12\columnwidth
%%%%%%%%
\begin{subfigure}[t]{0.25\linewidth}
\centering
\caption{
\begin{tikzpicture}[remember picture, overlay]
\path (0cm,0cm) node (node/NestedParts/a) [anchor=base] {\strut};
\path [draw,->,alice step] (node/DifferentParts/c) to[star] (-2em, 0cm |- node/NestedParts/a);
\end{tikzpicture}
}%
\label{fig:NestedParts:a}
\figureCode{\hole{} + w * v}%
\begin{tikzpicture}
\path (0cm,0cm) graph[graph style] {
 root[root style] -> {
  "\vWrapPlus" [> "\eWrapPlus"] -> {
   {},
   "\vSimpleTimes" [> "\eMoveTimes"] -> {
    "\vNestedPartsAlice" [> "\eNestedPartsAlice"', >alice edge, alice node],
    "\vDifferentPartsBob" [> "\eDifferentPartsBob"]
   }
  }
 }
};
\end{tikzpicture}
\end{subfigure}
%%%%%%%%
\hfill
%%%%%%%%
\begin{subfigure}[t]{0.25\linewidth}
\centering
\caption{
\begin{tikzpicture}[remember picture, overlay]
\path (0cm,0cm) node (node/NestedParts/b) [anchor=base] {\strut};
\path [draw,->,bob step] (node/DifferentParts/c) to [out=15,in=165,star] (-2em, 0cm |- node/NestedParts/b);
\end{tikzpicture}
}%
\label{fig:NestedParts:b}
\figureCode{u * v + \hole{}}%
\begin{tikzpicture}
\path (0cm,0cm) graph[graph style] {
 root[root style] -> {
  "\vWrapPlus" [> "\eWrapPlus"] -> {
   "\vSimpleTimes" [> "\eNestedPartsBob"', >bob edge] -> {
    "\vDifferentPartsAlice" [> "\eDifferentPartsAlice"'],
    "\vDifferentPartsBob" [> "\eDifferentPartsBob"]
   },
   {}
  }
 }
};
\end{tikzpicture}
\end{subfigure}
%%%%%%%%
\hfill
%%%%%%%%
\begin{subfigure}[t]{0.25\linewidth}
\centering
\caption{
\begin{tikzpicture}[remember picture, overlay]
\path (0cm,0cm) node (node/NestedParts/c) [anchor=base] {\strut};
\path [draw,->,merge step] (node/NestedParts/a) to [out=15,in=165] (-2em, 0cm |- node/NestedParts/c);
\path [draw,->,merge step] (node/NestedParts/b) to [out=15,in=165] (-2em, 0cm |- node/NestedParts/c);
\end{tikzpicture}
}%
\label{fig:NestedParts:c}
\figureCode{w * v + \hole{}}%
\begin{tikzpicture}
\path (0cm,0cm) graph[graph style] {
 root[root style] -> {
  "\vWrapPlus" [> "\eWrapPlus"] -> {
   "\vSimpleTimes" [> "\eNestedPartsBob"', >merge edge] -> {
    "\vNestedPartsAlice" [> "\eNestedPartsAlice"', >merge edge, merge node],
    "\vDifferentPartsBob" [> "\eDifferentPartsBob"]
   },
   {}
  }
 }
};
\end{tikzpicture}
\end{subfigure}
%%%%%%%%
\hfill{}
\caption{Example of Users Editing Nested Parts of the Code.}%
\label{fig:NestedParts}
\end{minipage}
\end{figure}
}

%%%%%%%%%%%%%%%%%%%%%%%%%%%%%%%%
%% \figureMultiChild
%%%%%%%%%%%%%%%%%%%%%%%%%%%%%%%%
\newedge{MultiChildAlice}{R}
\newvertex{MultiChildAlice}{x}
\newedge{MultiChildBob}{R}
\newvertex{MultiChildBob}{y}
\newcommand{\figureMultiChild}{
\begin{figure}
%\hfill
\hskip0.12\columnwidth
%%%%%%%%
\begin{subfigure}[t]{0.21\linewidth}
\centering
\caption{
\begin{tikzpicture}[remember picture, overlay]
\path (-7.5em,0cm) node (node/NestedParts/cx) [gray,anchor=base west] {{\textbf{\small Fig.~\ref*{fig:NestedParts:c}}}};
%%%%
\path (0cm,0cm) node (node/MultiChild/a) [anchor=base] {\strut};
\path [draw,->,alice step] (node/NestedParts/cx) to (-2em, 0cm |- node/MultiChild/a);
\end{tikzpicture}
}%
\label{fig:MultiChild:a}
\figureCode{w * v + x}%
\begin{tikzpicture}
\path (0cm,0cm) graph[graph style] {
 root[root style] -> {
  "\vWrapPlus" [> "\eWrapPlus"] -> {
   "\vSimpleTimes" [> "\eNestedPartsBob"'] -> {
    "\vNestedPartsAlice" [> "\eNestedPartsAlice"'],
    "\vDifferentPartsBob" [> "\eDifferentPartsBob"]
   },
   {
    "\vMultiChildAlice" [> "\eMultiChildAlice", >alice edge, alice node]
   }
  }
 }
};
\end{tikzpicture}
\end{subfigure}
%%%%%%%%
\hfill
%%%%%%%%
\begin{subfigure}[t]{0.21\linewidth}
\centering
\caption{
\begin{tikzpicture}[remember picture, overlay]
\path (0cm,0cm) node (node/MultiChild/b) [anchor=base] {\strut};
\path [draw,->,bob step] (node/NestedParts/cx) to[out=15,in=165] (-2em, 0cm |- node/MultiChild/b);
\end{tikzpicture}
}%
\label{fig:MultiChild:b}
\figureCode{w * v + y}%
\begin{tikzpicture}
\path (0cm,0cm) graph[graph style] {
 root[root style] -> {
  "\vWrapPlus" [> "\eWrapPlus"] -> {
   "\vSimpleTimes" [> "\eNestedPartsBob"'] -> {
    "\vNestedPartsAlice" [> "\eNestedPartsAlice"'],
    "\vDifferentPartsBob" [> "\eDifferentPartsBob"]
   },
   "\vMultiChildBob" [> "\eMultiChildBob", >bob edge, bob node]
  }
 }
};
\end{tikzpicture}
\end{subfigure}
%%%%%%%%
\hfill
%%%%%%%%
\begin{subfigure}[t]{0.4\linewidth}
\centering
\caption{
\begin{tikzpicture}[remember picture, overlay]
\path (0cm,0cm) node (node/MultiChild/c) [anchor=base] {\strut};
\path [draw,->,merge step] (node/MultiChild/a) to[out=15,in=165] (-2em, 0cm |- node/MultiChild/c);
\path [draw,->,merge step] (node/MultiChild/b) to (-2em, 0cm |- node/MultiChild/c);
\end{tikzpicture}
}%
\label{fig:MultiChild:c}
\figureCode{w * v + $\conflictHole{x,y}$}%
\begin{tikzpicture}
\path (0cm,0cm) graph[graph style] {
 root[root style] -> {
  "\vWrapPlus" [> "\eWrapPlus"] -> {
   "\vSimpleTimes" [> "\eNestedPartsBob"'anchor=-15] -> {
    "\vNestedPartsAlice" [> "\eNestedPartsAlice"'],
    "\vDifferentPartsBob" [> "\eDifferentPartsBob"]
   },
   "\vMultiChildAlice" [> "\eMultiChildAlice"anchor=153, >merge edge, merge node],
   {},
   "\vMultiChildBob" [> "\eMultiChildBob"anchor=195, >merge edge, merge node]
  }
 }
};
\end{tikzpicture}
\end{subfigure}
%%%%%%%%
%\hfill{}
\caption{Example of Multi-Child Conflicts.}%
\Description{This figure shows an example of multi-child conflicts}
\label{fig:MultiChild}
\end{figure}
}

%%%%%%%%%%%%%%%%%%%%%%%%%%%%%%%%
%% \figureMultiParent
%%%%%%%%%%%%%%%%%%%%%%%%%%%%%%%%
\newedge{MultiParentAlice}{R}
\newedge{MultiParentBob}{R}
\newcommand{\figureMultiParent}{
\begin{figure*}
\hfill
% \hskip0.1\columnwidth
%%%%%%%%
\begin{subfigure}[t]{0.12\linewidth}
\centering
\caption{
\begin{tikzpicture}[remember picture, overlay]
% \path (-7.5em,0cm) node (node/MultiChild/cx) [gray,anchor=base west] {{\textbf{\small Fig.~\ref*{fig:MultiChild:c}}}};
%%%%
\path (0cm,0cm) node (node/MultiParent/a) [anchor=base] {\strut};
% \path [draw,->,alice step] (node/MultiChild/cx) to[star] (-2em, 0cm |- node/MultiParent/a);
\end{tikzpicture}
}%
\label{fig:MultiParent:a}
\figureCode{w * \hole{} + \hole{}\otherVertexVskip\phantom{\multiVertex{58} = w}}%
\begin{tikzpicture}
\path (0cm,0cm) graph[graph style] {
 root[root style] -> {
  "\vWrapPlus" [> "\eWrapPlus"] -> {
   "\vSimpleTimes" [> "\eNestedPartsBob"'] -> {
    "\vNestedPartsAlice" [> "\eNestedPartsAlice"'],
    {}
   },
   {}
  }
 }
};
\end{tikzpicture}
\end{subfigure}
%%%%%%%%
\hfill
%%%%%%%%
\begin{subfigure}[t]{0.14\linewidth}
\centering
\caption{
\begin{tikzpicture}[remember picture, overlay]
\path (0cm,0cm) node (node/MultiParent/b) [anchor=base] {\strut};
\path [draw,->,alice step] (node/MultiParent/a) to[star] (-2em, 0cm |- node/MultiParent/b);
\end{tikzpicture}
}%
\label{fig:MultiParent:b}
\figureCode{\hole{} * w + \hole{}\otherVertexVskip\phantom{\multiVertex{58} = w}}%
\begin{tikzpicture}
\path (0cm,0cm) graph[graph style] {
 root[root style] -> {
  "\vWrapPlus" [> "\eWrapPlus"] -> {
   "\vSimpleTimes" [> "\eNestedPartsBob"'] -> {
    {},
    "\vNestedPartsAlice" [> "\eMultiParentAlice"', >alice edge]
   },
   {}
  }
 }
};
\end{tikzpicture}
\end{subfigure}
%%%%%%%%
\hfill
%%%%%%%%
\begin{subfigure}[t]{0.14\linewidth}
\centering
\caption{
\begin{tikzpicture}[remember picture, overlay]
\path (0cm,0cm) node (node/MultiParent/c) [anchor=base] {\strut};
\path [draw,->,bob step] (node/MultiParent/a) to[out=15,in=165,star] (-2em, 0cm |- node/MultiParent/c);
\end{tikzpicture}
}%
\label{fig:MultiParent:c}
\figureCode{\hole{} * \hole{} + w\otherVertexVskip\phantom{\multiVertex{58} = w}}%
\begin{tikzpicture}
\path (0cm,0cm) graph[graph style] {
 root[root style] -> {
  "\vWrapPlus" [> "\eWrapPlus"] -> {
   "\vSimpleTimes" [> "\eNestedPartsBob"'],
   "\vNestedPartsAlice" [> "\eMultiParentBob", >bob edge]
  }
 }
};
\end{tikzpicture}
\end{subfigure}
%%%%%%%%
\hfill
%%%%%%%%
\begin{subfigure}[t]{0.18\linewidth}
\centering
\caption{
\begin{tikzpicture}[remember picture, overlay]
\path (0cm,0cm) node (node/MultiParent/d) [anchor=base] {\strut};
\path [draw,->,merge step] (node/MultiParent/b) to[out=15,in=165] (-2em, 0cm |- node/MultiParent/d);
\path [draw,->,merge step] (node/MultiParent/c) to[out=15,in=165] (-2em, 0cm |- node/MultiParent/d);
\end{tikzpicture}
}%
\label{fig:MultiParent:d}
\figureCode{\hole{} * \multiVertex{58} + \multiVertex{58}\otherVertexVskip\multiVertex{58} = w}%
\begin{tikzpicture}
\path (0cm,0cm) graph[graph style] {
 root[root style] -> {
  a/{\vWrapPlus} [> "\eWrapPlus"] -> {
   "\vSimpleTimes" [> "\eNestedPartsBob"'] -> {
    {},
    b/"\vNestedPartsAlice" [> "\eMultiParentAlice"', >merge edge]
   },
   {}
  }
 },
};
\path [draw,->,merge edge] (a) to ["\eMultiParentBob",out=-60,in=60] (b);
\end{tikzpicture}
\end{subfigure}
%%%%%%%%
\hfill
%%%%%%%%
\begin{subfigure}[t]{0.13\linewidth}
\centering
\caption{
\begin{tikzpicture}[remember picture, overlay]
\path (0cm,0cm) node (node/MultiParent/e) [anchor=base] {\strut};
\path [draw,->,alice step] (node/MultiParent/d) to (-2em, 0cm |- node/MultiParent/e);
\end{tikzpicture}
}%
\label{fig:MultiParent:e}
\figureCode{\hole{} * w + \hole{}\otherVertexVskip\phantom{\multiVertex{58} = w}}%
\begin{tikzpicture}
\path (0cm,0cm) graph[graph style] {
 root[root style] -> {
  v8/"\vWrapPlus" [> "\eWrapPlus"] -> {
   v2/"\vSimpleTimes" [> "\eNestedPartsBob"'] -> {
    {},
    "\vNestedPartsAlice" [> "\eMultiParentAlice"']
   },
   {}
  }
 }
};
\end{tikzpicture}
\end{subfigure}
%%%%%%%%
\hfill{}
\caption{(a) We start in a state with a variable, \li{w}, and two holes. (b) Alice relocates \li{w} to the left hole. (c) Bob relocates \li{w} to the right hole. (d) After merging, vertex \vNestedPartsAlice{} has two incoming edges, i.e. it has a \emph{relocation conflict}. The corresponding decomposition leaves a \emph{relocation conflict reference} at both locations, partially addressing the \textbf{relocation conflict problem}. Terms that have a relocation conflict are tracked separately by decomposition. (e) The relocation conflict can be resolved by deleting all but one reference.}%
\Description{This figure depicts an example of relocation conflicts}
\label{fig:MultiParent}
\end{figure*}
}

%%%%%%%%%%%%%%%%%%%%%%%%%%%%%%%%
%% \figureCycle
%%%%%%%%%%%%%%%%%%%%%%%%%%%%%%%%
\newedge{MultiCycleTimes}{L}
\newvertex{MultiCycleTimes}{*}
\newedge{MultiCyclePlus}{R}
\newvertex{MultiCyclePlus}{+}
\newedge{MultiCycleAliceTimes}{R}
\newedge{MultiCycleAlicePlus}{L}
\newedge{MultiCycleBobPlus}{R}
\newedge{MultiCycleBobTimes}{L}
\newcommand{\figureCycle}{
\begin{figure*}
\hfill
% \hskip0.07\columnwidth
%%%%%%%%
\begin{subfigure}[t]{0.25\textwidth}
\centering
\caption{
\begin{tikzpicture}[remember picture, overlay]
% \path (-9em,0cm) node (node/MultiParent/ex) [gray,anchor=base west] {{\textbf{\small Fig.~\ref*{fig:MultiParent:e}}}};
%%%%
\path (0cm,0cm) node (node/Cycle/a) [anchor=base] {\strut};
% \path [draw,->,alice step] (node/MultiParent/ex) to[star] (-2em, 0cm |- node/Cycle/a);
\end{tikzpicture}
}%
\label{fig:Cycle:a}
\figureCode{\parens{\hole{} * \hole{}} * \parens{\hole{} + \hole{}} + \hole{}%
  \otherVertexVskip\phantom{\multiVertex{30} = \multiVertex{32} * \hole{}}%
  \\\phantom{\multiVertex{32} = \multiVertex{30} * \hole{}}}%
\begin{tikzpicture}
\path (0cm,0cm) graph[graph style] {
 root[root style] -> {
  "\vWrapPlus" [> "\eWrapPlus"] -> {
   "\vSimpleTimes" [> "\eNestedPartsBob"'] -> {
    "\vMultiCycleTimes" [> "\eMultiCycleTimes"'],
    "\vMultiCyclePlus" [> "\eMultiCyclePlus"]
   },
   {}
  }
 }
};
\end{tikzpicture}
\end{subfigure}
%%%%%%%%
\hfill
%%%%%%%%
\begin{subfigure}[t]{0.25\textwidth}
\centering
\caption{
\begin{tikzpicture}[remember picture, overlay]
\path (0cm,0cm) node (node/Cycle/b) [anchor=base] {\strut};
\path [draw,->,alice step] (node/Cycle/a) to[star] (-2em, 0cm |- node/Cycle/b);
\end{tikzpicture}
}%
\label{fig:Cycle:b}
\figureCode{\hole{} * \hole{} + \parens{\hole{} + \hole{}} * \hole{}%
  \otherVertexVskip\phantom{\multiVertex{30} = \multiVertex{32} * \hole{}}%
  \\\phantom{\multiVertex{32} = \multiVertex{30} * \hole{}}}%
\begin{tikzpicture}
\path (0cm,0cm) graph[graph style] {
 root[root style] -> {
  "\vWrapPlus" [> "\eWrapPlus"] -> {
   "\vSimpleTimes" [> "\eNestedPartsBob"'] -> {
   },
   {
    "\vMultiCycleTimes" [> "\eMultiCycleAliceTimes", >alice edge, alice node] -> {
     "\vMultiCyclePlus" [> "\eMultiCycleAlicePlus"', >alice edge, alice node],
     {}
    }
   }
  }
 }
};
\end{tikzpicture}
\end{subfigure}
%%%%%%%%
\hfill
%%%%%%%%
\begin{subfigure}[t]{0.25\textwidth}
\centering
\caption{
\begin{tikzpicture}[remember picture, overlay]
\path (0cm,0cm) node (node/Cycle/c) [anchor=base] {\strut};
\path [draw,->,bob step] (node/Cycle/a) to[out=15,in=165,star] (-2em, 0cm |- node/Cycle/c);
\end{tikzpicture}
}%
\label{fig:Cycle:c}
\figureCode{\hole{} * \hole{} + \parens{\hole{} * \hole{} + \hole{}}%
  \otherVertexVskip\phantom{\multiVertex{30} = \multiVertex{32} * \hole{}}%
  \\\phantom{\multiVertex{32} = \multiVertex{30} * \hole{}}}%
\begin{tikzpicture}
\path (0cm,0cm) graph[graph style] {
 root[root style] -> {
  "\vWrapPlus" [> "\eWrapPlus"] -> {
   "\vSimpleTimes" [> "\eNestedPartsBob"'] -> {
   },
   {
    "\vMultiCyclePlus" [> "\eMultiCycleBobPlus", >bob edge, bob node] -> {
     "\vMultiCycleTimes" [> "\eMultiCycleBobTimes"', >bob edge, bob node],
     {}
    }
   }
  }
 }
};
\end{tikzpicture}
\end{subfigure}
%%%%%%%%
\hfill
%%%%%%%%
\begin{subfigure}[t]{0.18\textwidth}
\centering
\caption{
\begin{tikzpicture}[remember picture, overlay]
\path (0cm,0cm) node (node/Cycle/d) [anchor=base] {\strut};
\path [draw,->,merge step] (node/Cycle/b) to[out=15,in=165] (-2em, 0cm |- node/Cycle/d);
\path [draw,->,merge step] (node/Cycle/c) to (-2em, 0cm |- node/Cycle/d);
\end{tikzpicture}
}%
\label{fig:Cycle:d}
\figureCode{\parens{\hole{} * \hole{}} + $\binaryConflictHole{\multiVertex{\vidMultiCycleTimes}}{\multiVertex{\vidMultiCyclePlus}}$%
  \otherVertexVskip\multiVertex{\vidMultiCycleTimes} = \multiVertex{\vidMultiCyclePlus} * \hole{}%
  \\\multiVertex{\vidMultiCyclePlus} = \multiVertex{\vidMultiCycleTimes} * \hole{}}%
\begin{tikzpicture}
\path (0cm,0cm) graph[graph style] {
 root[root style] -> {
  "\vWrapPlus" [> "\eWrapPlus"] -> {
   "\vSimpleTimes" [> "\eNestedPartsBob"'anchor=-15] -> {
   },
   v1/"\vMultiCycleTimes" [> "\eMultiCycleAliceTimes"anchor=155, >merge edge, merge node],
   {},
   v2/"\vMultiCyclePlus" [> "\eMultiCycleBobPlus"anchor=-165, >merge edge, merge node]
  }
 }
};
\path[draw,-{>[bend]},merge edge] (v1) to ["\eMultiCycleAlicePlus"anchor=south,out=-60,in=-120] (v2);
\path[draw,-{>[bend]},merge edge] (v2) to ["\eMultiCycleBobTimes"anchor=north,out=-90,in=-90] (v1);
\end{tikzpicture}
\end{subfigure}
%%%%%%%%
%\hfill{}
\caption{(a) We start with a tree with a multiplication,  \vMultiCycleTimes{}, and addition, \vMultiCyclePlus{}, at the leaves. (b) Alice relocates them both, such that \vMultiCycleTimes{} is the parent of \vMultiCyclePlus{}. (c) Bob relocates them both, such that \vMultiCyclePlus{} is the parent of the \vMultiCycleTimes{}. (d) In the merged state, there is a cycle in the graph. Because the terms have a common parent, there is a local conflict. Because the cycle is connected to the rest of the graph, the cycle is broken during decomposition by relocation conflict references as shown.}
% TODO replace screenshots with text
% TODO make ref code gray
% TODO move whole figure left
% TODO remove cursor from screenshots
% TODO add multiparent box
% TODO only show parens where needed - figure 10, 11 - by associativity
% TODO repplace cycle screenshot with cycles box %
\Description{This figures shows an example of a cycle}
\label{fig:Cycle}
\end{figure*}
}

%%%%%%%%%%%%%%%%%%%%%%%%%%%%%%%%
%% \figureDisconnect
%%%%%%%%%%%%%%%%%%%%%%%%%%%%%%%%
\newedge{DisconnectAlice}{R}
\newedge{DisconnectBob}{L}
\newcommand{\figureDisconnect}{
\begin{figure*}
%\hfill
\hskip0.12\columnwidth
%%%%%%%%
\begin{subfigure}[t]{0.21\textwidth}
\centering
\caption{
\begin{tikzpicture}[remember picture, overlay]
\path (-7.5em,0cm) node (node/Cycle/dx) [gray,anchor=base west] {{\textbf{\small Fig.~\ref*{fig:Cycle:d}}}};
%%%%
\path (0cm,0cm) node (node/Disconnect/a) [anchor=base] {\strut};
\path [draw,->,alice step] (node/Cycle/dx) to[star] (-2em, 0cm |- node/Disconnect/a);
\end{tikzpicture}
}%
\label{fig:Disconnect:a}
\figureCode{\hole{} * \hole{} + \hole{} * \hole{}\otherVertexVskip\strut}%
\begin{tikzpicture}
\path (0cm,0cm) graph[graph style] {
 root[root style] -> {
  "\vWrapPlus" [> "\eWrapPlus"] -> {
   "\vSimpleTimes" [> "\eNestedPartsBob"'],
   "\vMultiCycleTimes" [> "\eMultiCycleAliceTimes"]
  }
 }
};
\end{tikzpicture}
\end{subfigure}
%%%%%%%%
\hfill
%%%%%%%%
\begin{subfigure}[t]{0.21\textwidth}
\centering
\caption{
\begin{tikzpicture}[remember picture, overlay]
\path (0cm,0cm) node (node/Disconnect/b) [anchor=base] {\strut};
\path [draw,->,alice step] (node/Disconnect/a) to[star] (-2em, 0cm |- node/Disconnect/b);
\end{tikzpicture}
}%
\label{fig:Disconnect:b}
\figureCode{\hole{} * \parens{\hole{} * \hole{}} + \hole{}\otherVertexVskip\strut}%
\begin{tikzpicture}
\path (0cm,0cm) graph[graph style] {
 root[root style] -> {
  "\vWrapPlus" [> "\eWrapPlus"] -> {
   "\vSimpleTimes" [> "\eNestedPartsBob"'] -> {
    {},
    "\vMultiCycleTimes" [> "\eDisconnectAlice", >alice edge]
   },
   {}
  }
 }
};
\end{tikzpicture}
\end{subfigure}
%%%%%%%%
\hfill
%%%%%%%%
\begin{subfigure}[t]{0.21\textwidth}
\centering
\caption{
\begin{tikzpicture}[remember picture, overlay]
\path (0cm,0cm) node (node/Disconnect/c) [anchor=base] {\strut};
\path [draw,->,bob step] (node/Disconnect/a) to[out=15,in=165,star] (-2em, 0cm |- node/Disconnect/c);
\end{tikzpicture}
}%
\label{fig:Disconnect:c}
\figureCode{\hole{} + \parens{\hole{} * \hole{}} * \hole{}\otherVertexVskip\strut}%
\begin{tikzpicture}
\path (0cm,0cm) graph[graph style] {
 root[root style] -> {
  "\vWrapPlus" [> "\eWrapPlus"] -> {
   {},
   "\vMultiCycleTimes" [> "\eMultiCycleAliceTimes"] -> {
    "\vSimpleTimes" [> "\eDisconnectBob"', >bob edge],
    {}
   }
  }
 }
};
\end{tikzpicture}
\end{subfigure}
%%%%%%%%
\hfill
%%%%%%%%
\begin{subfigure}[t]{0.21\textwidth}
\centering
\caption{
\begin{tikzpicture}[remember picture, overlay]
\path (0cm,0cm) node (node/Disconnect/d) [anchor=base] {\strut};
\path [draw,->,merge step] (node/Disconnect/b) to[out=15,in=165] (-2em, 0cm |- node/Disconnect/d);
\path [draw,->,merge step] (node/Disconnect/c) to (-2em, 0cm |- node/Disconnect/d);
\end{tikzpicture}
}%
\label{fig:Disconnect:d}
\figureCode{\hole{} + \hole{}\otherVertexVskip\cycleVertex{2} = \hole{} * \parens{\cycleVertex{2} * \hole{}}}%
\begin{tikzpicture}
\path (0cm,0cm) graph[graph style] {
 root[root style] -> {
  "\vWrapPlus" [> "\eWrapPlus"]
 }
};
\path (-0.5cm,-2cm) graph[graph style] { plus/"\vSimpleTimes" };
\path ( 0.5cm,-2cm) graph[graph style] { times/"\vMultiCycleTimes" };
\path[draw,->,merge edge] (times) to ["\eMultiCycleBobTimes"'anchor=north,out=-135,in=-45] (plus);
\path[draw,->,merge edge] (plus) to ["\eMultiCycleBobPlus"anchor=south,out=45,in=135] (times);
\end{tikzpicture}
\end{subfigure}
%%%%%%%%
\hfill{}
\caption{Example of Disconnection.}%
\Description{This figure shows an example of a disconnected component}
\label{fig:Disconnection}
\end{figure*}
}

% \newvertex{DecompPlusA}{+}
\newedge{DecompTimesA}{L}
\newvertex{DecompTimesA}{*}
\newedge{DecompPlusAX}{R}
\newvertex{DecompX}{x}
\newedge{DecompTimesAX}{R}
%
\newedge{DecompPlusB}{L}
\newvertex{DecompPlusB}{+}
\newedge{DecompTimesB}{R}
\newvertex{DecompTimesB}{*}
\newedge{DecompY}{R}
\newvertex{DecompY}{y}
\newedge{DecompZ}{R}
\newvertex{DecompZ}{z}
%
\newedge{DecompTimes}{L}
%
\newcommand{\figureDecompExample}{
\begin{figure}
  \centering
  \begin{subfigure}{0.5\textwidth}
    \centering
    \begin{tikzpicture}
      \path (0cm,0cm) graph[graph style] {
        root[root style] -> 
        {
        a/{\vDecompPlusA} [> "\eSimpleTimes"] -> {
          "\vDecompTimesA" [> "\eDecompTimesA"'] -> {
          {},
          b/"\vDecompX" [> "\eDecompTimesAX"']
          },
          {}
        }
        },
      };
      \path [draw,->] (a) to ["\eDecompPlusAX",out=-60,in=60] (b);
      %
      \path (2cm,-0.85cm) graph[graph style] { plus/"\vDecompPlusB" };
      \path (3cm,-0.85cm) graph[graph style] { times/"\vDecompTimesB" };
      \path (3.25cm,-1.75cm) graph[graph style] { y/"\vDecompY" };
      \path[draw,->] (times) to ["\eDecompPlusB"'anchor=north,out=-135,in=-45] (plus);
      \path[draw,->] (plus) to ["\eDecompTimesB"anchor=south,out=45,in=135] (times);
      \path[draw,->] (times) to ["\eDecompY"anchor=west] (y);
      %
      \path (1.55cm,-2.1cm) graph[graph style] { plus/"\vDecompZ" };
    \end{tikzpicture}
    \caption{A graph}%
    \label{fig:Decomposition example graph}
  \end{subfigure}%
  \hspace{-18}
  \begin{subfigure}{.5\textwidth}
    \centering
    \begin{align*}
      \Theta = 
      \{ & (\emptyHole{(40,\texttt{*},L)}~\texttt{*}^{\id{40}}~(\eid{43}{}\multiVertex{43,(42,x)}))~\texttt{+}^{\id{38}}~(\eid{41}{}\multiVertex{41,(42,x)}), \\
      & \eid{52}{z}, \eid{42}{x}, \emptyHole{(46,\texttt{+},L)}~\texttt{+}^{\id{46}}~(\eid{45}{}\cycleVertex{45,(46,\texttt{+})}~\texttt{*}^{\id{48}}~(\eid{49}{}y^{\id{50}}))\} \\
    \end{align*}
    \caption{The grove corresponding to the graph}%
    \label{fig:Decomposition example grove}
  \end{subfigure}
  \begin{subfigure}{.5\textwidth}
    \centering
    \begin{align*}
      \text{t}_r & = \eid{1}{}  (\emptyHole{(40,\texttt{*},L)}~\texttt{*}^{\id{40}}~(\eid{43}{}\multiVertex{43,(42,x)}))~\texttt{+}^{\id{38}}~(\eid{41}{}\multiVertex{41,(42,x)})\\
      \text{NP} & = \{ \eid{52}{z} \} \\
      \text{MP} & = \{ \eid{42}{x} \} \\
      \text{U} & = \{ \emptyHole{(46,\texttt{+},L)}~\texttt{+}^{\id{46}}~(\eid{45}{}\cycleVertex{45,(46,\texttt{+})}~\texttt{*}^{\id{48}}~(\eid{49}{}y^{\id{50}})) \}
    \end{align*}
    \caption{The partitioned grove corresponding to the graph}%
    \label{fig:Decomposition example partitioned grove}
  \end{subfigure}
  \caption{Example of graph decomposition}%
  \Description{This figure describes the decomposition of a graph to its corresponding grove}
  \label{fig:Decomposition example}
\end{figure}
}

% \newcommand{\termMV}{t}
\newcommand{\subtermMV}{\overline{t}}

\newcommand{\figureTermSyntaxContent}{%

\[
\begin{array}{lclll}
     \termMV \in & Term & \coloneqq & 
        \eid{u}{\K}\ \subtermMV_1 ... \subtermMV_n
        \mid \multiref
        \mid \uniref \\
     \subtermMV \in & SubTerm & \coloneqq & 
        \ehole
        \mid \lexp{\termMV} 
        \mid \conflict{\termMV} \\  
\end{array}
\]
}

\newcommand{\figureTermSyntax}{%
\begin{figure}
  \figureTermSyntaxContent
  \caption{Syntax of terms}
  \Description{This figure describes the grammar for the syntax of generic terms}
  \label{fig:Syntax}
\end{figure}%
}
% \newcommand{\figureArityContent}{%
\[
  \arraycolsep=0pt
  \begin{array}{ll}
    \multicolumn{2}{l}{\arityOp : \K \rightarrow \wp(\P \times \{Exp, Pat, Typ\})} \\
    \hline
    \arity{\Root}={} & \left\{ (\Root, Exp) \right\} \\
    \arity{\PatVar(x)}={} & \left\{ \right\} \\
    \arity{\ExpVar(x)}={} & \left\{ \right\} \\
    \arity{\ExpLam}={} & \left\{ (\LamParam, Pat), (\mathtt{Type}, Typ), (\mathtt{Body}, Exp) \right\} \\
    \arity{\ExpApp}={} & \left\{ (\AppFun, Exp), (\AppArg, Exp) \right\} \\
    \arity{\ExpPlus}={} & \left\{ (\PlusLeft, Exp), (\PlusRight, Exp) \right\} \\
    \arity{\ExpTimes}={} & \left\{ (\PlusLeft, Exp), (\PlusRight, Exp) \right\} \\
    \arity{\ExpNum(n)}={} & \left\{ \right\} \\
    \arity{\TypArrow}={} & \left\{ (\ArrowArg, Typ), (\ArrowResult, Typ) \right\} \\
    \arity{\TypNum}={} & \left\{ \right\} \\
  \end{array}
\]%
}

\newcommand{\figureArity}{%
\begin{figure}
\figureArityContent
\caption{Constructors, Indexes and Arity}
\Description{This figure shows the mapping from constructors to their corresponding indexes and arity. For each constructor, such as \ExpLam or \ExpApp, the arity is defined in terms of the parts that make up the constructor. For example, \ExpLam has three parts: a pattern (\LamParam), a type (\mathtt{Type}), and a body (\mathtt{Body}), each with their corresponding expression, pattern, or type.}
\label{fig:Constructors, Indexes and Arity}
\end{figure}%
}
% 
\newcommand{\figureDecompositionDefHelpersContent}{%
\begin{align*}
  \ingraph{v} &= \SetOf{\e{=}(u, v', p, v) \mapsto G(\e) \SuchThat{\e \in \edges{G}}} \\
  \outedges{v}{p} &= \SetOf{\e{=}(u, v, p, v') \SuchThat{G(\e) = \Plus}} \\
  \parents{v} &= \SetOf{v' \SuchThat{\exists \e{=}(u, v', p, v) \land G(\e) = \Plus}} \\
  \ancestors{v} &= \mathopen{}\left( \lfp{\ancestorsPrimeOp} \right)\mathclose{}(v) \\
  \ancestorsPrime{v} &= \parents{v} \cup \ancestorsPrime{\parents{v}} \\
  \min{(u_1, k_1), \ldots, (u_n, k_n)} &= (u_j, k_j) \text{ s.t. } 1 \leq j \leq n \land u_j \le u_i \forall i = 1, \ldots, n \\
  \edges{G} &= \SetOf{\e \SuchThat{G(\e) \in \SetOf{\Plus, \Minus}}} \\
  \sources{G_t} &= \SetOf{(v, p) \SuchThat{(u, v, p, v_{G_t}) \in \edges{G_t} \land u \in \U}}
\end{align*}%
}

%%%%%%%%%%%%%%%%%%%%%%%%%%%%%%%%%%%%%%%%%%%%%%%%%%%%%%%%%%%%%%%%%%%%%%%%%%%%%%%%

\newcommand{\figureDecompositionDefDecomp}{%
\begin{align*}
  \decomp{G} &= (\Set[NP], \Set[MP], \Set[U]) \\
  \figureDecompositionDefDecompComponents
\end{align*}%
}

\newcommand{\figureDecompositionDefDecompComponents}{%
  \Set[NP] &= \SetOf{\decomp{(u, v, p, v')} \SuchThat{|\parents{v'}| = 0}} \\
  \Set[MP] &= \SetOf{\decomp{(u, v, p, v')} \SuchThat{|\parents{v'}| > 1}} \\
  \Set[U] &= \SetOf{\decomp{(u, v, p, v')} \SuchThat{|\parents{v'}| = 1 \land v' = \min{\ancestors{v'}}}}%
}

\newcommand{\figureDecompositionDefDecompTerm}{%
\begin{align*}
  \decomp{\e{=}(u, v, p, (u', k))} &= \begin{cases}
    \edecomp{\e} &\quad \sort{k} = Exp \\
    \pdecomp{\e} &\quad \sort{k} = Pat \\
    \tdecomp{\e} &\quad \sort{k} = Typ \\
  \end{cases}
\end{align*}
}

\newcommand{\figureDecompositionDefEdecomp}{%
\begin{align*}
  \edecomp{\e{=}(u, v, p, v'{=}(u', \ExpVar(x)))} &= \eVar{\ingraph{v'}}{x} \\
  \edecomp{\e{=}(u, v, p, v'{=}(u', \ExpLam))} &= \eFun{\ingraph{v'}}{q}{\tau}{e} \\
  & q = \pdecompPrime{\e}{\LamParam} \\
  & \tau = \tdecompPrime{\e}{\LamType} \\
  & e = \edecompPrime{\e}{\LamBody} \\
  \edecomp{\e{=}(u, v, p, v'{=}(u', \ExpApp))} &= \eApp{\ingraph{v'}}{e_\AppFun}{e_\AppArg} \\
  & e_\AppFun = \edecompPrime{\e}{\AppFun} \\
  & e_\AppArg = \edecompPrime{\e}{\AppArg} \\
  \edecomp{\e{=}(u, v, p, v'{=}(u', \ExpNum(n)))} &= \eNum{\ingraph{v'}}{n} \\
  \edecomp{\e{=}(u, v, p, v'{=}(u', \ExpPlus))} &= \ePlus{\ingraph{v'}}{e_\PlusLeft}{e_\PlusRight} \\
  & e_\PlusLeft = \edecompPrime{\e}{\PlusLeft} \\
  & e_\PlusRight = \edecompPrime{\e}{\PlusRight} \\
  \edecomp{\e{=}(u, v, p, v'{=}(u', \ExpTimes))} &= \eTimes{\ingraph{v'}}{e_\TimesLeft}{e_\TimesRight} \\
  & e_\TimesLeft = \edecompPrime{\e}{\TimesLeft} \\
  & e_\TimesRight = \edecompPrime{\e}{\TimesRight}
\end{align*}%
}

\newcommand{\figureDecompositionDefPdecomp}{%
\begin{align*}
  \pdecomp{\e{=}(u, v, p, v'{=}(u', \PatVar(x)))} &= \pVar{\ingraph{v'}}{x}
\end{align*}%
}

\newcommand{\figureDecompositionDefTdecomp}{%
\begin{align*}
  \tdecomp{\e{=}(u, v, p, v'{=}(u', \TypArrow))} &= \tArrow{\ingraph{v'}}{\tau_\ArrowArg}{\tau_\ArrowResult} \\
  & \tau_\ArrowArg = \tdecompPrime{\e}{\ArrowArg} \\
  & \tau_\ArrowResult = \tdecompPrime{\e}{\ArrowResult} \\
  \tdecomp{\e{=}(u, v, p, v'{=}(u', \TypNum))} &= \tNum{\ingraph{v'}}
\end{align*}%
}

\newcommand{\figureDecompositionDefEdecompPrime}{%
\[
  \begin{array}{l}
    \edecompPrime{\e{=}(u, v, p, v')}{p'} = \\
    \left\{
      \begin{array}{ll}
        \conflictHole{\edecomp{\e_i}}_{i \leq n} &
          \outedges{v'}{p'} = \SetOf{\e_1, \ldots, \e_n}
        \\
        \multiVertex{\e'} &
          \outedges{v'}{p'} = \SetOf{\e'{=}(u_1, v_1, p_1, v_1')}
          \land |\parents{v_1'}| > 1
        \\
        \cycleVertex{\e'} &
          \outedges{v'}{p'} = \SetOf{\e'{=}(u_1, v_1, p_1, v_1')}
          \land |\parents{v_1'}| = 1
          \\ & \quad
          \land v_1' = \min{\ancestors{v_1'}}
        \\
        \edecomp{\e'} &
          \outedges{v'}{p'} = \SetOf{\e'{=}(u_1, v_1, p_1, v_1')}
          \land |\parents{v_1'}| = 1
          \\ & \quad
          \land v_1' \neq \min{\ancestors{v_1'}}
        \\
        \emptyHole{v'}{p'} & \outedges{v'}{p'} = \varnothing \\
      \end{array}
    \right.
  \end{array}
\]%
}

\newcommand{\figureDecompositionDefPdecompPrime}{%
\[
  \begin{array}{l}
    \pdecompPrime{\e{=}(u, v, p, v')}{p'} = \\
    \left\{
      \begin{array}{ll}
        \conflictHole{\pdecomp{\e_i}}_{i \leq n} &
          \outedges{v'}{p'} = \SetOf{\e_1, \ldots, \e_n}
        \\
        \multiVertex{\e'} &
          \outedges{v'}{p'} = \SetOf{\e'{=}(u_1, v_1, p_1, v_1')}
          \land |\parents{v_1'}| > 1
        \\
        \cycleVertex{\e'} &
          \outedges{v'}{p'} = \SetOf{\e'{=}(u_1, v_1, p_1, v_1')}
          \land |\parents{v_1'}| = 1
          \\ & \quad
          \land v_1' = \min{\ancestors{v_1'}}
        \\
        \pdecomp{\e'} &
          \outedges{v'}{p'} = \SetOf{\e'{=}(u_1, v_1, p_1, v_1')}
          \land |\parents{v_1'}| = 1
          \\ & \quad
          \land v_1' \neq \min{\ancestors{v_1'}}
        \\
        \emptyHole{v'}{p'} & \outedges{v'}{p'} = \varnothing \\
      \end{array}
    \right.
  \end{array}
\]%
}

\newcommand{\figureDecompositionDefTdecompPrime}{%
\[
  \begin{array}{l}
    \tdecompPrime{\e{=}(u, v, p, v')}{p'} = \\
    \left\{
      \begin{array}{ll}
        \conflictHole{\tdecomp{\e_i}}_{i \leq n} &
          \outedges{v'}{p'} = \SetOf{\e_1, \ldots, \e_n}
        \\
        \multiVertex{\e'} &
          \outedges{v'}{p'} = \SetOf{\e'{=}(u_1, v_1, p_1, v_1')}
          \land |\parents{v_1'}| > 1
        \\
        \cycleVertex{\e'} &
          \outedges{v'}{p'} = \SetOf{\e'{=}(u_1, v_1, p_1, v_1')}
          \land |\parents{v_1'}| = 1
          \\ & \quad
          \land v_1' = \min{\ancestors{v_1'}}
        \\
        \tdecomp{\e'} &
          \outedges{v'}{p'} = \SetOf{\e'{=}(u_1, v_1, p_1, v_1')}
          \land |\parents{v_1'}| = 1
          \\ & \quad
          \land v_1' \neq \min{\ancestors{v_1'}}
        \\
        \emptyHole{v'}{p'} & \outedges{v'}{p'} = \varnothing \\
      \end{array}
    \right.
  \end{array}
\]%
}

%%%%%%%%%%%%%%%%%%%%%%%%%%%%%%%%%%%%%%%%%%%%%%%%%%%%%%%%%%%%%%%%%%%%%%%%%%%%%%%%

\newcommand{\figureDecompositionDef}{%
\begin{figure}

\figureDecompositionDefEdecomp

\figureDecompositionDefPdecomp

\figureDecompositionDefTdecomp

\figureDecompositionDefEdecompPrime

\figureDecompositionDefPdecompPrime

\figureDecompositionDefTdecompPrime

\caption{Graph decomposition.}
\label{fig:Graph decomposition definition}
\end{figure}%
}

\newcommand{\figureDecompositionDefHelpers}{%
\begin{figure}
\figureDecompositionDefHelpersContent
\caption{Graph decomposition helpers.}
\label{fig:Graph decomposition definition helpers}
\end{figure}%
}

\begin{document}

%% Title information
% \title[Grove]{Convergent Collaborative Structure Editing}
\title[Grove]{Grove: A Convergent Collaborative Structure-Editor Calculus}
%\subtitle{Subtitle}

%% Author information
\author{Michael D. Adams}
\orcid{0000-0003-3160-6972}

\author{Eric Griffis}
\orcid{0000-0003-1693-6172}

\author{Cyrus Omar}
\orcid{0000-0003-4502-7971}
\affiliation{
  %\position{Assistant Research Scientist}
  \department[0]{Computer Science and Engineering}
  \department[1]{Electrical Engineering and Computer Science}
  \department[2]{College of Engineering}
  \institution{University of Michigan}
  \streetaddress{Bob and Betty Beyster Building, 2260 Hayward Street}
  \city{Ann Arbor}
  \state{MI}
  \postcode{48109-2121}
  \country{USA}
}


%% Abstract
%% Note: \begin{abstract}...\end{abstract} environment must come
%% before \maketitle command
\begin{abstract}
  Text of abstract \ldots.
\end{abstract}


%% 2012 ACM Computing Classification System (CSS) concepts
%% Generate at 'http://dl.acm.org/ccs/ccs.cfm'.
%\begin{CCSXML}
%<ccs2012>
%<concept>
%<concept_id>10011007.10011006.10011008</concept_id>
%<concept_desc>Software and its engineering~General programming languages</concept_desc>
%<concept_significance>500</concept_significance>
%</concept>
%<concept>
%<concept_id>10003456.10003457.10003521.10003525</concept_id>
%<concept_desc>Social and professional topics~History of programming languages</concept_desc>
%<concept_significance>300</concept_significance>
%</concept>
%</ccs2012>
%\end{CCSXML}
%\ccsdesc[500]{Software and its engineering~General programming languages}
%\ccsdesc[300]{Social and professional topics~History of programming languages}
%% End of generated code


%% Keywords
%% comma separated list
% \keywords{keyword1, keyword2, keyword3}  %% \keywords are mandatory in final camera-ready submission


%% \maketitle
%% Note: \maketitle command must come after title commands, author
%% commands, abstract environment, Computing Classification System
%% environment and commands, and keywords command.
\maketitle


\section{Introduction}%
\label{sec:Introduction}

%%%%%%%%%%%%%%%%%%%%%%%%%%%%%%%%%%%%%%%%%%%%%%%%%%%%%%%%%%%%%%%%%%%%%%%%%%%%%%%%

Programming is an increasingly collaborative activity.
Today, most programming editors are not explicitly designed with collaboration in mind,
so programmers rely on ad hoc integrations of asynchronous diff-based version control systems like Git, Darcs, or Pijul.

Asynchronous diffing is not an ideal solution to collaboration because
the merging process must solve the \textbf{inverse problem}---inference of compatible edits in a simplistic edit action language.
For example, suppose Alice and Bob are collaboratively editing a file comprised of a single line \verb|A B|.
Alice changes the file to \verb|A B C| and communicates the change to Bob as a diff (Figure \ref{fig:inverse-problem:alice}).
At the same time, Bob changes the file to \verb|A D B| and sends Alice a diff (Figure \ref{fig:inverse-problem:bob}).
Both parties changed the same line, but in a different way,
so when Bob applies Alice's diff, a merge conflict arises and the contents of Bob's editor is mangled as in Figure \ref{fig:inferse-problem:diff}.

\begin{figure}[H]
  \begin{subfigure}{.45\linewidth}
\begin{verbatim}
1c1
< A B
---
> A B C
\end{verbatim}
    \label{fig:inverse-problem:alice}
    \subcaption{From Alice to Bob}
  \end{subfigure}
  \hfil
  \begin{subfigure}{.45\linewidth}
\begin{verbatim}
1c1
< A B
---
> A D B
\end{verbatim}
    \label{fig:inverse-problem:bob}
    \subcaption{From Bob to Alice}
  \end{subfigure}

  \begin{subfigure}{.45\linewidth}
\begin{verbatim}
<<<<<<< Alice
A B C
||||||| base
A B
=======
A D B
>>>>>>> Bob
\end{verbatim}
    \label{fig:inverse-problem:diff}
    \subcaption{Bob's editor (diff)}
  \end{subfigure}
  \hfil
  \begin{subfigure}{.45\linewidth}
\begin{verbatim}
A { B | D} { C | B}
\end{verbatim}
    \label{fig:inverse-problem:grove}
    \subcaption{Bob's editor (Grove)}
  \end{subfigure}
  \label{fig:inverse-problem}
  \caption{The effects of diff-and-merge on conflicting edit states}
\end{figure}

An alternative to solving the inverse problem is to build the editor on top of a distributed collaboration protocol such as OT or,
more recently, to model the shared edit state as a CRDT---a data structure implemented behind a fixed set of commutative update operations.
Commutativity is desirable because it reduces collaboration to a simple matter of record and replay.
Since CRDT imposes no other design constraints, it is potentially better suited for collaborative program editing than diff-and-merge.
For instance, the \textbf{conflict problem}---how to meaningfully represent conflicts without mangling edit state---%
becomes easier to solve as the language of edits becomes more precise (Figure \ref{fig:inverse-problem:grove}).

Although CRDT may offer a more natural medium for collaborative programming than diffs,
care must be taken to ensure the preservation of user intent.
Suppose Alice and Bob are collaboratively editing the line \verb|A B| with Alice's cursor on \verb|A| and Bob's on \verb|B|.
If Alice moves \verb|A| to the other side of \verb|B|, it isn't obvious where Bob's cursor should go.
Naively, since the cursor was originally on the second element (\verb|B|), it could simply stay there,
which in this case means it winds up on \verb|A|.
Realistically, \verb|A| and \verb|B| could be very long functions, in which case Bob's cursor would suddenly appear to move a great distance.
If Bob's cursor does not need to react to other users' activity, there is no risk of a disorienting jump.
One way to solve this \textbf{repositioning problem} and affect confusion-free movement is to uniquely identify every piece of editable content.

To address these problems we propose \emph{Grove}, a CRDT-backed collaborative structure editor.
Grove represents the edit state as a labeled, directed multigraph with vertex insertion and deletion operations such that deletion always wins.
In the graph, conflicts have a natural representation as edges with common origin or destination vertices.
Each label includes a unique identifier which enables not only a solution to the repositioning problem
but also a means to construct symbolic references into the edit state.
By leveraging an isomorphism between the graph and its corresponding \emph{grove}---%
a collection of disjoint forests connected via symbolic reference---%
views of the graph may be projected into a tree-native setting, for instance as terms in the Hazel editor,
without adding compexity to the underlying collaboration model.

% We are further motivated by the pursiut of a collaborative extension for Hazel, an integrated programming language and structure editor
% that provides semantic editor services by enforcing a sensibility invariant, i.e., every edit state is semantically valid,
% an interesting avenue of future work with potential implications for the scalability of collaborative programming editors.

% do this instead:
%   the standard method is diff-based version control.
%   diff is bad, evene for what it's indended for.
%   here's why: solving inverse problem ==> have to infer edits
%   alternative approach is to record the edits directly and then form a CRDT s.t. collaboration can be accomplished by sending your edits to your collaborators
%   primarily been explored in the setting of real time text editing
%   there are some aspects of this approach that don't make it suitable for programming:
%   1. the repositioning problem
%   2. the conflict problem (even json crdt does ad hoc conflict resolution)

% Note: you'd still want to solve these problems for text as well, but we're interested in making this work for Hazel
% Hazel is good for this because it already has a rich calculus of edit actions -- motivates the tree focus
% Even when it comes to rich text editing, we really wanna be working with trees: due to markup

% move to related work

% Operational Transformation (OT), on the other hand, goes perhaps too far in the opposite direction.
% OT is a distributed system architecture that coordinates the proliferation and maintainance of edits and edit states.
% Industrial-scale collaborative editors, such as Google Docs,
% employ OT for real-time applications requiring strong eventual consistency of causally ordered events.
% It does so by maintaining an event log it transforms upon each new entry in order to preserve the causal ordering.

% The Convergent Replicated Data Type (CRDT) was proposed as a simpler alternative to the OT architecture.
% CRDTs do not require causal ordering maintenance because the language of edits must form a join semilattice,
% thereby ensuring the edit state update function increases monotonically.
% CRDT places fewer constraints than OT on the language of edits,
% so calculi for editing tree-native documents can be simpler to model for CRDT than OT.
% Similarly, by eliminating all but one design constraint, edit calculi for CRDT can be easier to implement and extend.

% Unfortunately, monotonicity merely changes the nature of the complexity rather than eliminating it entirely.
% Although tree-native document editing calculi may ultimately admit simpler designs under CRDT than OT,
% there are no generic recipes for transforming a sequential tree editing calculus into one suitable for CRDT.
% Consequently, progress of CRDT-backed tree editors has slowed as the research community debates whether CRDT is living up to its promise.
% Thus, we are motivated to ask:
% Is there a more natural model for collaborative editing of trees?
% And how do we preserve user intent in the presence of concurrent edits in a tree-native editor?



% algorithms for collaborative text editing are well developed
% natural fit for text
% not a natural fit for trees
% so if you want to build a grove, a more natural approach is needed


% a lot of collaborative editing of other sorts (not necessarily programming) requires a tree model,
% but that problem doesn't seem to have been study directly (maybe except JSON CRDT).

% A \emph{collaborative structure editor} is an editor that
% (1) allows multiple concurrent users to work on a shared document, while also
% (2) providing structure-aware editor services such as projectional editing, syntax highlighting, or automatic code folding.
% %
% Collaborative editing research focuses on the design and implementation of real-time, multi-user, character-based communication systems,
% whereas structure editors typically presume a more complex document schema and then focus on some other aspect of the user experience.
% In both settings, preservation of user intent is a core technical challenge.
% %
% Although collaborative editors and structure editors have overlapping goals (optimal user experience)
% and complementary design challenges (subject-subject versus subject-object harmony),
% to our knowledge, there is no comprehensive, principled account of their combined use.

% Since collaborative editors are essentially distributed systems, existing work tends to focus on extensions to distribution protocols.
% Lots of examples using OT. (Google Docs)
% OT is complex and largely textual.
% OT can make sense for real time systems: users typically change one character at a time, and instant feedback can help to prevent conflicts.
% On the other hand, OT system designs can be difficult to extend.

% Alternatively, there's CRDT. (Peritext?)
% CRDT is easier to implement, but harder to design for.
% There's an All-CRDT editor---it turned out to be not so realistic. (what's it called again?)

% Structure editing has been a recurring theme in the computer science literature since at least Engelbart's ``Mother of All Demos.''
% Provides automation for domain experts and reduces the barrier to entry for everyone else.
% Popular for editing programs, i.e., for programming language-specific editors.

% However, modern program editors typically disable editor services (like what?) when the document is not in a consistent state,
% a phenomenon called the ``gap problem.''
% Of course, in the presence of multiple concurrent users, the problem gets worse.

% In a collaborative setting,
% Hazel is a structure editor with support for advanced editing services, (e.g., semantic actions?).

\newpage




% Motivation:

% - collaborative editing (both synchronous ala Google Docs and asynchronous version control)
% is good and important as computing grows

% - semantic structure editing is good because it solves the gap problem (semantic editor services
% are always available) -- cite Hazelnut papers (talk about holes)

% - previous approaches to collaborative editing have limitations

% - diff/merge based approaches (trying to solve the inverse problem based on final states --
% you lose the actual actions that were performed, and have to reconstruct them or an approx.
% of them i.e. add line/delete line actions -- would need to adapt this to structure editing,
% some papers have started to look at that, but fundamentally we don't want to throw away the
% knowledge we have about the edits!)

% - operational transforms (complexity, you have to patch previous actions based on new actions)

% - CRDT-based collaborative editing (that's all been on text, not PL semantics) -- this is good
% because it is relatively simple: you just send all the edits to all the replicas and they are
% convergent by design

% - we want to have the same convergence for a CRDT-based collaborative structure editor that maintains
% the sensibility invariant of Hazelnut, i.e. every editor state has meaning. mention that maintaining sensibility
% allows scaling of semantic editor services in the presence of large number of collaborators (in contrast,
% using VS Code or other collaborative text editors with large numbers of collaborators means that almost always
% the semantic editor services will be disabled because the program is going to be broken in multiple places
% transiently)

% this is tricky because:

% - some edits might be conflicting -- solve this with "conflict holes"

% - adding cut/paste or delete/restore allows for degenerate programs (cycles, multiple parents, etc.)

% - since we are commutative, we solve both synchronus and async collaborative editing

% - and this resolves issues around merges and conflicts

% - contribution of this paper is to solve these problems from type-theoretic first principles:

% - ...

% - Hazel

\subsection{Contributions and Paper Organization}%
\label{sec:Contributions and Paper Organization}


\section{Grove By Example}%
\label{sec:Grove By Example}

This section introduces collaborative structure editing in Grove by example.
\autoref{sub:Program Representation} describes how we use graphs to represent collaborative program sketches. 
\autoref{sub:Single-User Actions} then shows examples of edits being performed by a single user, Alice. 
\autoref{sub:Multi-user Interactions}-\ref{sub:Merge Conflicts} then describes a collaboration between two users, Alice and Bob,
as they edit their own branches of a program and periodically merge in each other's edits, starting with examples without conflicts, then considering the various kinds of conflicts that might arise.

For simplicity and concision, all of the examples in this section will be for a 
language of standard arithmetic operations, 
but our formalism in \autoref{sec:Formalism} and our implementation in \autoref{sec:Implementation} are parameterized by an arbitrary abstract syntax.

\figureSimple


Grove can form the basis for both a conventional version control workflow,
where edits are batched into commits, or real-time collaborative editing, 
where edits are communicated as they occur. This paper makes no assumptions about which batching mode is in use, nor do we consider the well-studied problem of reliably and efficiently communicating patches over a network.
%\autoref{sub:Cursors} discusses representing cursor locations when in real-time mode. 

\subsection{Representing Collaborative Program Sketches as Graphs}%
\label{sub:Program Representation}

The \textit{edit state} of a Grove branch is a directed multi-graph representing a \emph{collaborative program sketch}, meaning an incomplete program, i.e. one that may have \emph{holes} and (as we will return to) conflicts. 
For example, \autoref{fig:Simple:a} gives one such graph and its corresponding \emph{decomposition} into, in this case, a single syntax tree,~\texttt{x * \hole},
whose missing right operand is a hole, denoted $\hole$.

Each vertex represents a term in the specified language, except for a distinguished root vertex, 
and is labeled with a unique identifier (UID) and a \emph{constructor}. 
In \autoref{fig:Simple:a}, the root vertex has UID 0 and constructor~\textbullet. 
Vertices \vSimpleTimes{} and \vSimpleX{} have UIDs 2 and 4 and constructors~\texttt{*} and \texttt{var(}$x$\texttt{)}, respectively.
For clarity, we abbreviate \texttt{var(}$x$\texttt{)} as simply \texttt{x}; here, $x$ is a constructor parameter. We treat identifiers and literals as indivisible, but we discuss character-level editing in \autoref{sec:Discussion and Conclusion}.

An edge indicates that the destination vertex is a child of the origin vertex. 
Each edge is labeled with a UID~(e.g.,~1 and~3 in \autoref{fig:Simple:a})
and a \emph{position} (e.g., \texttt{Root} and \texttt{L} in \autoref{fig:Simple:a}). 
The parent vertex's constructor determines a set of valid {positions}. 
For instance, the~\texttt{*} constructor defines positions
~\texttt{L}~(for the left operand)
and~\texttt{R}~(for the right operand).
The~\texttt{var} constructor is a leaf so it defines no positions.
The root vertex constructor~\textbullet~has a single child position, \texttt{Root}.

%Visually, we indicate the position of an edge by the location of its origin.

Holes arise in the decomposition by the absence of a child at a valid position.
For example, in \autoref{fig:Simple:a} the absence of an \texttt{R} child under \vSimpleTimes{}
corresponds to the hole in the right operand of~\texttt{x * \hole}.

For clarity, we use odd numbers for vertex UIDs and even numbers for edge UIDs. In practice, UIDs would be generated by a mechanism
that effectively ensures that collaborators always generate distinct UIDs, e.g. by generating
universally unique IDs (UUIDs)~\cite{paskin1999toward}.

% \figureMove{}

\subsection{Structure Editing}%
\label{sub:Single-User Actions}
\figureWrapMove

Individual users perform \emph{edits} to evolve the edit state. We consider several standard edits, including insertion, deletion, cut-and-paste (relocation), copy-and-paste, and undo/redo. This paper abstracts over the user interface aspects of structure editors and makes no usability-related claims; these edits could be performed through, for example, drag-and-drop interactions (as in block-based editors like Scratch) or keyboard interactions (as in MPS and Hazel).  % User interface design for structure editors remains an active research area.
% The common thread is that structure editors maintain a program sketch throughout the editing process.

Each edit translates directly to a \emph{graph patch}, which consists of a sequence of \emph{patch commands}. The Grove patch language requires only two patch commands:  \emph{edge insertion} and \emph{edge deletion}. A vertex is inserted when it is included in an edge insertion command.

To illustrate the Grove patch language, let us consider a sequence of standard edits, found across structure editors, performed by a single user, Alice.

\subsubsection{Hole Filling}%
\label{sub:Construction}

First, Alice fills the hole in the right position of \texttt{x * \hole} from \autoref{fig:Simple:a} with the variable~\texttt{y}. The resulting edit state is shown in \autoref{fig:Simple:b}.
The patch corresponding to this hole filling action inserts an edge, labeled \eSimpleY{}, from the vertex corresponding to the parent term, \vSimpleTimes{}, to the newly constructed variable's vertex, \vSimpleY{}. 
The resulting graph decomposes to the term \texttt{x * y}.

\subsubsection{Deletion}%
\label{sub:Deletion}

% \figureMove{}

Next, Alice moves the cursor to~\texttt{x} in \autoref{fig:Simple:b} and deletes it, 
causing the deletion of edge~\eSimpleX{} and  resulting in the decomposition~\texttt{\hole{} * y} as shown in \autoref{fig:Simple:c}.

Once an edge with a particular identifier is deleted, it cannot be re-inserted.
For instance, if Alice performed an ``undo'' on this deletion,
a fresh edge between \vSimpleTimes{} and \vSimpleX{} would be created. (We can allow simple undo only if the patch has not yet been communicated to a collaborator). 

Notice that vertex \vSimpleX{} continues to exist (and if it had any children, they would remain connected to it; see below for the implications in the collaborative setting). In the remaining figures, we omit such orphaned vertices if they are not relevant to the exposition.

% The same state would arise if Alice cut \li{x}: we assume each user has an individually managed clipboard, so until a corresponding {paste} edit is performed the vertex is deleted but it can be restored by edge insertion.

\subsubsection{Wrapping}%
\label{sub:Wrapping}

Next, Alice moves the cursor to the parent term \vSimpleTimes{} in \autoref{fig:Simple:c}
with the intention of wrapping it in a binary addition expression with constructor~\texttt{+}.

Many structure editors define a primitive wrapping edit, choosing a position heuristically (e.g. favoring the left). Others 
require the user to cut the original term, construct the new outer term, then paste
the original term in the intended position. 

In either case, the corresponding sequence of patch commands would produce the edit states shown in \autoref{fig:Wrap}: the edge connecting the root to the original term is deleted (effectively cutting the original term), leaving \vSimpleTimes{} temporarily orphaned, then an edge to the new outer term is inserted, followed by an edge reconnecting the original term (effectively pasting the original term).

\subsubsection{Relocation}%
\label{sub:Repositioning}

Alice changes her mind and decides to relocate the multiplication from the left to the right position of the addition. A structure editor might support this using drag-and-drop or cut-and-paste affordances. In either case, the resulting patch commands will proceed through the two states shown in \autoref{fig:Move}: deleting the original incoming edge and then inserting an edge at the new location. Notice that the sub-graph corresponding to the relocated term itself is never deleted nor re-inserted, in contrast to conventional line-based patch languages.
See below for the implications in the collaborative setting.

\subsubsection{Copying} 
\label{sub:Copy}
A copy-and-paste, or a cut followed by multiple pastes, would of course involve copying the graph structure of the original term but generating fresh UIDs (not shown).


% \figureDifferentParts{}

%\figureCommutativity{}
\subsection{Collaboration}%
\label{sub:Multi-user Interactions}

We now turn our attention to how Grove handles collaboration.
The examples in this section generalize to collaborations between any number of users,
but for simplicity we consider only two: Alice and Bob.
Alice and Bob are each concurrently editing their own branches of the repository (or their own instance of a real-time collaborative editor), 
performing edits that translate to patches as described above. 
They periodically communicate these patches to one another. \autoref{fig:Commutativity} diagrams the Grove collaboration model.



\begin{figure}[h]
  \centering
  \begin{tikzpicture}
    \path (-3cm, 0cm) node (a) [align=center]            {Base \\ Branch};
    \path ( 0cm, 1cm) node (b) [align=center,alice node] {Alice's  \\ Branch};
    \path ( 0cm,-1cm) node (c) [align=center,bob node]   {Bob's    \\ Branch};
    \path ( 3cm, 0cm) node (d) [align=center]            {Merged \\ Branch};
    \path [draw,->,alice step] (a) -- node [pos=0.7,align=center,auto]      {Alice's \\ Edits} (b);
    \path [draw,->,bob step]   (a) -- node [pos=0.7,align=center,auto,swap] {Bob's \\ Edits}   (c);
    \path [draw,->,merge step] (b) -- node [pos=0.3,align=center,auto]      {Apply \\Bob's Patch} (d);
    \path [draw,->,merge step] (c) -- node [pos=0.3,align=center,auto,swap] {Apply \\Alice's Patch} (d);
  \end{tikzpicture}
  \caption{Collaboration in Grove is simple due to the commutativity of Grove's patch language.}
  \Description{This figures describes the commutativity of edits}
  \label{fig:Commutativity}
\end{figure}



\subsubsection{Commutativity}%
\figureDifferentPartsNestedParts

\label{sub:Commutativity:informal}
The Grove patch language is commutative, meaning that there is no need for a 
complex three-way merge algorithm (i.e. operational transform). Instead, each 
user can simply apply incoming patches to their own edit state as they arrive, no matter the order in 
which they arrive. If two users have received the same set of patches, their edit state will converge.

The key properties that make the Grove patch language commutative is that 
edge deletion is permanent and vertex insertion is permanent.
We establish commutativity formally in \autoref{sec:Formalism}.
For now, let us consider several example scenarios that demonstrate 
how Grove handles different collaborative editing scenarios, particularly 
the problematic situations outlined in \autoref{sec:Introduction}. 

\subsubsection{Solving the Granularity Problem}%
\label{sub:Editing Different Parts of the Code}

Alice and Bob start where Alice left off in \autoref{fig:Move:b} 
with the term \texttt{\hole{} + \hole{} * y}.
Alice then adds~\vDifferentPartsAlice{} as the left child of \vSimpleTimes{}.
Concurrently, Bob changes \vSimpleY{} to \vDifferentPartsBob{}.
Before sharing their patches,
Alice and Bob have the edit states \autoref{fig:DifferentParts:a} and \autoref{fig:DifferentParts:b}, respectively.
Note that the transition from \autoref{fig:Move:b} to \autoref{fig:DifferentParts:b}
represents multiple graph updates,
i.e., deleting~\eSimpleY{} and adding~\eDifferentPartsBob{} along with its child~\vDifferentPartsBob{}.
We thus mark the transition with a star.
Once Alice and Bob share their patches and apply each other's patch to their own edit state,
both edit states converge to the graph in \autoref{fig:DifferentParts:c}. 
Because Grove's patch language is structural rather than line-based, the fact that these edits happened to be close to one another (i.e. in the same arithmetic expression) does not run afoul of the \textbf{granularity problem} described in \autoref{sec:Introduction}.

% \figureNestedParts{}

\subsubsection{Solving the Relocation Modification Problem}%%
\label{sub:Editing Nested Parts of the Code}

After converging on \texttt{\hole{} + u * y} in \autoref{fig:DifferentParts:c}, 
Alice changes~\vDifferentPartsAlice{} to~\vNestedPartsAlice{},
producing the edit state in \autoref{fig:NestedParts:a}.
Meanwhile, Bob relocates~\vSimpleTimes{} (bringing along its children)
from the~\texttt{R} position of~\vWrapPlus{} to the~\texttt{L} position. As discussed above, this involves deleting~\eMoveTimes{} and adding~\eNestedPartsBob{}.
Bob's resulting edit state is shown in \autoref{fig:NestedParts:b}.

Although Alice has modified a term that Bob has concurrently relocated, the edits commute: Alice's modifications are relocated to the new location chosen by Bob.
This addresses the \textbf{relocation modification problem} described in \autoref{sec:Introduction}.
In a line-based setting, this kind of edit can lead to silent code duplication or spurious conflicts (the threat of which, in the author's experience, can inhibit development teams from performing useful code reorganizations).

% TODO: \todo{TODO}She can restore that code using a restoration action and add a fresh edge to it.

\subsubsection{Warning of Edits under Disconnected Terms}

If instead of relocating the multiplication in \autoref{fig:NestedParts}, Bob had deleted it (i.e. disconnected it from the root), then Alice's edits would still commute, but her edits would be applied under a deleted term. 

This situation could also arise in a real-time collaborative editor, where each individual edit might arrive at any time (rather than in atomic commits). If Alice, say, receives Bob's deletion of~\eMoveTimes{}, then makes her edits 
before receiving Bob's subsequent insertion of~\eNestedPartsBob{} to complete the relocation, 
Alice's edits would then temporarily be under a disconnected term. 

This does not present a formal problem or conflict. A subsequent edit might reconnect a disconnected term, so it is sensible for edits to these terms to be recorded. 
However, heuristically, a system might warn users, perhaps after a period of quiescence in a real-time setting, that Alice's edits were effectively deleted and provide affordances for interacting with disconnected terms.

\subsection{Conflicts}%
\label{sub:Merge Conflicts}

The collaborative edits discussed so far merge cleanly,
but in general, merging patches can lead to graphs that do not map cleanly to a conventional syntax tree. We identify several different motifs that might arise, all of which give rise to different kinds of conflicts in the graph decomposition: \emph{local conflicts}, \emph{relocation conflicts}, and \emph{unicyclic relocation conflicts}. 
As with merge conflicts in version-control systems such as git, these all require user intervention to resolve. 
%As we will return to in \autoref{sec:Type System}\todo{sec}, conflicts need not disable type-based editor services (the \textbf{semantic gap problem}). Instead, the system can type check around these conflict (and in some cases, infer a type for the conflict itself).

\subsubsection{Local Conflicts}%
\label{sub:Multi-child conflicts}

\figureMultiChild

Suppose Alice and Bob both start with the edit state \texttt{w * v + \hole} from \autoref{fig:NestedParts:c}.
Alice moves the cursor to the hole and constructs~\vMultiChildAlice{} as the~\texttt{R} child of~\vWrapPlus{}.
At the same time, Bob constructs~\vMultiChildBob{} at the same location.
Now Alice and Bob have the graphs in \autoref{fig:MultiChild:a} and \autoref{fig:MultiChild:b}, respectively.

When these patches are merged in \autoref{fig:MultiChild:c}, \emph{both} \eMultiChildAlice{} and \eMultiChildBob{}
appear in the merged graph. 
When decomposing this graph to a syntax tree, 
we resolve this conflict in the~\texttt{R} position of~\vWrapPlus~by decomposing to a \emph{local conflict}, {\binaryConflictHole{x}{y}}.

% TODO: "delete both"

Local conflicts can be resolved simply by deleting or relocating all but one of the conflicting terms (and editing the remaining term into the correctly merged value, if needed), which would remove the corresponding edge.
For example, Alice could resolve the problem by wrapping \li{x} and \li{y} with a multiplication, effectively moving them to non-conflicting locations.
No special edit actions are needed for conflict resolution.

% To ensure users can continue editing while in a conflicted state,
% all conflicts must be resolved before a program can be run.\footnote{It
%   may be possible to develop evaluation models that allow these sorts of conflicts,
%   but that is beyond the scope of this paper.})
% % TODO: need example of fully-automatic resolution and semi-automatic resolution.
% Note that as a convenience to the user, certain simple conflicts might be automatically resolved,
% but we consider this a higher-level, user-interface consideration.

It is worth noting one special situation: if Alice and Bob independently filled the hole with structurally identical terms, e.g. \li{x}, Grove would \emph{still} formally identify a conflict, because the terms have distinct UIDs. In this situation, it would be reasonable for the system to resolve the conflict without further coordination by deterministically choosing one of the two terms, e.g. the smallest.

When the conflicted terms are similar up to UID differences but not identical, it might be helpful  to give the developer the option to ``push down'' the conflicts as deeply as possible, using a tree differencing algorithm. However, this would increase the number of conflicts overall, so it may not always be preferable. Tree differencing is not fundamental to the collaboration model of Grove.


\subsubsection{Relocation Conflicts}%
\label{sub:Multi-parent conflicts}

\figureMultiParent

Grove's support for repositioning creates the possibility for \emph{relocation conflicts}. These occur when a merge causes a vertex to have multiple incoming edges, indicating that it does not have a uniquely determined location (as opposed to local conflicts, which occur when there are multiple outgoing edges at a specified location). 

For example, \autoref{fig:MultiParent} shows an example where Alice and Bob relocate a term, \li{w}, to two different locations (the two holes in \autoref{fig:MultiParent:a}). In both cases, the edits are modeled as one edge deletion followed by one edge addition.
Alice deletes~\eNestedPartsAlice{} and adds~\eMultiParentAlice{} in \autoref{fig:MultiParent:b}.
At the same time, Bob deletes~\eNestedPartsAlice{} and adds~\eMultiParentBob{} in \autoref{fig:MultiParent:c}.
Edge deletions is idempotent,
so the fact that Alice and Bob both deleted~\eNestedPartsAlice{} will not lead to a conflict. 
However, both~\eMultiParentAlice{} (added by Alice) and~\eMultiParentBob{} (added by Bob)
point to the same vertex.
Once these patches are merged, the resulting edit state is given in \autoref{fig:MultiParent:d}.
Notice~\vNestedPartsAlice{} has two edges pointing to it. 
When decomposing the graph, we leave a \emph{relocation conflict reference}, $\multiVertex{18}$, at each conflicting location. The conflicted term is separately tracked in the result of decomposition, which, due to conflicts like these, is formally a set of terms with references like these between them. We call this set a \emph{grove}. 
This approach partially addresses the \textbf{relocation conflict problem} from \autoref{sec:Introduction} (we also need to handle unicycles, see below, to fully address the problem).

To resolve relocation conflicts, a user can simply delete all but one of the relocation conflict references. This will cause the corresponding edges to be deleted,
and when only one edge remains, there will no longer be a conflict. An editor might provide a more convenient way of deleting all but a selected relocation conflict reference, and provide affordances for displaying these terms, e.g. by transcluding them inline at each location or showing them in a separate sidebar.

% \subsubsection{Single-User Cycles}
% \label{sub:Cycles}

% TODO: cut this section

% \figureSingleUserCycles{}

% Another case we must consider is when cycles appear in the graph.
% We categorize these into cycles caused by the action of a single user
% and cycles caused by the interaction of the actions of multiple users.

% In a single user context,
% normal insertion and deletion of code by the user cannot create cycles.
% However, it is possible with certain kinds of copy and paste.
% For example, suppose Alice is editing the code in \autoref{fig:Single-User Cycles:a}
% and uses copy and paste to copy \Vertex{TODO} to the right child of \Vertex{TODO}.
% There are two ways to interpret the paste action.
% The first interpretation is to create a deep copy of \Vertex{TODO}.
% This results in \autoref{fig:Single-User Cycles:b} and
% does not cause a cycle.
% The second interpretation is to simply add an edge to \Vertex{TODO}.
% This results in \autoref{fig:Single-User Cycles:b}
% and causes a cycle.

% Note that not all pastes should be deep copies.
% For example, Alice may have accomplished code move in \autoref{sub:Editing Nested Parts of the Code}
% by a cutting from the old position and pasting to the new position.
% Preserving Bob's nested edits requires that the paste be by reference instead of by copy.
% Distinguishing when a paste should be by reference versus by copy
% is ultimately a user interface question.
% Cycles caused by the local user's edits can be detected as soon as a user enters them
% by noting either that the graph would contain a cycle or the vertex
% already has a parent somewhere in the graph.

% Thus, as a user interface consideration, it might be best to either
% disallow such edits to at least warn users when their
% edits would create a cycle.

\subsubsection{Cycles}%
\label{sub:Multi-User Cycles}

\figureCycle
\figureDisconnect

Relocation in a collaborative setting can also lead to cycles in the graph. 
For example, consider the situation in \autoref{fig:Cycle}.
Starting in \autoref{fig:Cycle:a}, 
Alice relocates~\vMultiCycleTimes{} to the \texttt{R} child of~\vWrapPlus{}
and then~\vMultiCyclePlus{} underneath that in \autoref{fig:Cycle:b}.
Bob does the opposite, putting~\vMultiCycleTimes{} under~\vMultiCyclePlus{} in \autoref{fig:Cycle:c}. On their own, neither of these edits creates a cycle.
However, merging the two patches results in the graph
in \autoref{fig:Cycle:d}, which has a cycle
between~\vMultiCycleTimes{} and~\vMultiCyclePlus{}.

The main difficulty with cycles has to do with decomposition back to a term, i.e. a syntax tree. We do not want decomposition to traverse endlessly attempting to create an infinite tree, so we need to break the cycle somewhere.

If the cycle is connected to a larger term, then there will necessarily be at least one vertex along the cycle that has multiple incoming edges. In \autoref{fig:Cycle}, both \vMultiCycleTimes{} and \vMultiCyclePlus{} have multiple incoming edges. As described above, decomposition will leave relocation conflict references in these positions, thereby breaking the cycle. In this example, these references appear within a local conflict as well, because both vertices were relocated under a common parent vertex. As before, this cycle can be broken by deleting or otherwise modifying the terms until there are no longer any such conflicts.

It is also possible to merge patches such that a \emph{disconnected unicycle} emerges in the graph, even when neither patch disconnects any vertex from the root. \autoref{fig:Disconnection} shows a simple example of when this could occur: Alice relocates~\vMultiCycleTimes{} under~\vSimpleTimes{},
while Bob relocates~\vSimpleTimes{} under~\vMultiCycleTimes{}. 
This results in \autoref{fig:Disconnect:b} and \autoref{fig:Disconnect:c}, respectively.
Merging these causes both vertices to become disconnected, because they were both relocated. The inserted edges form a \emph{unicycle}, meaning a cycle where every vertex has in-degree $1$. In this case, we cannot rely on relocation references to break the cycle. Instead, we break the cycle by arbitrarily but deterministically choosing an edge along the unicycle, e.g. the edge with the smallest UID, and leaving a \emph{unicycle conflict reference} at that location, as shown in \autoref{fig:Disconnect:d}. A user can be notified of this situation when merging and resolve these conflicts again by relocating or deleting terms until the cycle no longer exists in the graph.

Relocation conflict references and unicycle conflict references together fully address the \textbf{relocation conflict problem} from \autoref{sec:Introduction}.


% TODO: technically a cycle

% TODO: in general detect when there are or were edits to something now deleted

% \todo{UI would show orphans only once edits orphans occur}

% \subsection{Cursors}%
% \label{sub:Cursors}

% \paragraph{Representation}

% For most editing a cursor is represented by a position within a vertex.
% For example, in TODO, Bob's cursor might be at the TODO position of the TODO:vertex.
% Representing it this way means that if Bob or some other user deletes TODO:vertex
% and replaces it with a different vertex, Bob's cursor is still valid.
% We specify a position within a vertex instead of a vertex because TODO.

% However, there is one case where we need a more precise cursor.
% That is when a vertex has a multi-child conflict (e.g. TODO in FIG:TODO).
% We want users to be able to put their cursor on either the conflict as a whole
% (e.g., TODO: \texttt{\{ x | y \}})
% or on an individual element (e.g., \texttt{x} or \texttt{y}).
% In the former case, a pair of a vertex and a child position suffices.
% For the latter case, we represent cursors by an edge identifier (e.g., TODO in TODO:FIG).

% \paragraph{Communication}

% In our system all edits use edge and vertex identifiers, and user
% cursors do not affect the interpretation of graph edits.
% Thus, cursors need not be communicated to other users.
% However, in a collaborative setting, seeing the cursors of other users
% can be useful.
% For this purpose, editors can announce their cursor position to other editors.
% At any point in time, the announcement
% with the most recent timestamp
% for a particular user
% is used.

% Note that this cursor position may refer to edges or verticies that
% do not yet exist on the receivers machine.
% In these cases, we can either not display the other user's cursor (and perhaps
% have a visual display flagging this fact), or we can display the most recent
% cursor that represents a valid position in the local graph (perhaps
% shown in a fadded color to show that this cursor is known to not be up to date).
% (Or perhaps, all cursors always fade/decay over time like on radar blips.)


\section{Formalism}%
\label{sec:Formalism}

% TODO: try to introduce groves in section 2?
% TODO: use key phrases in section 2, e.g.: "unresolved parentage"

We will now make the intuitions developed in the previous section precise by defining a collaborative structure editor calculus called Grove.
This section introduces
  graphs (\autoref{sub:Convergent Graphs})
  and commutative graph edits (\autoref{sub:Graph Edits}).
Users do not edit graphs directly.
Instead, they work with the decomposition of graphs into groves (\autoref{sub:Groves})%
---collections of terms with empty holes, conflict holes, and references to terms with unresolved parentage.
User actions on groves translate to corresponding graph edits (\autoref{sub:User Actions}).


TODO: point of this section is commutativity theorem

TODO: (though as we show in \autoref{sec:Formalism} creating a vertex is implicit in edge creation)

The formalism is rather simple, \autoref{sec:Grove By Example} shows
that it is sufficient to handle many common situations,
and this simplicity aids in the predictability.

\figureArity{}

\subsection{Convergent Graphs}%
\label{sub:Convergent Graphs}

Let $\U$ be a set of \emph{unique identifiers} equipped with a total ordering $\leq$,
and $\K$ the \emph{constructors} of terms in the source language,
and $\P$ the \emph{positions} of subterms in the source language,
and $\Sigma = \SetOf{\bot, \Plus, \Minus}$ the
  \emph{edge states} equipped with a total ordering $\bot \sqsubset \Plus \sqsubset \Minus$ that forms a lattice over $\Sigma$.
Now, let $\V = \U \times \K$ all possible \emph{vertices},
and $\S = \V \times \P$ all possible \emph{sources},
and $\E = \U \times \S \times \V$ all possible \emph{edges}.

A \emph{graph} $G : \E \rightarrow \Sigma$ is a function from edges to edge states,
  where $\E = \U \times \S \times \V$,
  unique identifiers are drawn from some suitable set $\U$ equipped with a total ordering $\leq$,
  sources are drawn from $\S$,
  vertices are drawn from $\V$,
  and edge states are drawn from $\Sigma$,
  all defined below.
Graphs that represent a complete edit state have a distinguished
  root vertex $\rootVertex$
  and root position $\rootPosition$.
The \emph{inedges} of $G$ are the edges on which $G$ returns $\Plus$.
The \emph{outedges} of a source are the inedges that have the source.

An \emph{edge} $\e{=}(u, (v, p), v^\prime) \in \E$ represents a directed multi-edge
identified by $u \in \U$, originating from source $(v, p) \in \S$, with destination vertex $v' \in \V$.

The \emph{edge states} $\Sigma = \SetOf{\bot, \Plus, \Minus}$ determine which edges are live in a graph.
For example, if $G(\e) = \Plus$ for some $G : \E \rightarrow \Sigma, \e \in \E$,
then we say $\e$ is \emph{live}, i.e., it corresponds to a term in the graph decomposition, as explained in \autoref{sub:Graph Decomposition}.
The total ordering $\bot \sqsubset \Plus \sqsubset \Minus$ forms a lattice over $\Sigma$.

A \emph{source} is a pair $(v, p) \in \S$,
  where $\S = \V \times \P$,
  vertices are drawn from $\V$,
  and positions are drawn from $\P$.

A \emph{vertex} $v = (u, k) \in \V = \U \times \K$ is a constructor instance,
  where unique identifiers are drawn from $\U$
  and constructors are drawn from $k \in \K$.
Vertices represent \emph{constructable terms}, i.e., terms that are not references or holes.
The \emph{inedges} of $v$ are the inedges that target $v$.
The \emph{ingraph} of $v$ is the subgraph on inedges that target $v$.
We write $G_v$ to denote the ingraph of $v$.
The \emph{parents} of $v$ are the source vertices of the inedges of $v$.
The \emph{ancestors} of $v$ is defined inductively as the parent vertices of $v$ plus their ancestors.
The \emph{least ancestor} of $v$ is the ancestor with the smallest unique identifier.

% % % % % % % % % % % % % % % % % % % % % % % % % % % % % % % % % % % % % % % % 

\subsubsection{Well sorted graphs}
\label{sub:Well sorted graphs}

% TODO: where does this go?
% 
% Each contructor has an arity, which we define as a set of position-sort pairs. A position $p
% \in \P$ identifies a location at some originating vertex from which edges to
% destination vertices may originate. The $\arityOp$ function in
% \autoref{fig:Constructors, Indexes and Arity} gives the complete mapping from
% constructors to potential edge positions with their corresponding sorts.

%%%%%%%%%%%%%%%%%%%%%%%%%%%%%%%%%%%%%%%%%%%%%%%%%%%%%%%%%%%%%%%%%%%%%%%%%%%%%%%%
% TODO: put this in a definition box
A graph $G$ is well sorted if
for all $\e = (u, (u_1, k_1), p, (u_2, k_2))$
such that $G(\e) \neq \bot$,
we have $(p, \sort (k_2)) \in \arity k_1$.

%%%%%%%%%%%%%%%%%%%%%%%%%%%%%%%%%%%%%%%%%%%%%%%%%%%%%%%%%%%%%%%%%%%%%%%%%%%%%%%%


In decomp section:
- If a graph is well sorted, then there exists a grove. (i.e., exhaustiveness check for expr)
  - want to show that expr is a total function on Exps
- If decomp(G) = Grove, then vertices(Grove) = vertices(edges(G))
  - every vertex should be part of the decomposition
- correctness of grove components:
  - $e_r$: rootVertex($e_r) = v_r$
  - MP: for every expression in MP, |parents(the root vertex)| > 1
  - NP: for every expression in NP, |parents(the root vertex)| = 0
  - U: for every expression in U, |parents(the root vertex)| = 1 and the root vertex = min(ancestors(the root vertex))
    - every other vertex notin U has exactly 1 parent and is not its own min-ancestor
  - we've accounted for every edge:
    - want to say edges(Grove) = edges(G), but we don't have a clean way to say that
    - there is a choice from (an infinite family of graphs) recomposition(Grove) (which are all structurally equivalent to G, modulo edge ids)



correctness covers:
- invertibility (w/ some identity assignment)
- parentage (things are multiparented correctly)

\vspace*{\baselineskip}

% TODO: should graph "recomposition" have an algorithm, too?
% TODO: update this to match our current understanding
A graph can be derived from an expression in the calculus by starting with an
empty graph, selecting unique ids $u, u_k$ and constructor $k$ for the outermost
expression, and creating edge $\e = (u, v_0, p_0, (u_k, k))$. Then, for each
sub-term at some position $p \in \arity k$, select unique id $u'$ and
constructor $k'$ and create edge $e' = (u', v, p, v')$ with $v = (u_k, k)$, $v'
= (u', k')$. Continuing in this manner, treat each sub-term as if it were the
outermost, starting with the ones at positions in $\arity k'$, and create edges
to the vertices representing their immediate sub-terms until only empty holes
remain. Note that empty holes are not modeled explicitly in the graph. Wherever
an empty hole appears in an expression, the corresponding graph position has no
edges originating from it.

% You can map from syntax to graph by selecting
% unique IDs $u$ (for both edges and vertexes).
% - holes are not explicit in the graph

\subsection{Graph Edits}%
\label{sub:Graph Edits}

We define a graph edit $\alpha \in \A = \left\{\Plus, \Minus\right\} \times \E$
as a pair of an edge state, excluding $\bot$, and an edge. As a shorthand, we
define $+\e$ to be $(\Plus, \e)$ and $-\e$ to be $(\Minus, \e)$.

We define the join operation $s_1 \sqcup s_2$ to be the least upper bound of
$s_1$ and $s_2$ with respect to the $\sqsubset$ ordering. Accordingly, $\Plus
\sqcup \bot = \Plus$ means that edges can only be created if they do not already
exist, and $\Plus \sqcup \Minus = \Minus = \Minus \sqcup \Minus$ means that once
an edge is deleted it can never be restored.

We define the semantics of graph actions via the following transition relation
between graphs.
\[
  G \overset{(s,\e)}{\longrightarrow} G\left[ \e \mapsto s \sqcup G(\e) \right]
\]
Applying graph action $\left(s, \e\right)$ to graph $G$ results in the
updated graph $G' = G\left[\e \mapsto s \sqcup G(\e) \right]$, wherein the edge
state associated with edge $\e$ in $G$ becomes the join of $s$ with the state of
$\e$ in $G$, and $G'(\e') = G(\e')$ for all other $\e' \ne \e$ in the domain of
$G$.

% Graph action semantics is a transition system between graphs.
% (join with whatever it was before)

% Intuitively, joining any edge state with $\bot$ is always the non-$\bot$ edge state (once an edge exists, it always exists).
% Joining anything with $\Minus$ is always $\Minus$ (once an edge is marked as deleted, it is permanently deleted).

\subsection{Commutativity}%
\label{sub:Commutativity:formal}

In order to define commutativity, we first define a lattice over $G$.
Then we show that for any graphs $G$ and $G'$ and graph action $\alpha$,
$G \overset{\alpha}\rightarrow G'$
iff $G \sqcup \llbracket\alpha\rrbracket = G'$.
Thus the commutativity of graph actions is established
by the commutativity of the join operations defining those graph actions.

Discuss uniqueness of uuid

\begin{lemma}[Join Commutativity]
  \label{lem:Join Commutativity}
\end{lemma}

\begin{lemma}[Joining]
  \label{lem:Joining}
  For all graphs $G$ and $G'$ and all edit actions $\alpha$,
  $G \overset{\alpha}{\longrightarrow} G'$
  iff $G \sqcup \llbracket\alpha_1\rrbracket = G'$.
\end{lemma}
\begin{proof}
  If $G \overset{\alpha_1\alpha_2}{\longrightarrow} G'$
\end{proof}


\begin{theorem}[Commutativity]
  \label{thm:Commutativity}
  For all graphs $G$ and $G'$ and all edit actions $\alpha_1$ and $\alpha_2$,
  if $G \overset{\alpha_1\alpha_2}{\longrightarrow} G'$,
  then $G \overset{\alpha_2\alpha_1}{\longrightarrow} G'$.
\end{theorem}
\begin{proof}
  Using \autoref{lem:Joining} to unfold $\rightarrow$ into $\sqcup$,
  $G \sqcup \llbracket\alpha_1 \sqcup \llbracket \sqcup \alpha_2 \rrbracket = G'$.
  Then by \autoref{lem:Join Commutativity},
  $G \sqcup \llbracket\alpha_2 \sqcup \llbracket \sqcup \alpha_1 \rrbracket = G'$.
  Finally using \autoref{lem:Joining} to fold $\sqcup$ into $\rightarrow$,
  $G \overset{\alpha_2\alpha_1}{\longrightarrow} G'$.
\end{proof}

\subsubsection{Agda Mechanization}%
\label{sub:Agda Mechanization}

\subsection{Groves}%
\label{sub:Groves}

\figureTermSyntax{}

% TODO: check for \hole --> \emptyHole

Our source language (\autoref{fig:Syntax}) is a simply typed lambda calculus with references, empty holes, and conflict holes.
The syntax of terms is sorted into
  expressions $e$,
  types $\tau$,
  and patterns $q$.
These can include
  graph annotations $G$,
  edge annotations $\e$,
  or source annotations $(v, p)$,
  which we introduce in \autoref{sub:Convergent Graphs}.
Expressions include
  variables $\eVar{G}{x}$,
  lambda abstractions with pattern arguments $\eFun{G}{q}{\tau}{e}$,
  applications $\eApp{G}{e_1}{e_2}$,
  and numbers $\eNum{G}{n}$ with addition $\ePlus{G}{e_1}{e_2}$ and multiplication $\eTimes{G}{e_1}{e_2}$.
Types include
  arrows $\tArrow{G}{\tau_1}{\tau_2}$
  and the type of numbers $\tNum{G}$.
Patterns include variables $\pVar{G}{x}$.
Each sort of term also contains
and empty holes $\hole_{(v, p)}$.
multiparent references $\multiVertex{\e}$,
unicycle references $\cycleVertex{\e}$,
and conflict holes $\conflictHole{t_i}_{i < n}$.
For conflict holes, it must be the case that $n \geq 2$.
We identify conflict holes up to reordering.

Each term that is not a conflict hole carries an annotation.
For terms that are not holes, the annotation corresponds to the graph or edge it recomposes to.
Multiparent references correspond to edges that target
  \emph{multiparent roots}---vertices targeted by multiple edges.
Unicycle references correspond to edges that target
  \emph{unicycle roots}---vertices that are their own least ancestor.
Empty holes, which denote the absence of any corresponding structure,
are annotated with the source needed to construct terms in their place.

Conflict holes correspond to multiple edges with a common source.
Since conflict holes cannot be nested or contain empty holes,
each term in a conflict hole is annotated with the graph or edge it recomposes to,
and the combined annotations correspond to the graph the entire conflict hole recomposes to.

\subsection{Graph Decomposition}%
\label{sub:Graph Decomposition}

% NOTE base grammar has holes. rooted expression grammar has refs, conflicts,
% and multiparents

% NOTE: this whole section is for describing the 4-tuple

Trees are often easier to work with than arbitrary graphs directly, for example to present the graph to users or for type checking.
A graph can be decomposed into a collection of expressions governed by the grammar in \autoref{fig:Syntax}, which we organize into a structure called a grove.

\figureDecompExample

% TODO: make sure we're not claiming to invent the term "unicycular graph"
% TODO: maybe add a forward reference at end here to Implementation section
% TODO: make ID of y lower than the unicycle root, to show what we're minimizing

A grove is a 3-tuple of the following form:
\[
  (NP, MP, U)
\]
where
\begin{align*}
  \figureDecompositionDefDecompComponents
\end{align*}
For example, the grove corresponding to the graph in \autoref{fig:Decomposition example graph} is
given in \autoref{fig:Decomposition example grove}.

\figureDecompositionDef

Let us now examine each of the four components in more detail.
%
The first component $NP$ is the set of expressions rooted at a vertex with no parents, including the main root vertex.
% TODO: change vRoot to dot w/ subscript
In the example, $e_r$ is the expression corresponding to the main root, $\vRoot$.
% TODO: change z_... to macro use w/ same appearance as in the graph
In addition, $z_{\id{52}}$ has no parents because it was deleted, so it also appears in $NP$.

The first component $e_r$ is the expression derived from the root vertex $v_0$.
\autoref{fig:Decomposition expr} defines $\expr(v)$ which determines the expression derived from $v$.
There is one rule for each term constructor.
For example, the expression derived from ${+}_{\id{38}} = (38, \ExpPlus)$ is $\expr'({+}_{\id{38}}, \PlusLeft) \text{+}^{\id{38}} \expr'({+}_{\id{38}}, \PlusRight)$.
In each child position, $\expr'(v, p)$ determines the subterm for the children of $v$ at position $p$.
An empty hole $\hole$ appears if there are no children at the given position.
A conflict hole $\conflictHole{e_1,\ldots,e_n}$ appears if there are multiple children.
If there is one child, the expression is determined recursively unless that child appears in $MP$ or $U$, discussed below, in which case we leave a reference, $\multiVertex{u}$ or $\cycleVertex{u}$, respectively.
For the $\ExpLam$ case, we define $\patt(q)$ and $\type(\tau)$ similarly.

The second component $MP$ is the set of expressions derived from a vertex with multiple parents.
Wherever an edge to such a vertex appears, we place a reference to it $\multiVertex{u}$.
In the example, $x_{\id{42}}$ is referenced twice in $e_r$.

The third component $U$ is the set of expressions derived from the remaining vertices which necessarily have exactly one parent.
These vertices form structures formally called unicycular graphs~\citep{DBLP:journals/algorithmica/KruskalRS90}, or \emph{unicycles} for short.
Every unicycle consists of a single cycle where each vertex in the cycle may have additional children that are not part of the cycle.
Breaking the cycle turns a unicyclic graph into a tree.
% TODO: try clearing up "the root" as e.g., "the component root" or "the root of this components of the grove"
% TODO: make sure we say "the graph root" when talking about the actual root, and "the component root" everywhere else.
To break the cycle, we arbitrarily choose the vertex on the cycle with the smallest id, $u$, as the root of that tree.
Formally, we choose the vertex $v$ such that $v = \min{\ancestors{v}}$ because
only a vertex on the cycle can be its own ancestor.
When an edge to this vertex appears, we place a reference to it $\cycleVertex{u}$.
In our example, ${+}_{\id{46}}$ is referenced by the $LEFT$ edge of ${*}_{\id{48}}$.

% XXX

decomposition produces sets of trees. As long as everything is well sorted, we
can use them to produce expressions in the main grammar. (alt: define wrt the
graph instead, once we establish the graph can be turned into expressions)

*** well-sortedness judgment on trees: the tree only has valid positions for the
constructors, and the child at each position has the sort corresponding to its
position, and edges in position maps all start at the given constructor.

Theorem: well-sorted graphs produce decompositions of well-sorted trees

``A graph has this property iff these conditions hold''

then we can define a total mapping from well-sorted trees to expressions
(because the tree structure is very generic, so this ties them back to the
earlier grammar.)


forest = 4-tuple

forest validity: each of the trees in the forest are well sorted, and all Refs
point to roots of other trees of the correct sort.

every valid forest corresponds to a well behaved grove (i.e., ``expression
forest'')

soundness: if you start with a valid graph, you get a well behaved grove that
connects every vertex (and only those vertices) in the graph. (i.e., if you turn
it back into a graph, it's the same graph (modulo edge IDs unless we add $\e$ to
Refs??? OR make Ref nullary and pull edge/vertex info from the position map and
add Root sort

OR:

use edge IDs instead of edges and build a mapping from vertices to
sorts---change the $\e$s to $u$s)

% TODO: maybe add a forward reference here to Implementation section

\subsection{User Actions}%
\label{sub:User Actions}

% TODO: define a grammar for user actions --> mapping to sequences of underlying graph edits
% TODO: define well-sorted graph & theorem: graph is well sorted => exists a decomposition

move = delete + add

other composite edits

remember commutativity diagram: we want edits on trees to be equivalent to edits on graphs

given a semantics for user actions, we can show that operating on the grove is equivalent to operating on the graph.
This way is more efficient (less annoying?) because then we don't have to deal with decomposition for every edit.
(we probably want to do it this way in Hazel)

OR

We could go the indirect route and avoid having to prove the theorem.
(this way is easier to start with, and it's how GRV works)

Mapping to b actions

\[
  \begin{array}{l}
    \textrm{User Actions}: \grave{e} \in Act ::= e \mid \#u \mid \conflict{\grave{e}}{\grave{e}}                                                       \\
    \s : \G \to Act \times \V^{\ast{}} \times \V^{\ast{}} \times \V^{\ast{}}                                                                                          \\
    \s(G) = (\grave{e}, \MP(G), \CC(G), \D(G))                                                                                                         \\
    \MP(G) = \left\{ v \in \vertexes(G) \mid \exists e_1,e_2 \in \liveEdges(G), e_1= \right.                                                           \\
    \qquad \left. (v,p_1,v_1), e_2=(v,p_2,v_2), e_1 \ne e_2 \right\}                                                                                   \\
    \CC(G) = \ldots                                                                                                                                    \\
    \D(G) = \orphans(G) \setminus \left\{ v_0 \right\}                                                                                                 \\
    \orphans(G) = \left\{ v \in \vertexes(G) \mid \forall e=(v^\prime,p,v^{\prime\prime}) \in \right.                                                  \\
    \qquad \left. \liveEdges(G), v^{\prime\prime} \ne v \right\}                                                                                       \\
    \vertexes(G) = \left\{ v \mid \exists e = (v^\prime, p, v^{\prime\prime}) \in \edges(G), v \in \left\{ v^\prime, v^{\prime\prime}\right\} \right\} \\
    \edges(G) = \left\{ e \mid G(e) \ne \bot \right\}                                                                                                  \\
    \liveEdges(G) = \left\{ e \mid G(e) = + \right\}                                                                                                   \\
  \end{array}
\]


\section{Implementation}%
\label{sec:Implementation}

% TODO: talk about code generation somewhere in here

% TODO: talk about:
% - decomp vertex set generation and unicycle traversal
% - how to choose which branch of expr' to take
% - lfp(ancestors') and min of it

% TODO: can we save any of this in a README or code comments?

% TODO: talk about performance: O(decomp) given "size of the graph" (probably linear in # edges)

% GRV -- how it is implemented, how it connects to the formalism, describe the graph,

We implemented the core Grove calculus of \autoref{sec:Formalism} as an OCaml library.
Our code is parameterized by a syntax specification for expressions that we generate automatically, which makes the implemented language of expressions easier to modify or replace.
We also built a Web browser-based Grove editor as a proof of concept.

Since we cannot implement an infinite mapping directly, the graph is a data structure, \texttt{Graph.t}, that maps from edges to the edge states $\left\{\Plus, \Minus\right\}$.
We do not represent edges that map to $\bot$.

% TODO: rename Decomposition.ml to Grove.ml

There's a function \verb!decompose! of type \verb!Graph.t) : t * Edge.Set.t Vertex.Map.t! that 

From this graph, we compute the set of all edges and the set of live edges with a linear scan.

% TODO: make this into a listing

\begin{verbatim}
  Graph.edges : Graph.t -> Edge.Set.t
  Graph.live_edges : Graph.t -> Edge.Set.t
\end{verbatim}

% TODO: keep connecting the code to the formalism like this

These vertices are partitioned into three sets: multi-parented, single-parented, or orphaned.
A vertex $v$ is considered a parent of another vertex $v'$ if there is a live edge from $v$ to $v'$.

For each vertex, we take the parent and child edge sets.

% TODO: fix the formalism for parents (to work with edges instead of vertices) and then check that it matches the implementation. Try renaming parents to vparents and adding parents for edges.

% TODO: Talk about how the graph is implemented (not as an actual function, but as a map from edges to edge states) so that we can denote the set of created + deleted edges clearly, as well as all of the vertices that have ever been created.

For a graph $G : \E \to \Sigma$ where $\E = \U \times \V \times \P \times \V$, our graph decomposition algorithm runs in $O(\abs{\V} \log \abs{\V} + \abs{\E} \log \abs{\V})$.
It begins with a scan of all edges that have been created or deleted $O(\abs{\E})$.
Their vertices at both ends are partitioned into three sets: multi-parented, single-parented, or orphaned.
Their relationships are recorded in maps for $O(\log \abs{\V})$ lookups of parent and child edge sets.
After the vertices have been partitioned, we traverse the various single-parented components and produce equivalent expressions $O(\abs{\V})$.
For unicycles, we traverse backwards until a vertex is seen twice $O(\abs{\V})$, then proceed forward to find the least vertex on the cycle $O(\abs{\V})$.

% TODO: is this still a thing?
% Compared to the core language, our implementation lacks support for lists and case expressions.

Optimizations: least fixed point


We determine $e$ by starting at the root of the graph and including every connected
vertex that has a single parent. In positions where there are no children, we
leave an empty hole $\hole$. In positions where there are multiple children, we
leave a conflict hole of the form $\conflictHole{e_1,\ldots,e_n}$. Whenever we
encounter a vertex with multiple parents, we add its corresponding expression to
the second component $MP$ and leave an indirect reference $\multiVertex{u}$, where $u$ is
the unique identifier of the referenced vertex

% here are some properties that show how to get one from the other, and to/from
% graph

algorithmic section: if we want to implement these conversions, here are the
steps (without introducing new concepts):

- linear scan
- breaking out roots
- ...

Graph decomposition occurs in three steps.



% TODO: use code generation instead: have Lang.ml generate expr / expr' / ... for us; then we don't need this figure.
% TODO: maybe show this figure and say that it's possible to handle generically without code generation

\begin{figure}
  \[
    \arraycolsep=0pt
    \begin{array}{lrlll}
      \textrm{Trees:\qquad}  & t & {}\in Tree & {}::={} & \textrm{Vertex}(v, m) \mid \textrm{Ref}(v) \\
      \textrm{Maps:}         & m & {}\in Map & {}::={} & \varnothing \mid p \mapsto [(\e, t), \ldots, (\e, t)]; m \\
    \end{array}
  \]
  \caption{Syntax of graph decomposition trees as a grammar.}
  \label{fig:Syntax of graph decomposition trees}
\end{figure}


lemma: a unicycle is a cycle with trees hanging off of it

lemma: unicycle traversal always produces a valid tree

Unicycular graph traversal occurs in two directions. A vertex is chosen as an arbitrary
starting point and its parents are traversed and marked as seen. Any vertex that
has been seen twice must be on the cycle. We take the first one as the root of
the unicycle and traverse its children to produce a tree that covers the unicycular graph.
Any remaining single-parented vertices must be parts of other unicycular graphs and are
unicycle-traversed until there are no more single-parented vertices left.


\section{Related Work}%
\label{sec:Related Work}

Contrast with a traditional diff

Similar to how git merged across moves.
A notable difference though, is that \texttt{git} detects
file moves by structural similarity rather than
our nominal equality.
(We exploit this in Subsection~REF:TODO).

Compare this to what happens when there are conflicts in a traditional,
line/diff-based version control system (e.g., \texttt{git}, \texttt{mercurial}, \texttt{svn}, etc.).
Those systems annotate files with difference markers (e.g., sequences of~\texttt{<},~\texttt{>}, or~\texttt{=})
when a merge conflict happens.
These show alternate versions of the code, and it is up to the user
to replace those alternates with the desired merged result.

Another place our model is different
is that changes are explicit rather than being inferred by a \texttt{diff} algorithm.
Finally, in our model, since all changes are in terms of
the structure of the code rather than lines,
all changes respect that structure.
This is unlike the line-based model where there is no guarantee
that a merge conflict follows the grammatical structure of the code.

\subsection{Trees}

\subsubsection{Structure editors}

Hazel~\citep{Omar:2019:10.1145/3290327}

\subsubsection{Tree diff}

% We are using action based instead of tree diff

% We have to deal with merge

% We have to deal with edits from multiple people

% J. W. Hunt and M. D. McIlroy. 1976. An Algorithm for Differential File Comparison. Technical Report CSTR 41. Bell Laboratories, Murray Hill, NJ.

% Type-directed diffing of structured data \url{https://dl.acm.org/doi/10.1145/3122975.3122976}

%   Approximating Tree Edit Distance through String Edit Distance \url{https://dl.acm.org/doi/10.5555/3118232.3118518}

%   Meaningful change detection in structured data \url{https://dl.acm.org/doi/10.1145/253260.253266}

%   An optimal decomposition algorithm for tree edit distance \url{https://dl.acm.org/doi/10.5555/2394539.2394560}

%   Diff/TS: A Tool for Fine-Grained Structural Change Analysis \url{https://dl.acm.org/doi/10.1109/WCRE.2008.44}

% An efficient algorithm for type-safe structural diffing \url{https://dl.acm.org/doi/10.1145/3341717}

%   Precise Version Control of Trees with Line-Based Version Control Systems \url{https://dl.acm.org/doi/10.1007/978-3-662-54494-5_9}

%   A survey on tree edit distance and related problems \url{https://dl.acm.org/doi/10.1016/j.tcs.2004.12.030}

%   Cycle-aware minimization of acyclic deterministic finite-state automata \url{https://dl.acm.org/doi/10.1016/j.dam.2013.08.003}

%   Computing the Edit-Distance between Unrooted Ordered Trees \url{https://dl.acm.org/doi/10.5555/647908.740125}

%   Type-safe diff for families of datatypes \url{https://dl.acm.org/doi/10.1145/1596614.1596624}

%   A Categorical Theory of Patches \url{https://dl.acm.org/doi/10.1016/j.entcs.2013.09.018}

%   Type-directed diffing of structured data \url{https://dl.acm.org/doi/10.1145/3122975.3122976}

%   The Semantics of Version Control \url{https://dl.acm.org/doi/10.1145/2661136.2661137}

%   The Tree-to-Tree Correction Problem \url{https://dl.acm.org/doi/10.1145/322139.322143}

%   Generic Diff3 for algebraic datatypes \url{https://dl.acm.org/doi/10.1145/2976022.2976026}

\subsection{Version Control}

% Git \url{https://git-scm.com/}

% Darcs \url{https://darcs.net/}

%   Darcs: distributed version management in haskell \url{https://dl.acm.org/doi/10.1145/1088348.1088349}

% Hg? \url{https://www.mercurial-scm.org/}

% SVN \url{https://subversion.apache.org/}

% Pijul and (Anu is a rewrite of Pijul and seems to have been subsumed into Pijul)

%   \url{https://pijul.org/}
%   \url{https://pijul.org/manual/theory.html}

%   \url{https://tahoe-lafs.org/~zooko/badmerge/simple.html}

\subsection{Collaborative Editing}

% Collaborative Structure Editing

% SmallTalk collaboration with images

% TouchDevelop papers

% Lots of list-of-chars or list-of-list-of-chars (we ignore these except to discuss them here)

% \subsection{CRDTs}
% (Are we a known CRDT?)

% List of CRTD papers: \url{https://crdt.tech/papers.html}

% Bottom = Tombstone

% \url{https://www.waitingforcode.com/big-data-algorithms/conflict-free-replicated-data-types-flags-graphs-maps/read}
%  - Add-Remove Partial Order data type
%  - 2P2P-Sets
%  - Replicated Growable Array

% \url{https://github.com/PsychoLlama/graph-crdt}
%  - Graph CRDT
%  - Uses a LWW-E-Set

% \url{https://martin.kleppmann.com/2020/07/06/crdt-hard-parts-hydra.html} (overview talk)
%  We don't have interleaving problems because
%    - everything is relative to a specific ID not a position
%    - we don't try to auto resolve
%    - we are tree not list
%    Part three: moving sub tree
%    - Last parent writter wins (prevents cycles)
%      Equivalent to us if we filter multiparent edges, different for cycles, no way to delete?

% A commutative replicated data type for cooperative editing
%   \url{https://hal.inria.fr/inria-00445975/document}
%   Describes TreeDoc but this uses a document model that is a list (tree is just how it is implemented)

% Logoot : a Scalable Optimistic Replication Algorithm for Collaborative Editing on P2P Networks
%   \url{http://pagesperso.lina.univ-nantes.fr/~molli-p/pmwiki/uploads/Main/weiss09.pdf}

% Specification and Complexity of Collaborative Text Editing
%   \url{https://www.microsoft.com/en-us/research/wp-content/uploads/2016/07/podc16-complete.pdf}

% LSEQ: an Adaptive Structure for Sequences in Distributed Collaborative Editing,
%   \url{https://hal.archives-ouvertes.fr/file/index/docid/921633/filename/fp025-nedelec.pdf}

% Data consistency for P2P collaborative editing
%   \url{https://hal.archives-ouvertes.fr/file/index/docid/108523/filename/OsterCSCW06.pdf}

% Interleaving anomalies in collaborative text editors
%   \url{https://martin.kleppmann.com/papers/interleaving-papoc19.pdf}

% Moving Elements in List CRDTs
%   \url{https://martin.kleppmann.com/papers/list-move-papoc20.pdf}

% A highly-available move operation for replicated trees and distributed filesystems
%   \url{https://martin.kleppmann.com/papers/move-op.pdf}

% ...
%  commutative replicated data types CmRDT
%  convergent replicated data types, or CvRDTs
%  Delta state CRDTs[12][13] (or simply Delta CRDTs

% The Causal Graph CRDT for Complex Document Structure
%   \url{https://dl.acm.org/doi/10.1145/3209280.3229110}

% \url{https://en.wikipedia.org/wiki/Conflict-free_replicated_data_type}

% G-Set
% PN-Set
% 2P-Set
% LWW-element-Set (Last-Write-Wins)
% OR-Set
% MV-Register: Multi-Value Register
% U-Set (this is what we are for edges, not OR-set due to duplicates)
% Add-Remove Partial Order data type
%   2P-Set for vertices, and a G-Set for edges.

% A comprehensive study of Convergent and Commutative Replicated Data Types (2011)
%   \url{https://hal.inria.fr/inria-00555588/document}

% CRDTs: Consistency without concurrency control
%   \url{https://arxiv.org/abs/0907.0929}

% https://medium.com/@amberovsky/crdt-conflict-free-replicated-data-types-b4bfc8459d26
%  removeVertex() has priority, all incident edges are removed
%  addEdge() has priority, all removed vertices are re-added
%  Delay removeVertex() execution till all concurrent removeVertex() are executed.
% First one is 2P2P-Set

% https://crdt.tech/papers.html

% Mahsa Najafzadeh, Marc Shapiro, and Patrick Eugster. Co-design and verification of an available file system. In 19th International Conference on Verification, Model Checking, and Abstract Interpretation, VMCAI 2018, pages 358--381. Springer LNCS volume 10747, January 2018. [ bib | DOI | .pdf ]
%   \url{http://dx.doi.org/10.1007/978-3-319-73721-8_17}
%   https://pages.lip6.fr/Marc.Shapiro/papers/VMCAI-2018-filesys.pdf

% Martin Kleppmann and Alastair R Beresford. A conflict-free replicated JSON datatype. IEEE Transactions on Parallel and Distributed Systems, 28(10):2733--2746, April 2017. [ bib | DOI | arXiv ]
%   http://dx.doi.org/10.1109/TPDS.2017.2697382
%   http://arxiv.org/abs/1608.03960

% Vinh Tao, Marc Shapiro, and Vianney Rancurel. Merging semantics for conflict updates in geo-distributed file systems. In 8th ACM International Systems and Storage Conference, SYSTOR 2015. ACM, May 2015. [ bib | DOI | .pdf ]
%   http://dx.doi.org/10.1145/2757667.2757683
%   https://pages.lip6.fr/Marc.Shapiro/papers/geodistr-FS-Systor-2015.pdf

% Mehdi Ahmed-Nacer, Stéphane Martin, and Pascal Urso. File system on CRDT. Research Report RR-8027, INRIA, July 2012. [ bib | arXiv | http ]
%   http://arxiv.org/abs/1207.5990
%   https://hal.inria.fr/hal-00720681/

% Stéphane Martin, Pascal Urso, and Stéphane Weiss. Scalable XML collaborative editing with undo. In On the Move to Meaningful Internet Systems (OTM), pages 507--514. Springer LNCS volume 6426, October 2010. [ bib | DOI | arXiv ]
%   \url{http://dx.doi.org/10.1007/978-3-642-16934-2_37}
%   http://arxiv.org/abs/1010.3615



% 2P-Sets
% Anomaly: Creating a lone deleted vertex requires create and delete of otherwise unneeded edge

% Operational Transforms

% Etherpad

% Live Share

\subsection{Synchronization}

% Unison
% \url{https://www.cis.upenn.edu/~bcpierce/unison/}
% \url{https://www.cis.upenn.edu/%7Ebcpierce/papers/index.shtml#File%20Synchronization}

\subsection{Homotopical Patch Theory}

% Homotopical Patch Theory: \url{https://www.cambridge.org/core/journals/journal-of-functional-programming/article/homotopical-patch-theory/42AD8BB8A91688BCAC16FD4D6A2C3FE7}
% Homotopical patch theory: \url{https://dl.acm.org/doi/10.1145/2628136.2628158}


\section{Discussion and Conclusion}%
\label{sec:Discussion and Conclusion}

Contribution 1 and Contribution 2 together result in a typed collaborative structure calculus
where, uniquely, all edits, including code relocations that stymie existing approaches, commute and where there are no semantic gaps: all possible editor states, including editor states with various kinds of unresolved conflicts, are semantically meaningful.

In practice, it may be helpful to garbage collect orphaned vertices once there is consensus across collaborators that the deletion is permanent, but we do not consider this consensus protocol formally in this paper. In an open-ended collaboration scenario (where the set of collaborators is not known, e.g. on GitHub), we simply retain all vertices.


TODO: leaves

TODO: our model supports treating these as a cons-list of characters

TODO: (Place somewhere) This move semantics gives us a richer structure than
when treating code as a list of lines.  We exploit
this in Section~REF:TODO in order merge edits that involve moving code.

TODO: Traditional diff: no relation (and indent might change)

TODO: memory usage

TODO: cache eviction algorithm

TODO: Finally, note that through we present several complex scenarios, this is merely for presentation.
In practice, these complex scenarios occur less frequently than portrayed here.

\subsection{Variable names, strings, and numbers}%
\label{sub:Variable names, strings, and numbers}

NOTE: we could implement each digit as a separate characters

GUI for string conflicts: use popups


%% Acknowledgments
\begin{acks}                            %% acks environment is optional
  %% contents suppressed with 'anonymous'
  %% Commands \grantsponsor{<sponsorID>}{<name>}{<url>} and
  %% \grantnum[<url>]{<sponsorID>}{<number>} should be used to
  %% acknowledge financial support and will be used by metadata
  %% extraction tools.
  This material is based upon work supported by the
  \grantsponsor{GS100000001}{National Science
    Foundation}{http://dx.doi.org/10.13039/100000001} under Grant
  No.~\grantnum{GS100000001}{nnnnnnn} and Grant
  No.~\grantnum{GS100000001}{mmmmmmm}.  Any opinions, findings, and
  conclusions or recommendations expressed in this material are those
  of the author and do not necessarily reflect the views of the
  National Science Foundation.
\end{acks}

%% Appendix
\appendix

%% Appendixes that will be published  go **BEFORE** the bibliography and count towards the page count

%% Bibliography
\bibliography{grove-paper}

%% Temporary appendixes that will not be published go **AFTER** the bibliography and do not count towards the page count

\clearpage % Put these appendixes on separate pages so we can easily remove them from the document.  Also ensure all figures have been placed.


\section{Supplemental Material: Complete Graph Sequences for All Figures}%
\label{apx:Supplemental Material: Complete Graph Sequences for All Figures}

TODO: put all intermediate states for all graphs along with the edge actions

\section{Formalism}

%%%%%%%%%%%%%%%%%%%%%%%%%%%%%%%%%%%%%%%%%%%%%%%%%%%%%%%%%%%%%%%%%%%%%%%%%%%%%%%%

\subsection{Terms}

\figureTermSyntaxContent

For conflict hole forms $\conflictHole{t_i}_{i \leq n}$, it must be the case that $n \geq 2$.
We identify conflict holes up to reordering.

% % % % % % % % % % % % % % % % % % % % % % % % % % % % % % % % % % % % % % % % 

\subsubsection{Term Constructors}

\[
  \arraycolsep=0pt
  \begin{array}{ll}
    \multicolumn{2}{l}{\sortOp : \K \to \SetOf{Exp, Pat, Typ}} \\
    \hline
    \sort{\Root}={} & Exp \\
    \sort{\ExpVar(x)}={} & Exp \\
    \sort{\ExpLam}={} & Exp \\
    \sort{\ExpApp}={} & Exp \\
    \sort{\ExpPlus}={} & Exp \\
    \sort{\ExpTimes}={} & Exp \\
    \sort{\ExpNum(n)}={} & Exp \\
    \sort{\PatVar(x)}={} & Pat \\
    \sort{\TypArrow}={} & Typ \\
    \sort{\TypNum}={} & Typ \\
  \end{array}
\]
%
\figureArityContent
%
\[
  \arraycolsep=0pt
  \begin{array}{ll}
    \multicolumn{2}{l}{\defaultposOp : \K \to \P} \\
    \hline
    \defaultpos{\Root}={} & \Root \\
    \defaultpos{\ExpLam}={} & \LamParam \\
    \defaultpos{\ExpApp}={} & \AppFun \\
    \defaultpos{\ExpPlus}={} & \PlusLeft \\
    \defaultpos{\ExpTimes}={} & \TimesLeft \\
    \defaultpos{\TypArrow}={} & \ArrowArg \\
  \end{array}
\]

We write $\defaultpos{k}$ undefined when there does not exist a $p$ such that $\defaultpos{k} = p$.

\noindent $\boxed{\constructor{t} = k}$
%
\begin{align*}
  \constructor{e} &= \econstructor{e} \\
  \constructor{p} &= \pconstructor{p} \\
  \constructor{\tau} &= \tconstructor{\tau}
\end{align*}

\noindent $\boxed{\econstructor{e} = k}$
%
\begin{align*}
  \econstructor{\eVar{G}{x}} &= \ExpVar(x) \\
  \econstructor{\eFun{G}{q}{\tau}{e}} &= \ExpLam \\
  \econstructor{\eApp{G}{e_1}{e_2}} &= \ExpApp \\
  \econstructor{\eNum{G}{n}} &= \ExpNum(n) \\
  \econstructor{\ePlus{G}{e_1}{e_2}} &= \ExpPlus \\
  \econstructor{\eTimes{G}{e_1}{e_2}} &= \ExpTimes
\end{align*}

\noindent $\boxed{\pconstructor{q} = k}$
%
\begin{align*}
  \pconstructor{\pVar{G}{x}} &= \PatVar(x)
\end{align*}

\noindent $\boxed{\tconstructor{\tau} = k}$
%
\begin{align*}
  \tconstructor{\tArrow{G}{\tau_1}{\tau_2}} &= \TypArrow \\
  \tconstructor{\tNum{G}} &= \TypNum
\end{align*}

%%%%%%%%%%%%%%%%%%%%%%%%%%%%%%%%%%%%%%%%%%%%%%%%%%%%%%%%%%%%%%%%%%%%%%%%%%%%%%%%

\subsection{Graphs}

Let $t$ denote a term and $\Set$ a set of terms.

\begin{definition}
  A \emph{constructor} $k$ names the form of a term that is not a hole.
\end{definition}

\begin{definition}
  A \emph{unique identifier} $u \in \U$ is an object that can be distiinguished from other unique identifiers by inspection.
\end{definition}

\begin{definition}
  A \emph{vertex} $v{=}(u, k)$ is an instance of constructor $k \in \K$ with unique identifier $u$.
\end{definition}

\begin{definition}
  A \emph{position} $p$ names an origin of directed edges.
\end{definition}

\begin{definition}
  An \emph{edge} $\e{=}(u, v, p, v^\prime)$ is a directed connection from vertex $v$ to vertex $v'$, originating from position $p$, and identified by $u$.
  We call $v$ the \emph{source} vertex and $v'$ the \emph{target} vertex.
\end{definition}

\begin{definition}
  An \emph{edge state} $s \in \Sigma{=}\SetOf{\Plus, \Minus}$ indicates whether an edge is live $\left(\Plus\right)$ or deleted $\left(\Minus\right)$.
\end{definition}

\begin{definition}
  A \emph{graph} $G : \E \rightarrow \Sigma$ is a partial function from edges to edge states.
  We write $G(\e) = \bot$ to mean that $G(\e)$ is undefined.
\end{definition}

\begin{definition}
  The \emph{in-graph} of a vertex $v$ is the subgraph containing all mappings from edges targeting $v$ to $\Plus$ or $\Minus$.
\end{definition}

\begin{definition}
  The \emph{out-edges} of a vertex $v$ and position $p$ is the set of all live edges with source $v$ originating from $p$.
\end{definition}

\begin{definition}
  The \emph{parents} of a vertex $v$ is the set of all live edges targeting $v$.
\end{definition}

\begin{definition}
  The \emph{ancestors} of a vertex $v$ is defined recursively as the parents of $v$ and their ancestors.
\end{definition}

% \begin{definition}
%   The \emph{in-graph} of a term $t$ is the subgraph associated with $t$, when it exists, which maps every edge targeting the root vertex of $t$.
%   For example, when $t$ is of the form $\eFun{G}{q}{\tau}{e}$, its in-graph is $G$.
%   If $t$ is of the form $\multiVertex{\e}$ or $\cycleVertex{\e}$, its in-graph maps the associated edge to $\Plus$ and all other edges to $\bot$.
%   If $t$ is not a reference and does not have an associated graph, then it is a conflict hole and therefore does not have an in-graph.
%   We write $G_t$ to denote the in-graph of $t$.
% \end{definition}

% Let $G_t$, called the \emph{in-graph} of term $t$, denote one of the following:
% \begin{itemize}
%   \item
%     The graph associated with term $t$, when such a graph exists.
%     For example, when $t$ is of the form $\eFun{G}{q}{\tau}{e}$, we have $G_t = G$.
%   \item
%     Otherwise,
%     if $t \in \SetOf{\multiVertex{\e}, \cycleVertex{\e}}$,
%     then $G_t = \SetOf{\e \mapsto \Plus}$.
% \end{itemize}

% Let $v_{G_t}$, called the \emph{in-vertex} of term $t$,
% denote the destination vertex of the edge(s) of $G_t$,
% provided that $t$ is well sorted.

Let $\fresh{u}$ denote a fresh id.

% % % % % % % % % % % % % % % % % % % % % % % % % % % % % % % % % % % % % % % % 

\subsubsection{Well-sortedness}

\begin{definition}
  A graph $G$ is well sorted if $\edges{G}$ are well sorted.
\end{definition}

\begin{definition}
  An edge $(u, (u_1, k_1), p, (u_2, k_2))$ is well sorted if $(p, \sort{k_2}) \in \arity{k_1}$.
\end{definition}

\begin{definition}
  A grove $(\Set[NP], \Set[MP], \Set[U])$ is well sorted if all of the following hold:
  \begin{enumerate}
    \item The terms of $\Set[NP], \Set[MP],$ and $\Set[U]$ are well sorted.
      % $\Set[NP]$, $\Set[MP]$, and $\Set[U]$ contain only well sorted terms.
    \item For all $t \in \Set[NP]$,
      we have $\SizeOf{\SetOf{\e \SuchThat{G_t(\e) = \Plus}}} = 0$.
      % The terms in $\Set[NP]$ correspond to edges with target vertices that have no parents.
    \item For all $t \in \Set[MP]$,
      we have $\SizeOf{\SetOf{\e \SuchThat{G_t(\e) = \Plus}}} > 1$.
      % The terms in $\Set[MP]$ correspond to edges with target vertices that have multiple parents.
    \item For all $t \in \Set[U]$, all of the following hold:
      \begin{enumerate}
        \item $\SizeOf{\SetOf{\e{=}(u, v, p, v') \SuchThat{G_t(\e) = \Plus}}} = 1$.
        \item $v' = \min{\ancestors{v'}}$.
      \end{enumerate}
      % The terms in $\Set[U]$ correspond to edges with target vertices that are unicycle roots.
    \item For all $t \in \Set[MP] \cup \Set[U]$,
      we have $\SizeOf{\SetOf{\e{=}(u, \rootVertex, \Root, v) \SuchThat{\e \in \edges{G_t} \land u \in \U \land v \in \V}}} = 0$.
      % Any terms corresponding to edges originating from $\rootVertex$ are in $\Set[NP]$.
  \end{enumerate}
\end{definition}

\begin{definition}
  A term $t$ is well sorted if one of the following holds:
  \begin{itemize}

    \item $t = \emptyHole{v{=}(u, k)}{p}$
      and there exists $s \in \SetOf{Exp, Pat, Typ}$
      such that $(p, s) \in \arity{k}$.
      % An empty hole is well sorted if the associated source vertex can have
      % children at the associated position.

    \item $t = \conflictHole{t_i}_{i \leq n}$ and all of the following hold:
      \begin{enumerate}
        \item For all $i = 1, \ldots, n$, we have $t_i$ is well sorted.
          % The conflicting terms are well sorted ...
        \item
          There exist $v_0 \in \V, p_0 \in \P$ such that,
          for all $i = 1, \ldots, n$,
          and all $\e{=}(u, v, p, v')$ such that $G_{t_i}(\e) = \Plus$,
          we have $v = v_0$ and $p = p_0$.
          % ... and all of their incoming edges have the same source vertex and position.
      \end{enumerate}

    \item $t = \multiVertex{\e}$ and all of the following hold:
      \begin{enumerate}
        \item $\e$ is well sorted.
          % The corresponding edge is well sorted ...
        \item $\SizeOf{\SetOf{\e' \SuchThat{G_t(\e') = \Plus}}} > 1$.
          % ... and its target vertex has multiple parents.
      \end{enumerate}

    \item $t = \cycleVertex{(u, v, p, v')}$ and all of the following hold:
      \begin{enumerate}
        \item $(u, v, p, v')$ is well sorted.
          % The corresponding edge is well sorted ...
        \item $v' = \min{\ancestors{v'}}$
          % ... and its target vertex is the root ...
        \item For all $\e{=}(u_{\e}, v_{\e}, p_{\e}, v_{\e}')$
          such that $v_{\e}' \in \ancestors{v'}$,
          we have $\SizeOf{\parents{v_{\e}'}} = 1$.
          % ... of a unicycle.
      \end{enumerate}

    \item Otherwise, $G_t$ exists and all of the following hold:
      \begin{enumerate}
        \item $G_t$ is well sorted.
          % All incoming edges are well sorted ...
        \item For all $\e{=}(u, v, p, v') \in \edges{G_t}$,
          we have $v' = v_{G_t}$.
          % ... all incoming edges have the same destination vertex (and we call it the in-vertex of t).
        \item $\SizeOf{\edges{G_t}} \geq 1$.
          % The in-graph of t is not empty.
        \item There exists $u_0 \in \U$ such that,
          for all $\e{=}(u, v, p, v') \in \edges{G_t}$,
          we have $v' = v_{G_t} = (u_0, \constructor{t})$.
          % The destination vertices of all incoming edges have the same id and constructor.
          % The constructor matches the term.
      \end{enumerate}
  \end{itemize}
\end{definition}

% TODO: update confliceHole usage
% TODO: rename multiVertex / cycleVertex

%%%%%%%%%%%%%%%%%%%%%%%%%%%%%%%%%%%%%%%%%%%%%%%%%%%%%%%%%%%%%%%%%%%%%%%%%%%%%%%%

\subsection{Decomposition and Recomposition}

\figureDecompositionDefHelpersContent

% % % % % % % % % % % % % % % % % % % % % % % % % % % % % % % % % % % % % % % % 

\subsubsection{Decomposition}

\begin{theorem}
  For any well sorted graph $G$
  there exists a well sorted grove $\Grove$
  such that \[\decomp{G} = \Grove.\]
\end{theorem}

% Proof: provide a witness that demonstrates the conclusion.

\noindent $\boxed{\decomp{G} = \Grove}$
%
\figureDecompositionDefDecomp

\noindent $\boxed{\decomp{\e} = t}$
%
\figureDecompositionDefDecompTerm

\noindent $\boxed{\edecomp{\e} = e}$
%
\figureDecompositionDefEdecomp

\noindent $\boxed{\pdecomp{\e} = q}$
%
\figureDecompositionDefPdecomp

\noindent $\boxed{\tdecomp{\e} = \tau}$
%
\figureDecompositionDefTdecomp

\noindent $\boxed{\edecompPrime{\e}{p} = e}$
%
\figureDecompositionDefEdecompPrime

\noindent $\boxed{\pdecompPrime{\e}{p} = q}$
%
\figureDecompositionDefPdecompPrime

\noindent $\boxed{\tdecompPrime{\e}{p} = \tau}$
%
\figureDecompositionDefTdecompPrime%

% % % % % % % % % % % % % % % % % % % % % % % % % % % % % % % % % % % % % % % % 

\subsubsection{Recomposition}\hspace*{\fill} \\

\begin{theorem}
  For any well sorted graph $G$,
  if $\decomp{G} = \Grove$ then $\recomp{\Grove} = G$.
\end{theorem}

\noindent $\boxed{\recomp{\Grove} = G}$
%
\begin{align*}
  \recomp{(\Set[NP], \Set[MP], \Set[U])} &= \bigcup_{t \in \Set[NP] \cup \Set[MP] \cup \Set[U]} \recomp{t}
\end{align*}

\noindent $\boxed{\recomp{t} = G}$
%
\begin{align*}
  \recomp{e} &= \erecomp{e} \\
  \recomp{q} &= \precomp{q} \\
  \recomp{\tau} &= \trecomp{\tau}
\end{align*}

\noindent $\boxed{\erecomp{e} = G}$
%
\begin{align*}
  \erecomp{\eVar{G}{x}} &= G
  \\
  \erecomp{\eFun{G}{q}{\tau}{e}}
    &= G \cup \precomp{q} \cup \trecomp{\tau} \cup \erecomp{e}
  \\
  \erecomp{\eApp{G}{e_1}{e_2}}
    &= G \cup \erecomp{e_1} \cup \erecomp{e_2}
  \\
  \erecomp{\eNum{G}{n}} &= G
  \\
  \erecomp{\ePlus{G}{e_1}{e_2}}
    &= G \cup \erecomp{e_1} \cup \erecomp{e_2}
  \\
  \erecomp{\eTimes{G}{e_1}{e_2}}
    &= G \cup \erecomp{e_1} \cup \erecomp{e_2}
  \\
  \erecomp{\conflictHole{e_i}_{i \leq n}}
  &= \bigcup_{i=1}^n \erecomp{e_i}
  \\
  \erecomp{\multiVertex{\e}} &= \SetOf{\e \mapsto \Plus}
  \\
  \erecomp{\cycleVertex{\e}} &= \SetOf{\e \mapsto \Plus}
  \\
  \erecomp{\emptyHole{v}{p}} &= \SetOf{}
\end{align*}

\noindent $\boxed{\precomp{q} = G}$
%
\begin{align*}
  \precomp{\pVar{G}{x}} &= G
  \\
  \precomp{\conflictHole{q_i}_{i \leq n}} &= \bigcup_{i=1}^n \precomp{q_i}
  \\
  \precomp{\multiVertex{\e}} &= \SetOf{\e \mapsto \Plus}
  \\
  \precomp{\cycleVertex{\e}} &= \SetOf{\e \mapsto \Plus}
  \\
  \precomp{\emptyHole{v}{p}} &= \SetOf{}
\end{align*}

\noindent $\boxed{\trecomp{\tau} = G}$
%
\begin{align*}
  \trecomp{\tArrow{G}{\tau_1}{\tau_2}}
    &= G \cup \trecomp{\tau_1} \cup \trecomp{\tau_2}
  \\
  \trecomp{\tNum{G}} &= G
  \\
  \trecomp{\conflictHole{\tau_i}_{i \leq n}} &= \bigcup_{i=1}^n \trecomp{\tau_i}
  \\
  \trecomp{\multiVertex{\e}} &= \SetOf{\e \mapsto \Plus}
  \\
  \trecomp{\cycleVertex{\e}} &= \SetOf{\e \mapsto \Plus}
  \\
  \trecomp{\emptyHole{v}{p}} &= \SetOf{}
\end{align*}

%%%%%%%%%%%%%%%%%%%%%%%%%%%%%%%%%%%%%%%%%%%%%%%%%%%%%%%%%%%%%%%%%%%%%%%%%%%%%%%%

\subsection{Cursors}

% % % % % % % % % % % % % % % % % % % % % % % % % % % % % % % % % % % % % % % % 

\subsubsection{Zippered Terms}

\[
  \arraycolsep=0pt
  \begin{array}{lcllll}
    \Z{t} & {}\in{} & ZTerm & {}::={} &
      \Z{e}
      \mid \Z{q}
      \mid \Z{\tau}
    \\
    \Z{e} & {}\in{} & ZExp & {}::={} &
      \cursor{e}
      \mid \eFun{G}{\Z{q}}{\tau}{e}
      \mid \eFun{G}{q}{\Z{\tau}}{e}
      \mid \eFun{G}{q}{\tau}{\Z{e}}
      \mid \eApp{G}{\Z{e}}{e}
      \mid \eApp{G}{e}{\Z{e}}
      \mid \ePlus{G}{\Z{e}}{e}
      \mid \ePlus{G}{e}{\Z{e}}
      \mid \eTimes{G}{\Z{e}}{e}
      \\
    &&&&
      \mid \eTimes{G}{e}{\Z{e}}
      \mid \conflictHole{\Z{e}, \SetOf{e_i}_{i \leq n}}
    \\
    \Z{q} & {}\in{} & ZPat & {}::={} &
      \cursor{q}
      \mid \conflictHole{\Z{q}, \SetOf{q_i}_{i \leq n}}
    \\
    \Z{\tau} & {}\in{} & ZTyp & {}::={} &
      \cursor{\tau}
      \mid \tArrow{G}{\Z{\tau}}{\tau}
      \mid \tArrow{G}{\tau}{\Z{\tau}}
      \mid \conflictHole{\Z{\tau}, \SetOf{\tau_i}_{i \leq n}}
    \\
  \end{array}
\]

Let $\Z{\Set} = (\Set, \Z{t})$ denote a set of terms paired with a zippered term.

% % % % % % % % % % % % % % % % % % % % % % % % % % % % % % % % % % % % % % % % 

\subsubsection{Zippered Groves}

\begin{gather*}
  \arraycolsep=0pt
  \begin{array}{lll}
    \Z{\gamma} & {}::={} &
      (\Z{\Set}_{NP}, \Set[MP], \Set[U])
      \mid (\Set[NP], \Z{\Set}_{MP}, \Set[U])
      \mid (\Set[NP], \Set[MP], \Z{\Set}_U)
  \end{array}
\end{gather*}

% % % % % % % % % % % % % % % % % % % % % % % % % % % % % % % % % % % % % % % % 

\subsubsection{Cursor Erasure}\hspace*{\fill} \\

\noindent $\boxed{\erase{\Z{\gamma}} = \gamma}$
%
\begin{align*}
  \erase{(\Z{\Set}_{NP}, \Set[MP], \Set[U])} &= (\erase{\Z{\Set}_{NP}}, \Set[MP], \Set[U]) \\
  \erase{(\Set[NP], \Z{\Set}_{MP}, \Set[U])} &= (\Set[NP], \erase{\Z{\Set}_{MP}}, \Set[U]) \\
  \erase{(\Set[NP], \Set[MP], \Z{\Set}_U)} &= (\Set[NP], \Set[MP], \erase{\Z{\Set}_U})
\end{align*}

\noindent $\boxed{\erase{\Z{\Set}} = \Set}$
%
\begin{align*}
  \erase{(\Set, \Z{t})} &= \Set \cup \SetOf{t}
\end{align*}

\noindent $\boxed{\erase{\Z{e}} = e}$
%
\begin{align*}
  \erase{\cursor{e}} &= e \\
  \erase{\left(\eFun{G}{\Z{q}}{\tau}{e}\right)} &= \eFun{G}{\erase{\Z{q}}}{\tau}{e} \\
  \erase{\left(\eFun{G}{q}{\Z{\tau}}{e}\right)} &= \eFun{G}{q}{\erase{\Z{\tau}}}{e} \\
  \erase{\left(\eFun{G}{q}{\tau}{\Z{e}}\right)} &= \eFun{G}{q}{\tau}{\erase{\Z{e}}} \\
  \erase{\eApp{G}{\Z{e}_1}{e_2}} &= \eApp{G}{\erase{\Z{e}_1}}{e_2} \\
  \erase{\eApp{G}{e_1}{\Z{e}_2}} &= \eApp{G}{e_1}{\erase{\Z{e}_2}} \\
  \erase{\ePlus{G}{\Z{e}_1}{e_2}} &= \ePlus{G}{\erase{\Z{e}_1}}{e_2} \\
  \erase{\ePlus{G}{e_1}{\Z{e}_2}} &= \ePlus{G}{e_1}{\erase{\Z{e}_2}} \\
  \erase{\eTimes{G}{\Z{e}_1}{e_2}} &= \eTimes{G}{\erase{\Z{e}_1}}{e_2} \\
  \erase{\eTimes{G}{e_1}{\Z{e}_2}} &= \eTimes{G}{e_1}{\erase{\Z{e}_2}} \\
  \erase{\conflictHole{\Z{e}, \SetOf{e_i}_{i \leq n}}} &= \conflictHole{\erase{\Z{e}}, \SetOf{e_i}_{i \leq n}}
\end{align*}

\noindent $\boxed{\erase{\Z{q}} = q}$
%
\begin{align*}
  \erase{\cursor{q}} &= q \\
  \erase{\conflictHole{\Z{q}, \SetOf{q_i}_{i \leq n}}} &= \conflictHole{\erase{\Z{q}}, \SetOf{q_i}_{i \leq n}}
\end{align*}

\noindent $\boxed{\erase{\Z{\tau}} = \tau}$
%
\begin{align*}
  \erase{\cursor{\tau}} &= \tau \\
  \erase{\left(\tArrow{G}{\Z{\tau}_1}{\tau_2}\right)} &= \tArrow{G}{\erase{\Z{\tau}_1}}{\tau_2} \\
  \erase{\left(\tArrow{G}{\tau_1}{\Z{\tau}_2}\right)} &= \tArrow{G}{\tau_1}{\erase{\Z{\tau}_2}} \\
  \erase{\conflictHole{\Z{\tau}, \SetOf{\tau_i}_{i \leq n}}} &= \conflictHole{\erase{\Z{\tau}}, \SetOf{\tau_i}_{i \leq n}}
\end{align*}

%%%%%%%%%%%%%%%%%%%%%%%%%%%%%%%%%%%%%%%%%%%%%%%%%%%%%%%%%%%%%%%%%%%%%%%%%%%%%%%%

\subsection{User Actions}

\begin{theorem}[Sensibility]
  For any zippered grove $\Z{\Grove}$
  such that $\erase{\Z{\Grove}}$ is well sorted
    and $\recomp{\erase{\Z{\Grove}}} = G$,
  and any user action $\alpha$,
  if
  \[
    \applyAction{\Z{\Grove}}{\alpha}{a^{*}},
  \]
  then $G \action{a^{*}} G'$ and $G'$ is well sorted.
\end{theorem}

\[
  \arraycolsep=0pt
  \begin{array}{llll}
    \alpha & {}::={} &
      \Construct{k}
      \mid \Delete
      \mid \Reposition{v}{p}
    \\
    c & {}::={} &
      \CVertex{v}
      \mid \CEdge{\e}
      \mid \CSource{v}{p}
      \mid \CSourceVertex{v}{p}{v}
      \mid \CSourceEdge{v}{p}{\e}
    \\
  \end{array}
\]

\noindent $\boxed{\applyActionX{ \Z{\Grove} }{ c }{ \alpha }{ \Q{a} }{ c' }}$
%
\begin{mathpar}
  \inferrule{
    \applyActionX{ \Z{\Set}_{NP} }{ c }{ \alpha }{ \Q{a} }{ c' }
  }{
    \applyActionX{ (\Z{\Set}_{NP}, \Set[MP], \Set[U]) }{ c }{ \alpha }{ \Q{a} }{ c' }
  }

  \inferrule{
    \applyActionX{ \Z{\Set}_{MP} }{ c }{ \alpha }{ \Q{a} }{ c' }
  }{
    \applyActionX{ (\Set[NP], \Z{\Set}_{MP}, \Set[U]) }{ c }{ \alpha }{ \Q{a} }{ c' }
  }

  \inferrule{
    \applyActionX{ \Z{\Set}_U }{ c }{ \alpha }{ \Q{a} }{ c' }
  }{
    \applyActionX{ (\Set[NP], \Set[MP], \Z{\Set}_U) }{ c }{ \alpha }{ \Q{a} }{ c' }
  }
\end{mathpar}

\noindent $\boxed{\applyActionX{ \Z{\Set} }{ c }{ \alpha }{ \Q{a} }{ c' }}$
%
\begin{mathpar}
  \inferrule{
    \applyActionX{ \Z{t} }{ c }{ \alpha }{ \Q{a} }{ c' }
  }{
    \applyActionX{ (\Set, \Z{t}) }{ c }{ \alpha }{ \Q{a} }{ c' }
  }
\end{mathpar}

\noindent $\boxed{\applyActionX{ \Z{t} }{ c }{ \alpha }{ \Q{a} }{ c' }}$
%
\begin{mathpar}
  \inferrule{
    \applyActionX{ \Z{q} }{ c }{ \alpha }{ \Q{a} }{ c' }
  }{
    \applyActionX{ (\eFun{G}{\Z{q}}{\tau}{e}) }{ c }{ \alpha }{ \Q{a} }{ c' }
  }

  \inferrule{
    \applyActionX{ \Z{\tau} }{ c }{ \alpha }{ \Q{a} }{ c' }
  }{
    \applyActionX{ (\eFun{G}{q}{\Z{\tau}}{e}) }{ c }{ \alpha }{ \Q{a} }{ c' }
  }

  \inferrule{
    \applyActionX{ \Z{e} }{ c }{ \alpha }{ \Q{a} }{ c' }
  }{
    \applyActionX{ (\eFun{G}{q}{\tau}{\Z{e}}) }{ c }{ \alpha }{ \Q{a} }{ c' }
  }

  \inferrule{
    \applyActionX{ \Z{e}_1 }{ c }{ \alpha }{ \Q{a} }{ c' }
  }{
    \applyActionX{ \eApp{G}{\Z{e}_1}{e_2} }{ c }{ \alpha }{ \Q{a} }{ c' }
  }

  \inferrule{
    \applyActionX{ \Z{e}_2 }{ c }{ \alpha }{ \Q{a} }{ c' }
  }{
    \applyActionX{ \eApp{G}{e_1}{\Z{e}_2} }{ c }{ \alpha }{ \Q{a} }{ c' }
  }

  \inferrule{
    \applyActionX{ \Z{e}_1 }{ c }{ \alpha }{ \Q{a} }{ c' }
  }{
    \applyActionX{ (\ePlus{G}{\Z{e}_1}{e_2}) }{ c }{ \alpha }{ \Q{a} }{ c' }
  }

  \inferrule{
    \applyActionX{ \Z{e}_2 }{ c }{ \alpha }{ \Q{a} }{ c' }
  }{
    \applyActionX{ (\ePlus{G}{e_1}{\Z{e}_2}) }{ c }{ \alpha }{ \Q{a} }{ c' }
  }

  \inferrule{
    \applyActionX{ \Z{e}_1 }{ c }{ \alpha }{ \Q{a} }{ c' }
  }{
    \applyActionX{ (\eTimes{G}{\Z{e}_1}{e_2}) }{ c }{ \alpha }{ \Q{a} }{ c' }
  }

  \inferrule{
    \applyActionX{ \Z{e}_2 }{ c }{ \alpha }{ \Q{a} }{ c' }
  }{
    \applyActionX{ (\eTimes{G}{e_1}{\Z{e}_2}) }{ c }{ \alpha }{ \Q{a} }{ c' }
  }

  \inferrule{
    \applyActionX{ \Z{\tau}_1 }{ c }{ \alpha }{ \Q{a} }{ c' }
  }{
    \applyActionX{ (\tArrow{G}{\Z{\tau}_1}{\tau_2}) }{ c }{ \alpha }{ \Q{a} }{ c' }
  }

  \inferrule{
    \applyActionX{ \Z{\tau}_2 }{ c }{ \alpha }{ \Q{a} }{ c' }
  }{
    \applyActionX{ (\tArrow{G}{\tau_1}{\Z{\tau}_2}) }{ c }{ \alpha }{ \Q{a} }{ c' }
  }

  \inferrule{
    \applyActionX{ \Z{t} }{ c }{ \alpha }{ \Q{a} }{ c' }
  }{
    \applyActionX{ \conflictHole{\Z{t}, \SetOf{t_i}_{i \leq n}} }{ c }{ \alpha }{ \Q{a} }{ c' }
  }
\end{mathpar}
%
\begin{mathpar}
  \inferrule[ConstructWrap]{
    \sources{G_t} = \SetOf{(v_{s_i}, p_{s_i})}_{i \leq n} \\
    \defaultpos{k} = p_k \\
    \sort{k} = \sort{k_{G_t}} \\
    (p_k, \sort{k}) \in \arity{k} \\
    v_k = (\fresh{u}_k, k) \\
    \SetOf{\e_i}_{i \leq n} = \SetOf{(\fresh{u}_i, v_{s_i}, p_{s_i}, v_k)}_{i \leq n} \\
    \e' = (\fresh{u}, v_k, p_k, v_{G_t})
  }{
    \applyActionX{ \cursor{t} }{ c }{ \Construct{k} }{
      \SetOf{\graphAction{\Minus}{\e}}_{\e \in \edges{G_t}} \cup
      \SetOf{\graphAction{\Plus}{\e_i}}_{i \leq n} \cup
      \SetOf{\graphAction{\Plus}{\e'}}
    }{ c }
  }
  
  \inferrule[ConstructConflict]{
    \sources{G_t} = \SetOf{(v_{s_i}, p_{s_i})}_{i \leq n} \\
    \defaultpos{k} \text{ undefined} \\
    \sort{k} = \sort{k_{G_t}}
  }{
    \applyAction{
      \cursor{t}
    }{\Construct{k}}{
      \SequenceOf{\graphAction{\Plus}{(\fresh{u}_{s_i}, v_{s_i}, p_{s_i}, (\fresh{u}_k, k))}}_{i \leq n}
    }
  }

  \inferrule[Construct]{
    (p_s, \sort{k}) \in \arity{k_s}
  }{
    \applyAction{
      \cursor{\emptyHole{v_s{=}(u_s, k_s)}{p_s}}
    }{\Construct{k}}{
      \graphAction{\Plus}{(\fresh{u}, v_s, p_s, (\fresh{u}_k, k))}
    }
  }

  \inferrule[Delete]{
    \edges{G_t} = \SetOf{\e_i}_{i \leq n}
  }{
    \applyAction{
      \cursor{t}
    }{\Delete}{
      \SequenceOf{\graphAction{\Minus}{\e_i}}_{i \leq n}
    }
  }

  \inferrule[Reposition]{
    \edges{G_t} = \SetOf{\e_i}_{i \leq n} \\
    (p, \sort{k_{G_t}}) \in \arity{k}
  }{
    \applyAction{
      \cursor{t}
    }{\Reposition{v{=}(u, k)}{p}}{
      \SequenceOf{\graphAction{\Minus}{\e_i}}_{i \leq n};
      \graphAction{\Plus}{(\fresh{u}, v, p, v_{G_t})}
    }
  }

  \inferrule[Conflict]{
    \SetOf{\applyAction{\cursor{t_i}}{\alpha}{a_i^{*}}}_{i \leq n}
  }{
    \applyAction{
      \cursor{\conflictHole{t_i}_{i \leq n}}
    }{\alpha}{
      \SequenceOf{a_i^{*}}_{i \leq n}
    }
  }
\end{mathpar}


\end{document}