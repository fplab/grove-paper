
\section{The Grove Workbench}%
\label{sec:Implementation}

We implemented the core Grove calculus of the previous section as an \texttt{OCaml} library called the Grove Workbench and a corresponding web-based collaborative structure editor written using \texttt{js\_of\_ocaml}~\cite{DBLP:journals/spe/VouillonB14} primarily intended to demonstrate the collaborative features of the workbench and serve as a companion to the formal developments in this paper. The library is parameterized by a syntax specification for expressions, with the necessary data structures generated automatically.

\subsection{The Grove Workbench}
\label{sub:impl-grv}

On opening up the \emph{Grove Workbench}, the user is met with two almost identical panels side-by-side, emulating a collaborative editor environment between two users. Additional collaborators can be generated on command. In each case, a cursor is placed on an empty hole at the root of the displayed term decomposition, which is displayed as a graph visualization of the same graph structure so that the UI resembles the figures in \autoref{sec:Grove By Example}. Below this are buttons for various user edit actions, and commands to send queued commands to specific other users. In addition, we have separate panels for multi-parented, deleted, and unicycle panels, corresponding to the partitioned grove datastructure in the previous section. We will now examine them in correspondence to the formalism described in \autoref{sec:Formalism}. 

% As the graph forms a CmRDT, sharing edits across instances is a matter of record and replay. When the user performs some edits, which correspond to patches formally, the \emph{actions} panel displays the local history of edit actions. These patches can be shared across editor instances, and the local history panel is flushed upon synchronization. The user edits the state of the graph, which is displayed in the graph visualization, and upon these edits, the graph gets \emph{decomposed} into a grove. Under the hood, the model is able to \emph{classify} the vertices as one of $NP$, $MP$, or $U$ roots or not a root. The terms in $NP$, $MP$, and $U$ root cases correspond to the deleted, multiparented, and unicycle panels in the workbench, where the latter two are instances of location-based conflicts. The deleted panel also allows the user to select a deleted term and restore it to the cursor location in the editor, which is attached using a fresh edge, as edge deletion is permanent. In addition to performing edits and sharing them, the user can also \emph{clone} and \emph{drop} editor instances to add more collaborative users into the mix. This is a brief overview of the features of the workbench.


% patches -> history of edit actions -> actions panel
% user interacts with tree structure editor backed by the graph data-structure and recomp-decomp
% generates dsa in Ocaml given a syntax tree spec

\subsection{Graph Implementation}
\label{sec: Graph Implementation}
The graph data structure is implemented as \texttt{OCaml Map} data structure with insertion and selection operations whose asymptotic worst-case complexity is logarithmic with respect to the size of the map. Since we cannot implement an infinite mapping directly, the graph only maps live or deleted edges to edge states $\left\{\Plus, \Minus\right\}$. We do not represent edges that map to $\bot$.

For a graph $G : (\E = \U \times \V \times \P \times \V) \to \Sigma$, our graph decomposition algorithm runs in $O(\abs{\V} \log \abs{\V} + \abs{\E} \log \abs{\V})$.
It begins with a scan of all edges that have been created or deleted $O(\abs{\E})$.
Their vertices at both ends are partitioned into three sets: multi-parented, single-parented, or orphaned.
Their relationships are recorded in maps for $O(\log \abs{\V})$ lookups of parent and child edge sets.
After the vertices have been partitioned, we traverse the various single-parented components and produce equivalent expressions $O(\abs{\V})$.
For unicycles, we traverse backward until a vertex is seen twice $O(\abs{\V})$, then proceed forward to find the least vertex on the cycle $O(\abs{\V})$. Once the least vertex is found, decomposition of unicycle begins with it, thus ensuring any edges to it become references to a unicycle root.

% \subsection{Grove Implementation}
% \label{sec: Grove Implementation}


% \subsection{Parameterized Code Generation}
% \label{sec: Parameterized Code Generation}

% How do you give it the corresponding language definition to generate child positions, arities


% % TODO: rename Decomposition.ml to Grove.ml

% There's a function \verb!decompose! of type \verb!Graph.t) : t * Edge.Set.t Vertex.Map.t! that 

% From this graph, we compute the set of all edges and the set of live edges with a linear scan.

% % TODO: make this into a listing

% \begin{verbatim}
%   Graph.edges : Graph.t -> Edge.Set.t
%   Graph.live_edges : Graph.t -> Edge.Set.t
% \end{verbatim}

% % TODO: keep connecting the code to the formalism like this

% These vertices are partitioned into three sets: multi-parented, single-parented, or orphaned.
% A vertex $v$ is considered a parent of another vertex $v'$ if there is a live edge from $v$ to $v'$.

% For each vertex, we take the parent and child edge sets.

% % TODO: fix the formalism for parents (to work with edges instead of vertices) and then check that it matches the implementation. Try renaming parents to vparents and adding parents for edges.

% % TODO: Talk about how the graph is implemented (not as an actual function, but as a map from edges to edge states) so that we can denote the set of created + deleted edges clearly, as well as all of the vertices that have ever been created.



% TODO: is this still a thing?
% Compared to the core language, our implementation lacks support for lists and case expressions.

% Optimizations: least fixed point


% We determine $e$ by starting at the root of the graph and including every connected
% vertex that has a single parent. In positions where there are no children, we
% leave an empty hole $\hole$. In positions where there are multiple children, we
% leave a conflict hole of the form $\conflictHole{e_1,\ldots,e_n}$. Whenever we
% encounter a vertex with multiple parents, we add its corresponding expression to
% the second component $MP$ and leave an indirect reference $\multiVertex{u}$, where $u$ is
% the unique identifier of the referenced vertex

% here are some properties that show how to get one from the other, and to/from
% graph

% algorithmic section: if we want to implement these conversions, here are the
% steps (without introducing new concepts):

% - linear scan
% - breaking out roots
% - ...

% Graph decomposition occurs in three steps.



% TODO: use code generation instead: have Lang.ml generate expr / expr' / ... for us; then we don't need this figure.
% TODO: maybe show this figure and say that it's possible to handle generically without code generation

% \begin{figure}
%   \[
%     \arraycolsep=0pt
%     \begin{array}{lrlll}
%       \textrm{Trees:\qquad}  & t & {}\in Tree & {}::={} & \textrm{Vertex}(v, m) \mid \textrm{Ref}(v) \\
%       \textrm{Maps:}         & m & {}\in Map & {}::={} & \varnothing \mid p \mapsto [(\e, t), \ldots, (\e, t)]; m \\
%     \end{array}
%   \]
%   \caption{Syntax of graph decomposition trees as a grammar.}
%   \Description{This figure describes the graph decomposition grammar}
%   \label{fig:Syntax of graph decomposition trees}
% \end{figure}


% lemma: a unicycle is a cycle with trees hanging off of it

% lemma: unicycle traversal always produces a valid tree

% Unicycular graph traversal occurs in two directions. A vertex is chosen as an arbitrary
% starting point and its parents are traversed and marked as seen. Any vertex that
% has been seen twice must be on the cycle. We take the first one as the root of
% the unicycle and traverse its children to produce a tree that covers the unicycular graph.
% Any remaining single-parented vertices must be parts of other unicycular graphs and are
% unicycle-traversed until there are no more single-parented vertices left.



% TODO: Talk about Uuid Generation
% First, we set up \texttt{Uuid.ml}, a module to generate well-founded monotonically increasing unique IDs for all the parts of our code that require unique identifiers.This module has features that allow it to both generate and check for the well-foundedness of a given ID.. Our programmatically generated target language syntax, explained further in \autoref{sec: Parameterized Code Generation}, provides a \texttt{Lang.Constructor.t} corresponding to $k \in \K$ from \autoref{sec:Grove Formalism}.Further, we have a \texttt{Vertex.ml} module that \emph{wraps} a given \texttt{Lang.Constructor.t} with a UUID, such that, $\V = \K \times \U$. This also allows for the vertices to be ordered by their UUIDs. Internally, we represent edge-sources ($s = (v,p)$) as a record containing \texttt{Vertex.t}\textit{s} and \texttt{Lang.Position.t}\textit{s}, where the latter encodes the child positions and arities of the various constructors, in \texttt{Cursor.ml}. That is, a cursor (or a source) serves as a point of focus for graph actions and it references a position relative to some vertex. 

% Then we have edges corresponding to $\E = \U \times \S \times \V$, implemented in \texttt{Egde.ml} UUID wrapped \emph{source} from a source to a target. Edge states ($\Sigma$), are expressed as variant type with two variants \texttt{Created} and \texttt{Deleted} representing $\Plus$ and $\Minus$ respectively.Since we do not directly implement infinite mappings in \texttt{Graph}, $\bot$ does not have a corresponding variant. Further, trees are defined in \texttt{Tree.ml} to represent disjointed graph components. A tree is a vertex with a (possibly empty) set of children or is a reference to one. The children are a list of \emph{child}s, where a child is a tree paired with an edge UUID. We also have a helper \texttt{child\_map : (Lang.Position.t * (Uuid.Id.t $*$ t) list) list $->$ children Position\_map.t} that takes in pairs of target language positions and lists of pairs Uuid and trees to return a map of the children. Here, a position\_map is a structure that \emph{maps} \texttt{Lang.position.t}s to their appropriate child representations.

% Finally, the logic and infrastructure for a grove that a given graph decomposes into using the decomposition algorithm is implemented in \texttt{Grove.ml}.

% TODO: Tree.ml, Position_map.mk, Grove.ml, Graph_action.ml

% TODO: talk about code generation somewhere in here

% TODO: talk about:
% - decomp vertex set generation and unicycle traversal
% - how to choose which branch of expr' to take
% - lfp(ancestors') and min of it

% TODO: can we save any of this in a README or code comments?

% TODO: talk about performance: O(decomp) given "size of the graph" (probably linear in # edges)

% GRV -- how it is implemented, how it connects to the formalism, describe the graph,
