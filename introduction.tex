
\section{Introduction}%
\label{sec:Introduction}

%%%%%%%%%%%%%%%%%%%%%%%%%%%%%%%%%%%%%%%%%%%%%%%%%%%%%%%%%%%%%%%%%%%%%%%%%%%%%%%%

Programming is an increasingly collaborative activity.
Today, most programming editors are not explicitly designed with collaboration in mind,
so programmers rely on ad hoc integrations of asynchronous diff-based version control systems like Git, Darcs, or Pijul.

Asynchronous diffing is not an ideal solution to collaboration because
the merging process must solve the \textbf{inverse problem}---inference of compatible edits in a simplistic edit action language.
For example, suppose Alice and Bob are collaboratively editing a file comprised of a single line \verb|A B|.
Alice changes the file to \verb|A B C| and communicates the change to Bob as a diff (Figure \ref{fig:inverse-problem:alice}).
At the same time, Bob changes the file to \verb|A D B| and sends Alice a diff (Figure \ref{fig:inverse-problem:bob}).
Both parties changed the same line, but in a different way,
so when Bob applies Alice's diff, a merge conflict arises and the contents of Bob's editor is mangled as in Figure \ref{fig:inferse-problem:diff}.

\begin{figure}[H]
  \begin{subfigure}{.45\linewidth}
\begin{verbatim}
1c1
< A B
---
> A B C
\end{verbatim}
    \label{fig:inverse-problem:alice}
    \subcaption{From Alice to Bob}
  \end{subfigure}
  \hfil
  \begin{subfigure}{.45\linewidth}
\begin{verbatim}
1c1
< A B
---
> A D B
\end{verbatim}
    \label{fig:inverse-problem:bob}
    \subcaption{From Bob to Alice}
  \end{subfigure}

  \begin{subfigure}{.45\linewidth}
\begin{verbatim}
<<<<<<< Alice
A B C
||||||| base
A B
=======
A D B
>>>>>>> Bob
\end{verbatim}
    \label{fig:inverse-problem:diff}
    \subcaption{Bob's editor (diff)}
  \end{subfigure}
  \hfil
  \begin{subfigure}{.45\linewidth}
\begin{verbatim}
A { B | D} { C | B}
\end{verbatim}
    \label{fig:inverse-problem:grove}
    \subcaption{Bob's editor (Grove)}
  \end{subfigure}
  \label{fig:inverse-problem}
  \caption{The effects of diff-and-merge on conflicting edit states}
\end{figure}

An alternative to solving the inverse problem is to build the editor on top of a distributed collaboration protocol such as OT or,
more recently, to model the shared edit state as a CRDT---a data structure implemented behind a fixed set of commutative update operations.
Commutativity is desirable because it reduces collaboration to a simple matter of record and replay.
Since CRDT imposes no other design constraints, it is potentially better suited for collaborative program editing than diff-and-merge.
For instance, the \textbf{conflict problem}---how to meaningfully represent conflicts without mangling edit state---%
becomes easier to solve as the language of edits becomes more precise (Figure \ref{fig:inverse-problem:grove}).

Although CRDT may offer a more natural medium for collaborative programming than diffs,
care must be taken to ensure the preservation of user intent.
Suppose Alice and Bob are collaboratively editing the line \verb|A B| with Alice's cursor on \verb|A| and Bob's on \verb|B|.
If Alice moves \verb|A| to the other side of \verb|B|, it isn't obvious where Bob's cursor should go.
Naively, since the cursor was originally on the second element (\verb|B|), it could simply stay there,
which in this case means it winds up on \verb|A|.
Realistically, \verb|A| and \verb|B| could be very long functions, in which case Bob's cursor would suddenly appear to move a great distance.
If Bob's cursor remained on \verb|B| instead, there would be no disorienting jump.
To solve this \textbf{repositioning problem} and affect confusion-free movement, all editable content must be uniquely identifiable.


% To address these concerns, we propose \emph{Grove}, a CRDT-backed collaborative structure editing calculus.
% Grove models the shared edit state as a labeled multidigraph with a distinguished root vertex
% and a pair of commutative vertex insertion and deletion operations such that deletion always wins.
% In this setting, collaboration is replay and conflicting edits have a natural representation as edges with a common origin or destination.
% We establish an isomorphism between a graph and its cooresponding \emph{grove}, or collection of disjoint forests,
% enabling a full range of existing tree-based interactivity, for instance as terms in a programming language editor.

% We are further motivated by the pursiut of a collaborative extension for Hazel, an integrated programming language and structure editor
% that provides semantic editor services by enforcing a sensibility invariant, i.e., every edit state is semantically valid,
% an interesting avenue of future work with potential implications for the scalability of collaborative programming editors.

% do this instead:
%   the standard method is diff-based version control.
%   diff is bad, evene for what it's indended for.
%   here's why: solving inverse problem ==> have to infer edits
%   alternative approach is to record the edits directly and then form a CRDT s.t. collaboration can be accomplished by sending your edits to your collaborators
%   primarily been explored in the setting of real time text editing
%   there are some aspects of this approach that don't make it suitable for programming:
%   1. the repositioning problem
%   2. the conflict problem (even json crdt does ad hoc conflict resolution)

% Note: you'd still want to solve these problems for text as well, but we're interested in making this work for Hazel
% Hazel is good for this because it already has a rich calculus of edit actions -- motivates the tree focus
% Even when it comes to rich text editing, we really wanna be working with trees: due to markup

% move to related work

% Operational Transformation (OT), on the other hand, goes perhaps too far in the opposite direction.
% OT is a distributed system architecture that coordinates the proliferation and maintainance of edits and edit states.
% Industrial-scale collaborative editors, such as Google Docs,
% employ OT for real-time applications requiring strong eventual consistency of causally ordered events.
% It does so by maintaining an event log it transforms upon each new entry in order to preserve the causal ordering.

% The Convergent Replicated Data Type (CRDT) was proposed as a simpler alternative to the OT architecture.
% CRDTs do not require causal ordering maintenance because the language of edits must form a join semilattice,
% thereby ensuring the edit state update function increases monotonically.
% CRDT places fewer constraints than OT on the language of edits,
% so calculi for editing tree-native documents can be simpler to model for CRDT than OT.
% Similarly, by eliminating all but one design constraint, edit calculi for CRDT can be easier to implement and extend.

% Unfortunately, monotonicity merely changes the nature of the complexity rather than eliminating it entirely.
% Although tree-native document editing calculi may ultimately admit simpler designs under CRDT than OT,
% there are no generic recipes for transforming a sequential tree editing calculus into one suitable for CRDT.
% Consequently, progress of CRDT-backed tree editors has slowed as the research community debates whether CRDT is living up to its promise.
% Thus, we are motivated to ask:
% Is there a more natural model for collaborative editing of trees?
% And how do we preserve user intent in the presence of concurrent edits in a tree-native editor?



% algorithms for collaborative text editing are well developed
% natural fit for text
% not a natural fit for trees
% so if you want to build a grove, a more natural approach is needed


% a lot of collaborative editing of other sorts (not necessarily programming) requires a tree model,
% but that problem doesn't seem to have been study directly (maybe except JSON CRDT).

% A \emph{collaborative structure editor} is an editor that
% (1) allows multiple concurrent users to work on a shared document, while also
% (2) providing structure-aware editor services such as projectional editing, syntax highlighting, or automatic code folding.
% %
% Collaborative editing research focuses on the design and implementation of real-time, multi-user, character-based communication systems,
% whereas structure editors typically presume a more complex document schema and then focus on some other aspect of the user experience.
% In both settings, preservation of user intent is a core technical challenge.
% %
% Although collaborative editors and structure editors have overlapping goals (optimal user experience)
% and complementary design challenges (subject-subject versus subject-object harmony),
% to our knowledge, there is no comprehensive, principled account of their combined use.

% Since collaborative editors are essentially distributed systems, existing work tends to focus on extensions to distribution protocols.
% Lots of examples using OT. (Google Docs)
% OT is complex and largely textual.
% OT can make sense for real time systems: users typically change one character at a time, and instant feedback can help to prevent conflicts.
% On the other hand, OT system designs can be difficult to extend.

% Alternatively, there's CRDT. (Peritext?)
% CRDT is easier to implement, but harder to design for.
% There's an All-CRDT editor---it turned out to be not so realistic. (what's it called again?)

% Structure editing has been a recurring theme in the computer science literature since at least Engelbart's ``Mother of All Demos.''
% Provides automation for domain experts and reduces the barrier to entry for everyone else.
% Popular for editing programs, i.e., for programming language-specific editors.

% However, modern program editors typically disable editor services (like what?) when the document is not in a consistent state,
% a phenomenon called the ``gap problem.''
% Of course, in the presence of multiple concurrent users, the problem gets worse.

% In a collaborative setting,
% Hazel is a structure editor with support for advanced editing services, (e.g., semantic actions?).

\newpage




Motivation:

- collaborative editing (both synchronous ala Google Docs and asynchronous version control)
is good and important as computing grows

- semantic structure editing is good because it solves the gap problem (semantic editor services
are always available) -- cite Hazelnut papers (talk about holes)

- previous approaches to collaborative editing have limitations

- diff/merge based approaches (trying to solve the inverse problem based on final states --
you lose the actual actions that were performed, and have to reconstruct them or an approx.
of them i.e. add line/delete line actions -- would need to adapt this to structure editing,
some papers have started to look at that, but fundamentally we don't want to throw away the
knowledge we have about the edits!)

- operational transforms (complexity, you have to patch previous actions based on new actions)

- CRDT-based collaborative editing (that's all been on text, not PL semantics) -- this is good
because it is relatively simple: you just send all the edits to all the replicas and they are
convergent by design

- we want to have the same convergence for a CRDT-based collaborative structure editor that maintains
the sensibility invariant of Hazelnut, i.e. every editor state has meaning. mention that maintaining sensibility
allows scaling of semantic editor services in the presence of large number of collaborators (in contrast,
using VS Code or other collaborative text editors with large numbers of collaborators means that almost always
the semantic editor services will be disabled because the program is going to be broken in multiple places
transiently)

this is tricky because:

- some edits might be conflicting -- solve this with "conflict holes"

- adding cut/paste or delete/restore allows for degenerate programs (cycles, multiple parents, etc.)

- since we are commutative, we solve both synchronus and async collaborative editing

- and this resolves issues around merges and conflicts

- contribution of this paper is to solve these problems from type-theoretic first principles:

- ...

- Hazel

\subsection{Contributions and Paper Organization}%
\label{sec:Contributions and Paper Organization}
