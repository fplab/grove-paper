
\section{Marking System}%
\label{sec:Marking System}

[Motivation about how groves may be ill-typed because of arbitrary actions and conflicts, causing gaps in semantic services (rehash some of intro)]

[POPL 24] introduce the \emph{marked lambda calculus} to ...

We enrich groves and their underlying syntax of terms by \emph{marking} them, that is, systematically adding error marks during the type-checking process to enable \emph{total error localization and recovery}, refining the work from [popl 24]\todo{cite} to fit Grove's collaborative needs. This allows the editor to optimistically recover from errors by formalizing the cases in which they can occur, and also provide more helpful error messages and editor services.

In addition to type-checking and marking expressions, we also gather type-inference constraints and \emph{location conflict contexts} , further augmenting the editor with the ability to inspect and mark \emph{multi-location} and \emph{cycle} conflicts against the different contexts their occurrences can appear in, providing valuable information to the user while resolving these conflicts.

Next, we revisit the marked lambda calculus and establish all the required machinery to type-check and mark expressions using a conventional bidirectional system \todo{cite TAPL?}. Next, in \autoref{sub:marking-groves}, we introduce the constructs required to handle the various conflicts that can arise from the graph-based setting when type-checking a tree-native surface language. And finally, we describe the mechanization for the marking system in \autoref{sub:marking-agda} done using the Agda \todo{cite} proof assistant.

[Recap marking judgments using old lam and ap rules]

\subsection{Background: The Marked Lambda Calculus}

The marked lambda calculus is a bidirectionally typed rewriting system that takes an arbitrary syntactically well-formed expression and \emph{marks} it by localizing type errors, producing a \emph{marked expression}. The key judgments are synthetic and analytic marking judgments that mark and type-check expressions. For example, the rules relevant to functions and function applications are shown in \autoref{inf:marking-rules}.

\begin{mathpar}
\label{inf:marking-rules}
\inferrule[MKALam1]{
        \matchedArrow{\TMV_3}{\TMV_1}{\TMV_2} \\
        \consistent{\TMV}{\TMV_1} \\
        \ctxAnaFixedInto{\extendCtx{\ctx}{x}{\TMV}}{\EMV}{\ECMV}{\TMV_2}
    }{
        \ctxAnaFixedInto{\ctx}{\eFun{}{x}{\TMV}{\EMV}}{\eFun{}{x}{\TMV}{\ECMV}}{\TMV_3}
    }

     \inferrule[MKALam2]{
        \notMatchedArrow{\TMV_3} \\
        \ctxAnaFixedInto{\extendCtx{\ctx}{x}{\TMV}}{\EMV}{\ECMV}{\TUnknown}
    }{
        \ctxAnaFixedInto{\ctx}{\eFun{}{x}{\TMV}{\EMV}}{\ECLamAnaNonMatchedArrow{x}{\TMV}{\ECMV}}{\TMV_3}
    }
\end{mathpar}

The first rule for analyzing functions against their expected type requires the matched arrow judgment and a type consistency check between the ascription and the input of the expected type, after which we can analyze the body against the output while recursively marking it. What happens when the expected type is not a matched arrow type? Conventional type checkers cannot proceed in this case, and type-checking halts. However, whenever we encounter an error during type-checking, that is, one of the premises fails, we mark the term as such, indicating that the expected type is not a matched arrow type in the analytic position. Yet we are still able to recursively mark the body and analyze it against the unknown type without stopping type-checking on failure.
...

In some situations, a unique type cannot be determined. In these cases, the system recovers by using the unknown type, a.k.a. the type hole, from gradual typing. For example, the rules for marking variables are given below:

...

Marking is total, meaning every well-formed expression can be marked, producing a well-typed marked expression. Once marked, the system layers on unification-based type inference to opportunistically fill type holes. The typing judgment for marked expressions generates constraints: <judgement format>
<Say something about users interactively choosing from partial solutions when the constraints for a type hole are in conflict.>

\subsection{Marking Groves}
\label{sub:marking-groves}
We extend the marked lambda calculus to groves. Most of the machinery remains largely the same as in [POPL 24]; we focus our attention on conflicts.

[Discuss the unmarked language as an instantiation of the generic term language described in Sec 3.2; probably can't do this properly until all the language for section 3 is figured out]

\subsubsection{The unmarked language}
\label{sub:unmarked-lang}
First, we introduce the syntax of the \emph{unmarked} language which comprises two mutually defined expression sorts: $Exp$ \& $ExpList$, denoted by $\EMV$ and $\ELV$ respectively as displayed in \autoref{fig:unmarked-language}. In addition, this unmarked language is an instantiation of the generic term syntax introduced in \autoref{sub:Groves}, with $\EMV$ \& $\TMV$ being the expression and type sorts along with their mutually defined list counterparts $\ELV$ \& $\TLMV$ that allow us to model holes (empty list), \emph{regular} singleton expressions, and local conflicts. Type-checking this base unmarked language is straightforward using conventional bidirectional type-checking rules and we introduce the sort of semantic types $STyp$, denoted by $\sigma$, which builds on ideas from \todo{cite GTLC} Gradually Typed Lambda Calculus (GTLC). This is achieved using 4 mutual type-checking judgments as displayed in \autoref{fig:unmarked-judgments}. The rules are not very different from most simply typed lambda calculi except for the cases where we treat \emph{conflicts} as holes. For the sake of brevity, we have chosen to omit the rules here, but they are expressed in detail in \autoref{sec:marked-unmarked-expressions}. 

\begin{figure}
    \judgbox{\ctxSynTypeU{\ctx}{\EMV}{\sigma}} 
    \judgbox{\ctxSynTypeU{\ctx}{\ELV}{\sigma}}
    \judgbox{\ctxAnaTypeU{\ctx}{\EMV}{\sigma}} 
    \judgbox{\ctxAnaTypeU{\ctx}{\ELV}{\sigma}}
    \centering
    \label{fig:unmarked-judgments}
\end{figure}

\begin{figure}
    \[
    \begin{array}{lclll}
     \EMV \in & Exp & \coloneqq & 
        \var{x}
        \mid  \num 
        \mid \plus{\ELV_1}{\ELV_2} 
        \mid \mult{\ELV_1}{\ELV_2}
        \mid \lam{x}{\TLMV}{\ELV}
        \mid \app{\ELV_1}{\ELV_2}
        \mid \multiref
        \mid \uniref \\
     \ELV \in & ExpList & \coloneqq & 
        \ehole
        \mid \lexp{\EMV} 
        \mid \conflict{\EMV} \\
    \TMV \in & \Typ & \coloneqq & 
        \tnum 
        \mid \tarrow 
        \mid \multiref 
        \mid \uniref \\ 
    \TLMV \in & TypList & \coloneqq &
        \ehole
        \mid \lexp{\TMV}
        \mid \conflict{\TMV} \\
    \end{array}
    \]
    \centering
    \caption{Unmarked Language}
    \label{fig:unmarked-language}
\end{figure}

\begin{figure}
    \[
    \begin{array}{lclll}
    \ECMV \in & MExp & \coloneqq &
        \var{x}
        \mid \num
        \mid \plus{\ELMV_1}{\ELMV_2}
        \mid \mult{\ELMV_1}{\ELMV_2}
        \mid \lam{x}{\TLMV}{\ELV}
        \mid \app{\ELMV_1}{\ELMV_2}
        \mid \multiref
        \mid \uniref \\ 
        & & & \ECFree{\var{x}} 
        \mid \incontype
        \mid \lamasc
        \mid \lamanamarr
        \mid \apsynmarr \\ 
        % \mid \lamge \\
     \ELMV \in & MExpList & \coloneqq &
        \ehole
        \mid \lexp{\ECMV}
        \mid \conflict{\ECMV} \\
    \end{array}
    \]
    \centering
    \caption{Marked Language}
    \label{fig:marked-language}
\end{figure}

\begin{figure}
    \[
    \begin{array}{lclll}
    \sigma \in & STyp & \coloneqq & 
        \TUnknown
        \mid \tNum{}
        \mid \starrow \\
    \end{array}
    \]
    \judgbox{\graphErase{\TLMV} = \sigma}
    \[\begin{array}{rrcl}
        \Prov & \Provp & \Coloneqq & u \mid exp(u) \mid \rightarrow_L(\Provp) \mid \rightarrow_R(\Provp)\\
        \TMName  & \TMV  & \Coloneqq & \dots \mid \TUnknown^{p} \mid \TUnknown^{(v,p)}\\
    \end{array}\]
    \judgbox{\ensuremath{\matchedArrowConstraint{\sigma}{\sigma_1}{\sigma_2}{C}}}
    \centering
    \caption{Other machinery}
    \label{fig:marking-machinery}
\end{figure}



\subsubsection{The marked language}
\label{sub:marked-lang}
Next, we introduce the heart of the marking machinery in \autoref{fig:marking-judgments} and \autoref{fig:marked-language}, again in the form of 4 mutually defined judgments along with the marked language denoted by $\ECMV$ \& $\ELMV$. 

\begin{figure}
    \[
    \begin{array}{lclll}
    \ECMV \in & MExp & \coloneqq &
        \var{x}
        \mid \num
        \mid \plus{\ELMV_1}{\ELMV_2}
        \mid \mult{\ELMV_1}{\ELMV_2}
        \mid \lam{x}{\TLMV}{\ELV}
        \mid \app{\ELMV_1}{\ELMV_2}
        \mid \multiref
        \mid \uniref \\ 
        & & & \ECFree{\var{x}} 
        \mid \incontype
        \mid \lamasc
        \mid \lamanamarr
        \mid \apsynmarr \\ 
        % \mid \lamge \\
     \ELMV \in & MExpList & \coloneqq &
        \ehole
        \mid \lexp{\ECMV}
        \mid \conflict{\ECMV} \\
    \end{array}
    \]
    \centering
    \caption{Marked Language}
    \label{fig:marked-language}
\end{figure}

\begin{figure}
    \judgbox{\synMarkConstraint{\ctx}{\EMV}{\ECMV}{\sigma}{\cConstraint}{\mConstraint}} 
    \judgbox{\synMarkConstraint{\ctx}{\ELV}{\ELMV}{\sigma}{\cConstraint}{\mConstraint}} 
    \judgbox{\anaMarkConstraint{\ctx}{\EMV}{\ECMV}{\sigma}{\cConstraint}{\mConstraint}} 
    \judgbox{\anaMarkConstraint{\ctx}{\ELV}{\ELMV}{\sigma}{\cConstraint}{\mConstraint}} 
    \centering
    \caption{Marking Expressions}
    \label{fig:marking-judgments}
\end{figure}

\subsubsection{Marking Lambda Abstractions}
\label{sub:mark-lam}
Now, we examine the marking rules for lambda abstractions and function applications. In \autoref{inf:lam-ana}, in the first case, we successfully mark the lambda abstraction by first \emph{graph-erasing} its type and then using the matched arrow judgment and consistency relations to recursively mark its body. When all the premises hold, the abstraction is analyzed with no issues. What happens when the analyzed type is not a matched arrow type or the erased type is \emph{not} consistent with the expected input type? This is where the next rules help localize these errors to the node where the failure occurs. In $MKACLam2$, the expected type is not a matched arrow type. Similarly, in $MKACLam3$, the consistency check between the erased ascription and input of the expected type fails, and again, we mark the abstraction as such, depicting the inconsistency in the ascription of the lambda in $\ECLamInconAsc{\var{x}}{\TLMV}{\ELMV}$.

\begin{mathpar}
\label{inf:mark-lam-ana}
\caption{Marking and Analyzing Lambda Abstractions}
\inferrule[MKALam1]{
        \graphErase{\TLMV} = \sigma \\
        \matchedArrow{\sigma_3}{\sigma_1}{\sigma_2} \\
        \consistent{\sigma}{\sigma_1} \\
        \anaMarkConstraint{\extendCtx{\ctx}{x}{\sigma}}{\ELV}{\ELMV}{\sigma_2}
    }{
        \anaMarkConstraint{\ctx}{\lam{x}{\TLMV}{\EMV}}{\lam{x}{\TLMV}{\ELMV}}{\sigma_3}
    }

     \inferrule[MKALam2]{
        \graphErase{\TLMV} = \sigma \\
        \notMatchedArrow{\sigma_3} \\
        \anaMarkConstraint{\extendCtx{\ctx}{x}{\sigma}}{\ELV}{\ELMV}{\TUnknown}
    }{
        \anaMarkConstraint{\ctx}{\reglam}{\ECLamAnaNonMatchedArrow{\var{x}}{\TLMV}{\ELMV}}{\sigma_3}
    }

    \inferrule[MKALam3]{
        \erase{\TLMV} = \sigma \\
        \matchedArrow{\sigma_3}{\sigma_1}{\sigma_2} \\
        \inconsistent{\sigma}{\sigma_1} \\
        \anaMarkConstraint{\extendCtx{\ctx}{x}{\sigma_1}}{\ELV}{\ELMV}{\sigma_2}
    }{
        \anaMarkConstraint{\ctx}{\reglam}{\ECLamInconAsc{\var{x}}{\TLMV}{\ELMV}}{\sigma_3}
    }
\end{mathpar}

\subsubsection{Marking Function Applications}
\label{sub:mark-ap-syn}
\begin{mathpar}
\label{inf:ap-syn}
\label{Marking and Synthesizing Function Applications}
\inferrule[MKSAp1]{
    \synMarkConstraint{\ctx}{\ELV_1}{\ELMV_1}{\sigma} \\
    \matchedArrow{\sigma}{\sigma_1}{\sigma_2}\\
    \anaMarkConstraint{\ctx}{\ELV_2}{\ELMV_2}{\sigma_1} \\
  }{
    \synMarkConstraint{\ctx}{\eApp{u}{\ELV_1}{\ELV_2}}{\eApp{u}{\ELMV_1}{\ELMV_2}}{\sigma_2}
  }

  \inferrule[MKSAp2]{
    \synMarkConstraint{\ctx}{\ELV_1}{\ELMV_1}{\sigma} \\
    \notMatchedArrow{\sigma} \\
    \anaMarkConstraint{\ctx}{\ELV_2}{\ELMV_2}{\TUnknown}
  }{
    \synMarkConstraint{\ctx}{\eApp{u}{\ELV_1}{\ELV_2}}{\app{\ECApSynNonMatchedArrow{\ELMV_1}}{\ELMV_2}}{\TUnknown^{\rightarrow_{R}{(exp(u))}}}
  }
\end{mathpar}

Next, in \autoref{inf:ap-syn}, we look at the case when marking an application succeeds. For this to happen, we first synthesize a type for the first sub-term and then match it with an arrow type before analyzing the argument against the expected input type. What if the synthesized is not a matched arrow type? We insert an error mark again, indicating the failed premise in the synthetic mode in $\app{\ECApSynNonMatchedArrow{\ELMV_1}}{\ELMV_2}$.

- core unmarked language
- marked language
- semantic types
- matched arrows from GTLC
- local-info stuff
- provenance and inference stuff

% \subsubsection{Holes} TODO holes aren't really new?

\subsubsection{Local Conflicts}
To mark local conflicts, we recursively mark all subterms, gather the constraint and location conflict contexts, and synthesize the unknown type; such conflicts are treated essentially as holes. 

\[
\inferrule[MKSCConflict]{ 
    \Setof{\synMarkConstraint{\ctx}{\EMV}{\ECMV}{\sigma_i}{\cConstraint_i}{\mConstraint_i}}_{i<n}
}{
    \synMarkConstraint{\ctx}{\conflict{\EMV}}{\conflict{\ECMV}}{\TUnknown^{(v,p)}}{\Setof{\constrain{\TUnknown^{(exp(u_i))}}{\sigma_i}}_{i<n} \cup \bigcup_{i<n} \cConstraint_i}{\bigcup_{i<n} \mConstraint_1}
}
\]

\subsubsection{Multi-Location Conflicts}
\todo{fix rule names here}
\[
\inferrule[MKSCMultiParent]{ }{
    \synMarkConstraint{\ctx}{\multiref}{\multiref}{\TUnknown^{exp(id-of(v))}}{\constraintNil{}}{(id-of(v), w , \ctx, syn)}
}
\]

\subsubsection{Location Cycle Conflicts}

\[
\inferrule[MKSCUnicycle]{ }{
    \synMarkConstraint{\ctx}{\uniref}{\uniref}{\TUnknown^{exp(id-of(v)}}{\constraintNil{}}{(id-of(v),w , \ctx, syn)}
}  
\]

We observe that both  \uniref and \multiref are instances of \emph{location} conflicts; they are terms that can be marked and type-checked in the various locations where they occur. At first, this can confuse the user when they encounter these new conflict constructs while programming collaboratively. As a result, we want them to be able to access these different instances \emph{and} their contexts to resolve them in an informed manner. This is precisely the motivation behind \emph{location conflict contexts} $(\L)$ which are sets of 4-tuples of vertex-IDs, edge-IDs, contexts, and mode denoted by $v$, $w$, \ctx, and $m$ respectively. The goal is to have an inspection window in the editor that gathers the relevant location when a user clicks on one of these location conflicts to display their other instances starting from the clicked context, allowing the user to explore them interactively to resolve the conflicts. However, we have left that to the UI considerations of future work while formalizing the machinery required. Otherwise, these conflicts are treated as holes during type-checking, with their provenance determined by the vertex-ID $(v)$ they carry.

\subsection{Agda Mechanization}
\label{sub:marking-agda}
Our extension of the marked lambda calculus has been mechanized in the Agda proof assistant, taking the standard approach of modeling judgments as inductive datatypes and inference rules as constructors. Marked terms are implicitly typed, allowing local info mappings to be computed as a function of these terms.


\subsection{Inference}
\label{sub:inference}

I don't know what to write here lol \ldots

% TODO discuss how no type hole inference/constraint generation so 
% TODO discuss how it is separate from the grove agda but a correspondence can be proven?

% Questions while porting over marking system
% - new judgments for marking expressions
% - need to prove totality and unicity of marking
% - then type checking both regular and marked expressions
% - need to check if we need to introduce new marked errors
% - typeset all of this
% - get all of this working in Agda?