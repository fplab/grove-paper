
\section{Related Work}%
\label{sec:Related Work}

Contrast with a traditional diff

Similar to how git merged across moves.
A notable difference though is that \texttt{git} detects
file moves by structural similarity rather than
our nominal equality.
(We exploit this in Subsection~REF:TODO).

Compare this to what happens when there are conflicts in a traditional,
line/diff-based version control system (e.g., \texttt{git}, \texttt{mercurial}, \texttt{svn}, etc.).
Those systems annotate files with difference markers (e.g., sequences of~\texttt{<},~\texttt{>}, or~\texttt{=})
when a merge conflict happens.
These show alternate versions of the code, and it is up to the user
to replace those alternates with the desired merged result.

Another place our model is different
is that changes are explicit rather than being inferred by a \texttt{diff} algorithm.
Finally, in our model, since all changes are in terms of
the structure of the code rather than lines,
all changes respect that structure.
This is unlike the line-based model where there is no guarantee
that a merge conflict follows the grammatical structure of the code.

\subsection{Trees}

\subsubsection{Structure editors}

Hazel~\citep{Omar:2019:10.1145/3290327}

\subsubsection{Tree diff}

We are using action based instead of tree diff

We have to deal with merge

We have to deal with edits from multiple people

J. W. Hunt and M. D. McIlroy. 1976. An Algorithm for Differential File Comparison. Technical Report CSTR 41. Bell Laboratories, Murray Hill, NJ.

Type-directed diffing of structured data \url{https://dl.acm.org/doi/10.1145/3122975.3122976}

  Approximating Tree Edit Distance through String Edit Distance \url{https://dl.acm.org/doi/10.5555/3118232.3118518}

  Meaningful change detection in structured data \url{https://dl.acm.org/doi/10.1145/253260.253266}

  An optimal decomposition algorithm for tree edit distance \url{https://dl.acm.org/doi/10.5555/2394539.2394560}

  Diff/TS: A Tool for Fine-Grained Structural Change Analysis \url{https://dl.acm.org/doi/10.1109/WCRE.2008.44}

An efficient algorithm for type-safe structural diffing \url{https://dl.acm.org/doi/10.1145/3341717}

  Precise Version Control of Trees with Line-Based Version Control Systems \url{https://dl.acm.org/doi/10.1007/978-3-662-54494-5_9}

  A survey on tree edit distance and related problems \url{https://dl.acm.org/doi/10.1016/j.tcs.2004.12.030}

  Cycle-aware minimization of acyclic deterministic finite-state automata \url{https://dl.acm.org/doi/10.1016/j.dam.2013.08.003}

  Computing the Edit-Distance between Unrooted Ordered Trees \url{https://dl.acm.org/doi/10.5555/647908.740125}

  Type-safe diff for families of datatypes \url{https://dl.acm.org/doi/10.1145/1596614.1596624}

  A Categorical Theory of Patches \url{https://dl.acm.org/doi/10.1016/j.entcs.2013.09.018}

  Type-directed diffing of structured data \url{https://dl.acm.org/doi/10.1145/3122975.3122976}

  The Semantics of Version Control \url{https://dl.acm.org/doi/10.1145/2661136.2661137}

  The Tree-to-Tree Correction Problem \url{https://dl.acm.org/doi/10.1145/322139.322143}

  Generic Diff3 for algebraic datatypes \url{https://dl.acm.org/doi/10.1145/2976022.2976026}

\subsection{Version Control}

Git \url{https://git-scm.com/}

Darcs \url{https://darcs.net/}

  Darcs: distributed version management in haskell \url{https://dl.acm.org/doi/10.1145/1088348.1088349}

Hg? \url{https://www.mercurial-scm.org/}

SVN \url{https://subversion.apache.org/}

Pijul and (Anu is a rewrite of Pijul and seems to have been subsumed into Pijul)

  \url{https://pijul.org/}
  \url{https://pijul.org/manual/theory.html}

  \url{https://tahoe-lafs.org/~zooko/badmerge/simple.html}

\subsection{Collaborative Editing}

Collaborative Structure Editing

SmallTalk collaboration with images

TouchDevelop papers

Lots of list-of-chars or list-of-list-of-chars (we ignore these except to discuss them here)

\subsection{CRDTs}
(Are we a known CRDT?)

List of CRTD papers: \url{https://crdt.tech/papers.html}

Bottom = Tombstone

https://www.waitingforcode.com/big-data-algorithms/conflict-free-replicated-data-types-flags-graphs-maps/read
 - Add-Remove Partial Order data type
 - 2P2P-Sets
 - Replicated Growable Array

\url{https://github.com/PsychoLlama/graph-crdt}
 - Graph CRDT
 - Uses a LWW-E-Set

\url{https://martin.kleppmann.com/2020/07/06/crdt-hard-parts-hydra.html} (overview talk)
 We don't have interleaving problems because
   - everything is relative to a specific ID not a position
   - we don't try to auto resolve
   - we are tree not list
   Part three: moving sub tree
   - Last parent writter wins (prevents cycles)
     Equivalent to us if we filter multiparent edges, different for cycles, no way to delete?

A commutative replicated data type for cooperative editing
  \url{https://hal.inria.fr/inria-00445975/document}
  Describes TreeDoc but this uses a document model that is a list (tree is just how it is implemented)

Logoot : a Scalable Optimistic Replication Algorithm for Collaborative Editing on P2P Networks
  \url{http://pagesperso.lina.univ-nantes.fr/~molli-p/pmwiki/uploads/Main/weiss09.pdf}

Specification and Complexity of Collaborative Text Editing
  \url{https://www.microsoft.com/en-us/research/wp-content/uploads/2016/07/podc16-complete.pdf}

LSEQ: an Adaptive Structure for Sequences in Distributed Collaborative Editing,
  \url{https://hal.archives-ouvertes.fr/file/index/docid/921633/filename/fp025-nedelec.pdf}

Data consistency for P2P collaborative editing
  \url{https://hal.archives-ouvertes.fr/file/index/docid/108523/filename/OsterCSCW06.pdf}

Interleaving anomalies in collaborative text editors
  \url{https://martin.kleppmann.com/papers/interleaving-papoc19.pdf}

Moving Elements in List CRDTs
  \url{https://martin.kleppmann.com/papers/list-move-papoc20.pdf}

A highly-available move operation for replicated trees and distributed filesystems
  \url{https://martin.kleppmann.com/papers/move-op.pdf}

...
 commutative replicated data types CmRDT
 convergent replicated data types, or CvRDTs
 Delta state CRDTs[12][13] (or simply Delta CRDTs

The Causal Graph CRDT for Complex Document Structure
  \url{https://dl.acm.org/doi/10.1145/3209280.3229110}

\url{https://en.wikipedia.org/wiki/Conflict-free_replicated_data_type}

G-Set
PN-Set
2P-Set
LWW-element-Set (Last-Write-Wins)
OR-Set
MV-Register: Multi-Value Register
U-Set (this is what we are for edges, not OR-set due to duplicates)
Add-Remove Partial Order data type
  2P-Set for vertices, and a G-Set for edges.

A comprehensive study of Convergent and Commutative Replicated Data Types (2011)
  \url{https://hal.inria.fr/inria-00555588/document}

CRDTs: Consistency without concurrency control
  \url{https://arxiv.org/abs/0907.0929}

https://medium.com/@amberovsky/crdt-conflict-free-replicated-data-types-b4bfc8459d26
 removeVertex() has priority, all incident edges are removed
 addEdge() has priority, all removed vertices are re-added
 Delay removeVertex() execution till all concurrent removeVertex() are executed.
First one is 2P2P-Set

https://crdt.tech/papers.html

Mahsa Najafzadeh, Marc Shapiro, and Patrick Eugster. Co-design and verification of an available file system. In 19th International Conference on Verification, Model Checking, and Abstract Interpretation, VMCAI 2018, pages 358--381. Springer LNCS volume 10747, January 2018. [ bib | DOI | .pdf ]
  \url{http://dx.doi.org/10.1007/978-3-319-73721-8_17}
  https://pages.lip6.fr/Marc.Shapiro/papers/VMCAI-2018-filesys.pdf

Martin Kleppmann and Alastair R Beresford. A conflict-free replicated JSON datatype. IEEE Transactions on Parallel and Distributed Systems, 28(10):2733--2746, April 2017. [ bib | DOI | arXiv ]
  http://dx.doi.org/10.1109/TPDS.2017.2697382
  http://arxiv.org/abs/1608.03960

Vinh Tao, Marc Shapiro, and Vianney Rancurel. Merging semantics for conflict updates in geo-distributed file systems. In 8th ACM International Systems and Storage Conference, SYSTOR 2015. ACM, May 2015. [ bib | DOI | .pdf ]
  http://dx.doi.org/10.1145/2757667.2757683
  https://pages.lip6.fr/Marc.Shapiro/papers/geodistr-FS-Systor-2015.pdf

Mehdi Ahmed-Nacer, Stéphane Martin, and Pascal Urso. File system on CRDT. Research Report RR-8027, INRIA, July 2012. [ bib | arXiv | http ]
  http://arxiv.org/abs/1207.5990
  https://hal.inria.fr/hal-00720681/

Stéphane Martin, Pascal Urso, and Stéphane Weiss. Scalable XML collaborative editing with undo. In On the Move to Meaningful Internet Systems (OTM), pages 507--514. Springer LNCS volume 6426, October 2010. [ bib | DOI | arXiv ]
  \url{http://dx.doi.org/10.1007/978-3-642-16934-2_37}
  http://arxiv.org/abs/1010.3615



2P-Sets
Anomaly: Creating a lone deleted vertex requires create and delete of otherwise unneeded edge

Operational Transforms

Etherpad

Live Share

\subsection{Synchronization}

Unision
\url{https://www.cis.upenn.edu/~bcpierce/unison/}
\url{https://www.cis.upenn.edu/%7Ebcpierce/papers/index.shtml#File%20Synchronization}

\subsection{TODO}

Homotopical Patch Theorey: \url{https://www.cambridge.org/core/journals/journal-of-functional-programming/article/homotopical-patch-theory/42AD8BB8A91688BCAC16FD4D6A2C3FE7}
Homotopical patch theory: \url{https://dl.acm.org/doi/10.1145/2628136.2628158}
