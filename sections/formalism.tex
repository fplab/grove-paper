
\section{The Grove Calculus}%
\label{sec:Formalism}

We now formalize the ideas presented informally above by defining the Grove Calculus. The calculus consists of a commutative patch language on graphs, defined in \autoref{sub:Convergent Graphs}, a decomposition procedure to go from graphs to groves (i.e. sets of terms with conflicts) in \autoref{sub:Groves}, and a user action language that operates on groves and is defined by translation to the patch language in \autoref{sub:User Actions}.

\subsection{A Commutative Graph Patch Language}%
\label{sub:Convergent Graphs}

\subsubsection{Graphs}
Let $\U$ and $\W$ be separate sets of \emph{unique identifiers} (UIDs) for vertices and edges, respectively. Assume each is equipped with a total ordering $\leq$. 

Let $\K$ be the set of \emph{constructors} of terms in the source language, and let $\P$ be the set of \emph{positions} of subterms in the source language. Assume each constructor $k\in\K$ is associated with a finite set of positions $\mathsf{arity}(k)\subseteq \P$, representing the set of positions that children of $k$ can inhabit.

For example, if the language is the simply typed lambda calculus, $\K$ will include constructors for function abstractions ($\textsc{Lambda}_x$), function applications ($\textsc{Ap}$), variables ($\textsc{Var}_x$), and constants ($\textsc{Const}_c$), as well as function type constructors ($\textsc{Arrow}$) and base types ($\textsc{Base}_b$). The arity of each constructor will reflect the child positions of these constructors, e.g. $\mathsf{arity}(\textsc{Ap}) = \{\textsc{Function} , \textsc{Argument}\}$, $\mathsf{arity}(\textsc{Lambda}_x) = \{\textsc{Annotation} , \textsc{Body}\}$, and $\mathsf{arity}(\textsc{Var}_x) = \emptyset$.

A \emph{vertex} $v = (u, k) \in \V = \U \times \K$ is labeled with a UID, $u$, and a constructor, $k$. 

A \emph{location} $\ell = (v, p) \in \L$ represents the origin of an edge, where $v=(u,k)$ is drawn from $\V$ and position $p$ is drawn from $\mathsf{arity}(k)$.

An \emph{edge} $\e = (w, \ell, v) \in \E = \W \times \L \times \V$ represents a directed multi-edge 
    identified by $w$, 
    originating from location $\ell$, 
    with destination vertex $v$.

A \emph{graph} $g : \E \rightarrow \Sigma$ is a function from edges to \emph{edge states} $s \in \Sigma = \SetOf{\bot, \Plus, \Minus}$. If $g(\e) = \bot$, then $\e$ has not yet been constructed in the graph. If $g(\e) = \Plus$ then $\e$ is live in the graph. If $g(\e) = \Minus$, then $\e$ has been deleted and cannot be constructed again. The total ordering $\bot \sqsubset \Plus \sqsubset \Minus$ forms a lattice over $\Sigma$. The state of each edge can only progress along this ordering over time, from not yet constructed, to live, to deleted. The \textit{realized} edges in a graph are those assigned to $\Plus$ or $\Minus$. We assume that there are finitely many realized edges in a graph. 

\subsubsection{Graph Patch Commands}
\label{subsub:Graph Patches}

We define a graph patch command $\pi = (s,\e) \in \left\{\Plus, \Minus\right\} \times \E$
as a pair of an edge state (excluding $\bot$) and an edge. As a shorthand, we use $+\e$ to denote the construction command $(\Plus, \e)$ and $-\e$ to denote the destruction command $(\Minus, \e)$.

For $s_1$, $s_2 \in \Sigma$, we define the join operation $s_1 \sqcup s_2$ to be the least upper bound of $s_1$ and $s_2$ with respect to the $\sqsubset$ ordering. Accordingly, $\Plus \sqcup \bot = \Plus$ means that an edge can be created if it has never existed, and $\Plus \sqcup \Minus = \Minus$ means that once an edge is deleted it can never be restored.

We define the semantics of patch commands via the following transition relation
between graphs.
\[
  g \overset{(s,\e)}{\longrightarrow} g\left[ \e \mapsto s \sqcup g(\e) \right]
\]
Applying patch command $\pi = \left(s, \e\right)$ to graph $g$ results in the updated graph $g' = g\left[\e \mapsto s \sqcup g(\e) \right]$, wherein the edge
state associated with edge $\e$ in $g$ becomes the join of $s$ with the state of $\e$ in $g$, and $g'(\e') = g(\e')$ for any $\e' \ne \e$. Essentially, these patch command semantics act as a transition system between graphs, \emph{joining} the new edge state with the corresponding edge state of the graph to which the patch command is being applied. Since a patch command can only change the value of $g$ on one input, applying the patch command preserves the property that $g$ maps all but finitely many edges to $\bot$.  

A \textit{graph patch} $\Q{\pi}$ is a sequence of graph patch commands, and the transition relation is extended to patches as the composition of the transitions of the constituent patch commands. 

\subsubsection{Commutativity}%
\label{subsub:Commutativity:formal}

Using these definitions, commutativity of patch commands can be established using the commutativity and associativity of the join operation on edge states. 

\begin{lemma}[Join Semilattice]
  \label{lem:Join Semilattice}
  $(\Sigma, \sqsubset)$ with $\sqcup$ forms a join semilattice. That is, $\sqcup$ is associative, commutative, and idempotent. 
\end{lemma}
\begin{proof}
    % Since $\Sigma$ is finite, 
    Follows directly from the total ordering $(\Sigma, \sqsubset)$. 
\end{proof}

\begin{theorem}[Commutativity]
  \label{thm:Commutativity}
  For all graphs $g$ and $g'$ and all graph patch commands $\pi_1 = (s_1, \e_1)$ and  $\pi_2 = (s_2, \e_2)$, 
  if $g \overset{\pi_1 \pi_2}{\longrightarrow} g'$,
  then $g \overset{\pi_2 \pi_1}{\longrightarrow} g'$
\end{theorem}
\begin{proof}
  If $g \overset{\pi_1 \pi_2}{\longrightarrow} g'$, then $g' = g[\e_1 \mapsto s_1 \sqcup g(\e_1)][\e_2 \mapsto s_2 \sqcup g(\e_2)]$. We want to show that this equals $g'' = g[\e_1 \mapsto s_1 \sqcup g(\e_1)][\e_2 \mapsto s_2 \sqcup g(\e_2)]$. For any $\e$, if $\e \neq \e_1$ and $\e \neq \e_2$, then $g'(\e) = g''(\e) = g(\e)$. If $\e = \e_1$ but $\e\neq \e_2$, then $g'(\e) = g''(\e) = s_1 \sqcup g(\e)$. Similarly if $\e=\e_2$ but not $\e_1$. Finally, if $\e=\e_1=\e_2$, then $g'(\e)=s_2\sqcup(s_1\sqcup g(\e))$, and $g''(\e)=s_1\sqcup(s_2\sqcup g(\e))$. These are equal by \autoref{lem:Join Semilattice}. Since $g'(\e) = g''(\e)$ for all $\e$, $g' = g''$ and  $g \overset{\pi_2 \pi_1}{\longrightarrow} g'$.
  % $ $\e_1 \ne \e_2$, we have $g'(\e_1) = s_1 \sqcup g(\e_1)$ and $g'(\e_2) = s_2 \sqcup g(\e_2)$, and if $\e_1 = \e_2 = \e$, we have $g'(\e) = s_1 \sqcup s_2 \sqcup g(\e)$. Hence, for any edge \e, if $\e_1$ and $\e_2$ are not equal we have $g'(\e) = g[\e_1 \mapsto s_1 \sqcup g(\e_1)][\e_2 \mapsto s_2 \sqcup g(\e_2)](\e) = g[\e_2 \mapsto s_2 \sqcup g(\e_2)][\e_1 \mapsto s_1 \sqcup g(\e_1)](\e)$, and otherwise by \autoref{lem:Join Commutativity}, we have $g'(\e) = s_1 \sqcup s_2 \sqcup g(\e) = s_2 \sqcup s_1 \sqcup g(\e)$
\end{proof}

The commutativity of patch commands generalizes directly to the commutativity of patches. 

\subsubsection{Interpretation as a CmRDT}

The definition above can be understood operationally as a CmRDT. In particular, the edges can be understood as forming a two-phase set (2P-Set), because deletion is permanent~\cite{shapiro2011conflict}. The set of realized edges forms a grow-only set, as does the set of realized vertices (those included in a realized edge).

\subsection{Groves}%
\label{sub:Groves}

Structure editors and other tools operate on trees, not on arbitrary graphs. When performing these operations on a graph $g$, only the set of edges $\e$ such that $g(\e) = \Plus$ should be considered. We call this the \textit{live subgraph} of $g$.

We present \textit{groves}, a tree-based representation of a live subgraph. A grove is a collection of mutually referential terms annotated with vertex and edge UIDs governed by the grammar in \autoref{fig:Syntax}. Groves correspond precisely to live subgraphs in the sense that a live subgraph can be \emph{decomposed} into a grove, and the grove can be \emph{recomposed} into the original live graph.

\subsubsection{Terms}%
\label{subsub:Terms}

\figureTermSyntax

Given a source language parameterized by a set of constructors $k \in \K$, we define an augmented term language as specified in \autoref{fig:Syntax}. Terms are extended with cases for relocation conflict references and unicycle conflict references. Additionally, since any position of any constructor may have zero, one, or multiple outgoing edges, we define a secondary sort \textit{ChildTerm} to represent these possibilities. A location with no outgoing edges corresponds to an empty hole, a location with one outgoing edge corresponds to an ordinary subterm, and a location with multiple outgoing edges corresponds to a local conflict between multiple terms. 

These terms also carry information to support additional operations. To enable the reconstruction of the original graph, each constructor carries the UID of its original vertex, each reference carries the vertex that it refers to, and each subterm carries the UID of the edge that leads to it. To support type hole inference (as discussed in \autoref{sub:marking-groves}), each empty hole and local conflict carries its location. Finally, to support user navigation of the grove, each reference also carries the UID of the edge that leads to it. 

\subsubsection{Graph Decomposition}%
\label{subsub:Graph Decomposition}

\figureDecompExample

A live subgraph $G$, which is just a finite set of edges, decomposes into a grove $\Theta$, which is a finite set of terms. In a tree, each node except the root has a unique parent, and there are no cycles. There is no guarantee of these properties in a graph. Decomposition operates by allowing each `problematic' vertex, in the sense of having multiple parents or forming a cycle, to be the root of its own entry in the grove, and to use references at special term leaves to encode the original graph structure.

Formally, the parents of a vertex $v$ in a graph is defined as the set of origin vertices $v'$ such that the graph includes an edge from $v'$ to $v$. Conversely, the children of a location $\ell$ is the set of $v$ such that the graph includes an edge from $\ell$ to $v$. It is convenient to consider these children to be paired with its edge's UID . 

\begin{definition}
\begin{align*}
    \mathsf{parents}_{G}(v) &= \SetOf{v' \in \V \SuchThat{(w, (v', p), v) \in G \land w \in \W \land p \in \P}}\\
    \mathsf{children}_{G}(\ell) &= \SetOf{(w,v) \in \W\times\V \SuchThat{(w, \ell, v) \in G}}
\end{align*}
\end{definition}

To help define decomposition, each vertex in a graph can be classified as either an NP root (if it has no parents), an MP root (if it has multiple parents), a U root (if there is a sequence of unique parents from it to itself, and it has the minimal UID in that sequence), or not a root. These classes are mutually exclusive.

\begin{definition}
    A vertex $v$ is an \textit{NP root} in $G$ if $\mathsf{parents}_{G}(v)=\emptyset$
\end{definition}

\begin{definition}
    A vertex $v$ is an \textit{MP root} in $G$ if $\abs{\mathsf{parents}_{G}(v)} > 1$
\end{definition}

\begin{definition}
    A vertex $v$ is a \textit{U root} in $G$ if $\exists \{v_i\}_{1 \leq i \leq n}$ such that $v_1 = v_n = v$, $\mathsf{parents}_{G}(v_i)=\{v_{i+1}\}$ for all $1 \leq i < n$, and the UID of $v$ is the minimum of the UIDs of $\{v_i\}_{1 \leq i \leq n}$.
\end{definition}

\begin{definition}
    $\mathsf{root}_G(v)$ if $v$ is an NP root, an MP root, or a U root in $G$.
\end{definition}

\begin{lemma}[Unique Classification]
\label{lem:uniq-classification}
    For any live subgraph $G\subseteq \E$ and vertex $v\in V$, exactly one of the following holds: 
    \begin{itemize}
        \item $v$ is an NP root in $G$
        \item $v$ is an MP root in $G$
        \item $v$ is an U root in $G$
        \item $v$ is not a root in $G$
    \end{itemize}
\end{lemma}

The decomposition of a vertex is specified intuitively as the term obtained by traversing the descendants of the vertex until a leaf or a root vertex of some kind is reached, which is not further decomposed, but is denoted in the term as the appropriate reference. The decomposition of a graph is simply the set obtained by decomposing each vertex classified as a root, provided the vertex actually appears as the origin of an edge in $G$.

Formally, the decomposition of a vertex is given by the function $\mathsf{decompVertex}$, which returns a term with the same UID and constructor as the vertex. At each position in the arity of the constructor, we obtain a child term from the function $\mathsf{decompLoc}$, which decomposes a location in the graph to a corresponding child term. Consider the multiplication constructor with id 40 in \autoref{fig:Decomposition example}. It decomposes to a multiplication term with left and right child terms, the result of $\mathsf{decompLoc}$ applied to its left and right locations. 

\begin{definition}
    $\mathsf{decompVertex}_{G}(v = (u,k)) = \genericTerm{u}{\mathsf{decompLoc}_{G}((v,p))}$
\end{definition}

The decomposition $\mathsf{decompLoc}(\ell)$ proceeds by inspecting the children of location $\ell$. If there are no children, the resulting child term is an empty hole. This hole is annotated with $\ell$ to act as a unique identifier. If $\ell$ has children, each is annotated with its edge UID and decomposed using the function $\mathsf{decompChild}$. If there are multiple children, they are placed in a local conflict, also annotated with $\ell$. To continue our example, the left position of vertex 40 has no edges, and therefore the result is a hole annotated with the source $(40, \texttt{*}, L)$. The right position has one edge, so the right argument to the multiplication term is the result of applying $\mathsf{decompChild}$ to the destination vertex of this edge. 

\begin{definition}
\begin{align*}
    \mathsf{decompLoc}_{G}(\ell) &= \begin{cases}
        \ehole &\quad \mathsf{children}_{G}(\ell) = \emptyset \\ \eid{w}{}{\mathsf{decompChild}_G(w, v')} &\quad abs{\mathsf{children}_{G}(\ell)} = 1\\ \conflictHoleTerm{\ell}{\eid{w}{}{\mathsf{decompChild}_G(w, v')} \SuchThat{(w,v') \in \mathsf{children}_{G}(\ell)}} &\quad \abs{\mathsf{children}_{G}(\ell)} > 1 \\
    \end{cases}
\end{align*}
\end{definition}

A call to $\mathsf{decompChild}$ is like a call to $\mathsf{decompVertex}$, except that roots are not decomposed further, and a reference is returned instead. Without this case, decomposition could traverse cycles in the graph and never terminate. Instead, decomposition results in well-behaved trees that nevertheless store enough information to enable rich exploration and analysis. In our example, vertex 42 has multiple parents, and therefore is an MP root. Its decomposition is a reference storing the edge UID 41 and the destination vertex.

\begin{definition}
\begin{align*}
    \mathsf{decompChild}_G(w, v)  &= \begin{cases}
        \multiref  &\quad v\text{ is an MP root in } G\\
        \uniref  &\quad v\text{ is a U root in} G\\
        \mathsf{decompVertex}_{G}(v) &\quad v\text{ is not a root in} G
    \end{cases} 
\end{align*}
\end{definition}

Be decomposing each root vertex that appears in $G$, we obtain the whole grove. 

\begin{definition}
    $\mathsf{vertices}(G) = \SetOf{v \in \V \SuchThat{(w, (v,p), v') \in G \land w \in \W \land v' \in \V \land p \in \P}}$
\end{definition}

\begin{definition}
    $\mathsf{decomp}(G) = \SetOf{\mathsf{decompVertex}_{G}(v) \SuchThat{\mathsf{root}_G(v)} \land v \in \mathsf{vertices}(G)}$
\end{definition}

The process can now be reversed and the same live subgraph can be obtained from the grove. At a high level, a term may be considered a tree, and therefore a graph, and the recomposition of a grove may be considered the simple union of each tree in the grove. Formally, the recomposition of a constructor applied to children is a set of edges that includes an edge from the parent to each child, as well as the recomposition of each child. Terms store vertices on constructors and edge UIDs for each child to enable the completeness of this reconstruction. 

The recomposition of a constructor term is the set obtained by adding edges from each child of the term, according to the function $\mathsf{recompChildTerm}$, whereas references contain no graph structure of their own and recompose to the empty set. 

\begin{definition}
    \begin{align*}
        \mathsf{recompTerm}(\genericTerm{u}{\subtermMV_p}) &= \bigcup_{p\in \mathsf{arity}(k)}\mathsf{recompChildTerm}(((u,k),p), \subtermMV_p)\\
        \mathsf{recompTerm}(\multiref) &= \emptyset\\
        \mathsf{recompTerm}(\uniref) &= \emptyset
    \end{align*}
\end{definition}

The recomposition of a child term at location $\ell$ includes, for each term in $\ell$, an edge from the $\ell$ to the term's corresponding vertex, as well as the term's recomposition. In the case of an empty hole, there are no terms, and this the recomposition is empty. If there are one or more terms, each term contributes its own edge and recomposition. 

\begin{definition}
    \begin{align*}
    \mathsf{recompChildTerm}(\ell, \ehole) &= \emptyset\\
    \mathsf{recompChildTerm}(\ell, \eid{w}{}\termMV) &= \SetOf{(w, \ell, \mathsf{vertexOfTerm}(\termMV))}\cup \mathsf{recompTerm}(t)\\
    \mathsf{recompChildTerm}(\ell, \conflictHoleTerm{\ell}{\eid{w_i}{}\termMV_i}_{i<n}) &= \bigcup_{i<n} (\SetOf{(w_i, \ell, \mathsf{vertexOfTerm}(\termMV_i))}\cup \mathsf{recompTerm}(t_i))
    \end{align*}
\end{definition}    

The vertex of a term is used as the destination of edges from the term's parent location. Thus the vertex of a constructor term is the vertex with the same UID and constructor, and the vertex of a reference is the vertex it refers to. 

\begin{definition}
    \begin{align*}
    \mathsf{vertexOfTerm}(\genericTerm{u}{\subtermMV_p}) &= (u,k)\\
    \mathsf{vertexOfTerm}(\multiref) &= v\\
    \mathsf{vertexOfTerm}(\uniref) &= v
    \end{align*}
\end{definition}    


The recomposition of a grove is simply the union of the recompositions of its terms.

\begin{definition}
    $\mathsf{recomp}(T) = \bigcup_{\termMV \in T}\mathsf{recompTerm}(\termMV)$
\end{definition}

The \autoref{thm:Recomposability} states that recomposition recovers the entirety of the original live subgraph, and therefore that groves faithfully represent the underlying data structure.

\begin{theorem}[Recomposability]
\label{thm:Recomposability}
    For all subgraphs $G$, $\mathsf{recomp}(\mathsf{decomp}(G))=G$.
\end{theorem}

\subsubsection{Partitioned Groves}

In order to facilitate user interaction with groves, we defined a presentation of a grove called a \textit{partitioned grove}. A partitioned grove contains three classes, corresponding to the the three kinds of roots: NP, MP, and U. A partitioned grove also designates a distinguished child term as the primary component of the program.

Formally, a partitioned grove $\gamma$ is a quadruple $(\subtermMV_r, NP, MP, U)$ where $\subtermMV_r \in \textit{ChildTerm}$ and $NP, MP, U$ are finite sets of terms. The construction of a partitioned grove from the grove $\mathsf{decomp}(G)$ requires the designation of a \textit{distinguished root} location $\ell_r=(v_r=(u_r,k_r),p_r)$ such that $v_r$ is an NP root in $G$ and $\mathsf{arity}(k_r)=\SetOf{p_r}$. The distinguished child term can be thought of as the decomposition of the distinguished root location. The NP, MP, and U partitions are delineated by the class of the root vertex of each term. The root vertex of a term is like the vertex of a term, except that the root vertex of a reference is the \textit{source} of the corresponding edge, rather than the destination. A term rooted at the distinguished root is excluded from the NP class, since its contents are already identified as the distinguished child term. \autoref{fig:Decomposition example partitioned grove} provides an example of a partitioned grove, where the position $\mathsf{Root}$ of vertex 0 is the distinguished root location.  

\begin{definition}
\begin{align*}
    \mathsf{distinguishedChildTerm}(\ell_r=((u_r,k_r),p_r), \Theta) &= \subtermMV_r &\quad \text{ if }
    \eid{u_r}{k_r}\ {\subtermMV_r}\in \Theta \\
    \mathsf{distinguishedChildTerm}(\ell_r=((u_r,k_r),p_r), \Theta) &= \emptyHole{\ell_r} &\quad \text{ otherwise}
\end{align*}
\end{definition}

\begin{definition}
    \begin{align*}
    \mathsf{rootVertexOfTerm}_{G}(\genericTerm{u}{\subtermMV_p}) &= (u,k)\\
    \mathsf{rootVertexOfTerm}_{G}(\multiref) &= v'\mathsf{ where }(w, (v',p), v)\in G\\
    \mathsf{rootVertexOfTerm}_{G}(\uniref) &=  v'\mathsf{ where }(w, (v',p), v)\in G
    \end{align*}
\end{definition}    

\begin{definition}
    \begin{align*}
    \mathsf{grove}(\ell_r=(v_r,p_r), \Theta, G) = (&\mathsf{distinguishedChildTerm}(\ell_r, \Theta),\\ &\SetOf{t\in \Theta \SuchThat{ \mathsf{rootVertexOfTerm}_{G}(t)\neq v_r\text{ and is an NP root in }G}},\\
    &\SetOf{t\in \Theta \SuchThat{ \mathsf{rootVertexOfTerm}_{G}(t)\text{ is an MP root in }G}},\\ 
    &\SetOf{t\in \Theta \SuchThat{ \mathsf{rootVertexOfTerm}_{G}(t)\text{ is a U root in }G}}) 
    \end{align*}
\end{definition}

\figureUserActionSyntax

\subsection{User Edit Actions}%
\label{sub:User Actions}

We have defined a patch language directly in terms of graph edges and edge states. Since the user interacts with the more intuitive, tree-based grove representation of the program, we require an equally intuitive system of \textit{user edit actions} (edits) that can be translated into the patch language. The new sorts are defined in \autoref{fig:User Action Syntax}.

Edit actions operate on zipper terms~\cite{DBLP:journals/jfp/Huet97,DBLP:conf/popl/OmarVHAH17}. The definitions of zipper terms and child terms are standard, except that the cursor cannot select a child term of the form $\eid{w}{}\termMV$, but is forced to select the term $\termMV$ directly. While inspecting a grove, the user may select an edit action $\alpha$ while the cursor is represented by $\Z{\termMV}$ or $\Z{\subtermMV}$, and this will result in a graph patch $\Q{\pi}$ to be applied directly to the  graph.

$\Construct{k}$ constructs a new edge whose destination is determined by the cursor's location. If the cursor is on an empty hole, the hole is filled by constructing a fresh vertex with constructor $k$. If the cursor is on an existing term or child term, we \emph{wrap} the existing term or child term by deleting it from its parent, constructing the new term in its place, then re-connecting the original term as a child of the new term.

$\Delete$ deletes any edges that pass from constructors above the cursor to constructors below the cursor. If an empty hole is selected, deletion is a no-op. If a local conflict is selected, each edge from that location to a conflicting child is deleted. If a constructor is selected, every edge targeting that constructor is deleted. If a reference is selected, the corresponding edge is deleted. 

$\Relocate{\ell}$ combines edge construction and deletion into a single \emph{atomic} operation. Provided the location $\ell$ is empty, the edges through the cursor are deleted, as described above, and new edges are simultaneously constructed with the same destinations as the old, but originating at location $\ell$. This is not equivalent to the composition of a sequence of deletions and constructions, because the UIDs of the relocated vertices remain the same. 

Formally, these patches are defined in terms of the function $\mathsf{edges}_G$ (\autoref{def:edges}) which produces the set of edges in the live subgraph $G$ passing through a cursor at the given term or child term. The live subgraph $G$ is therefore an additional input to the patch construction judgment.

\begin{definition}
\label{def:edges}
    \[
    \begin{array}{ccc}
    \mathsf{edges}_G(\ehole) & = & \emptyset\\
    \mathsf{edges}_G(\conflict{\termMV}) & = & \SetOf{(w_i,\ell,v)\SuchThat{(w_i,\ell,v)\in G\land i<n\land \ell\in\L\land v\in\V}}\\
    \mathsf{edges}_G(\genericTerm{u}{\subtermMV_p}) & = & \SetOf{(w,\ell,(u,k))\SuchThat{(w,\ell,(u,k))\in G\land w\in\W\land \ell\in\L}}\\
    \mathsf{edges}_G(\multiref) & = & \SetOf{(w,\ell,v)\SuchThat{(w,\ell,v)\in G\land\ell\in\L}}\\
    \mathsf{edges}_G(\uniref) & = & \SetOf{(w,\ell,v)\SuchThat{(w,\ell,v)\in G\land\ell\in\L}}\\
    \end{array}
\]
\end{definition}

For the purpose of the construction action, it is assumed that each constructor $k\in\K$ in the language is equipped with a designated position $\mathsf{defaultPos}(k)\in \mathsf{arity}(k)$. The formal patch generation judgment is defined in terms of $\mathsf{edges}$ and $\mathsf{defaultPos}$ in \autoref{fig:User Action Transitions}.

\figureUserActionTransitions

\subsection{Mechanized Metatheory}

The definitions and theorems in \autoref{sub:Convergent Graphs} and \autoref{subsub:Graph Decomposition} have been mechanized in the Agda proof assistant, with the exception that termination is justified separately for some definitions and proofs. In particular, \autoref{thm:Commutativity} and \autoref{thm:Recomposability} are formalized and proven. \autoref{lem:uniq-classification} follows from the more powerful theorem of classification correctness and completeness, which is mechanized and used to prove \autoref{thm:Recomposability}.


