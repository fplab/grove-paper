
\section{Total Type Error Localization and Recovery for Groves}%
\label{sec:Marking System}

Resolving conflicts is often a time-consuming process 
and requires understanding the semantics of the program
even while it still has conflicts. Traditionally, conflicts are indicated by inserting extra-linguistic conflict markers into files. 

These markers can limit the operation of language services that need a well-formed or well-typed term, 
e.g., type error reporting. Ignoring the markers is not sufficient because 
multiple conflicting versions of the code may appear between the markers. This can complicate variable resolution and type checking.

In the previous sections, we developed a system for explicitly representing various forms of syntactic conflict directly in the grove resulting from graph decomposition. We now consider the static semantics of 
groves.

Groves are sets of terms with holes, local conflicts, relocation conflict references, and unicycle conflict references between them. These terms may not yet be well-typed during the course of a collaboration. 
As such, we need to consider the problem of type error localization and recovery. 

We build on recent work on the marked lambda calculus~\cite{DBLP:journals/pacmpl/ZhaoMDBPO24}, which we briefly review in \autoref{sec:Background: The Marked Lambda Calculus}. 
The marked lambda calculus specifies a \emph{total} 
type error localization system, i.e. one where every syntactically well-formed term can be \emph{marked} with errors to produce a statically meaningful term.

In this section, we extend the marked lambda calculus to allow reasoning about conflicted programs, i.e. groves (instantiated with the syntax of the simply typed lambda calculus). 
The \emph{marked grove calculus} maintains the key totality property, so \emph{every edit state that can arise in the course of a collaboration is statically meaningful}. Downstream language services can, therefore, continue to provide support while users resolve conflicts. This addresses the \textbf{semantic gap problem} described in \autoref{sec:Introduction}.

We follow the marked lambda calculus in basing our \emph{marked grove calculus} in bidirectional type checking~\cite{10.1145/3450952} and deploying gradual typing~\cite{siek2015} when conflicts do not allow a single type to be inferred, then layer on a unification-based type inference system to opportunistically fill holes or suggest partial solutions when there are conflicting types due to conflicting syntax. We give an overview of the key judgments and rules in \autoref{sub:marking-groves} but due to space considerations and because much of the development is based directly on the prior work, we leave the full details to the supplemental material and the mechanized metatheory, which is briefly outlined in \ref{sub:marking-agda}.

\subsection{Background: The Marked Lambda Calculus}
\label{sec:Background: The Marked Lambda Calculus}
The marked lambda calculus is a gradual bidirectionally typed rewriting system that takes an arbitrary syntactically well-formed expression and \emph{marks} it by localizing type errors, producing a \emph{marked expression}. 

The key judgments are synthetic and analytic marking judgments that mark and type-check expressions: $\ctxSynFixedInto{\ctx}{\EMV}{\ECMV}{\TMV}$ and $\ctxAnaFixedInto{\ctx}{\EMV}{\ECMV}{\TMV}$.  
The synthetic judgment is used when a type is to be locally inferred from $\EMV$, while the analytic judgment is used when surrounding type annotations determine an expected type for $\EMV$. A subsumption rule, which is not shown, allows analysis at any type consistent with the synthesized type.

For example, the rules for marking variables are reproduced below:
\begin{mathpar}
  \inferrule[MKSVar]{
    \inCtx{\ctx}{x}{\TMV}
  }{
     \synMarkConstraint{\ctx}{\eVar{}{x}}{\eVar{}{x}}{\TMV}
  }

  \inferrule[MKSFree]{
    \notInCtx{\ctx}{x}
  }{
    \synMarkConstraint{\ctx}{\eVar{}{x}}{\ECFree{\eVar{}{x}}}{\TUnknown}
  }
\end{mathpar}
The first rule handles bound variables. The second rule handles free variables, which would normally be statically meaningless. Here, we instead mark it explicitly as a free variable and synthesize the unknown type, $\TUnknown$. 

The rules for marking lambda abstractions when an expected type is provided similarly handle both well-typed and ill-typed cases (which is key to proving totality):
\begin{mathpar}
\label{inf:marking-rules}
\inferrule[MKALam1]{
        \matchedArrow{\TMV_3}{\TMV_1}{\TMV_2} \\
        \consistent{\TMV}{\TMV_1} \\
        \ctxAnaFixedInto{\extendCtx{\ctx}{x}{\TMV}}{\EMV}{\ECMV}{\TMV_2}
    }{
        \ctxAnaFixedInto{\ctx}{\eFun{}{x}{\TMV}{\EMV}}{\eFun{}{x}{\TMV}{\ECMV}}{\TMV_3}
    }
\end{mathpar}
\begin{mathpar}
     \inferrule[MKALam2]{
        \notMatchedArrow{\TMV_3} \\
        \ctxAnaFixedInto{\extendCtx{\ctx}{x}{\TMV}}{\EMV}{\ECMV}{\TUnknown}
    }{
        \ctxAnaFixedInto{\ctx}{\eFun{}{x}{\TMV}{\EMV}}{\ECLamAnaNonMatchedArrow{x}{\TMV}{\ECMV}}{\TMV_3}
    }

    \inferrule[MKALam3]{
        \matchedArrow{\TMV_3}{\TMV_1}{\TMV_2} \\
        \inconsistent{\TMV}{\TMV_1} \\
        \ctxAnaFixedInto{\extendCtx{\ctx}{x}{\TMV_1}}{\EMV}{\ECMV}{\TMV_2}
    }{
        \ctxAnaFixedInto{\ctx}{\eFun{}{x}{\TMV}{\EMV}}{\ECLamInconAsc{x}{\TMV}{\ECMV}}{\TMV_3}
    }

\end{mathpar}

The first rule invokes the matched arrow judgment (which handles the situation where the expected type is unknown \cite{siek2015}) and performs a type consistency check with the argument type annotation, after which we can analyze the body against the expected output type while recursively marking it.

If there is not a matched arrow type, the second rule marks the lambda with an error (while recovering recursively into the body at an unknown type).
If the argument type annotation is inconsistent, the third rule similarly marks the lambda with a different error. 
The subscript and superscript are formal analogs of type error messages.

Marking is total, meaning every well-formed expression can be marked in this way, producing a well-typed marked expression as governed by the bidirectional typing judgments for marked expressions shown here: $\ctxSynTypeM{\ctx}{\ECMV}{\TMV}$, $\ctxAnaTypeM{\ctx}{\ECMV}{\TMV}$. 

% \todo{TODO: maybe put the statement of totality here?}

These typing rules emit constraints because once marked, the system layers on unification-based type inference to opportunistically fill the unknown types, i.e., the type holes, that arise. In case the constraints governing a type hole cannot be solved, the system suggests hole fillings that satisfy a subset of the constraints, asking the user to resolve the issue (rather than attempting to heuristically decide which constraints were intended). When the user hovers over a choice, the system returns control to the bidirectional marking system.

\subsection{Marking Groves}
\label{sub:marking-groves}

We extend the marked lambda calculus to groves. The machinery related to type errors remains largely unchanged. We focus our attention on type error localization in the presence of conflicts.

\subsubsection{Syntax}
\label{sub:unmarked-lang}
\autoref{fig:marking-syntax} introduces the syntax of the marked grove calculus. We define an \emph{unmarked} language, which comprises unmarked expressions, $\EMV$,  child expressions, $\ELV$, types, $\TMV$, and  child types $\TLMV$.
We also define a corresponding \emph{marked} language,  which comprises marked expressions, $\ECMV$, child expressions, $\ELMV$, types, $\TCMV$, and child types, $\TCLV$. This syntax is an instantiation of the generic term syntax introduced in \autoref{sub:Groves}.
The marked language differs from the unmarked language only by the inclusion of the marks that arise due to type errors taken directly from the prior work. 

\subsubsection{Graph Erasure} The presence of conflicts and UIDs requires us to distinguish syntactic types from semantic types, $\STMV$, which are related by a graph erasure function, $\graphErase{\TMV} = \STMV$ and $\graphErase{\TLMV} = \STMV$, that replaces holes and conflicts with the unknown type and erases vertex and edge UIDs (not shown, see supplement for the full definition).

\subsubsection{Marking}
\label{sub:marking}
Marking is performed bidirectionally using four mutually defined judgments:
\synMarkConstraint{\ctx}{\EMV}{\ECMV}{\STMV},
\synMarkConstraint{\ctx}{\ELV}{\ELMV}{\STMV},
\anaMarkConstraint{\ctx}{\EMV}{\ECMV}{\STMV}, and
\synMarkConstraint{\ctx}{\ELV}{\ELMV}{\STMV}.


\begin{figure}
    \[\begin{array}{lcllll}
    \TMV & \in & \TMName & \Coloneqq & 
        \tnum 
        \mid \tarrow 
        \mid \multiref 
        \mid \uniref \\ 
    \TLMV & \in & \TLMName & \Coloneqq &
        \ehole
        \mid \lexp{\TMV}
        \mid \conflict{\TMV} \\
     \EMV & \in & \EMName & \Coloneqq & 
        \var{x}
        \mid \num 
        \mid \plus{\ELV_1}{\ELV_2} 
        \mid \mult{\ELV_1}{\ELV_2}
        \mid \lam{x}{\TLMV}{\ELV}
        \mid \app{\ELV_1}{\ELV_2}
        \mid \multiref
        \mid \uniref \\
     \ELV & \in & \ELMName & \Coloneqq & 
        \ehole
        \mid \lexp{\EMV} 
        \mid \conflict{\EMV} \\
        & & & \\
     \ECMV & \in & \ECMName & \Coloneqq &
        \var{x}
        \mid \num
        \mid \plus{\ELMV_1}{\ELMV_2}
        \mid \mult{\ELMV_1}{\ELMV_2}
        \mid \lam{x}{\TLMV}{\ELV}
        \mid \app{\ELMV_1}{\ELMV_2}
        \mid \multiref
        \mid \uniref \\ 
        & & & &
          \ECFree{\var{x}} 
        \mid \lamasc
        \mid \lamanamarr
        \mid \apsynmarr
        \mid \incontype \\
    \ELMV & \in & \ELMVName & \Coloneqq &
        \ehole
        \mid \lexp{\ECMV}
        \mid \conflict{\ECMV} \\
        & & & \\
    \STMV & \in & \STMName & \Coloneqq & 
        \TUnknown
        \mid \tNum{}
        \mid \TArrow{\STMV_1}{\STMV_2} \\
        & & & \\
    \PTMV & \in & \PTMName & \Coloneqq & 
        \TUnknown^{\Provp}
        \mid \tNum{}
        \mid \TArrow{\PTMV_1}{\PTMV_2} \\
    \Provp & \in & \Prov & \Coloneqq &
        \mathsf{typ}(\ell)
        \mid \mathsf{exp}(\ell)
        \mid \mathsf{ref}(w)
        \mid \mathsf{mark}(u)
        \mid ~\rightarrow_L(\Provp) \mid ~\rightarrow_R(\Provp)
        \mid \mathsf{anon} \\
    m & \in & \mathsf{Mode} & \Coloneqq & \mathsf{syn} \mid \mathsf{ana}(\STMV)
\end{array}\]
    \vspace{-10px}
    \caption{Syntax}
    \label{fig:marking-syntax}
\end{figure}

For standard forms in the simply typed lambda calculus, the rules correspond directly to those from the marked lambda calculus. For the sake of brevity, we include only the rules related to lambda expressions below; the full set of rules are in the supplemental material.

\begin{mathpar}
\label{inf:mark-lam-ana}
\inferrule[MKALam1]{
        \graphErase{\TLMV} = \sigma \\
        \matchedArrow{\sigma_3}{\sigma_1}{\sigma_2} \\
        \consistent{\sigma}{\sigma_1} \\
        \anaMarkConstraint{\extendCtx{\ctx}{x}{\sigma}}{\ELV}{\ELMV}{\sigma_2}
    }{
        \anaMarkConstraint{\ctx}{\lam{x}{\TLMV}{\EMV}}{\lam{x}{\TLMV}{\ELMV}}{\sigma_3}
    }
\end{mathpar}
\begin{mathpar}
     \inferrule[MKALam2]{
        \graphErase{\TLMV} = \sigma \\\\
        \notMatchedArrow{\sigma_3} \\
        \anaMarkConstraint{\extendCtx{\ctx}{x}{\sigma}}{\ELV}{\ELMV}{\TUnknown}
    }{
        \anaMarkConstraint{\ctx}{\reglam}{\ECLamAnaNonMatchedArrow{\var{x}}{\TLMV}{\ELMV}}{\sigma_3}
    }
\end{mathpar}
\begin{mathpar}
     \inferrule[MKALam3]{
        \erase{\TLMV} = \sigma \\\\
        \matchedArrow{\sigma_3}{\sigma_1}{\sigma_2} \\
        \inconsistent{\sigma}{\sigma_1} \\
        \anaMarkConstraint{\extendCtx{\ctx}{x}{\sigma_1}}{\ELV}{\ELMV}{\sigma_2}
    }{
        \anaMarkConstraint{\ctx}{\reglam}{\ECLamInconAsc{\var{x}}{\TLMV}{\ELMV}}{\sigma_3}
    }
\end{mathpar}

These rules differ from the original rules, reproduced in \autoref{sec:Background: The Marked Lambda Calculus}, only in that they graph-erase the type annotation and recurse using the marking rules for child expressions. For empty holes and singleton expressions, these are straightforward. We return to local conflicts below.%
% \todo{synthetic rules for child expressions?}
\begin{mathpar}
    \inferrule[MKSHole]{ }{
        \synMarkConstraint{\ctx}{\emptyHole{\ell}}{\emptyHole{\ell}}{\TUnknown}
    }

    \inferrule[MKSOnly]{ 
        \synMarkConstraint{\ctx}{\EMV}{\ECMV}{\STMV}
    }{ 
        \synMarkConstraint{\ctx}{\lexp{\EMV}}{\lexp{\ECMV}}{\STMV}
    }
    
    \inferrule[MKAHole]{ }{
        \anaMarkConstraint{\ctx}{\emptyHole{\ell}}{\emptyHole{\ell}}{\STMV}
    }

    \inferrule[MKAOnly]{ 
        \anaMarkConstraint{\ctx}{\EMV}{\ECMV}{\STMV}
    }{ 
        \anaMarkConstraint{\ctx}{\lexp{\EMV}}{\lexp{\ECMV}}{\STMV}
    }
\end{mathpar}

Once marked, the corresponding marked expression typing judgments, $\synConstraint{\ctx}{\ECMV}{\PTMV}{\cConstraint}{\mConstraint}$, $\synConstraint{\ctx}{\ELMV}{\PTMV}{\cConstraint}{\mConstraint}$, 
$\anaConstraint{\ctx}{\ECMV}{\PTMV}{\cConstraint}{\mConstraint}$, and $\anaConstraint{\ctx}{\ELMV}{\PTMV}{\cConstraint}{\mConstraint}$ emit type inference constraints, $\cConstraint$, as in prior work, and, recursively collect location conflict contexts, $\mConstraint$, discussed further below. When the system generates inference constraints and tries to unify them, it encounters unknown types that need to be distinguished. This motivates the need for $\PTMV$ in \autoref{fig:marking-syntax}, which extends the sort of semantic types to enable linking generated unknown types to their associated provenance denoted by $q$. Provenances help locate the origin of the unknown type by associating them with their location ($\ell$), edge-ID ($w$), or vertex-ID ($u$).

% \todo{mention why we need provenances}
% \todo{marked expression typing rules for lambdas and marked lambdas}

\begin{mathpar}
    \inferrule[MALam1]{
        \matchedArrowConstraint{\PTMV_3}{\PTMV_1}{\PTMV_2}{\cConstraint_1} \\
        \graphErase{\TLMV} = \PTMV \\
        \consistent{\PTMV}{\PTMV_1} \\
        \anaConstraint{\extendCtx{\ctx}{x}{\PTMV}}{\ELMV}{\PTMV_2}{\cConstraint_2}{\mConstraint_1}
    }{
        \anaConstraint{\ctx}{\lam{x}{\TLMV}{\ELMV}}{\PTMV_3}{\cConstraint_1 \cup \cConstraint_2 \cup \Setof{\constrain{\PTMV}{\PTMV_1}}}{\mConstraint_1}
    }
\end{mathpar}
\begin{mathpar}
    \inferrule[MALam2]{
        \notMatchedArrow{\PTMV_3} \\
        \graphErase{\TLMV} = \PTMV \\
        \anaConstraint{\extendCtx{\ctx}{x}{\PTMV}}{\ELMV}{\TUnknown^{\mathsf{anon}}}{\cConstraint}{\mConstraint}
    }{
        \anaConstraint{\ctx}{\ECLamAnaNonMatchedArrow{\var{x}}{\TLMV}{\ELMV}}{\PTMV_3}{\cConstraint \cup \Setof{\constrain{\TUnknown^{\mathsf{mark}(u)}}{\PTMV_3}}}{\mConstraint}
  }
\end{mathpar}
\begin{mathpar}
    \inferrule[MALam3]{
        \matchedArrowConstraint{\PTMV_3}{\PTMV_1}{\PTMV_2}{\cConstraint_1} \\
        \graphErase{\TLMV} = \PTMV \\
        \inconsistent{\PTMV}{\PTMV_1} \\
        \anaConstraint{\extendCtx{\ctx}{x}{\PTMV_1}}{\ELMV}{\PTMV_2}{\cConstraint_2}{\mConstraint}
    }{
        \anaConstraint{\ctx}{\ECLamInconAsc{\var{x}}{\TLMV}{\ELMV}}{\PTMV_3}{\cConstraint_1 \cup \cConstraint_2 \cup \Setof{\constrain{\TUnknown^{\mathsf{mark}(u)}}{\PTMV_3}}}{\mConstraint}
    }
\end{mathpar}

\subsubsection{Local Conflicts}
To mark local conflicts, we recursively mark all conflicted terms under the given context and with the same expected type when available. The local conflict as a whole synthesizes the unknown type, i.e. it is essentially a ``conflict hole'':
\begin{mathpar}
    \inferrule[MKSLocalConflict]{ 
        \Setof{\synMarkConstraint{\ctx}{\EMV}{\ECMV}{\sigma_i}}_{i<n}
    }{
        \synMarkConstraint{\ctx}{\conflict{\EMV}}{\conflict{\ECMV}}{\TUnknown}
    }

    \inferrule[MKALocalConflict]{ 
      \Setof{\anaMarkConstraint{\ctx}{\EMV_i}{\ECMV_i}{\STMV}}_{i<n}
    }{
        \anaMarkConstraint{\ctx}{\conflict{\EMV}}{\conflict{\ECMV}}{\STMV}
    }
\end{mathpar}

After marking, we generate type inference constraints that constrain this hole (identified using a provenance based on the location of conflict, $\ell$) using all of the conflicted terms.
\begin{mathpar}
    \inferrule[MSLocalConflict]{ 
      \Setof{\synConstraint{\ctx}{\ECMV}{\PTMV_i}{\cConstraint_i}{\mConstraint_i}}_{i<n}
    }{
      \synConstraint{\ctx}{\conflict{\ECMV}}{\TUnknown^{\mathsf{exp}(\ell)}}{\bigcup_{i<n} \cConstraint_i \cup \Setof{\constrain{\TUnknown^{\mathsf{exp}(\ell)}}{\PTMV_i}}_{i<n}}{\bigcup_{i<n} \mConstraint_i}
    }

    \inferrule[MALocalConflict]{ 
        \Setof{\anaConstraint{\ctx}{\ECMV_i}{\PTMV}{\cConstraint_i}{\mConstraint_i}}_{i<n}
    }{ 
      \anaConstraint{\ctx}{\conflict{\ECMV}}{\PTMV}{\bigcup_{i<n} \cConstraint_i}{\bigcup_{i<n} \mConstraint_i}
    }
\end{mathpar}

 When the conflicted terms have a consistent type, inference will succeed. When it fails, we can fall back to the unknown type and rely on the type hole inference system developed in prior work to allow users to choose from partial hole solutions. When a partial solution is selected (effectively annotating the conflict), the mode becomes analytic, and the conflicted terms are remarked, allowing users to identify where errors would arise if the conflict were resolved at a particular type.

% \todo{check rule names}

\subsubsection{Relocation and Unicycle Conflict References}
%
Relocation and unicycle conflict references also synthesize the unknown type and operate similarly: 
\begin{mathpar}
    \inferrule[MKSRelocationConflict]{ }{
        \synMarkConstraint{\ctx}{\multiref}{\multiref}{\TUnknown}
    }
    
    \inferrule[MSRelocationConflict]{ }{
        \synConstraint{\ctx}{\multiref}{\TUnknown^{\mathsf{ref}(w)}}{\constraintNil{}}{(v, w, \ctx, \mathsf{syn})}
    }
    
    \inferrule[MARelocationConflict]{
    }{
      \anaConstraint{\ctx}{\multiref}{\PTMV}{\Setof{\constrain{\TUnknown^{\mathsf{ref}(v)}}{\PTMV}}}{(v, w ,\ctx, \mathsf{ana}(\PTMV))}
    }
\end{mathpar}
\begin{mathpar}
   \inferrule[MKSCycleLocationConflict]{ }{
    \synMarkConstraint{\ctx}{\uniref}{\uniref}{\TUnknown}
  }
  
  \inferrule[MSUnicycleConflict]{ }{
    \synConstraint{\ctx}{\uniref}{\TUnknown^{\mathsf{ref}(w)}}{\constraintNil{}}{(v ,w, \ctx, \mathsf{syn})}
  }
  
  \inferrule[MAUnicycleConflict]{ 
    }{
      \anaConstraint{\ctx}{\uniref}{\PTMV}{\Setof{\constrain{\TUnknown^{\mathsf{ref}(v)}}{\PTMV}}}{(v, w ,\ctx, \mathsf{ana}(\PTMV))}
    }
\end{mathpar}


Because the referenced term has multiple possible locations, it also has multiple possible typing contexts and typing modes, so we cannot immediately mark it.
Instead, we gather this information in the location conflict context, $L$, as shown above. This maps a location (identified by an edge UID, $w$, and also for the vertex, $v$, for operational simplicity) to a pair of a typing context, $\Gamma$, and a typing mode, $m$. 
This information can be used by the system to provisionally mark the conflicted term under an explicitly selected context and mode (e.g. in response to which location the user's cursor is on or using some other user interface affordance).

The emitted constraints assume an unknown type for the referenced vertex and constrain it with any expected types that appear at any of the locations where the corresponding term appears. Again, solutions to this unknown type can be presented to the user to help them decide which location might be most sensible.

% The user interface design space for these kinds of affordances is large, so we do not aim to contribute a particular design here. Instead, our intention here was to demonstrate that conflicts can give rise to an interesting static semantics by incorporating ideas from bidirectional typing, gradual typing, and type inference.

\subsection{Mechanized Metatheory}
\label{sub:marking-agda}
The key meta-theoretic property that tells us that we did not miss any cases is \emph{totality}:
% \todo{add marked typing to right side of each clause below}

% TODO: totality statement\todo{totality statement}

\begin{theorem}[name=Marking Totality] \
  \begin{enumerate}
    \item For all $\ctx$ and $\EMV$, there exist $\ECMV$ and $\STMV$ such that
      $\ctxSynFixedInto{\ctx}{\EMV}{\ECMV}{\STMV}$ and $\synConstraint{\ctx}{\ECMV}{\PTMV}{\cConstraint}{\mConstraint}$ and $\provErase{\PTMV} = \STMV$.
    % \item For all $\ctx$ and $\ELV$, there exist $\ELMV$ and $\STMV$ such that
      % $\ctxSynFixedInto{\ctx}{\ELV}{\ELMV}{\STMV}$ and $\synConstraint{\ctx}{\ELMV}{\PTMV}{\cConstraint}{\mConstraint}$ and $\provErase{\PTMV} = \STMV$.
    \item For all $\ctx$, $\EMV$, and $\STMV$, there exists $\ECMV$ such that
      $\ctxAnaFixedInto{\ctx}{\EMV}{\ECMV}{\STMV}$ and $\anaConstraint{\ctx}{\ECMV}{\PTMV}{\cConstraint}{\mConstraint}$ and $\provErase{\PTMV} = \STMV$.
    % \item For all $\ctx$, $\ELV$, and $\STMV$, there exists $\ELMV$ such that
      % $\ctxAnaFixedInto{\ctx}{\ELV}{\ELMV}{\STMV}$ and $\anaConstraint{\ctx}{\ELMV}{\PTMV}{\cConstraint}{\mConstraint}$ and $\provErase{\PTMV} = \STMV$.
  \end{enumerate}
\end{theorem}


In addition, the prior work defined a number of other auxiliary metatheorems that help sanity check these definitions: well-formedness (marking preserves syntactic structure), and marking unicity (marking is deterministic). 

We have mechanized~\cite{grove-artifact} our extension of the marked lambda calculus and these metatheorems using the Agda proof assistant, taking the standard approach of modeling judgments as inductive datatypes and inference rules as constructors. Marked terms are intrinsically typed, allowing $L$ to be computed as a function of the term.

% TODO discuss how no type hole inference/constraint generation so 
% TODO discuss how it is separate from the grove agda but a correspondence can be proven?
% \begin{figure}
% \label{fig:unmarked-judgments}
% \caption{Typing Unmarked Expressions}
%     \judgbox{\ctxSynTypeU{\ctx}{\EMV}{\sigma}} 
%     \judgbox{\ctxSynTypeU{\ctx}{\ELV}{\sigma}}
%     \judgbox{\ctxAnaTypeU{\ctx}{\EMV}{\sigma}} 
%     \judgbox{\ctxAnaTypeU{\ctx}{\ELV}{\sigma}}
%     \centering   
% \end{figure}

% \begin{figure}
%     \[
%     \begin{array}{lclll}
%      \EMV \in & Exp & \coloneqq & 
%         \var{x}
%         \mid  \num 
%         \mid \plus{\ELV_1}{\ELV_2} 
%         \mid \mult{\ELV_1}{\ELV_2}
%         \mid \lam{x}{\TLMV}{\ELV}
%         \mid \app{\ELV_1}{\ELV_2}
%         \mid \multiref
%         \mid \uniref \\
%      \ELV \in & ExpList & \coloneqq & 
%         \ehole
%         \mid \lexp{\EMV} 
%         \mid \conflict{\EMV} \\
%     \TMV \in & \Typ & \coloneqq & 
%         \tnum 
%         \mid \tarrow 
%         \mid \multiref 
%         \mid \uniref \\ 
%     \TLMV \in & TypList & \coloneqq &
%         \ehole
%         \mid \lexp{\TMV}
%         \mid \conflict{\TMV} \\
%     \end{array}
%     \]
%     \centering
%     \caption{Unmarked Language}
%     \label{fig:unmarked-language}
% \end{figure}

% \begin{figure}
%     \[
%     \begin{array}{lclll}
%     \ECMV \in & MExp & \coloneqq &
%         \var{x}
%         \mid \num
%         \mid \plus{\ELMV_1}{\ELMV_2}
%         \mid \mult{\ELMV_1}{\ELMV_2}
%         \mid \lam{x}{\TLMV}{\ELV}
%         \mid \app{\ELMV_1}{\ELMV_2}
%         \mid \multiref
%         \mid \uniref \\ 
%         & & & \ECFree{\var{x}} 
%         \mid \incontype
%         \mid \lamasc
%         \mid \lamanamarr
%         \mid \apsynmarr \\ 
%         % \mid \lamge \\
%      \ELMV \in & MExpList & \coloneqq &
%         \ehole
%         \mid \lexp{\ECMV}
%         \mid \conflict{\ECMV} \\
%     \end{array}
%     \]
%     \centering
%     \caption{Marked Language}
%     \label{fig:marked-language}
% \end{figure}

% \begin{figure}
%     \[
%     \begin{array}{lclll}
%     \sigma \in & STyp & \coloneqq & 
%         \TUnknown
%         \mid \tNum{}
%         \mid \starrow \\
%     \end{array}
%     \]
%     \judgbox{\graphErase{\TLMV} = \sigma}
%     \[\begin{array}{rrcl}
%         \Prov & \Provp & \Coloneqq & u \mid exp(u) \mid \rightarrow_L(\Provp) \mid \rightarrow_R(\Provp)\\
%         \TMName  & \TMV  & \Coloneqq & \dots \mid \TUnknown^{p} \mid \TUnknown^{(v,p)}\\
%     \end{array}\]
%     \judgbox{\ensuremath{\matchedArrowConstraint{\sigma}{\sigma_1}{\sigma_2}{C}}}
%     \centering
%     \caption{Other machinery}
%     \label{fig:marking-machinery}
% \end{figure}




% \begin{figure}
%     \[
%     \begin{array}{lclll}
%     \ECMV \in & MExp & \coloneqq &
%         \var{x}
%         \mid \num
%         \mid \plus{\ELMV_1}{\ELMV_2}
%         \mid \mult{\ELMV_1}{\ELMV_2}
%         \mid \lam{x}{\TLMV}{\ELV}
%         \mid \app{\ELMV_1}{\ELMV_2}
%         \mid \multiref
%         \mid \uniref \\ 
%         & & & \ECFree{\var{x}} 
%         \mid \incontype
%         \mid \lamasc
%         \mid \lamanamarr
%         \mid \apsynmarr \\ 
%         % \mid \lamge \\
%      \ELMV \in & MExpList & \coloneqq &
%         \ehole
%         \mid \lexp{\ECMV}
%         \mid \conflict{\ECMV} \\
%     \end{array}
%     \]
%     \centering
%     \caption{Marked Language}
%     \label{fig:marked-language}
% \end{figure}

% \begin{figure}
%     \judgbox{\synMarkConstraint{\ctx}{\EMV}{\ECMV}{\STMV}}
%     \judgbox{\synMarkConstraint{\ctx}{\ELV}{\ELMV}{\STMV}}
%     \judgbox{\anaMarkConstraint{\ctx}{\EMV}{\ECMV}{\STMV}}
%     \judgbox{\anaMarkConstraint{\ctx}{\ELV}{\ELMV}{\STMV}}
%     \centering
%     \caption{Marking Expressions}
%     \label{fig:marking-judgments}
% \end{figure}