%
% utilities
%
\newcommand{\colorSideJudge}{Black!50}
\newcommand{\colorCText}{\colorUText}
\newcommand{\colorCBkgSyn}{\colorUBkgSyn}
\newcommand{\colorCBkgAna}{\colorUBkgAna}
\definecolor{hole}{RGB}{162,85,162}
\definecolor{cursorhighlight}{RGB}{230,255,230}
\definecolor{cursor}{RGB}{76,170,76}
\newcommand{\mathcolorbox}[2]{{\fboxsep=1pt\colorbox{#1}{$\displaystyle #2$}}}

%
% color
%
\newtcolorbox{highlightbox}[2][]{
  on line,
  hbox,
  boxsep=0pt,
  left=1pt,
  right=1pt,
  top=1pt,
  bottom=1pt,
  colframe=white,
  colback=#2
  #1,
}

\DeclareMathOperator{\?}{?}
\DeclareMathOperator{\dom}{dom}


\input{symbols/types}
\input{symbols/expressions}
\input{symbols/marked}

%
% judgments
%

% judgments
\newcommand{\judgment}[3]{\inferrule[#3]{#1}{#2}}
\newcommand{\judgbox}[1]{\noindent \fbox{$#1$}}


% constraints
\newcommand{\constrain}[2]{{#1} \approx {#2}}
\newcommand{\constraintCons}[2]{\normalfont\textsf{cons}(#1, #2)}
\newcommand{\constraintNil}{\{\}}
\newcommand{\maps}[2]{{#1} \mapsto {#2}}
\newcommand{\mapNil}[0]{\cdot}

\newcommand{\matchedArrowConstraint}[4]{
    \matchedArrow{#1}{#2}{#3}\goodcolor{\colorSideJudge}{~|~} {#4}
}
\newcommand{\matchedProdConstraint}[4]{
    \matchedProd{#1}{#2}{#3}\goodcolor{\colorSideJudge}{~|~} {#4}
}

\newcommand{\constraintTurn}{\goodcolor{\colorCText}{\vdash}}
\newcommand{\constraintBar}{\goodcolor{\colorCText}{~|~}}
\newcommand{\constraintSyn}{\goodcolor{\colorCText}{\Rightarrow}}
\newcommand{\constraintAna}{\goodcolor{\colorCText}{\Leftarrow}}
\newcommand{\synConstraint}[5]{
    \begin{mybox}{\colorCBkgSyn}
    \ensuremath{{#1} \constraintTurn {#2} \constraintSyn {#3} \constraintBar {#4} \constraintBar {#5}}
    \end{mybox}
}
\newcommand{\anaConstraint}[5]{
    \begin{mybox}{\colorCBkgAna}
    \ensuremath{{#1} \constraintTurn {#2} \constraintAna {#3} \constraintBar {#4} \constraintBar {#5}}
    \end{mybox}
}
\newcommand{\synMarkConstraint}[4]{
    \begin{mybox}{\colorCBkgSyn}
      \ensuremath{{#1} \constraintTurn {#2} \goodcolor{\colorMKText}        {\looparrowright} {#3}
      \constraintSyn {#4}}
    \end{mybox}
}
\newcommand{\anaMarkConstraint}[4]{
    \begin{mybox}{\colorCBkgAna}
      \ensuremath{{#1} \constraintTurn {#2} \goodcolor{\colorMKText}{\looparrowright} {#3}
    \constraintAna {#4}}
    \end{mybox}
}

\newcommand{\mConstraint}[0]{M}
\newcommand{\cConstraint}[0]{C}


\newcommand{\meet}[2]{{#1} \sqcap {#2}}


% type incomplete
\newcommand{\incomplete}[1]{#1 ~ \normalfont\textsf{incomplete}}

% potential type
\newcommand{\PTUnknownVar}[1]{\ensuremath{#1}}
\newcommand{\PTNum}{\ensuremath{{\normalfont\textsf{N}}}}
\newcommand{\PTBool}{\ensuremath{{\normalfont\textsf{B}}}}
\newcommand{\PTArrow}[2]{\ensuremath{#1 \to #2}}

% potential types
\newcommand{\ptypCons}[2]{\normalfont\textsf{cons}(#1, #2)}
\newcommand{\ptypSingle}[1]{\normalfont\textsf{single}(#1)}

% potential type set
\newcommand{\ptsRep}[1]{\normalfont\textsf{rep}(#1)}
\newcommand{\ptsLead}[1]{\normalfont\textsf{lead}(#1)}

% PTGraph Elements and Membership
\newcommand{\ptypGraphNode}[2]{#1\normalfont\textsf{ : }#2}
\newcommand{\underGraph}[2]{#1~|~#2}
\DeclareMathOperator{\cod}{cod}
\newcommand{\isIn}[2]{#1 \in #2}
\newcommand{\isNotIn}[2]{#1 \notin #2}

% PTGraph Update
\newcommand{\ptypGraphUpdate}[4]{
    \underGraph{#1}{\normalfont\textsf{update}~#2:#3 ~\hookrightarrow~ #4}
}

% PTGraph add node
\newcommand{\ptypGraphAdd}[3]{
    \underGraph{#1}{\normalfont\textsf{add}~#2 ~\hookrightarrow~ #3}
}

% PTSet Leader
\newcommand{\leader}[4]{
    \underGraph{#1}{\normalfont\textsf{leader}~#2 \equiv \ptypGraphNode{#3}{#4}}
}

% Unions
\newcommand{\unionOf}[3]{#2 \cup_{#1} #3}

%   Type Union
\newcommand{\typUnion}[2]{\unionOf{\tau}{#1}{#2}}
\newcommand{\ptypGraphUnion}[4]{
    \underGraph{#1}{\typUnion{#2}{#3}} ~\hookrightarrow~ #4
}

% PTyps Union
\newcommand{\ptypsUnion}[2]{#1~\Cup~#2}
\newcommand{\specSubset}[3]{#1~\Subset_{#3}~#2}
\newcommand{\ptypesSubset}[2]{\specSubset{#1}{#2}{u}}
\newcommand{\ptypSetSubset}[3]{\underGraph{#1}{\specSubset{#2}{#3}{s}}}
\newcommand{\snapSubset}[2]{\specSubset{#1}{#2}{j}}
\newcommand{\snapElement}[2]{#1 ~\in_{h}~ #2}

% unification
\newcommand{\unify}[3]{\underGraph{#1}{#2~\amalg~#3}}

% Snapshot PTS
\newcommand{\ptsSnapshot}[3]{\underGraph{#1}{\normalfont\textsf{snapshot}_s~#2 \equiv #3}}

% Snapshot List
\newcommand{\listSnapshot}[3]{\underGraph{#1}{\normalfont\textsf{snapshot}_l~#2 \equiv #3}}

% Solution
\newcommand{\solution}[2]{\normalfont\textsf{solution}~#1 \equiv #2}

% List
\newcommand{\listCons}[2]{\normalfont\textsf{cons}(#1, #2)}
\newcommand{\listNil}{\normalfont\textsf{nil}}

% fresh
\newcommand{\markFresh}[1]{#1~\normalfont\textsf{fresh}}
