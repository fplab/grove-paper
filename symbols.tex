%
% utilities
%
\newcommand{\colorSideJudge}{Black!50}
\newcommand{\colorCText}{\colorUText}
\newcommand{\colorCBkgSyn}{\colorUBkgSyn}
\newcommand{\colorCBkgAna}{\colorUBkgAna}
\definecolor{hole}{RGB}{162,85,162}
\definecolor{cursorhighlight}{RGB}{230,255,230}
\definecolor{cursor}{RGB}{76,170,76}
\newcommand{\mathcolorbox}[2]{{\fboxsep=1pt\colorbox{#1}{$\displaystyle #2$}}}

%
% color
%
\newtcolorbox{highlightbox}[2][]{
  on line,
  hbox,
  boxsep=0pt,
  left=1pt,
  right=1pt,
  top=1pt,
  bottom=1pt,
  colframe=white,
  colback=#2
  #1,
}

\DeclareMathOperator{\?}{?}
\DeclareMathOperator{\dom}{dom}


\newcommand{\colorOkSideJudge}{Black!50}
\newcommand{\colorFailSideJudge}{red}

%
% relations
%

% equality
\newcommand{\markEqual}[2]{\ensuremath{#1 = #2}}
\newcommand{\notEqual}[2]{\ensuremath{#1 \neq #2}}

% consistency
\newcommand{\consistentRel}{\ensuremath{\sim}}
\newcommand{\consistent}[2]{\ensuremath{#1 \goodcolor{\colorOkSideJudge}{\consistentRel} #2}}
\newcommand{\inconsistentRel}{\ensuremath{\nsim}}
\newcommand{\inconsistent}[2]{\ensuremath{\goodcolor{\colorFailSideJudge}{#1 \inconsistentRel #2}}}

% matching
\newcommand{\matchedRel}[1]{\ensuremath{\blacktriangleright_{#1}}}
\newcommand{\notMatchedRel}[1]{\ensuremath{\blacktriangleright_{\centernot{#1}}}}
\newcommand{\matchedArrowRel}{\ensuremath{\matchedRel{\to}}}
\newcommand{\notMatchedArrowRel}{\ensuremath{\notMatchedRel{\to}}}
\newcommand{\matchedArrow}[3]{\ensuremath{#1 \goodcolor{\colorOkSideJudge}{\matchedArrowRel} \TArrow{#2}{#3}}}
\newcommand{\notMatchedArrow}[1]{\ensuremath{\goodcolor{\colorFailSideJudge}{#1 \notMatchedArrowRel}}}
\newcommand{\matchedProdRel}{\ensuremath{\matchedRel{\times}}}
\newcommand{\notMatchedProdRel}{\ensuremath{\notMatchedRel{\times}}}
\newcommand{\matchedProd}[3]{\ensuremath{#1 \goodcolor{\colorOkSideJudge}{\matchedProdRel} \TProd{#2}{#3}}}
\newcommand{\notMatchedProd}[1]{\ensuremath{\goodcolor{\colorFailSideJudge}{#1 \notMatchedProdRel}}}

% base types
\newcommand{\base}[1]{\ensuremath{#1 ~{\normalfont\textsf{base}}}}

% 
% syntax
%
\newcommand{\TMName}{{\normalfont\textsf{UType}}}
\newcommand{\TMSet}{\ensuremath{T}}
\newcommand{\TMV}{\ensuremath{\tau}}
\newcommand{\TLMName}{{\normalfont\textsf{UChildType}}}
\newcommand{\TLMV}{\bar{\TMV}}

\newcommand{\STMName}{{\normalfont\textsf{Type}}}
\newcommand{\STMV}{\ensuremath{\sigma}}

\newcommand{\PTMName}{{\normalfont\textsf{TypeWithProvenance}}}
\newcommand{\PTMV}{\ensuremath{\dot{\sigma}}}

\newcommand{\meetRel}{\sqcap}
\newcommand{\noMeetRel}{\centernot\sqcap}
\newcommand{\TMeet}[2]{\ensuremath{#1 \meetRel #2}}

\newcommand{\TUnknown}{\ensuremath{\?}}
\newcommand{\TUnknownSwitch}{\ensuremath{\?^{\Rightarrow}}}
\newcommand{\TNum}{\ensuremath{{\normalfont\textsf{num}}}}
\newcommand{\TBool}{\ensuremath{{\normalfont\textsf{bool}}}}
\newcommand{\TArrow}[2]{\ensuremath{#1 \to #2}}
\newcommand{\TProd}[2]{\ensuremath{#1 \times #2}}

%
% contexts
%
\newcommand{\ctx}{\ensuremath{\Gamma}}
\newcommand{\extendCtx}[3]{\ensuremath{#1 , ~\assignType{#2}{#3}}}
\newcommand{\inCtx}[3]{\ensuremath{\assignType{#2}{#3} \in #1}}
\newcommand{\notInCtx}[2]{\ensuremath{\goodcolor{\colorFailSideJudge}{#2 \not\in \dom(#1)}}}
\newcommand{\withCtx}[2]{\ensuremath{#1 \vdash #2}}

%
% typing
%
\newcommand{\assignType}[2]{\ensuremath{#1 : #2}}
\newcommand{\synType}[2]{\ensuremath{#1 \Rightarrow #2}}
\newcommand{\notSynType}[2]{\ensuremath{#1 \not\Rightarrow #2}}
\newcommand{\ctxSynType}[3]{\ensuremath{\withCtx{#1}{\synType{#2}{#3}}}}
\newcommand{\ctxNotSynType}[3]{\ensuremath{\withCtx{#1}{\notSynType{#2}{#3}}}}
\newcommand{\anaType}[2]{\ensuremath{#1 \Leftarrow #2}}
\newcommand{\notAnaType}[2]{\ensuremath{#1 \goodcolor{\colorFailSideJudge}{\not\Leftarrow} #2}}
\newcommand{\ctxAnaType}[3]{\ensuremath{\withCtx{#1}{\anaType{#2}{#3}}}}
\newcommand{\ctxNotAnaType}[3]{\ensuremath{\withCtx{#1}{\notAnaType{#2}{#3}}}}

% !requires types

%
% external expressions
%
\newcommand{\EMName}{{\normalfont\textsf{UExp}}}
\newcommand{\EMSet}{\ensuremath{E}}
\newcommand{\EMV}{\ensuremath{e}}
\newcommand{\ELMName}{{\normalfont\textsf{UChildExp}}}
\newcommand{\ELV}{\bar{\EMV}}

% holes
\newcommand{\EEHole}{\ensuremath{\ECEHole}}

% integers
\newcommand{\ENum}[1]{\ensuremath{\ECNum{#1}}}
\newcommand{\ENumMV}{\ensuremath{\ECNumMV}}
\newcommand{\EOpPlus}{\ensuremath{\ECOpPlus}}
\newcommand{\EPlus}[2]{\ensuremath{\ECPlus{#1}{#2}}}

% booleans
\newcommand{\ETrue}{\ensuremath{\ECTrue}}
\newcommand{\EFalse}{\ensuremath{\ECFalse}}
\newcommand{\EIf}[3]{\ensuremath{\ECIf{#1}{#2}{#3}}}

% lambdas
\newcommand{\ELam}[3]{\ensuremath{\ECLam{#1}{#2}{#3}}}
\newcommand{\EAp}[2]{\ensuremath{\ECAp{#1}{#2}}}

% pairs
\newcommand{\EPair}[2]{\ensuremath{\ECPair{#1}{#2}}}
\newcommand{\EProjL}[1]{\ensuremath{\ECProjL{#1}}}
\newcommand{\EProjR}[1]{\ensuremath{\ECProjR{#1}}}

% let
\newcommand{\ELet}[3]{\ensuremath{\ECLet{#1}{#2}{#3}}}

%
% marked expressions
%
\newcommand{\ECMName}{{\normalfont\textsf{MExp}}}
\newcommand{\ECMSet}{\ensuremath{\check{E}}}
\newcommand{\ECMV}{\ensuremath{\check{e}}}
\newcommand{\ELMVName}{{\normalfont\textsf{MChildExp}}}
\newcommand{\ELMV}{\check{\bar{e}}}

% holes
\definecolor{hole}{RGB}{162,85,162}
\newcommand{\ECEHole}{\ensuremath{\textcolor{hole}{\bm{\llparenthesis}}\textcolor{hole}{\bm{\rrparenthesis}}}}

% integers
\newcommand{\ECNum}[1]{\ensuremath{\underline{#1}}}
\newcommand{\ECNumMV}{\ensuremath{\ECNum{n}}}
\newcommand{\ECOpPlus}{\ensuremath{+}}
\newcommand{\ECPlus}[2]{\ensuremath{#1 \ECOpPlus #2}}

% booleans
\newcommand{\ECTrue}{\ensuremath{{\normalfont\textsf{tt}}}}
\newcommand{\ECFalse}{\ensuremath{{\normalfont\textsf{ff}}}}
\newcommand{\ECIf}[3]{\ensuremath{\textsf{if}~ #1 ~\textsf{then}~ #2 ~\textsf{else}~ #3}}

% lambdas
\newcommand{\ECLam}[3]{\ensuremath{\lambda #1 : #2. ~#3}}
\newcommand{\ECAp}[2]{\ensuremath{#1 ~#2}}

% pairs
\newcommand{\ECPair}[2]{\ensuremath{(#1, #2)}}
\newcommand{\ECProjL}[1]{\ensuremath{\pi_1 #1}}
\newcommand{\ECProjR}[1]{\ensuremath{\pi_2 #1}}

% let
\newcommand{\ECLet}[3]{\ensuremath{\textsf{let}~ #1 = #2 ~\textsf{in}~ #3}}

% errors
\newcommand{\MRFree}{\ensuremath{\mathsmaller{\mathsmaller{\mathsmaller{\square}}}}}
\newcommand{\MRInconType}{\ensuremath{\mathsmaller{\inconsistentRel}}}
\newcommand{\MRGraphErase}{\ensuremath{\mathsmaller{\graphErase}}}
\newcommand{\MRInconBr}{\ensuremath{\mathlarger{\noMeetRel}}}
\newcommand{\MRInconAsc}{\ensuremath{\bm{:}}}
\newcommand{\MRSynNonMatchedArrow}{\ensuremath{\mathsmaller{\notMatchedArrowRel}}}
\newcommand{\MRSynNonMatchedProd}{\ensuremath{\mathsmaller{\notMatchedProdRel}}}
\newcommand{\MRAnaNonMatchedArrow}{\ensuremath{\mathsmaller{\notMatchedArrowRel}}}
\newcommand{\MRAnaNonMatchedProd}{\ensuremath{\mathsmaller{\notMatchedProdRel}}}
\newcommand{\MRFreeTypeVar}{\ensuremath{\MRFree}}

\newcommand{\MRSyn}{\ensuremath{\mathsmaller{\Rightarrow}}}
\newcommand{\MRAna}{\ensuremath{\mathsmaller{\Leftarrow}}}

\newcommand{\ECMarked}[3]{\ensuremath{\textcolor{red}{\bm{\llparenthesis}#1\bm{\rrparenthesis}_{#2}^{#3}}}}
\newcommand{\ECMarkedFree}[1]{\ensuremath{\ECMarked{#1}{\MRFree}{}}}
\newcommand{\ECMarkedInconType}[1]{\ensuremath{\ECMarked{#1}{\MRInconType}{}}}
\newcommand{\ECMarkedInconBr}[1]{\ensuremath{\ECMarked{#1}{\MRInconBr}{}}}
\newcommand{\ECMarkedInconAsc}[1]{\ensuremath{\ECMarked{#1}{\MRInconAsc}{}}}
\newcommand{\ECMarkedSynNonMatchedArrow}[1]{\ensuremath{\ECMarked{#1}{\MRSynNonMatchedArrow}{\MRSyn}}}
\newcommand{\ECMarkedSynNonMatchedProd}[1]{\ensuremath{\ECMarked{#1}{\MRSynNonMatchedProd}{\MRSyn}}}
\newcommand{\ECMarkedAnaNonMatchedArrow}[1]{\ensuremath{\ECMarked{#1}{\MRAnaNonMatchedArrow}{\MRAna}}}
\newcommand{\ECMarkedAnaNonMatchedProd}[1]{\ensuremath{\ECMarked{#1}{\MRAnaNonMatchedProd}{\MRAna}}}

\newcommand{\ECFree}[1]{\ensuremath{\ECMarkedFree{#1}}}
\newcommand{\ECInconType}[1]{\ensuremath{\ECMarkedInconType{#1}}}
\newcommand{\ECInconBr}[3]{\ensuremath{\ECMarkedInconBr{\ECIf{#1}{#2}{#3}}}}
\newcommand{\ECInconAsc}[1]{\ensuremath{\ECMarkedInconAsc{#1}}}
\newcommand{\ECSynNonMatchedArrow}[1]{\ensuremath{\ECMarkedSynNonMatchedArrow{#1}}}
\newcommand{\ECSynNonMatchedProd}[1]{\ensuremath{\ECMarkedSynNonMatchedProd{#1}}}
\newcommand{\ECAnaNonMatchedArrow}[1]{\ensuremath{\ECMarkedAnaNonMatchedArrow{#1}}}
\newcommand{\ECAnaNonMatchedProd}[1]{\ensuremath{\ECMarkedAnaNonMatchedProd{#1}}}
\newcommand{\ECLamGraphErasure}[1]{\ECMarked{#1}{\triangle^{\times}}{}}

\newcommand{\ECLamInconAsc}[3]{\ensuremath{\ECInconAsc{\ECLam{#1}{#2}{#3}}}}
\newcommand{\ECApSynNonMatchedArrow}[2]{\ensuremath{\ECAp{\ECSynNonMatchedArrow{#1}}{#2}}}
\newcommand{\ECProjLSynNonMatchedProd}[1]{\ensuremath{\ECProjL{\ECSynNonMatchedProd{#1}}}}
\newcommand{\ECProjRSynNonMatchedProd}[1]{\ensuremath{\ECProjR{\ECSynNonMatchedProd{#1}}}}
\newcommand{\ECLamAnaNonMatchedArrow}[3]{\ensuremath{\ECAnaNonMatchedArrow{\ECLam{#1}{#2}{#3}}}}
\newcommand{\ECPairAnaNonMatchedProd}[2]{\ensuremath{\ECAnaNonMatchedProd{\ECPair{#1}{#2}}}}

% !requires types

%
% typing
%

% unmarked
\newcommand{\colorUText}{PineGreen!90}
\newcommand{\colorUBkgSyn}{Gray!5}
\newcommand{\colorUBkgAna}{Gray!5}

\newcommand{\withCtxU}[2]{\ensuremath{#1 \goodcolor{\colorUText}{\vdash_{\hspace{-3.25pt}\scaleto{U}{3.1pt}}} #2}}
\newcommand{\synTypeU}[2]{\ensuremath{#1 \goodcolor{\colorUText}{\Rightarrow} #2}}
\newcommand{\notSynTypeU}[2]{\ensuremath{#1 \not\Rightarrow #2}}
\newcommand{\ctxSynTypeU}[3]{\ensuremath{\withCtxU{#1}{\synTypeU{#2}{#3}}}}
\newcommand{\ctxNotSynTypeU}[3]{\ensuremath{\withCtxU{#1}{\notSynTypeU{#2}{#3}}}}

\newcommand{\anaTypeU}[2]{\ensuremath{#1 \goodcolor{\colorUText}{\Leftarrow} #2}}
\newcommand{\notAnaTypeU}[2]{\ensuremath{#1 \goodcolor{\colorFailSideJudge}{\not\Leftarrow} #2}}
\newcommand{\ctxAnaTypeU}[3]{\ensuremath{\withCtxU{#1}{\anaTypeU{#2}{#3}}}}
\newcommand{\ctxNotAnaTypeU}[3]{\ensuremath{\withCtxU{#1}{\notAnaTypeU{#2}{#3}}}}

% marked
\newcommand{\colorMText}{DarkOrchid}
\newcommand{\colorMBkgSyn}{Gray!5}
\newcommand{\colorMBkgAna}{Gray!5}

\newcommand{\withCtxM}[2]{\ensuremath{#1 \goodcolor{\colorMText}{\vdash_{\hspace{-3.25pt}\scaleto{M}{3.1pt}}} #2}}
\newcommand{\synTypeM}[2]{\ensuremath{#1 \goodcolor{\colorMText}{\Rightarrow} #2}}
\newcommand{\notSynTypeM}[2]{\ensuremath{#1 \not\Rightarrow #2}}
\newcommand{\ctxSynTypeM}[3]{\ensuremath{\withCtxM{#1}{\synTypeM{#2}{#3}}}}
\newcommand{\ctxNotSynTypeM}[3]{\ensuremath{\withCtxM{#1}{\notSynTypeM{#2}{#3}}}}

\newcommand{\anaTypeM}[2]{\ensuremath{#1 \goodcolor{\colorMText}{\Leftarrow} #2}}
\newcommand{\notAnaTypeM}[2]{\ensuremath{#1 \goodcolor{\colorFailSideJudge}{\not\Leftarrow} #2}}
\newcommand{\ctxAnaTypeM}[3]{\ensuremath{\withCtxM{#1}{\anaTypeM{#2}{#3}}}}
\newcommand{\ctxNotAnaTypeM}[3]{\ensuremath{\withCtxM{#1}{\notAnaTypeM{#2}{#3}}}}

% marking
\newcommand{\colorMKText}{Bittersweet}
\newcommand{\colorMKBkgSyn}{Gray!5}
\newcommand{\colorMKBkgAna}{Gray!5}

\newcommand{\withCtxMK}[2]{\ensuremath{#1 \goodcolor{\colorMKText}{\vdash} #2}}
\newcommand{\synTypeMK}[2]{\ensuremath{#1 \goodcolor{\colorMKText}{\Rightarrow} #2}}
\newcommand{\anaTypeMK}[2]{\ensuremath{#1 \goodcolor{\colorMKText}{\Leftarrow} #2}}
\newcommand{\synFixedInto}[3]{\ensuremath{\synTypeMK{#1 \goodcolor{\colorMKText}{\looparrowright} #2}{#3}}}
\newcommand{\ctxSynFixedInto}[4]{\ensuremath{{\withCtxMK{#1}{\synFixedInto{#2}{#3}{#4}}}}}
\newcommand{\anaFixedInto}[3]{\ensuremath{\anaTypeMK{#1 \goodcolor{\colorMKText}{\looparrowright} #2}{#3}}}
\newcommand{\ctxAnaFixedInto}[4]{\ensuremath{\withCtxMK{#1}{\anaFixedInto{#2}{#3}{#4}}}}

%
% judgments
%

% subsumable
\newcommand{\subsumable}[1]{\ensuremath{#1 ~{\normalfont\textsf{subsumable}}}}

% markless
\newcommand{\markless}[1]{\ensuremath{#1 ~{\normalfont\textsf{markless}}}}

% mark erasure
% \newcommand{\markErase}[1]{\ensuremath{#1^{\square}}}
\newcommand{\erasesTo}[2]{\ensuremath{#1^{\square} = #2}}




%
% judgments
%

% judgments
\newcommand{\judgment}[3]{\inferrule[#3]{#1}{#2}}
\newcommand{\judgbox}[1]{\noindent \fbox{$#1$}}


% constraints
\newcommand{\constrain}[2]{{#1} \approx {#2}}
\newcommand{\constraintCons}[2]{\normalfont\textsf{cons}(#1, #2)}
\newcommand{\constraintNil}{\{\}}
\newcommand{\maps}[2]{{#1} \mapsto {#2}}
\newcommand{\mapNil}[0]{\cdot}

\newcommand{\matchedArrowConstraint}[4]{
    \matchedArrow{#1}{#2}{#3}\goodcolor{\colorSideJudge}{~|~} {#4}
}
\newcommand{\matchedProdConstraint}[4]{
    \matchedProd{#1}{#2}{#3}\goodcolor{\colorSideJudge}{~|~} {#4}
}

\newcommand{\constraintTurn}{\goodcolor{\colorCText}{\vdash}}
\newcommand{\constraintBar}{\goodcolor{\colorCText}{~|~}}
\newcommand{\constraintSyn}{\goodcolor{\colorCText}{\Rightarrow}}
\newcommand{\constraintAna}{\goodcolor{\colorCText}{\Leftarrow}}
\newcommand{\synConstraint}[5]{
    \begin{mybox}{\colorCBkgSyn}
    \ensuremath{{#1} \constraintTurn {#2} \constraintSyn {#3} \constraintBar {#4} \constraintBar {#5}}
    \end{mybox}
}
\newcommand{\anaConstraint}[5]{
    \begin{mybox}{\colorCBkgAna}
    \ensuremath{{#1} \constraintTurn {#2} \constraintAna {#3} \constraintBar {#4} \constraintBar {#5}}
    \end{mybox}
}
\newcommand{\synMarkConstraint}[4]{
    \begin{mybox}{\colorCBkgSyn}
      \ensuremath{{#1} \constraintTurn {#2} \goodcolor{\colorMKText}        {\looparrowright} {#3}
      \constraintSyn {#4}}
    \end{mybox}
}
\newcommand{\anaMarkConstraint}[4]{
    \begin{mybox}{\colorCBkgAna}
      \ensuremath{{#1} \constraintTurn {#2} \goodcolor{\colorMKText}{\looparrowright} {#3}
    \constraintAna {#4}}
    \end{mybox}
}

\newcommand{\mConstraint}[0]{M}
\newcommand{\cConstraint}[0]{C}


\newcommand{\meet}[2]{{#1} \sqcap {#2}}


% type incomplete
\newcommand{\incomplete}[1]{#1 ~ \normalfont\textsf{incomplete}}

% potential type
\newcommand{\PTUnknownVar}[1]{\ensuremath{#1}}
\newcommand{\PTNum}{\ensuremath{{\normalfont\textsf{N}}}}
\newcommand{\PTBool}{\ensuremath{{\normalfont\textsf{B}}}}
\newcommand{\PTArrow}[2]{\ensuremath{#1 \to #2}}

% potential types
\newcommand{\ptypCons}[2]{\normalfont\textsf{cons}(#1, #2)}
\newcommand{\ptypSingle}[1]{\normalfont\textsf{single}(#1)}

% potential type set
\newcommand{\ptsRep}[1]{\normalfont\textsf{rep}(#1)}
\newcommand{\ptsLead}[1]{\normalfont\textsf{lead}(#1)}

% PTGraph Elements and Membership
\newcommand{\ptypGraphNode}[2]{#1\normalfont\textsf{ : }#2}
\newcommand{\underGraph}[2]{#1~|~#2}
\DeclareMathOperator{\cod}{cod}
\newcommand{\isIn}[2]{#1 \in #2}
\newcommand{\isNotIn}[2]{#1 \notin #2}

% PTGraph Update
\newcommand{\ptypGraphUpdate}[4]{
    \underGraph{#1}{\normalfont\textsf{update}~#2:#3 ~\hookrightarrow~ #4}
}

% PTGraph add node
\newcommand{\ptypGraphAdd}[3]{
    \underGraph{#1}{\normalfont\textsf{add}~#2 ~\hookrightarrow~ #3}
}

% PTSet Leader
\newcommand{\leader}[4]{
    \underGraph{#1}{\normalfont\textsf{leader}~#2 \equiv \ptypGraphNode{#3}{#4}}
}

% Unions
\newcommand{\unionOf}[3]{#2 \cup_{#1} #3}

%   Type Union
\newcommand{\typUnion}[2]{\unionOf{\tau}{#1}{#2}}
\newcommand{\ptypGraphUnion}[4]{
    \underGraph{#1}{\typUnion{#2}{#3}} ~\hookrightarrow~ #4
}

% PTyps Union
\newcommand{\ptypsUnion}[2]{#1~\Cup~#2}
\newcommand{\specSubset}[3]{#1~\Subset_{#3}~#2}
\newcommand{\ptypesSubset}[2]{\specSubset{#1}{#2}{u}}
\newcommand{\ptypSetSubset}[3]{\underGraph{#1}{\specSubset{#2}{#3}{s}}}
\newcommand{\snapSubset}[2]{\specSubset{#1}{#2}{j}}
\newcommand{\snapElement}[2]{#1 ~\in_{h}~ #2}

% unification
\newcommand{\unify}[3]{\underGraph{#1}{#2~\amalg~#3}}

% Snapshot PTS
\newcommand{\ptsSnapshot}[3]{\underGraph{#1}{\normalfont\textsf{snapshot}_s~#2 \equiv #3}}

% Snapshot List
\newcommand{\listSnapshot}[3]{\underGraph{#1}{\normalfont\textsf{snapshot}_l~#2 \equiv #3}}

% Solution
\newcommand{\solution}[2]{\normalfont\textsf{solution}~#1 \equiv #2}

% List
\newcommand{\listCons}[2]{\normalfont\textsf{cons}(#1, #2)}
\newcommand{\listNil}{\normalfont\textsf{nil}}

% fresh
\newcommand{\markFresh}[1]{#1~\normalfont\textsf{fresh}}
